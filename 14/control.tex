\secrel{14 Control Operations 102}\secdown

The term control refers to any programming language instruction that causes
evaluation to proceed, because it “controls” the program counter of the machine.
In that sense, even a simple arithmetic expression should qualify as “control”,
and operations such as sequential program execution, or function calls and
returns, most certainly do. However, in practice we use the term to refer
primarily to those operations that cause non-local transfer of control,
especially beyond that of mere functions and procedures, and the next step up,
namely exceptions. We will study such operations in this chapter.

As we study the following control operators, it’s worth remembering that even
without them, we still have languages that are Turing-complete, and therefore
have no more “power”. Therefore, what control operators do is change and
potentially improve the way we express our intent, and therefore enhance the
structure of programs. Thus, it pays to being our study by focusing on program
structure.

\input{14/1/onweb}
\input{14/2/contpass}
\secrel{14.3 Generators   . . 117}
\secdown
\secrel{14.3.1 Design Variations   117}
\secrel{14.3.2 Implementing Generators  . . 119}
\secup

\secrel{14.4 Continuations and Stacks   121}

Surprising as it may seem, CPS conversion actually provides tremendous insight
into the nature of the program execution stack. The first thing to understand is
that every continuation is actually the stack itself. This might seem odd, given
that stacks are low-level machine primitives while continuations are seemingly
complex procedures. But what is the stack, really?
\begin{itemize}
  \item 
It’s a record of what remains to be done in the computation. So is the
continuation.
  \item 
It’s traditionally thought of as a list of stack frames. That is, each frame has
a reference to the frames remaining after it finishes. Similarly, each
continuation is a small procedure that refers to—and hence closes over—its own
continuation. If we had chosen a different representation for program
instructions, combining this with the data structure representation of closures,
we would obtain a continuation representation that is essentially the same as
the machine stack.
  \item 
Each stack frame also stores procedure parameters. This is implicitly managed by
the procedural representation of continuations, whereas this was done explicitly
in the data stucture representation (using bind).
  \item 
Each frame also has space for “local variables”. In principle so does the
continuation, though by using the macro implementation of local binding, we’ve
effectively reduced everything to procedure parameters. Conceptually, however,
some of these are “true” procedure parameters while others are local bindings
turned into procedure parameters by a macro.
  \item 
The stack has references to, but does not close over, the heap. Thus changes to
the heap are visible across stack frames. In precisely the same way, closures
refer to, but do not close over, the store, so changes to the store are visible
across closures.
\end{itemize}
Therefore, traditionally the stack is responsible for maintaining lexical scope,
which we get automatically because we are using closures in a statically-scoped
language.

Now we can study the conversion of various terms to understand the mapping to
stacks. For instance, consider the conversion of a function application \ref{}:
\lsts{14/4/1.rkt}{rkt}
How do we ``read'' this? As follows:
\begin{itemize}
  \item 
Let’s use k to refer to the stack present before the function application begins
to evaluate.
  \item 
When we begin to evaluate the function position (f), create a new stack frame
((lambda (fv) ...)). This frame has one free identifier: k. Thus its closure
needs to record one element of the environment, namely the rest of the stack.
  \item 
The code portion of the stack frame represents what is left to be done once we
obtain a value for the function: evaluate the argument, and perform the
application, and return the result to the stack expecting the result of the
application: k.
  \item 
When evaluation of f completes, we begin to evaluate a, which also creates a
stack frame: (lambda (av) ...). This frame has two free identifiers: k and fv.
This tells us:
\begin{itemize}
  \item 
We no longer need the stack frame for evaluating the function position, but
  \item 
we now need a temporary that records the value—hopefully a function value—of
evaluating the function position.
\end{itemize}
  \item 
The code portion of this second frame also represents what is left to be done:
invoke the function value with the argument, in the stack expecting the value of
the application.
\end{itemize}
Let us apply similar reasoning to conditionals:
\lsts{14/4/2.rkt}{rkt}
It says that to evaluate the conditional expression we have to create a new
stack frame. This frame closes over the stack expecting the value of the entire
conditional. This frame makes a decision based on the value of the conditional
expression, and invokes one of the other expressions. Once we have examined this
value the frame created to evaluate the conditional expression is no longer
necessary, so evaluation can proceed in k.

Viewed through this lens, we can more easily provide an operational explanation
for generators. Each generator has its own private stack, and when execution
attempts to return past its end, our implementation raises an error. On
invocation, a generator stores a reference to the stack of the “rest of the
program” in where-to-go, and resumes its own stack, which is referred to by
resumer. On yielding, the system swaps references to stacks. Coroutines,
threads, and generators are all conceptually similar: they are all mechanisms to
create “many little stacks” instead of having a single, global stack.

\secrel{14.5 Tail Calls   . . 123}

Observe that the stack patterns above add a frame to the current stack, perform
some evaluation, and eventually always return to the current stack. In
particular, observe that in an application, we need stack space to evaluate the
function position and then the arguments, but once all these are evaluated, we
resume computation using the stack we started out with before the application.
In other words, function calls do not themselves need to consume stack space: we
only need space to compute the arguments.

However, not all languages observe or respect this property. In languages that
do, programmers can use recursion to obtain iterative behavior: i.e., a sequence
of function calls can consume no more stack space than no function calls at all.
This removes the need to create special looping constructs; indeed, loops can
simply be expressed as a syntactic sugar.

Of course, this property does not apply in general. If a call to f is performed
to compute an argument to a call to g, the call to f is still consuming space
relative to the context surrounding g. Thus, we should really speak of a
relationship between expressions: one expression is in tail position relative to
another if its evaluation requires no additional stack space beyond the other.
In our CPS macro, every expression that uses k as its continuation—such as a
function application after all the sub-expressions have been evaluated, or the
then- and else-branches of a conditional—are all in tail position relative to
the enclosing application (and perhaps recursively further up). In contrast,
every expression that has to create a new stack frame is not in tail position.

Some languages have special support for tail recursion: when a procedure calls
itself in tail position relative to its body. This is obviously useful, because
it enables recursion to efficiently implement loops. However, it hurts “loops”
that cannot be squeezed into a single recursive function. For instance, when
implementing a scanner or other state machine, it is most convenient to have a
set of functions each representing one state, and transitioning to other states
by making (tail) function calls. It is onerous (and misses the point) to turn
these into a single recursive function. If, however, a language recognizes tail
calls as such, it can optimize these cross-function calls just as much as it
does intra-function ones.

Racket, in particular, promises to implement tail calls without allocating
additional stack space. Though some people refer to this as “tail call
optimization”, this term is misleading: an optimization is optional, whereas
whether or not a language promises to properly implement tail calls is a
semantic feature. Developers need to know how the language will behave because
it affects how they program.

Because of this feature, observe something interesting about the program after
CPS transformation: all of its function applications are themselves tail calls!
You can see this starting with the read-number/suspend example that began this
chapter: any pending computation was put into the continuation argument.
Assuming the program might terminate at any call is tantamount to not using any
stack space at all (because the stack would get wiped out).

\Exercise{
How is the program able to function in the absence of a stack?
}

\secrel{14.6 Continuations as a Language Feature  124}\secdown

With this insight into the connection between continuations and stacks, we can
now return to the treatment of procedures: we ignored the continuation at the
point of closure creation and instead only used the one at the point of closure
invocation. This of course corresponds to normal procedure behavior. But now we
can ask, what if we use the creation-time continuation instead? This would
correspond to maintaining a reference to (a copy of) the stack at the point of
“procedure” creation, and when the procedure is applied, ignoring the dynamic
evaluation and going back to the point of procedure creation.

In principle, we are trying to leave lambda intact and instead give ourselves a
language construct that corresponds to this behavior:
\note{cc = “current continuation”}
\lsts{14/6/1.rkt}{rkt}

What this says is that either way, control will return to the expression that
immediately surrounds the let/cc: either by falling through (because the
continuation of the body, b, is k) or—more interestingly—by invoking the
continuation, which discards the dynamic continuation dyn/k by simply ignoring
it and returning to k instead.

Here’s the simplest test:
\lsts{14/6/2.rkt}{rkt}
This confirms that if we never use the continuation, evaluation of the body
proceeds as if the let/cc weren’t there at all (because of the ((cps b) k)). If
we use it, the value given to the continuation returns to the point of creation:
\lsts{14/6/3.rkt}{rkt}
This example, of course, isn’t revealing, but consider this one:
\lsts{14/6/4.rkt}{rkt}
This confirms that the addition actually happens. But what about the dynamic
continuation?
\lsts{14/6/5.rkt}{rkt}
This shows that the addition by 2 never happens, i.e., the dynamic continuation
is indeed ignored. And just to be sure that the continuation at the point of
creation is respected, observe:
\lsts{14/6/6.rkt}{rkt}

From these examples, you have probably noticed a familiar pattern: esc here is
behaving like an exception. That is, if you do not throw an exception (in this
case, invoke a continuation) it’s as if it’s not there, but if you do throw it,
all pending intermediate computation is ignored and computation returns to the
point of exception creation.

\Exercise{
Using let/cc and macros, create a throw/catch mechanism.
}

However, these examples only scratch the surface of available power, because the
continuation at the point of invocation is always an extension of one at the
point of creation: i.e., the latter is just earlier in the stack than the
former. However, nothing actually demands that k and dyn-k be at all related.
That means they are in fact free to be unrelated, which means each can be a
distinct stack, so we can in fact easily implement stack-switching procedures
with them.

\Exercise{
To be properly analogous to lambda, we should have introduced a construct
called, say, cont-lambda with the following expansion:
\lsts{14/6/7.rkt}{rkt}
Why didn’t we? Consider both the static typing implications, and also how we
might construct the above exception-like behaviors using this construct instead.
}

\secrel{14.6.1 Presentation in the Language  125}

\secrel{14.6.2 Defining Generators  . 126}

\secrel{14.6.3 Defining Threads   127}

Having done generators, let’s do another, similar primitive: threads. That is,
let’s assume we want to be able to write a program such as this:
\lsts{14/6/3/1.rkt}{rkt}
and expect the output
\lst{14/6/3/2.rkt}
We’ll build all the pieces necessary to achieve this.

Let’s start by defining the thread scheduler. It consumes a list of “threads”,
whose interface we assume will be a procedure that consumes a continuation to
which it eventually yields control. Each time the scheduler reactivates the
thread, it supplies it with a continuation. The scheduler might be choose
between threads in a simple round-robin manner, or it might use some more
complex algorithm; the details of how it chooses don’t concern us here.

As with generators, we’ll assume that yielding is done by invoking a procedure
named by the user: y, above. We could use name capture \ref{}\ to automatically
bind a name like yield.

More importantly, notice that the user of the thread system manually yields
control. This is called cooperative multitasking. Instead, we could have chosen
to have a timer or other intrinsic mechanism automtically yield without the
user’s permission; this is called preemptive multitasking (because the system
“pre-empts”—i.e., wrests control from—the thread). While the distinction is
important for buildling systems, it is not interesting from the perspective of
setting up the continuations.

\Exercise{
After we are done building cooperative multitasking, implement preemptive
multitasking. What changes?
}

With our stated constraints, we can write a first scheduler. It consumes a lists
of threads and continues executing so long as there are threads remaining. Each
time, it applies the thread procedure to a continuation that represents
returning to the scheduler and proceeding to the next thread:
\lsts{14/6/3/3.rkt}{rkt}
When the recipient thread invokes the continuation bound to after-thread,
control returns to the end of the first statement in the begin sequence. As a
result, the value supplied to the continuation is ignored (and can hence be any
dummy value; we’ll chose 'dummy, so that we can easily spot it if it shows up in
undesired places). Control then proceeds with the rest of the scheduler loop
after appending the most recently invoked thread to the end of the list of
threads (i.e., treating the list as a circular queue).

Now let’s define a thread. As we’ve said, it will be a procedure of one
argument, the scheduler’s continuation. Because the thread needs to resume,
i.e., continue where it left off, presumably it must store where it last was:
we’ll call this thread-resumer. Initially this is the entire thread body, but on
subsequent instances it will be a continuation: specifically, the continuation
of invoking yield. Thus, we obtain the following skeleton:
\lsts{14/6/3/4.rkt}{rkt}
That still leaves the yielder. It needs to be a procedure of no arguments that
stores the thread’s continuation in thread-resumer, and then invoke the
scheduler continuation with 'dummy. However, which scheduler continuation does
it need to invoke? Not the one provided on thread initiation, but rather the
most recent one. Thus, we must somehow “thread” the value in sched-k to the
yielder. There are many ways to accomplish it, but the simplest, perhaps most
brutal, way is to simply reconstruct the yielder on each thread resumption,
always closed over the most recent value of sched-k:
\lsts{14/6/3/5.rkt}{rkt}
When we run this ensemble, we get:
\lst{14/6/3/6.rkt}
Hey, that’s what we wanted! But it continues:
\lst{14/6/3/7.rkt}
Hmmm.

What’s happening? Well, we’ve been quiet all along about what needs to happen
when a thread reaches its end. In fact, control just returns to the thread
scheduler, which appends the thread to the end of the queue, and when the thread
comes to the head of the queue again, control resumes from the same previously
stored continuation: the one corresponding to printing the third value. This
prints, control returns, the thread is appended...ad infinitum.

Clearly, we need the thread scheduler to be notified when a thread has
terminated, so that the scheduler can remove it from the thread queue. We’ll
create a simple datatype to represent this signal:
\lsts{14/6/3/8.rkt}{rkt}
(In a real system, of course, these status messages might also carry informative
values from the computation.) We must now modify our scheduler to actually check
for and use these values:
\lsts{14/6/3/9.rkt}{rkt}
We have to now modify our thread representation in two ways: it must provide
Tsus- pended to the scheduler’s continuation on intermediate returns, and
provide Tdone when it terminates. Where does it terminate? After executing the
code in the body, b .... Observe, finally, that the termination process must be
sure to use the latest scheduler continuation, just as yielding does. Thus:
\lsts{14/6/3/10.rkt}{rkt}
If we now replace scheduler-loop-0 with scheduler-loop-1 and thread-0 with
thread-1 and re-run our example program above, we get just the output we desire.

\secrel{14.6.4 Better Primitives for Web Programming  131}

Finally, to tie the knot back to where we began, let’s return to read-number:
observe that, if the language running the server program has call/cc, instead of
having to CPS the entire program, we can simply capture the current continuation
and save it in the hash table, leaving the program structure again intact.

\secup

\secup
