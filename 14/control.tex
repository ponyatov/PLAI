\secrel{14 Control Operations 102}\secdown

The term control refers to any programming language instruction that causes
evaluation to proceed, because it “controls” the program counter of the machine.
In that sense, even a simple arithmetic expression should qualify as “control”,
and operations such as sequential program execution, or function calls and
returns, most certainly do. However, in practice we use the term to refer
primarily to those operations that cause non-local transfer of control,
especially beyond that of mere functions and procedures, and the next step up,
namely exceptions. We will study such operations in this chapter.

As we study the following control operators, it’s worth remembering that even
without them, we still have languages that are Turing-complete, and therefore
have no more “power”. Therefore, what control operators do is change and
potentially improve the way we express our intent, and therefore enhance the
structure of programs. Thus, it pays to being our study by focusing on program
structure.

\secrel{14.1 Control on the Web   102}

Let us begin our study by examining the structure of Web programs. Consider the
following program:
\note{ Henceforth, we’ll call this our “addition server”.
You should, of course, understand this as a stand-in for more sophisticated
applications. For instance, the two prompts might ask for starting and ending
points for a trip, and in place of addition we might compute a route or compute
airfares. There might even be computation between the two steps: e.g., after
entering the first city, the airline might prompt us with choices of where it
flies from there.}
\lsts{14/1/1.rkt}{rkt}
To test these ideas, here’s an implementation of read-number:
\lsts{14/1/2.rkt}{rkt}
When run at the console or in DrRacket, this program prompts us for one number,
then another, and then displays their sum.

Now suppose we want to run this on a Web server. We immediately encounter a
difficulty: the structure of server-side Web programs is such that they generate
a single Web page—such as the one asking for the first number—and then halt. As
a result, the rest of the program—which in this case prompts for the second
number, then adds them, and then prints that result, is lost.

\DoNow{
Why do Web servers behave in such a strange way?
}

There are at least two reasons for this behavior: one perhaps historical, and
the other technical. The historical reason is that Web servers were initially
designed to serve pages, i.e., static content. Any program that ran had to
generate its output to a file, from which a server could offer it. Naturally,
developers wondered why that same program couldn’t run on demand. This made Web
content dynamic. Terminating the program after generating a single piece of
output was the simplest incremental step towards programs, not pages, on the
Web.

The more important reason—and the one that has stayed with us—is technical.
Imagine our addition server has generated its first prompt. Recall that there is
considerable pending computation: the second prompt, the addition, and the
display of the result. This computation must suspend waiting for the user’s
input. If there are millions of users, then millions of computations must be
suspended, creating an enormous performance problem. Furthermore, suppose a user
does not actually complete the computation—analogous to searching at an on-line
bookstore or airline site, but not completing the purchase. How does the server
know when or even whether to terminate the computation? And until it does, the
resources associated with that computation remain in use.

Conceptually, therefore, the Web protocol was designed to be stateless: it would
not store state on the server associated with intermediate computations.
Instead, Web program developers would be forced to maintain all necessary state
elsewhere, and each request would need to be able to resume the computation in
full. In practice, the Web has not proven to be stateless at all, but it still
hews largely in this direction, and studying the structure of such programs is
very instructive.

Now consider client-side Web programs: those that run inside the browser,
written in or compiled to JavaScript. Suppose such a computation needs to
communicate with a server. The primitive for this is called XMLHttpRequest. The
user makes an instance of this primitive and invokes its send method to send a
message to the server. Communicating with a server is not, however,
instantaneous (and indeed may never complete, depending on the state of the
network). This leaves the sending process suspended.

The designers of JavaScript decided to make the language single-threaded: i.e.,
there would be only one thread of execution at a time. This avoids the various
perils that arise from combining mutation with threads. As a result, however,
the JavaScript process locks up awaiting the response, and nothing else can
happen: e.g., other handlers on the page no longer respond.
\note{Due to the structuring problems this causes, there are now various
proposals to, in effect, add “safe” threads to JavaScript. The ideas described
in this chapter can be viewed as an alternative that offer similar structuring
benefits.}

To avoid this problem, the design of XMLHttpRequest demands that the developer
provide a procedure that responds to the request if and when it arrives. This
callback procedure is registered with the system. It needs to embody the rest of
the processing of that request. Thus, for entirely different reasons—not
performance, but avoiding the problems of synchronization, non-atomicity, and
deadlocks—the client-side Web has evolved to demand the same pattern of
developers. Let us now better understand that pattern.

\secdown
\secrel{14.1.1 Program Decomposition into Now and Later 104}

Let us consider what it takes to make our above program work in a stateless
setting, such as on a Web server. First we have to determine the first
interaction. This is the prompt for the first number, because Racket evaluates
arguments from left to right. It is instructive to divide the program into two
parts: what happens to generate the first interaction (which can all run now),
and what needs to happen after it (which must be “remembered” somehow). The
former is easy:
\lsts{14/1/1/1.rkt}{rkt}
We’ve already explained in prose what’s left, but now it’s time to write it as a
program. It seems to be something like
\note{We’re intentionally
ignoring
read-number for
now, but we’ll
return to it. For
now, let’s pretend
it’s built-in.}
\lsts{14/1/1/2.rkt}{rkt}
A Web server can’t execute the above, however, because it evidently isn’t a
program. We instead need some way of writing this as one.

Let’s observe a few characteristics of this computation:
\begin{itemize}[nosep]
  \item 
It needs to be a legitimate program.
  \item 
It needs to stay suspended until the request comes in.
  \item 
It needs a way—such as a parameter—to refer to the value from the first
interaction.
\end{itemize}

Put together these characteristics and we have a clear representation—a
function:
\lsts{14/1/1/3.rkt}{rkt}

\secrel{14.1.2 A Partial Solution   104}

On the Web, there is an additional wrinkle: each Web page with input elements
needs to refer to a program stored on the Web, which will receive the data from
the form and process it. This program is named in the action field of a form.
Thus, imagine that the server generates a fresh label, stores the above function
in a table associated with that label, and refers to the table in the action
field. If and when the client actually submits the form, the server extracts the
associated function, supplies it with the form’s values, and thus resumes
execution.

\DoNow{
Is the solution above stateless?
}

Let’s imagine that we have a custom Web server that maintains the above table.
In such a server, we might have a special version of read-number—call it read-
number/suspend—that records the rest of the program:
\lsts{14/1/2/1.rkt}{rkt}
To test this, let’s implement such a procedure. First, we need a representation
for labels; numbers are an easy substitute:
\lsts{14/1/2/2.rkt}{rkt}
Let’s say new-label generates a fresh label on each invocation.
\Exercise{
Define new-label. You might use new-loc for inspiration.
}

We need a table to store the procedures representing the rest of the program.
\lsts{14/1/2/3.rkt}{rkt}
Now we can store these procedures:
\lsts{14/1/2/4.rkt}{rkt}
If we now run the above invocation of read-number/suspend, the system prints
First number To enter it, use the action field label 1 This is tantamount to
printing the prompt in a Web page, and putting the label 1 in the action field.
Because we’re simulating it, we need something to represent the browser’s
submission process. This needs both the label (from the action field) and the
value entered in the form. Given these two values, this procedure needs to
extract the relevant procedure from the table, and apply it to the form value.
\lsts{14/1/2/5.rkt}{rkt}

With this, we can now simulate the act of entering 3 and clicking on a “Submit”
button by running:
\lst{14/1/2/6.rkt}
where 1 is the label and 3 is the user’s input. Unfortunately, this simply
produces another prompt, because we haven’t fully converted the program. If we
delete read- number, we’re forced to convert the entire program:
\lsts{14/1/2/7.rkt}{rkt}
Just to be safe, we can also make sure the computation terminates after each
output by adding an error invocation at the end of read-number/suspend (to truly
ensure the most extreme form of “suspension”).

When we execute this program, we have to use resume twice:
\lst{14/1/2/8.rkt}
where the two user inputs are 3 and 10, giving a total of 13, and the halting
messages are generated by the error command we inserted.

We’ve purposely played a little coy with the types of the interesting parts of
our program. Let’s examine what these types should be. The second argument to
read- number/suspend needs to be a procedure that consumes numbers and returns
whatever the computation eventually produces: (number -> 'a). Similarly, the
return type of resume is the same 'a. How do these 'as communicate with one
another? This is done by table, which maps labels to (number -> 'a). That is, at
every step the computation makes progress towards the same outcome.
read-number/suspend writes into this table, and resume reads from it.

\secrel{14.1.3 Achieving Statelessness  . . 106}

We haven’t actually achieved statelessness yet, because we have this large table
residing on the server, with no clear means to remove entries from it. It would
be better if we could avoid the server state entirely. This means we have to
move the relevant state to the client.

There are actually two ways in which the server holds state. One is that we have
reserved the right to create as many entries in the hash table as we wish,
rather than a constant number (i.e., linear in the size of the program itself).
The other is what we’re storing in the table: honest-to-goodness closures, which
might be holding on to an arbitrary amount of state. We’ll see this more clearly
soon.

Let’s start by eliminating the closure. Instead, let’s have each of the funtion
arguments to be named, top-level functions (which immediately forces us to have
only a fixed number of them, because the program’s size cannot be unbounded):
\lsts{14/1/3/1.rkt}{rkt}
Observe how each code block refers only to the name of the next, rather than to
a real closure. The value of the argument comes from the form. There’s just one
problem: v1 in prog2 is a free identifier!

The way to fix this problem is, instead of creating a closure after one step, to
send v1 to the client to be stored there. Where do we store this? The browser
offers two mechanisms for doing this: cookies and hidden fields. Which one do we
use?

\secrel{14.1.4 Interaction with State  . 107}

The fundamental difference between cookies and hidden fields is that all pages
share the same cookie, but each page has its own hidden fields.

First, let’s consider a sequence of interactions with the existing program that
uses read-number/suspend (at both interaction points). It looks like this:
\lsts{14/1/4/1.rkt}{rkt}
Thus, resuming with label 2 appears to represent adding 3 to the given argument
(i.e., form field value). To be sure,
\lsts{14/1/4/2.rkt}{rkt}
So far, so good. Now suppose we use label 1 again:
\lsts{14/1/4/3.rkt}{rkt}
Observe that this new program execution needs to be resumed by using label 3,
not 1.
Indeed,
\lsts{14/1/4/4.rkt}{rkt}
But we ought to ask, what happens if we reuse label 2?

\DoNow{
Try (resume 2 10).
}

Doing this is tantamount to resuming the old computation. We therefore expect it
produce the same answer as before:
\lsts{14/1/4/5.rkt}{rkt}

Now let’s create a stateful implementation. We can simulate this by observing
that each closure has its own environment, but all closures share the same
mutable state. We can simulate this using our existing read-number/suspend by
making sure we don’t rely on the closure behavior of lambda, i.e., by not having
any free identifiers in the body.
\lsts{14/1/4/6.rkt}{rkt}

\Exercise{
What do we expect for the same sequence as before?
}

\DoNow{
What happens?
}

Initially, nothing seems different:
\lsts{14/1/4/7.rkt}{rkt}
When we reuse the initial computation, we indeed get a new resumption label:
\lsts{14/1/4/8.rkt}{rkt}
which, when used, computes what we’d expect:
\lsts{14/1/4/9.rkt}{rkt}
Now we come to the critical step:
\lsts{14/1/4/10.rkt}{rkt}
It is unsurprising that the two resumptions of label 2 would produce different
answers, given that they rely on mutable state. The reason it’s problematic is
because of what happens when we translate the same behavior to the Web.

Imagine visiting a hotel reservation Web site and searching for hotels in a
city. In return, you are shown a list of hotels and the label 1. You explore one
of them in a new tab or window; this produces information on that hotel, and
label 2 to make the reservation. You decide, however, to return to the hotel
listing and explore another hotel in a fresh tab or window. This produces the
second hotel’s information, with label 3 to reserve at that hotel. You decide,
however, to choose the first hotel, return to the first hotel’s page, and choose
its reservation button—i.e., submit label 2. Which hotel did you expect to be
booked into? Though you expected a reservation at the first hotel, on most
travel sites, this will either reserve at the second hotel—i.e., the one you
last viewed, but not the one on the page whose reservation button you clicked—
or produce an error. This is because of the pervasive use of cookies on Web
sites, a practice encouraged by most Web APIs.

\secup

\secrel{14.2 Continuation-Passing Style  . . 109}

The functions we’ve been writing have a name. Though we’ve presented ideas in
terms of theWeb, we’re relying on a much older idea: the functions are called
continuations, and this style of programs is called continuation-passing style
(CPS). This is worth studying in its own right, because it is the basis for
studying a variety of other nontrivial control operations—such as generators.
\note{We will take the liberty of using CPS as both a noun and verb: a
particular structure of code and the process that converts code into it.}

Earlier, we converted programs so that no Web input operation was nested inside
another. The motivation was simple: when the program terminates, all nested
computations are lost. A similar argument applies, in a more local sense, in the
case of XMLHttpRequest: any computation depending on the result of a response
from aWeb server needs to reside in the callback associated with the request to
the server.

In fact, we don’t need to transform every expression. We only care about
expressions that involve actual Web interaction. For example, if we computed a
more complex mathematical expression than just addition, we wouldn’t need to
transform it. If, however, we had a function call, we’d either have to be
absolutely certain the function didn’t have any Web invocations either inside
it, or in the functions in invokes, or the ones they invoke...or else, to be
defensive, we should transform them all. Therefore, we have to transform every
expression that we can’t be sure performs noWeb interactions.

The heart of our transformation is therefore to turn every one-argument
function, f, into one with an extra argument. This extra argument is the
continuation, which represents the rest of the computation. The continuation is
itself a function of one argument. This argument takes the value that would have
been returned by f and passes it to the rest of the computation. f, instead of
returning a value, instead passes the value it would have returned to its
continuation.

CPS is a general transformation, which we can apply to any program. Because it’s
a program transformation, we can think of it as a special kind of desugaring: in
particular, instead of transforming programs from a larger language to a smaller
one (as macros do), or from one language to entirely another (as compilers do),
it transforms programs within the same language: from the full language to a
more restricted version that obeys the pattern we’ve been discussing. As a
result, we can reuse an evaluator for the full language to also evaluate
programs in the CPS subset.

\secdown
\secrel{14.2.1 Implementation by Desugaring  . . 110}

Because we already have good support for desguaring, let’s use to define the CPS
transform. Concretely, we’ll implement a CPS macro [REF]. To more cleanly
separate the source language from the target, we’ll use slightly different names
for most of the language constructs: a one-armed with and rec instead of let and
letrec; lam instead of lambda; cnd instead of if; seq for begin; and set for
set!. We’ll also give ourselves a sufficiently rich language to write some
interesting programs!
\note{The presentation that follows orders the cases of the macro from what I
believe are easiest to hardest. However, the code in the macro must avoid
non-overlapping patterns, and hence follows a diffent order.}
\lsts{14/2/1/1.rkt}{rkt}

Our representation in CPS will be to turn every expression into a procedure of
one argument, the continuation. The converted expression will eventually either
supply a value to the continuation or will pass the continuation on to some
other expression that will—by preserving this invariant inductively—supply it
with a value. Thus, all output from CPS will look like (lambda (k) ...) (and we
will rely on hygiene [REF] to keep all these introduced k’s from clashing with
one another).

First let’s dispatch with the easy case, which is atomic values. Though
conceptually easiest, we have written this last because otherwise this pattern
would shadow all the other cases. (Ideally, we should have written it first and
provided a guard expression that precisely defines the syntactic cases we want
to treat as atomic. We’re playing loose here becuase our focus is on more
interesting cases.) In the atomic case, we already have a value, so we simply
need to supply it to the continuation:
\lsts{14/2/1/2.rkt}{rkt}
Similarly for quoted constants:
\lsts{14/2/1/3.rkt}{rkt}

Also, we already know, from [REF] and [REF], that we can treat with and rec as
macros, respectively:
\lsts{14/2/1/4.rkt}{rkt}
\lsts{14/2/1/5.rkt}{rkt}

Mutation is easy: we have to evaluate the new value, and then perform the actual
update:
\lsts{14/2/1/6.rkt}{rkt}

Sequencing is also straightforward: we perform each operation in turn. Observe
how this preserves the semantics of sequencing: not only does it obey the order
of operations, the value of the first sub-term (e1) is not mentioned anywhere in
the body of the second (e2), so the name given to the identifier holding its
value is irrelevant.
\lsts{14/2/1/7.rkt}{rkt}

When handling conditionals, we need to create a new continuation to remember
that we are waiting for the test expression to evaluate. Once we have its value,
however, we can dispatch on the result and return to the existing continuations:
\lsts{14/2/1/8.rkt}{rkt}

When we get to applications, we have two cases to consider. We absolutely need
to handle the treatment of procedures created in the language: those with one
argument. For the purposes of writing example programs, however, it is useful to
be able to employ primitives such as + and *. Thus, we will assume for
simplicity that oneargument procedures are written by the user, and hence need
conversion to CPS, while two-argument ones are primitives that will not perform
any Web or other control operations and hence can be invoked directly; we will
also assume that the primitive will be written in-line (i.e., the application
position will not be a complex expression that can itself, say, perform a Web
interaction).

For an application we have to evaluate both the function and argument
expressions. Once we’ve obtained these, we are ready to apply the function.
Therefore, it is tempting to write
\lsts{14/2/1/9.rkt}{rkt}

\DoNow{
Do you see why this is wrong?
}

The problem is that, though the function is now a value, that value is a closure
with a potentially complicated body: evaluating the body can, for example,
result in further Web interactions, at which point the rest of the function’s
body, as well as the pending (k ...) (i.e., the rest of the program), will all
be lost. To avoid this, we have to supply k to the function’s value, and let the
inductive invariant ensure that k will eventually be invoked with the value of
applying fv to av:
\lsts{14/2/1/10.rkt}{rkt}

Treating the special case of built-in binary operations is easier:
\lsts{14/2/1/11.rkt}{rkt}

The very pattern we could not use for user-defined procedures we employ here,
because we assume that the application of f will always return without any
unusual transfers of control.

A function is itself a value, so it should be returned to the pending
computation. The application case above, however, shows that we have to
transform functions to take an extra argument, namely the continuation at the
point of invocation. This leaves us with a quandary: which continuation do we
supply to the body?
\lsts{14/2/1/12.rkt}{rkt}

That is, in place of ..., which continuation do we supply: k or dyn-k?

\DoNow{
Which continuation should we supply?
}

The former is the continuation at the point of closure creation. The latter is
the continuation at the point of closure invocation. In other words, the former
is “static” and the latter is “dynamic”. In this case, we need to use the
dynamic continuation, otherwise something very strange would happen: the program
would return to the point where the closure was created, rather than where it is
being used! This would result in seemingly very strange program behavior, so we
wish to avoid it. Observe that we are consciously choosing the dynamic
continuation just as, where scope was concerned, we chose the static
environment.
\lsts{14/2/1/13.rkt}{rkt}

Finally, for the purpose of modeling Web programming, we can add our input and
output procedures. Output follows the application pattern we’ve already seen:
\lsts{14/2/1/14.rkt}{rkt}

Finally, for input, we can use the pre-existing read-number/suspend, but this
time generate its uses rather than force the programmer to construct them:
\lsts{14/2/1/15.rkt}{rkt}

Notice that the continuation bound to k is precisely the continuation that we
need to stash at the point of a Web interaction.

Testing any code converted to CPS is slightly annoying because all CPS terms
expect a continuation. The initial continuation is one that simply either (a)
consumes a value and returns it, or (b) consumes a value and prints it, or (c)
consumes a value, prints it, and gets ready for another computation (as the
prompt in the DrRacket Interactions window does). All three of these are
effectively just the identity function in various guises. Thus, the following
definition is helpful for testing:
\lsts{14/2/1/16.rkt}{rkt}
For instance,
\lsts{14/2/1/17.rkt}{rkt}
We can also test our old Web program:
\lsts{14/2/1/18.rkt}{rkt}

Lest you get lost in the myriad of code, let me highlight the important lesson
here: We’ve recovered our code structure. That is, we can write the program in
direct style, with properly nested expressions, and a compiler—in this case, the
CPS converter— takes care of making it work with a suitable underlying API. This
is what good programming languages ought to do!

\secrel{14.2.2 Converting the Example  . . 114}

Let’s consider the example above and see what it converts to. You can either do
this by hand, or take the easy way out and employ the Macro Stepper of DrRacket.
Assuming we include the application to identity contained in run, we get:
\note{For now, you need to put the code in \#lang racket to get the full force
of the Macro Stepper.}
\lsts{14/2/2/1.rkt}{rkt}
What! This isn’t at all the version we wrote by hand!

In fact, this program is full of so-called administrative lambdas that were
introduced by the particular CPS algorithm we used. Fear not! If we stepwise
apply each of these lambdas and substitute, however\ ---
\note{Designing better
CPS algorithms, that
eliminate needless
administrative
lambdas, is
therefore an
ongoing and open
research question.}
\DoNow{Do it!}
--- the program reduces to
\lsts{14/2/2/2.rkt}{rkt}
which is precisely what we wanted.

\secrel{14.2.3 Implementation in the Core  115}

\secup

\secrel{14.3 Generators   . . 117}

Many programming languages now have a notion of generators. A generator is like
a procedure, in that one can invoke it in an application. Whereas a regular
procedure always begins execution at the beginning, a generator resumes from
where it last left off. Of course, that means a generator needs a notion of
“exiting before it’s done”. This is known as yielding, namely returning control
to whatever called it.

\secdown
\secrel{14.3.1 Design Variations   117}

\secrel{14.3.2 Implementing Generators  . . 119}

To implement generators, it will be especially useful to employ our CPS macro
language. Let’s first decide where we stand regarding the above design
decisions. We will use the applicative representation of generators: that is,
asking for the next value from the generator is done by applying it to any
necessary arguments. Similarly, the yielder will also be an applicable value and
will in turn be an expression. Though we have already seen how macros can
automatically capture a name \ref{}, let’s make the yielder’s name explicit to
keep the macro simpler. Finally, we’ll raise an error when the generator is done
executing.

How do generators work? To yield, a generator must
\begin{itemize}[nosep]
\item remember where in its execution it currently is, and
\item know where in its caller it should return to.
\end{itemize}
while, when invoked, it should
\begin{itemize}[nosep]
  \item 
remember where in its execution its caller currently is, and
  \item 
know where in its body it should return to.
\end{itemize}
Observe the duality between invocation and yielding.

As you might guess, these “where”s correspond to continuations.

Let’s build up the generator rule of the cps macro incrementally. First a header
pattern:
\lsts{14/3/2/1.rkt}{rkt}

The beginning of the body is easy: all code in CPS needs to consume a
continuation, and because a generator is a value, this value should be supplied
to the continuation:
\lsts{14/3/2/2.rkt}{rkt}

Now we’re ready to tackle the heart of the generator.

Recall that a generator is an applicable value. That means it can occur in an
application position, and must therefore have the same “interface” as a
procedure: a procedure of two arguments, the first a value and the second the
continuation at the point of application. What should this procedure do? We’ve
described just this above. First the generator must remember where the caller is
in its execution, which is precisely the continuation at the point of
application; “remember” here most simply means “must be stored in state”. Then,
the generator should return to where it previously was, i.e., its own
continuation, which must clearly have been stored. Therefore the core of the
applicable value is:
\lsts{14/3/2/3.rkt}{rkt}

Here, where-to-go records the continuation of the caller, to resume it upon
yielding; resumer is the local continuation of the generator. Let’s think about
what their initial values must be:
\begin{itemize}
  \item 
where-to-go has no initial value (because the generator has yet to be invoked),
so it needs to throw an error if ever used. Fortunately this error will never
occur, because where-to-go is mutated on the first entry into the generator, so
the error is just a safeguard against bugs in the implementation.
  \item 
Initially, the rest of the generator is the whole generator, so resumer should
be bound to the (CPS of) b. What is its continuation? This is the continuation
of the entire generator, i.e., what to do when the generator finishes. We’ve
agreed that this should also signal an error (except in this case the error
truly can occur, in case the generator is asked to produce more values than it’s
equipped to).
\end{itemize}

We still need to bind yield. It is, as we’ve pointed out, symmetric to generator
resumption: save the local continuation in resumer and return by applying
where-to-go.

Putting together these pieces, we get:
\lsts{14/3/2/4.rkt}{rkt}

\DoNow{
Why this pattern of let and letrec instead of let?
}

Observe the dependencies between these code fragments. where-to-go doesn’t
depend on either of resumer or yield. yield clearly depends on both where-to-go
and resumer. But why are resumer and yield mutually referential?

\DoNow{
Try the alternative!
}

The subtle dependency you may be missing is that resumer contains b, the body of
the generator, which may contain references to yield. Therefore, it needs to be
closed over the binding of the yielder.

\Exercise{
How do generators differ from coroutines and threads? Implement coroutines and
threads using a similar strategy.
}

\secup

\secrel{14.4 Continuations and Stacks   121}

Surprising as it may seem, CPS conversion actually provides tremendous insight
into the nature of the program execution stack. The first thing to understand is
that every continuation is actually the stack itself. This might seem odd, given
that stacks are low-level machine primitives while continuations are seemingly
complex procedures. But what is the stack, really?
\begin{itemize}
  \item 
It’s a record of what remains to be done in the computation. So is the
continuation.
  \item 
It’s traditionally thought of as a list of stack frames. That is, each frame has
a reference to the frames remaining after it finishes. Similarly, each
continuation is a small procedure that refers to—and hence closes over—its own
continuation. If we had chosen a different representation for program
instructions, combining this with the data structure representation of closures,
we would obtain a continuation representation that is essentially the same as
the machine stack.
  \item 
Each stack frame also stores procedure parameters. This is implicitly managed by
the procedural representation of continuations, whereas this was done explicitly
in the data stucture representation (using bind).
  \item 
Each frame also has space for “local variables”. In principle so does the
continuation, though by using the macro implementation of local binding, we’ve
effectively reduced everything to procedure parameters. Conceptually, however,
some of these are “true” procedure parameters while others are local bindings
turned into procedure parameters by a macro.
  \item 
The stack has references to, but does not close over, the heap. Thus changes to
the heap are visible across stack frames. In precisely the same way, closures
refer to, but do not close over, the store, so changes to the store are visible
across closures.
\end{itemize}
Therefore, traditionally the stack is responsible for maintaining lexical scope,
which we get automatically because we are using closures in a statically-scoped
language.

Now we can study the conversion of various terms to understand the mapping to
stacks. For instance, consider the conversion of a function application \ref{}:
\lsts{14/4/1.rkt}{rkt}
How do we ``read'' this? As follows:
\begin{itemize}
  \item 
Let’s use k to refer to the stack present before the function application begins
to evaluate.
  \item 
When we begin to evaluate the function position (f), create a new stack frame
((lambda (fv) ...)). This frame has one free identifier: k. Thus its closure
needs to record one element of the environment, namely the rest of the stack.
  \item 
The code portion of the stack frame represents what is left to be done once we
obtain a value for the function: evaluate the argument, and perform the
application, and return the result to the stack expecting the result of the
application: k.
  \item 
When evaluation of f completes, we begin to evaluate a, which also creates a
stack frame: (lambda (av) ...). This frame has two free identifiers: k and fv.
This tells us:
\begin{itemize}
  \item 
We no longer need the stack frame for evaluating the function position, but
  \item 
we now need a temporary that records the value—hopefully a function value—of
evaluating the function position.
\end{itemize}
  \item 
The code portion of this second frame also represents what is left to be done:
invoke the function value with the argument, in the stack expecting the value of
the application.
\end{itemize}
Let us apply similar reasoning to conditionals:
\lsts{14/4/2.rkt}{rkt}
It says that to evaluate the conditional expression we have to create a new
stack frame. This frame closes over the stack expecting the value of the entire
conditional. This frame makes a decision based on the value of the conditional
expression, and invokes one of the other expressions. Once we have examined this
value the frame created to evaluate the conditional expression is no longer
necessary, so evaluation can proceed in k.

Viewed through this lens, we can more easily provide an operational explanation
for generators. Each generator has its own private stack, and when execution
attempts to return past its end, our implementation raises an error. On
invocation, a generator stores a reference to the stack of the “rest of the
program” in where-to-go, and resumes its own stack, which is referred to by
resumer. On yielding, the system swaps references to stacks. Coroutines,
threads, and generators are all conceptually similar: they are all mechanisms to
create “many little stacks” instead of having a single, global stack.

\secrel{14.5 Tail Calls   . . 123}

\secrel{14.6 Continuations as a Language Feature  124}\secdown

With this insight into the connection between continuations and stacks, we can
now return to the treatment of procedures: we ignored the continuation at the
point of closure creation and instead only used the one at the point of closure
invocation. This of course corresponds to normal procedure behavior. But now we
can ask, what if we use the creation-time continuation instead? This would
correspond to maintaining a reference to (a copy of) the stack at the point of
“procedure” creation, and when the procedure is applied, ignoring the dynamic
evaluation and going back to the point of procedure creation.

In principle, we are trying to leave lambda intact and instead give ourselves a
language construct that corresponds to this behavior:
\note{cc = “current continuation”}
\lsts{14/6/1.rkt}{rkt}

What this says is that either way, control will return to the expression that
immediately surrounds the let/cc: either by falling through (because the
continuation of the body, b, is k) or—more interestingly—by invoking the
continuation, which discards the dynamic continuation dyn/k by simply ignoring
it and returning to k instead.

Here’s the simplest test:
\lsts{14/6/2.rkt}{rkt}
This confirms that if we never use the continuation, evaluation of the body
proceeds as if the let/cc weren’t there at all (because of the ((cps b) k)). If
we use it, the value given to the continuation returns to the point of creation:
\lsts{14/6/3.rkt}{rkt}
This example, of course, isn’t revealing, but consider this one:
\lsts{14/6/4.rkt}{rkt}
This confirms that the addition actually happens. But what about the dynamic
continuation?
\lsts{14/6/5.rkt}{rkt}
This shows that the addition by 2 never happens, i.e., the dynamic continuation
is indeed ignored. And just to be sure that the continuation at the point of
creation is respected, observe:
\lsts{14/6/6.rkt}{rkt}

From these examples, you have probably noticed a familiar pattern: esc here is
behaving like an exception. That is, if you do not throw an exception (in this
case, invoke a continuation) it’s as if it’s not there, but if you do throw it,
all pending intermediate computation is ignored and computation returns to the
point of exception creation.

\Exercise{
Using let/cc and macros, create a throw/catch mechanism.
}

However, these examples only scratch the surface of available power, because the
continuation at the point of invocation is always an extension of one at the
point of creation: i.e., the latter is just earlier in the stack than the
former. However, nothing actually demands that k and dyn-k be at all related.
That means they are in fact free to be unrelated, which means each can be a
distinct stack, so we can in fact easily implement stack-switching procedures
with them.

\Exercise{
To be properly analogous to lambda, we should have introduced a construct
called, say, cont-lambda with the following expansion:
\lsts{14/6/7.rkt}{rkt}
Why didn’t we? Consider both the static typing implications, and also how we
might construct the above exception-like behaviors using this construct instead.
}

\secrel{14.6.1 Presentation in the Language  125}

Writing programs in our little toy languages can soon become frustrating.
Fortunately, Racket already provides a construct called call/cc that reifies
continuations. call/cc is a procedure of one argument, which is itself a
procedure of one argument, which Racket applies to the current
continuation—which is a procedure of one argument. Got that?

Fortunately, we can easily write let/cc as a macro over call/cc and program with
that instead. Here it is:
\lsts{14/6/1/1.rkt}{rkt}
To be sure, all our old tests still pass:
\lsts{14/6/1/2.rkt}{rkt}

\secrel{14.6.2 Defining Generators  . 126}

Now we can start to create interesting abstractions. For instance, let’s build
generators. Whereas previously we needed to both CPS expressions and pass around
continuations, now this is done for us automatically by call/cc. Therefore,
whenever we need the current continuation, we simply conjure it up without
having to transform the program. Thus the extra ...-k parameters can disappear
with a let/cc in the same place to capture the same continuation:
\lsts{14/6/2/1.rkt}{rkt}

Observe the close similarity between this code and the implementation of
generators by desugaring into CPS code. Specifically, we can drop the extra
continuation arguments, and replace them with invocations of let/cc that will
capture precisely the same continuations. The rest of the code is fundamentally
unchanged.

\Exercise{
What happens if we move the let/ccs and mutation to be the first statement
inside the begins instead?
}

We can, for instance, write a generator that iterates from the initial value
upward:
\lsts{14/6/2/2.rkt}{rkt}
whose behavior is:
\lst{14/6/2/3.rkt}
Because the body refers only to the initial value, ignoring that returned by
invoking yield, the values we pass on subsequent invocations are irrelevant. In
contrast, consider this generator:
\lsts{14/6/2/4.rkt}{rkt}
On its first invocation, it returns whatever value it was supplied. On
subsequent invocations, this value is added to that provided on re-entry into
the generator. In other words, this generator additively accumulates all values
given to it:
\lst{14/6/2/5.rkt}

\Exercise{
Now that you’ve seen how to implement generators using call/cc and let/cc,
implement coroutines and threads as well.
}

\secrel{14.6.3 Defining Threads   127}

\secrel{14.6.4 Better Primitives for Web Programming  131}

\secup

\secup
