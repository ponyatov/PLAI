\secrel{14.2.3 Implementation in the Core  115}

Now that we’ve seen how CPS can be implemented through desguaring, we should ask
whether it can be put in the core instead.

Recall that we’ve said that CPS applies to all programs. We have one program we
are especially interested in: the interpreter. Sure enough, we can apply the CPS
transformation to it, making available what are effectively the same
continuations.

First, we’ll find it convenient to use a procedural representation of closures
\ref{}. We’ll have the interpreter take an extra argument, which consumes values
(those given to the continuation) and eventually returns them:
\lsts{14/2/3/1.rkt}{rkt}
In the easy cases, instead of returning a value we need to simply pass it to the
continuation argument:
\lsts{14/2/3/2.rkt}{rkt}
(Note that multC is handled entirely analogous to plusC.)

Let’s start with the easy case, plusC. First we interpret the left
sub-expression. The continuation for this evaluation interprets the right
sub-expression. The continuation for that adds the result. What should happen to
the result of addition? In interp, it was returned to whichever computation
caused the plusC to be interpreted. Now, remember, we no longer return values;
instead we pass them to the continuation:
\lsts{14/2/3/3.rkt}{rkt}
\Exercise{
Implement the code for multC.
}
This leaves the two difficult, and related, pieces.

In an application, we again have to interpret the two sub-expressions, and then
apply the resulting closure to the argument. But we’ve already agreed that every
application needs a continuation argument. Therefore, we have to update our
definition of a value:
\lsts{14/2/3/4.rkt}{rkt}

Now we have to decide what continuation to pass. In an application, it’s the
continuation given to the interpreter:
\lsts{14/2/3/5.rkt}{rkt}

Finally, the lamC case. We have to create a closV using a lambda, as before.
However, this procedure needs to take two arguments: the actual value of the
argument, and the continuation of the application. The critical question is,
what is this latter value?

We have essentially two choices. k represents the static continuation: the one
active at the point of closure construction. However, what we want is the
continuation at the point of closure invocation: the dynamic continuation.
\lsts{14/2/3/6.rkt}{rkt}

To test this revised interpreter, we need to invoke interp/k with some kind of
initial continuation value. This needs to be a procedure that represents nothing
remaining in the computation. A natural representation for this is the identity
function:
\lsts{14/2/3/7.rkt}{rkt}
To signify that this is strictly a top-level interface to interp/k, we’ve
dropped the environment parameter and pass the empty environment automatically.
If we want to be especially sure we haven’t accidentally used this procedure
recursively, we could insert a call to error at its end to prevent it from
returning and its return value being used.
