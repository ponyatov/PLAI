\secrel{14.2 Continuation-Passing Style  . . 109}

The functions we’ve been writing have a name. Though we’ve presented ideas in
terms of theWeb, we’re relying on a much older idea: the functions are called
continuations, and this style of programs is called continuation-passing style
(CPS). This is worth studying in its own right, because it is the basis for
studying a variety of other nontrivial control operations—such as generators.
\note{We will take the liberty of using CPS as both a noun and verb: a
particular structure of code and the process that converts code into it.}

Earlier, we converted programs so that no Web input operation was nested inside
another. The motivation was simple: when the program terminates, all nested
computations are lost. A similar argument applies, in a more local sense, in the
case of XMLHttpRequest: any computation depending on the result of a response
from aWeb server needs to reside in the callback associated with the request to
the server.

In fact, we don’t need to transform every expression. We only care about
expressions that involve actual Web interaction. For example, if we computed a
more complex mathematical expression than just addition, we wouldn’t need to
transform it. If, however, we had a function call, we’d either have to be
absolutely certain the function didn’t have any Web invocations either inside
it, or in the functions in invokes, or the ones they invoke...or else, to be
defensive, we should transform them all. Therefore, we have to transform every
expression that we can’t be sure performs noWeb interactions.

The heart of our transformation is therefore to turn every one-argument
function, f, into one with an extra argument. This extra argument is the
continuation, which represents the rest of the computation. The continuation is
itself a function of one argument. This argument takes the value that would have
been returned by f and passes it to the rest of the computation. f, instead of
returning a value, instead passes the value it would have returned to its
continuation.

CPS is a general transformation, which we can apply to any program. Because it’s
a program transformation, we can think of it as a special kind of desugaring: in
particular, instead of transforming programs from a larger language to a smaller
one (as macros do), or from one language to entirely another (as compilers do),
it transforms programs within the same language: from the full language to a
more restricted version that obeys the pattern we’ve been discussing. As a
result, we can reuse an evaluator for the full language to also evaluate
programs in the CPS subset.

\secdown
\secrel{14.2.1 Implementation by Desugaring  . . 110}

\secrel{14.2.2 Converting the Example  . . 114}

\secrel{14.2.3 Implementation in the Core  115}

\secup
