\secrel{14.3.1 Design Variations   117}

There are many variations between generators. The points of variation,
predictably, have to do with how to enter and exit a generator:
\begin{itemize}
  \item 
In some languages a generator is an object that is instantiated like any other
object, and its execution is resumed by invoking a method (such as next in
Python). In others it is just like a procedure, and indeed it is re-entered by
applying it like a function.
\note{In languages where values in addition to regular procedures can be used in
an application, all such values are collectively called applicables.}
  \item 
In some languages the yielding operation—such as Python’s yield—is available
only inside the syntactic body of the generator. In others, such as Racket,
yield is an applicable value bound in the body, but by virtue of being a value,
it can be passed to abstractions, stored in data structures, and so on.
\end{itemize}

Python’s design represents an extreme point in that a generator is simply any
function that contains the keyword yield in its body. In addition, Python’s
yield cannot be passed as a parameter to another function that performs the
yielding on behalf of the generator.

There is also a small issue of naming. In many languages with generators, the
yielder is automatically called word yield: either as a keyword (as in Python)
or as an identifier bound to an applicable value (as in Racket). Another
possibility is that the user of the generator must indicate in the generator
expression what name to give the yielder. That is, a use might look like
\note{Curiously, Python expects users to determine what to call self or this in
objects, but it does not provide the same flexibility for yield, because it has
no other way to determine which functions are generators!}
\lsts{14/3/1/1.rkt}{rkt}
but it might equivalently be
\lsts{14/3/1/2.rkt}{rkt}
and if the yielder is an actual value, a user can also abstract over yielding:
\lsts{14/3/1/3.rkt}{rkt}
where yield-helper will presumably perform the actual yielding.

There are actually two more design decisions:
\begin{enumerate}
  \item 
Is yield a statement or expression? In many languages it is actually an
expression, meaning it has a value: the one supplied when resuming the
generator. This makes the generator more flexible because the user of a
generator can use the parameter( s) to alter the generator’s behavior, rather
than being forced to use state to communicate desired changes.
  \item 
What happens at the end of the generator’s execution? In many languages, a
generator raises an exception to signal its completion.
\end{enumerate}
