\secrel{14.3.2 Implementing Generators  . . 119}

To implement generators, it will be especially useful to employ our CPS macro
language. Let’s first decide where we stand regarding the above design
decisions. We will use the applicative representation of generators: that is,
asking for the next value from the generator is done by applying it to any
necessary arguments. Similarly, the yielder will also be an applicable value and
will in turn be an expression. Though we have already seen how macros can
automatically capture a name \ref{}, let’s make the yielder’s name explicit to
keep the macro simpler. Finally, we’ll raise an error when the generator is done
executing.

How do generators work? To yield, a generator must
\begin{itemize}[nosep]
\item remember where in its execution it currently is, and
\item know where in its caller it should return to.
\end{itemize}
while, when invoked, it should
\begin{itemize}[nosep]
  \item 
remember where in its execution its caller currently is, and
  \item 
know where in its body it should return to.
\end{itemize}
Observe the duality between invocation and yielding.

As you might guess, these “where”s correspond to continuations.

Let’s build up the generator rule of the cps macro incrementally. First a header
pattern:
\lsts{14/3/2/1.rkt}{rkt}

The beginning of the body is easy: all code in CPS needs to consume a
continuation, and because a generator is a value, this value should be supplied
to the continuation:
\lsts{14/3/2/2.rkt}{rkt}

Now we’re ready to tackle the heart of the generator.

Recall that a generator is an applicable value. That means it can occur in an
application position, and must therefore have the same “interface” as a
procedure: a procedure of two arguments, the first a value and the second the
continuation at the point of application. What should this procedure do? We’ve
described just this above. First the generator must remember where the caller is
in its execution, which is precisely the continuation at the point of
application; “remember” here most simply means “must be stored in state”. Then,
the generator should return to where it previously was, i.e., its own
continuation, which must clearly have been stored. Therefore the core of the
applicable value is:
\lsts{14/3/2/3.rkt}{rkt}

Here, where-to-go records the continuation of the caller, to resume it upon
yielding; resumer is the local continuation of the generator. Let’s think about
what their initial values must be:
\begin{itemize}
  \item 
where-to-go has no initial value (because the generator has yet to be invoked),
so it needs to throw an error if ever used. Fortunately this error will never
occur, because where-to-go is mutated on the first entry into the generator, so
the error is just a safeguard against bugs in the implementation.
  \item 
Initially, the rest of the generator is the whole generator, so resumer should
be bound to the (CPS of) b. What is its continuation? This is the continuation
of the entire generator, i.e., what to do when the generator finishes. We’ve
agreed that this should also signal an error (except in this case the error
truly can occur, in case the generator is asked to produce more values than it’s
equipped to).
\end{itemize}

We still need to bind yield. It is, as we’ve pointed out, symmetric to generator
resumption: save the local continuation in resumer and return by applying
where-to-go.

Putting together these pieces, we get:
\lsts{14/3/2/4.rkt}{rkt}

\DoNow{
Why this pattern of let and letrec instead of let?
}

Observe the dependencies between these code fragments. where-to-go doesn’t
depend on either of resumer or yield. yield clearly depends on both where-to-go
and resumer. But why are resumer and yield mutually referential?

\DoNow{
Try the alternative!
}

The subtle dependency you may be missing is that resumer contains b, the body of
the generator, which may contain references to yield. Therefore, it needs to be
closed over the binding of the yielder.

\Exercise{
How do generators differ from coroutines and threads? Implement coroutines and
threads using a similar strategy.
}
