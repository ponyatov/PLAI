\secrel{14.1.1 Program Decomposition into Now and Later 104}

Let us consider what it takes to make our above program work in a stateless
setting, such as on a Web server. First we have to determine the first
interaction. This is the prompt for the first number, because Racket evaluates
arguments from left to right. It is instructive to divide the program into two
parts: what happens to generate the first interaction (which can all run now),
and what needs to happen after it (which must be “remembered” somehow). The
former is easy:
\lsts{src/14/1/1/1.rkt}{rkt}
We’ve already explained in prose what’s left, but now it’s time to write it as a
program. It seems to be something like
\note{We’re intentionally
ignoring
read-number for
now, but we’ll
return to it. For
now, let’s pretend
it’s built-in.}
\lsts{src/14/1/1/2.rkt}{rkt}
A Web server can’t execute the above, however, because it evidently isn’t a
program. We instead need some way of writing this as one.

Let’s observe a few characteristics of this computation:
\begin{itemize}[nosep]
  \item 
It needs to be a legitimate program.
  \item 
It needs to stay suspended until the request comes in.
  \item 
It needs a way—such as a parameter—to refer to the value from the first
interaction.
\end{itemize}

Put together these characteristics and we have a clear representation—a
function:
\lsts{src/14/1/1/3.rkt}{rkt}
