\secrel{14.1 Control on the Web   102}

Let us begin our study by examining the structure of Web programs. Consider the
following program:
\note{ Henceforth, we’ll call this our “addition server”.
You should, of course, understand this as a stand-in for more sophisticated
applications. For instance, the two prompts might ask for starting and ending
points for a trip, and in place of addition we might compute a route or compute
airfares. There might even be computation between the two steps: e.g., after
entering the first city, the airline might prompt us with choices of where it
flies from there.}
\lsts{src/14/1/1.rkt}{rkt}
To test these ideas, here’s an implementation of read-number:
\lsts{src/14/1/2.rkt}{rkt}
When run at the console or in DrRacket, this program prompts us for one number,
then another, and then displays their sum.

Now suppose we want to run this on a Web server. We immediately encounter a
difficulty: the structure of server-side Web programs is such that they generate
a single Web page—such as the one asking for the first number—and then halt. As
a result, the rest of the program—which in this case prompts for the second
number, then adds them, and then prints that result, is lost.

\DoNow{
Why do Web servers behave in such a strange way?
}

There are at least two reasons for this behavior: one perhaps historical, and
the other technical. The historical reason is that Web servers were initially
designed to serve pages, i.e., static content. Any program that ran had to
generate its output to a file, from which a server could offer it. Naturally,
developers wondered why that same program couldn’t run on demand. This made Web
content dynamic. Terminating the program after generating a single piece of
output was the simplest incremental step towards programs, not pages, on the
Web.

The more important reason—and the one that has stayed with us—is technical.
Imagine our addition server has generated its first prompt. Recall that there is
considerable pending computation: the second prompt, the addition, and the
display of the result. This computation must suspend waiting for the user’s
input. If there are millions of users, then millions of computations must be
suspended, creating an enormous performance problem. Furthermore, suppose a user
does not actually complete the computation—analogous to searching at an on-line
bookstore or airline site, but not completing the purchase. How does the server
know when or even whether to terminate the computation? And until it does, the
resources associated with that computation remain in use.

Conceptually, therefore, the Web protocol was designed to be stateless: it would
not store state on the server associated with intermediate computations.
Instead, Web program developers would be forced to maintain all necessary state
elsewhere, and each request would need to be able to resume the computation in
full. In practice, the Web has not proven to be stateless at all, but it still
hews largely in this direction, and studying the structure of such programs is
very instructive.

Now consider client-side Web programs: those that run inside the browser,
written in or compiled to JavaScript. Suppose such a computation needs to
communicate with a server. The primitive for this is called XMLHttpRequest. The
user makes an instance of this primitive and invokes its send method to send a
message to the server. Communicating with a server is not, however,
instantaneous (and indeed may never complete, depending on the state of the
network). This leaves the sending process suspended.

The designers of JavaScript decided to make the language single-threaded: i.e.,
there would be only one thread of execution at a time. This avoids the various
perils that arise from combining mutation with threads. As a result, however,
the JavaScript process locks up awaiting the response, and nothing else can
happen: e.g., other handlers on the page no longer respond.
\note{Due to the structuring problems this causes, there are now various
proposals to, in effect, add “safe” threads to JavaScript. The ideas described
in this chapter can be viewed as an alternative that offer similar structuring
benefits.}

To avoid this problem, the design of XMLHttpRequest demands that the developer
provide a procedure that responds to the request if and when it arrives. This
callback procedure is registered with the system. It needs to embody the rest of
the processing of that request. Thus, for entirely different reasons—not
performance, but avoiding the problems of synchronization, non-atomicity, and
deadlocks—the client-side Web has evolved to demand the same pattern of
developers. Let us now better understand that pattern.

\secdown
\secrel{14.1.1 Program Decomposition into Now and Later 104}

Let us consider what it takes to make our above program work in a stateless
setting, such as on a Web server. First we have to determine the first
interaction. This is the prompt for the first number, because Racket evaluates
arguments from left to right. It is instructive to divide the program into two
parts: what happens to generate the first interaction (which can all run now),
and what needs to happen after it (which must be “remembered” somehow). The
former is easy:
\lsts{src/14/1/1/1.rkt}{rkt}
We’ve already explained in prose what’s left, but now it’s time to write it as a
program. It seems to be something like
\note{We’re intentionally
ignoring
read-number for
now, but we’ll
return to it. For
now, let’s pretend
it’s built-in.}
\lsts{src/14/1/1/2.rkt}{rkt}
A Web server can’t execute the above, however, because it evidently isn’t a
program. We instead need some way of writing this as one.

Let’s observe a few characteristics of this computation:
\begin{itemize}[nosep]
  \item 
It needs to be a legitimate program.
  \item 
It needs to stay suspended until the request comes in.
  \item 
It needs a way—such as a parameter—to refer to the value from the first
interaction.
\end{itemize}

Put together these characteristics and we have a clear representation—a
function:
\lsts{src/14/1/1/3.rkt}{rkt}

\secrel{14.1.2 A Partial Solution   104}

\secrel{14.1.3 Achieving Statelessness  . . 106}

\secrel{14.1.4 Interaction with State  . 107}

\secup
