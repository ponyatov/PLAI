\secrel{14.1.2 A Partial Solution   104}

On the Web, there is an additional wrinkle: each Web page with input elements
needs to refer to a program stored on the Web, which will receive the data from
the form and process it. This program is named in the action field of a form.
Thus, imagine that the server generates a fresh label, stores the above function
in a table associated with that label, and refers to the table in the action
field. If and when the client actually submits the form, the server extracts the
associated function, supplies it with the form’s values, and thus resumes
execution.

\DoNow{
Is the solution above stateless?
}

Let’s imagine that we have a custom Web server that maintains the above table.
In such a server, we might have a special version of read-number—call it read-
number/suspend—that records the rest of the program:
\lsts{src/14/1/2/1.rkt}{rkt}
To test this, let’s implement such a procedure. First, we need a representation
for labels; numbers are an easy substitute:
\lsts{src/14/1/2/2.rkt}{rkt}
Let’s say new-label generates a fresh label on each invocation.
\Exercise{
Define new-label. You might use new-loc for inspiration.
}

We need a table to store the procedures representing the rest of the program.
\lsts{src/14/1/2/3.rkt}{rkt}
Now we can store these procedures:
\lsts{src/14/1/2/4.rkt}{rkt}
If we now run the above invocation of read-number/suspend, the system prints
First number To enter it, use the action field label 1 This is tantamount to
printing the prompt in a Web page, and putting the label 1 in the action field.
Because we’re simulating it, we need something to represent the browser’s
submission process. This needs both the label (from the action field) and the
value entered in the form. Given these two values, this procedure needs to
extract the relevant procedure from the table, and apply it to the form value.
\lsts{src/14/1/2/5.rkt}{rkt}

With this, we can now simulate the act of entering 3 and clicking on a “Submit”
button by running:
\lst{src/14/1/2/6.rkt}
where 1 is the label and 3 is the user’s input. Unfortunately, this simply
produces another prompt, because we haven’t fully converted the program. If we
delete read- number, we’re forced to convert the entire program:
\lsts{src/14/1/2/7.rkt}{rkt}
Just to be safe, we can also make sure the computation terminates after each
output by adding an error invocation at the end of read-number/suspend (to truly
ensure the most extreme form of “suspension”).

When we execute this program, we have to use resume twice:
\lst{src/14/1/2/8.rkt}
where the two user inputs are 3 and 10, giving a total of 13, and the halting
messages are generated by the error command we inserted.

We’ve purposely played a little coy with the types of the interesting parts of
our program. Let’s examine what these types should be. The second argument to
read- number/suspend needs to be a procedure that consumes numbers and returns
whatever the computation eventually produces: (number -> 'a). Similarly, the
return type of resume is the same 'a. How do these 'as communicate with one
another? This is done by table, which maps labels to (number -> 'a). That is, at
every step the computation makes progress towards the same outcome.
read-number/suspend writes into this table, and resume reads from it.
