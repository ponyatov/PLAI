\secrel{14.1 Control on the Web   102}

Let us begin our study by examining the structure of Web programs. Consider the
following program:
\note{ Henceforth, we’ll call this our “addition server”.
You should, of course, understand this as a stand-in for more sophisticated
applications. For instance, the two prompts might ask for starting and ending
points for a trip, and in place of addition we might compute a route or compute
airfares. There might even be computation between the two steps: e.g., after
entering the first city, the airline might prompt us with choices of where it
flies from there.}
\lsts{14/1/1.rkt}{rkt}
To test these ideas, here’s an implementation of read-number:
\lsts{14/1/2.rkt}{rkt}
When run at the console or in DrRacket, this program prompts us for one number,
then another, and then displays their sum.

Now suppose we want to run this on a Web server. We immediately encounter a
difficulty: the structure of server-side Web programs is such that they generate
a single Web page—such as the one asking for the first number—and then halt. As
a result, the rest of the program—which in this case prompts for the second
number, then adds them, and then prints that result, is lost.

\DoNow{
Why do Web servers behave in such a strange way?
}

There are at least two reasons for this behavior: one perhaps historical, and
the other technical. The historical reason is that Web servers were initially
designed to serve pages, i.e., static content. Any program that ran had to
generate its output to a file, from which a server could offer it. Naturally,
developers wondered why that same program couldn’t run on demand. This made Web
content dynamic. Terminating the program after generating a single piece of
output was the simplest incremental step towards programs, not pages, on the
Web.

The more important reason—and the one that has stayed with us—is technical.
Imagine our addition server has generated its first prompt. Recall that there is
considerable pending computation: the second prompt, the addition, and the
display of the result. This computation must suspend waiting for the user’s
input. If there are millions of users, then millions of computations must be
suspended, creating an enormous performance problem. Furthermore, suppose a user
does not actually complete the computation—analogous to searching at an on-line
bookstore or airline site, but not completing the purchase. How does the server
know when or even whether to terminate the computation? And until it does, the
resources associated with that computation remain in use.

Conceptually, therefore, the Web protocol was designed to be stateless: it would
not store state on the server associated with intermediate computations.
Instead, Web program developers would be forced to maintain all necessary state
elsewhere, and each request would need to be able to resume the computation in
full. In practice, the Web has not proven to be stateless at all, but it still
hews largely in this direction, and studying the structure of such programs is
very instructive.

Now consider client-side Web programs: those that run inside the browser,
written in or compiled to JavaScript. Suppose such a computation needs to
communicate with a server. The primitive for this is called XMLHttpRequest. The
user makes an instance of this primitive and invokes its send method to send a
message to the server. Communicating with a server is not, however,
instantaneous (and indeed may never complete, depending on the state of the
network). This leaves the sending process suspended.

The designers of JavaScript decided to make the language single-threaded: i.e.,
there would be only one thread of execution at a time. This avoids the various
perils that arise from combining mutation with threads. As a result, however,
the JavaScript process locks up awaiting the response, and nothing else can
happen: e.g., other handlers on the page no longer respond.
\note{Due to the structuring problems this causes, there are now various
proposals to, in effect, add “safe” threads to JavaScript. The ideas described
in this chapter can be viewed as an alternative that offer similar structuring
benefits.}

To avoid this problem, the design of XMLHttpRequest demands that the developer
provide a procedure that responds to the request if and when it arrives. This
callback procedure is registered with the system. It needs to embody the rest of
the processing of that request. Thus, for entirely different reasons—not
performance, but avoiding the problems of synchronization, non-atomicity, and
deadlocks—the client-side Web has evolved to demand the same pattern of
developers. Let us now better understand that pattern.

\secdown
\secrel{14.1.1 Program Decomposition into Now and Later 104}

Let us consider what it takes to make our above program work in a stateless
setting, such as on a Web server. First we have to determine the first
interaction. This is the prompt for the first number, because Racket evaluates
arguments from left to right. It is instructive to divide the program into two
parts: what happens to generate the first interaction (which can all run now),
and what needs to happen after it (which must be “remembered” somehow). The
former is easy:
\lsts{14/1/1/1.rkt}{rkt}
We’ve already explained in prose what’s left, but now it’s time to write it as a
program. It seems to be something like
\note{We’re intentionally
ignoring
read-number for
now, but we’ll
return to it. For
now, let’s pretend
it’s built-in.}
\lsts{14/1/1/2.rkt}{rkt}
A Web server can’t execute the above, however, because it evidently isn’t a
program. We instead need some way of writing this as one.

Let’s observe a few characteristics of this computation:
\begin{itemize}[nosep]
  \item 
It needs to be a legitimate program.
  \item 
It needs to stay suspended until the request comes in.
  \item 
It needs a way—such as a parameter—to refer to the value from the first
interaction.
\end{itemize}

Put together these characteristics and we have a clear representation—a
function:
\lsts{14/1/1/3.rkt}{rkt}

\secrel{14.1.2 A Partial Solution   104}

On the Web, there is an additional wrinkle: each Web page with input elements
needs to refer to a program stored on the Web, which will receive the data from
the form and process it. This program is named in the action field of a form.
Thus, imagine that the server generates a fresh label, stores the above function
in a table associated with that label, and refers to the table in the action
field. If and when the client actually submits the form, the server extracts the
associated function, supplies it with the form’s values, and thus resumes
execution.

\DoNow{
Is the solution above stateless?
}

Let’s imagine that we have a custom Web server that maintains the above table.
In such a server, we might have a special version of read-number—call it read-
number/suspend—that records the rest of the program:
\lsts{14/1/2/1.rkt}{rkt}
To test this, let’s implement such a procedure. First, we need a representation
for labels; numbers are an easy substitute:
\lsts{14/1/2/2.rkt}{rkt}
Let’s say new-label generates a fresh label on each invocation.
\Exercise{
Define new-label. You might use new-loc for inspiration.
}

We need a table to store the procedures representing the rest of the program.
\lsts{14/1/2/3.rkt}{rkt}
Now we can store these procedures:
\lsts{14/1/2/4.rkt}{rkt}
If we now run the above invocation of read-number/suspend, the system prints
First number To enter it, use the action field label 1 This is tantamount to
printing the prompt in a Web page, and putting the label 1 in the action field.
Because we’re simulating it, we need something to represent the browser’s
submission process. This needs both the label (from the action field) and the
value entered in the form. Given these two values, this procedure needs to
extract the relevant procedure from the table, and apply it to the form value.
\lsts{14/1/2/5.rkt}{rkt}

With this, we can now simulate the act of entering 3 and clicking on a “Submit”
button by running:
\lst{14/1/2/6.rkt}
where 1 is the label and 3 is the user’s input. Unfortunately, this simply
produces another prompt, because we haven’t fully converted the program. If we
delete read- number, we’re forced to convert the entire program:
\lsts{14/1/2/7.rkt}{rkt}
Just to be safe, we can also make sure the computation terminates after each
output by adding an error invocation at the end of read-number/suspend (to truly
ensure the most extreme form of “suspension”).

When we execute this program, we have to use resume twice:
\lst{14/1/2/8.rkt}
where the two user inputs are 3 and 10, giving a total of 13, and the halting
messages are generated by the error command we inserted.

We’ve purposely played a little coy with the types of the interesting parts of
our program. Let’s examine what these types should be. The second argument to
read- number/suspend needs to be a procedure that consumes numbers and returns
whatever the computation eventually produces: (number -> 'a). Similarly, the
return type of resume is the same 'a. How do these 'as communicate with one
another? This is done by table, which maps labels to (number -> 'a). That is, at
every step the computation makes progress towards the same outcome.
read-number/suspend writes into this table, and resume reads from it.

\secrel{14.1.3 Achieving Statelessness  . . 106}

We haven’t actually achieved statelessness yet, because we have this large table
residing on the server, with no clear means to remove entries from it. It would
be better if we could avoid the server state entirely. This means we have to
move the relevant state to the client.

There are actually two ways in which the server holds state. One is that we have
reserved the right to create as many entries in the hash table as we wish,
rather than a constant number (i.e., linear in the size of the program itself).
The other is what we’re storing in the table: honest-to-goodness closures, which
might be holding on to an arbitrary amount of state. We’ll see this more clearly
soon.

Let’s start by eliminating the closure. Instead, let’s have each of the funtion
arguments to be named, top-level functions (which immediately forces us to have
only a fixed number of them, because the program’s size cannot be unbounded):
\lsts{14/1/3/1.rkt}{rkt}
Observe how each code block refers only to the name of the next, rather than to
a real closure. The value of the argument comes from the form. There’s just one
problem: v1 in prog2 is a free identifier!

The way to fix this problem is, instead of creating a closure after one step, to
send v1 to the client to be stored there. Where do we store this? The browser
offers two mechanisms for doing this: cookies and hidden fields. Which one do we
use?

\secrel{14.1.4 Interaction with State  . 107}

The fundamental difference between cookies and hidden fields is that all pages
share the same cookie, but each page has its own hidden fields.

First, let’s consider a sequence of interactions with the existing program that
uses read-number/suspend (at both interaction points). It looks like this:
\lsts{14/1/4/1.rkt}{rkt}
Thus, resuming with label 2 appears to represent adding 3 to the given argument
(i.e., form field value). To be sure,
\lsts{14/1/4/2.rkt}{rkt}
So far, so good. Now suppose we use label 1 again:
\lsts{14/1/4/3.rkt}{rkt}
Observe that this new program execution needs to be resumed by using label 3,
not 1.
Indeed,
\lsts{14/1/4/4.rkt}{rkt}
But we ought to ask, what happens if we reuse label 2?

\DoNow{
Try (resume 2 10).
}

Doing this is tantamount to resuming the old computation. We therefore expect it
produce the same answer as before:
\lsts{14/1/4/5.rkt}{rkt}

Now let’s create a stateful implementation. We can simulate this by observing
that each closure has its own environment, but all closures share the same
mutable state. We can simulate this using our existing read-number/suspend by
making sure we don’t rely on the closure behavior of lambda, i.e., by not having
any free identifiers in the body.
\lsts{14/1/4/6.rkt}{rkt}

\Exercise{
What do we expect for the same sequence as before?
}

\DoNow{
What happens?
}

Initially, nothing seems different:
\lsts{14/1/4/7.rkt}{rkt}
When we reuse the initial computation, we indeed get a new resumption label:
\lsts{14/1/4/8.rkt}{rkt}
which, when used, computes what we’d expect:
\lsts{14/1/4/9.rkt}{rkt}
Now we come to the critical step:
\lsts{14/1/4/10.rkt}{rkt}
It is unsurprising that the two resumptions of label 2 would produce different
answers, given that they rely on mutable state. The reason it’s problematic is
because of what happens when we translate the same behavior to the Web.

Imagine visiting a hotel reservation Web site and searching for hotels in a
city. In return, you are shown a list of hotels and the label 1. You explore one
of them in a new tab or window; this produces information on that hotel, and
label 2 to make the reservation. You decide, however, to return to the hotel
listing and explore another hotel in a fresh tab or window. This produces the
second hotel’s information, with label 3 to reserve at that hotel. You decide,
however, to choose the first hotel, return to the first hotel’s page, and choose
its reservation button—i.e., submit label 2. Which hotel did you expect to be
booked into? Though you expected a reservation at the first hotel, on most
travel sites, this will either reserve at the second hotel—i.e., the one you
last viewed, but not the one on the page whose reservation button you clicked—
or produce an error. This is because of the pervasive use of cookies on Web
sites, a practice encouraged by most Web APIs.

\secup
