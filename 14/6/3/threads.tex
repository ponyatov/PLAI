\secrel{14.6.3 Defining Threads   127}

Having done generators, let’s do another, similar primitive: threads. That is,
let’s assume we want to be able to write a program such as this:
\lsts{14/6/3/1.rkt}{rkt}
and expect the output
\lst{14/6/3/2.rkt}
We’ll build all the pieces necessary to achieve this.

Let’s start by defining the thread scheduler. It consumes a list of “threads”,
whose interface we assume will be a procedure that consumes a continuation to
which it eventually yields control. Each time the scheduler reactivates the
thread, it supplies it with a continuation. The scheduler might be choose
between threads in a simple round-robin manner, or it might use some more
complex algorithm; the details of how it chooses don’t concern us here.

As with generators, we’ll assume that yielding is done by invoking a procedure
named by the user: y, above. We could use name capture \ref{}\ to automatically
bind a name like yield.

More importantly, notice that the user of the thread system manually yields
control. This is called cooperative multitasking. Instead, we could have chosen
to have a timer or other intrinsic mechanism automtically yield without the
user’s permission; this is called preemptive multitasking (because the system
“pre-empts”—i.e., wrests control from—the thread). While the distinction is
important for buildling systems, it is not interesting from the perspective of
setting up the continuations.

\Exercise{
After we are done building cooperative multitasking, implement preemptive
multitasking. What changes?
}

With our stated constraints, we can write a first scheduler. It consumes a lists
of threads and continues executing so long as there are threads remaining. Each
time, it applies the thread procedure to a continuation that represents
returning to the scheduler and proceeding to the next thread:
\lsts{14/6/3/3.rkt}{rkt}
When the recipient thread invokes the continuation bound to after-thread,
control returns to the end of the first statement in the begin sequence. As a
result, the value supplied to the continuation is ignored (and can hence be any
dummy value; we’ll chose 'dummy, so that we can easily spot it if it shows up in
undesired places). Control then proceeds with the rest of the scheduler loop
after appending the most recently invoked thread to the end of the list of
threads (i.e., treating the list as a circular queue).

Now let’s define a thread. As we’ve said, it will be a procedure of one
argument, the scheduler’s continuation. Because the thread needs to resume,
i.e., continue where it left off, presumably it must store where it last was:
we’ll call this thread-resumer. Initially this is the entire thread body, but on
subsequent instances it will be a continuation: specifically, the continuation
of invoking yield. Thus, we obtain the following skeleton:
\lsts{14/6/3/4.rkt}{rkt}
That still leaves the yielder. It needs to be a procedure of no arguments that
stores the thread’s continuation in thread-resumer, and then invoke the
scheduler continuation with 'dummy. However, which scheduler continuation does
it need to invoke? Not the one provided on thread initiation, but rather the
most recent one. Thus, we must somehow “thread” the value in sched-k to the
yielder. There are many ways to accomplish it, but the simplest, perhaps most
brutal, way is to simply reconstruct the yielder on each thread resumption,
always closed over the most recent value of sched-k:
\lsts{14/6/3/5.rkt}{rkt}
When we run this ensemble, we get:
\lst{14/6/3/6.rkt}
Hey, that’s what we wanted! But it continues:
\lst{14/6/3/7.rkt}
Hmmm.

What’s happening? Well, we’ve been quiet all along about what needs to happen
when a thread reaches its end. In fact, control just returns to the thread
scheduler, which appends the thread to the end of the queue, and when the thread
comes to the head of the queue again, control resumes from the same previously
stored continuation: the one corresponding to printing the third value. This
prints, control returns, the thread is appended...ad infinitum.

Clearly, we need the thread scheduler to be notified when a thread has
terminated, so that the scheduler can remove it from the thread queue. We’ll
create a simple datatype to represent this signal:
\lsts{14/6/3/8.rkt}{rkt}
(In a real system, of course, these status messages might also carry informative
values from the computation.) We must now modify our scheduler to actually check
for and use these values:
\lsts{14/6/3/9.rkt}{rkt}
We have to now modify our thread representation in two ways: it must provide
Tsus- pended to the scheduler’s continuation on intermediate returns, and
provide Tdone when it terminates. Where does it terminate? After executing the
code in the body, b .... Observe, finally, that the termination process must be
sure to use the latest scheduler continuation, just as yielding does. Thus:
\lsts{14/6/3/10.rkt}{rkt}
If we now replace scheduler-loop-0 with scheduler-loop-1 and thread-0 with
thread-1 and re-run our example program above, we get just the output we desire.
