\secrel{14.6.2 Defining Generators  . 126}

Now we can start to create interesting abstractions. For instance, let’s build
generators. Whereas previously we needed to both CPS expressions and pass around
continuations, now this is done for us automatically by call/cc. Therefore,
whenever we need the current continuation, we simply conjure it up without
having to transform the program. Thus the extra ...-k parameters can disappear
with a let/cc in the same place to capture the same continuation:
\lsts{14/6/2/1.rkt}{rkt}

Observe the close similarity between this code and the implementation of
generators by desugaring into CPS code. Specifically, we can drop the extra
continuation arguments, and replace them with invocations of let/cc that will
capture precisely the same continuations. The rest of the code is fundamentally
unchanged.

\Exercise{
What happens if we move the let/ccs and mutation to be the first statement
inside the begins instead?
}

We can, for instance, write a generator that iterates from the initial value
upward:
\lsts{14/6/2/2.rkt}{rkt}
whose behavior is:
\lst{14/6/2/3.rkt}
Because the body refers only to the initial value, ignoring that returned by
invoking yield, the values we pass on subsequent invocations are irrelevant. In
contrast, consider this generator:
\lsts{14/6/2/4.rkt}{rkt}
On its first invocation, it returns whatever value it was supplied. On
subsequent invocations, this value is added to that provided on re-entry into
the generator. In other words, this generator additively accumulates all values
given to it:
\lst{14/6/2/5.rkt}

\Exercise{
Now that you’ve seen how to implement generators using call/cc and let/cc,
implement coroutines and threads as well.
}
