\secrel{14.6 Continuations as a Language Feature  124}\secdown

With this insight into the connection between continuations and stacks, we can
now return to the treatment of procedures: we ignored the continuation at the
point of closure creation and instead only used the one at the point of closure
invocation. This of course corresponds to normal procedure behavior. But now we
can ask, what if we use the creation-time continuation instead? This would
correspond to maintaining a reference to (a copy of) the stack at the point of
“procedure” creation, and when the procedure is applied, ignoring the dynamic
evaluation and going back to the point of procedure creation.

In principle, we are trying to leave lambda intact and instead give ourselves a
language construct that corresponds to this behavior:
\note{cc = “current continuation”}
\lsts{14/6/1.rkt}{rkt}

What this says is that either way, control will return to the expression that
immediately surrounds the let/cc: either by falling through (because the
continuation of the body, b, is k) or—more interestingly—by invoking the
continuation, which discards the dynamic continuation dyn/k by simply ignoring
it and returning to k instead.

Here’s the simplest test:
\lsts{14/6/2.rkt}{rkt}
This confirms that if we never use the continuation, evaluation of the body
proceeds as if the let/cc weren’t there at all (because of the ((cps b) k)). If
we use it, the value given to the continuation returns to the point of creation:
\lsts{14/6/3.rkt}{rkt}
This example, of course, isn’t revealing, but consider this one:
\lsts{14/6/4.rkt}{rkt}
This confirms that the addition actually happens. But what about the dynamic
continuation?
\lsts{14/6/5.rkt}{rkt}
This shows that the addition by 2 never happens, i.e., the dynamic continuation
is indeed ignored. And just to be sure that the continuation at the point of
creation is respected, observe:
\lsts{14/6/6.rkt}{rkt}

From these examples, you have probably noticed a familiar pattern: esc here is
behaving like an exception. That is, if you do not throw an exception (in this
case, invoke a continuation) it’s as if it’s not there, but if you do throw it,
all pending intermediate computation is ignored and computation returns to the
point of exception creation.

\Exercise{
Using let/cc and macros, create a throw/catch mechanism.
}

However, these examples only scratch the surface of available power, because the
continuation at the point of invocation is always an extension of one at the
point of creation: i.e., the latter is just earlier in the stack than the
former. However, nothing actually demands that k and dyn-k be at all related.
That means they are in fact free to be unrelated, which means each can be a
distinct stack, so we can in fact easily implement stack-switching procedures
with them.

\Exercise{
To be properly analogous to lambda, we should have introduced a construct
called, say, cont-lambda with the following expansion:
\lsts{14/6/7.rkt}{rkt}
Why didn’t we? Consider both the static typing implications, and also how we
might construct the above exception-like behaviors using this construct instead.
}

\secrel{14.6.1 Presentation in the Language  125}

Writing programs in our little toy languages can soon become frustrating.
Fortunately, Racket already provides a construct called call/cc that reifies
continuations. call/cc is a procedure of one argument, which is itself a
procedure of one argument, which Racket applies to the current
continuation—which is a procedure of one argument. Got that?

Fortunately, we can easily write let/cc as a macro over call/cc and program with
that instead. Here it is:
\lsts{14/6/1/1.rkt}{rkt}
To be sure, all our old tests still pass:
\lsts{14/6/1/2.rkt}{rkt}

\secrel{14.6.2 Defining Generators  . 126}

Now we can start to create interesting abstractions. For instance, let’s build
generators. Whereas previously we needed to both CPS expressions and pass around
continuations, now this is done for us automatically by call/cc. Therefore,
whenever we need the current continuation, we simply conjure it up without
having to transform the program. Thus the extra ...-k parameters can disappear
with a let/cc in the same place to capture the same continuation:
\lsts{14/6/2/1.rkt}{rkt}

Observe the close similarity between this code and the implementation of
generators by desugaring into CPS code. Specifically, we can drop the extra
continuation arguments, and replace them with invocations of let/cc that will
capture precisely the same continuations. The rest of the code is fundamentally
unchanged.

\Exercise{
What happens if we move the let/ccs and mutation to be the first statement
inside the begins instead?
}

We can, for instance, write a generator that iterates from the initial value
upward:
\lsts{14/6/2/2.rkt}{rkt}
whose behavior is:
\lst{14/6/2/3.rkt}
Because the body refers only to the initial value, ignoring that returned by
invoking yield, the values we pass on subsequent invocations are irrelevant. In
contrast, consider this generator:
\lsts{14/6/2/4.rkt}{rkt}
On its first invocation, it returns whatever value it was supplied. On
subsequent invocations, this value is added to that provided on re-entry into
the generator. In other words, this generator additively accumulates all values
given to it:
\lst{14/6/2/5.rkt}

\Exercise{
Now that you’ve seen how to implement generators using call/cc and let/cc,
implement coroutines and threads as well.
}

\secrel{14.6.3 Defining Threads   127}

\secrel{14.6.4 Better Primitives for Web Programming  131}

\secup
