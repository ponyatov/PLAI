\secrel{6 From Substitution to Environments 25}\secdown

Though we have a working definition of functions, you may feel a slight unease
about it. When the interpreter sees an identifier, you might have had a sense
that it needs to “look it up”. Not only did it not look up anything, we defined
its behavior to be an error! While absolutely correct, this is also a little
surprising. More importantly, we write interpreters to understand and explain
languages, and this implementation might strike you as not doing that, because
it doesn’t match our intuition.

There’s another difficulty with using substitution, which is the number of times
we traverse the source program. It would be nice to have to traverse only those
parts of the program that are actually evaluated, and then, only when necessary.
But substitution traverses everything—unvisited branches of conditionals, for
instance—and forces the program to be traversed once for substitution and once
again for interpretation.

\Exercise{
Does substitution have implications for the time complexity of evaluation?
}

There’s yet another problem with substitution, which is that it is defined in
terms of representations of the program source. Obviously, our interpreter has
and needs access to the source, to interpret it. However, other
implementations—such as compilers— have no need to store it for that purpose.
It would be nice to employ a mechanism that is more portable across
implementation strategies.
\note{Compilers might store versions of or information about the source for
other reasons, such as reporting runtime errors, and JITs may need it to
re-compile on demand.}

\secrel{6.1 Introducing the Environment  . 26}

The intuition that addresses the first concern is to have the interpreter “look
up” an identifier in some sort of directory. The intuition that addresses the
second concern is to defer the substitution. Fortunately, these converge nicely
in a way that also addresses the third. The directory records the intent to
substitute, without actually rewriting the program source; by recording the
intent, rather than substituting immediately, we can defer substitution; and the
resulting data structure, which is called an environment, avoids the need for
source-to-source rewriting and maps nicely to low-level machine representations.
Each name association in the environment is called a binding.

Observe carefully that what we are changing is the implementation strategy for
the programming language, not the language itself. Therefore, none of our
datatypes for representing programs should change, nor even should the answers
that the interpreter provides. As a result, we should think of the previous
interpreter as a “reference implementation” that the one we’re about to write
should match. Indeed, we should create a generator that creates lots of tests,
runs them through both interpreters, and makes sure their answers are the same.
Ideally, we should prove that the two interpreters behave the same, which is a
good topic for advanced study.

Let’s first define our environment data structure. An environment is a list of
pairs of names associated with...what?
\note{One subtlety is in defining precisely what “the same” means, especially
with regards to failure.}

\DoNow{
A natural question to ask here might be what the environment maps names to. But
a better, more fundamental, question is: How to determine the answer to the
“natural” question?
}

Remember that our environment was created to defer substitutions. Therefore, the
answer lies in substitution. We discussed earlier (Oh Wait, There’s More!) that
we want substitution to map names to answers, corresponding to an eager function
application strategy. Therefore, the environment should map names to answers.
\lsts{src/6/6_1_1.rkt}{rkt}
\secrel{6.2 Interpreting with Environments  27}

Now we can tackle the interpreter. One case is easy, but we should revisit all
the others:
\lsts{src/6/6_2_1.rkt}{rkt}
The arithmetic operations are easiest. Recall that before, the interpreter
recurred without performing any new substitutions. As a result, there are no new
deferred substitutions to perform either, which means the environment does not
change:
\lsts{src/6/6_2_2.rkt}{rkt}

Now let’s handle identifiers. Clearly, encountering an identifier is no longer
an error: this was the very motivation for this change. Instead, we must look up
its value in the directory:
\lsts{src/6/6_2_3.rkt}{rkt}

\DoNow{
Implement lookup.
}

Finally, application. Observe that in the substitution interpreter, the only
case that caused new substitutions to occur was application. Therefore, this
should be the case that constructs bindings. Let’s first extract the function
definition, just as before:
\lsts{src/6/6_2_4.rkt}{rkt}
Previously, we substituted, then interpreted. Because we have no substitution
step, we can proceed with interpretation, so long as we record the deferral of
substitution.
\lsts{src/6/6_2_5.rkt}{rkt}
That is, the set of function definitions remains unchanged; we’re interpreting
the body of the function, as before; but we have to do it in an environment that
binds the formal parameter. Let’s now define that binding process:
\lsts{src/6/6_2_6.rkt}{rkt}
the name being bound is the formal parameter (the same name that was substituted
for, before). It is bound to the result of interpreting the argument (because
we’ve decided on an eager application semantics). And finally, this extends the
environment we already have. Type-checking this helps to make sure we got all
the little pieces right.

Once we have a definition for lookup, we’d have a full interpreter. So here’s
one:
\lsts{src/6/6_2_7.rkt}{rkt}
Observe that looking up a free identifier still produces an error, but it has
moved from the interpreter—which is by itself unable to determine whether or not
an identifier is free—to lookup, which determines this based on the content of
the environment.

Now we have a full interpreter. You should of course test it make sure it works
as you’d expect. For instance, these tests pass:
\lsts{src/6/6_2_8.rkt}{rkt}
So we’re done, right?

\DoNow{
Spot the bug.
}

\secrel{6.3 Deferring Correctly   29}

\secrel{6.4 Scope   . 30}

The broken environment interpreter above implements what is known as dynamic
scope. This means the environment accumulates bindings as the program executes.
As a result, whether an identifier is even bound depends on the history of
program execution. We should regard this unambiguously as a flaw of programming
language design. It adversely affects all tools that read and process programs:
compilers, IDEs, and humans.

In contrast, substitution—and environments, done correctly—give us lexical scope
or static scope. “Lexical” in this context means “as determined from the source program”, while “static” in computer science means “without running the program”, so these are appealing to the same intuition. When we examine an identifier, we want to know two things: (1) Is it bound? (2) If so, where? By “where” we mean: if there are multiple bindings for the same name, which one governs this identifier? Put differently, which one’s substitution will give a value to this identifier? In general, these questions cannot be answered statically in a dynamically-scoped language: so your IDE, for instance, 
cannot overlay arrows to show you this information (as DrRacket does).
Thus, even though the rules of scope become more complex as the space of names
becomes richer (e.g., objects, threads, etc.), we should always strive to preserve the spirit of static 
scoping.
\note{A different way to think about it is that in a dynamically-scoped
language, the answer to these questions is the same for all identifiers, and it
simply refers to the dynamic environment. In other words, it provides no useful
information.}

\secdown
\secrel{6.4.1 How Bad Is It?   . 30}

You might look at our running example and wonder whether we’re creating a
tempest in a teapot. In return, you should consider two situations:
\begin{enumerate}

\item To understand the binding structure of your program, you may need to look
at the whole program. No matter how much you’ve decomposed your program into
small, understandable fragments, it doesn’t matter if you have a free identifier
anywhere.

\item Understanding the binding structure is not only a function of the size of
the program but also of the complexity of its control flow. Imagine an
interactive program with numerous callbacks; you’d have to track through every
one of them, too, to know which binding governs an identifier.

\end{enumerate}

Need a little more of a nudge? Let’s replace the expression of our example
program with this one:
\lsts{src/6/6_4_1.rkt}{rkt}

Suppose moon-visible? is a function that presumably evaluates to false on
new-moon nights, and true at other times. Then, this program will evaluate to an
answer except on new-moon nights, when it will fail with an unbound identifier
error.

\Exercise{
What happens on cloudy nights?
}

\secrel{6.4.2 The Top-Level Scope  . 31}

Matters become more complex when we contemplate top-level definitions in many
languages. For instance, some versions of Scheme (which is a paragon of lexical scoping) allow you to write this:
\lsts{src/6/6_4_2.rkt}{rkt}
which seems to pretty clearly suggest where the y in the body of f will come
from, except:
\lsts{src/6/6_4_3.rkt}{rkt}
is legal and (f 10) produces 12. Wait, you might think, always take the last
one! But:
\lsts{src/6/6_4_4.rkt}{rkt}

Here, z is bound to the first value of y whereas the inner y is bound to the
second value. There is actually a valid explanation of this behavior in terms of
lexical scope, but it can become convoluted, and perhaps a more sensible option is to
prevent such redefinition. Racket does precisely this, thereby offering the
convenience of a top-level without its pain.
\note{Most “scripting” languages exhibit similar problems. As a result, on the
Web you will find enormous confusion about whether a certain language is
statically- or dynamically-scoped, when in fact readers are comparing behavior
inside functions (often static) against the top-level (usually dynamic).
Beware!}

\secup

\secrel{6.5 Exposing the Environment  . . 31}

If we were building the implementation for others to use, it would be wise and a
courtesy for the exported interpreter to take only an expression and list of
function definitions, and invoke our defined interp with the empty environment.
This both spares users an implementation detail, and avoids the use of an
interpreter with an incorrect environment. In some contexts, however, it can be
useful to expose the environment parameter. For instance, the environment can
represent a set of pre-defined bindings: e.g., if the language wishes to provide
pi automatically bound to 3.2 (in Indiana).

\secup

