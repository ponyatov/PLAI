\secrel{11.4.1 Overview   . 84}

The key idea behind all GC algorithms is to traverse memory by following
references between values. Traversal begins at a root set, which is all the
places from which a program can possibly refer to a value in the store.
Typically the root set consists of every bound variable in the environment, and
any global variables. In an actual working implementation, the implementor must
be careful to also note ephemeral values such as references in registers. From
this root set, the algorithm walks all accessible values using a variety of
algorithms that are usually variations on depth-first search to identify
everything that is live (i.e., usable through some sequence of program
operations). Everything else, by definition, is garbage. Again, different
algorithms address the recovery of this space in different ways.
\note{Depth-first search is generally preferred because it works well with
stack-based implementations. Of course, you might (and should) wonder where the
GC’s own stack is stored!}
