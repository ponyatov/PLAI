\secrel{11.5 Convervative Garbage Collection  . . 86}

We’ve explained that the typical root set consists of the environment, global
variables, and some choice ephemerals. Where else might references reside?

In most languages, nowhere else. But some languages (I’m looking at you, C and
C++) allow references to be turned into arbitrary numbers, and arbitrary numbers
to be turned back into references. As a result, in principle, any numeric value
in the program (which, because of the nature of C and C++’s types, virtually any
value in the program) could potentially be treated as a reference.

This is problematic for two reasons. First, it means the GC can no longer limit
its attention to the small root set; instead, the entire store is now
potentially the root set. Second, if the GC tries to modify the object in any
way—e.g., to record a “visited” bit during traversal—then it is potentially
changing non-reference values: e.g., it might actually be changing an innocent
numeric constant in the program. As a result, the particular confluence of
features in languages like C and C++ conspire to make sound, efficient GC
extremely difficult.

But not impossible. A stimulating line of research called conservative GC has
managed to create reasonably effective GC systems for such languages. The
principle behind conservative GC notes that, while in principle every store
location might be a root, in practice many of them are not. It then proceeds
through a series of increasingly clever observations to deduce what must not be
a reference (the opposite of a traditional GC) and can hence be safely ignored:
for instance, on a word-aligned architecture, no odd number can never be a
reference. By skipping most of the store, by making some basic assumptions about
program behavior (e.g., that it will not manufacture certain kinds of
references), and by being careful to not modify the store—e.g., changing bits in
values, or moving data around—it can actually yield a reasonably effective GC
strategy.
\note{Nevertheless, it is a bit of a dog walking on its hind legs.}

Conservative GC is often popular with programming language implementations that
are written in, or rely on a base of code in, C or C++. For instance, early
versions of \racket\ relied exclusively on it. There are many good reasons for
this:
\begin{enumerate}
  \item 
It offers a quick bootstrapping technique, so the language implementor can focus
on other, more innovative, features.
  \item 
A language that controls all references (as Racket does) can easily create
memory representations that are especially conducive to increasing the
effectiveness of the GC (e.g., padding all true numbers with a one in the
least-significant-bit).
  \item 
It enables easy interoperation with useful libraries written in C and C++
(provided, of course, that they too meet the expectations of the technique).
\end{enumerate}

A word on vocabulary is in order. As we have argued [REF], all practical GC
techniques are “conservative” in that they approximate truth with reachability.
The word “conservative” has, however, become a term-of-art to refer to a GC
technique that operates in an uncooperative (and hopefully not hostile) run-time
system.
