\secrel{11.4.3 Central Assumptions  . 85}

Being able to soundly perform GC depends on two critical assumptions. The first
is one about the language’s implementation and the other about the language’s
semantics.
\begin{enumerate}

  \item 
When confronted with a value, the GC needs to know what kind of value it is, and
how its memory representation is laid out. For instance, when the traversal
reaches a cons cell, it must know:
\begin{enumerate}
  \item   
that this is a cons cell; and hence,
  \item   
that the first is at, say, a four byte offset, and
  \item   
that the rest is at, say, an eight byte offset.
\end{enumerate}
Obviously, this property must hold recursively, thus enabling a traversal
algorithm to correctly map the values in memory.

  \item 
That programs cannot manufacture references in two ways:
\begin{enumerate}
  \item 
Object references cannot reside outside the implementation’s pre-defined root
set.
  \item 
Object references can only refer to well-defined points in objects.
\end{enumerate}
When the second property is violated, the GC can effectively go haywire,
misinterpreting data. The first property sounds obvious: when it is violated, it
seems the run-time system has clearly failed to obey the language’s semantics.
However, the consequences of this property are subtle, as we discuss below
\ref{}.
  
\end{enumerate}