\secrel{11.3.1 The Cost of Fully-Manual Reclamation  82}

Let’s start by asking what the cost of these operations is. We might begin by
assuming that malloc has an associated register pointing into the store (like
new-loc \ref{}), and on every allocation it simply obtains the next free
locations. This model is extremely simple—in fact, deceptively so. The problem
arises when we free these values. Provided the first free is the last malloc, we
would encounter no problem; but store data often do not follow a stack
discipline. When we free anything but the most recently allocated value, we
leave holes in the store. These holes lead to fragmentation, and in the worst
case we become unable to allocate any objects even though there is ample space
in the store—just split up across many fragments, no one of which is large
enough.

\Exercise{
In principle, we could eliminate fragmentation by making all the free space be
contiguous. What does it take to do so? Think through all the consequences and
sketch whether you can in fact do this manually.
}

While fragmentation remains an insuperable problem in most manual memory
management schemes, there is more to consider even in this seemingly simple
discipline. What happens when we free a value? The run-time system has to
somehow record that it is now available for future allocation. It does by
maintaining a free list: a linked-list of the free spaces. A little reflection
immediately suggests a question: where is the free list itself stored, and who
manages its memory? The answer is that the free list references are stored in
the freed cells, which immediately implies a minimum size for each allocation.

In principle, then, each malloc must now traverse the free list to find a
suitable freed spot. I say “suitable” because the allocator must make a complex
decision. Should it take the first slot that matches or a later one? And either
way, what does “matches” mean? Should it take only slots the exact right size,
or take larger slots and break them up into smaller ones (thereby increasing the
likelihood of creating unusably small holes)? And so on.

Developers like allocation to be cheap. Therefore, in practice, allocation
systems
\note{Failing to make allocation cheap makes developers try to encode tricks
based on reusing values, thereby reducing clarity and quite possibly also
correctness.}
tend to use just a fixed set of sizes, often in powers of two. This makes it
possible to maintain not one but many free lists, each of holes of the same size
(which is a power of two). A table refers to each of these lists, and indexing
in the table is cheap by using bit-shifting. In return, developers sacrifice
space, because objects not a power-of-two size will end up being needlessly
padded. (This is a classic computer science trade-off: trading space for time.)
Also, free must put the freed memory in the right slot, and perhaps even break
up larger blocks into multiple smaller blocks to prepare for future allocations.
Nothing about this model is inherently as cheap as it seems.
\note{In particular, free is not free.}

Of course, all this assumes that developers can function in even a sound, much
less complete, fashion. But they don’t.
