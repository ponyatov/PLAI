\secrel{8 Mutation: Structures and Variables 41}\secdown

It’s time for another

\bigskip
\textbf{Which of these is the same?}
\begin{itemize}
  \item \verb|f = 3|
  \item \verb|o.f = 3|
  \item \verb|f = 3|
\end{itemize}

Assuming all three are in Java, the first and third could behave exactly like
each other or exactly like the second: it all depends on whether f is a local
identifier (such as a parameter) or a field of the object (i.e., the code is
really this.f = 3).

In either case, we are asking the evaluator to permanently change the value
bound to f. This has important implications for other observers. Until now, for
a given set of inputs, a computation always returned the same value. Now, the
answer depends on when it was invoked: above, it depends on whether it was
invoked before or after the value of f was changed. The introduction of time has
profound effects on reasoning about programs.

However, there are really two quite different notions of change buried in the
uniform syntax above. Changing the value of a field (o.f = 3 or this.f = 3) is
extremely different from changing that of an identifier (f = 3 where f is bound
inside the method, not by the object). We will explore these in turn. We’ll
tackle fields below, and return to identifiers in section \ref{8_2_vars}.

\secrel{8.1 Mutable Structures   41}
\secdown
\input{8_1_1_model}
\input{8_1_2_scaffold}
\secrel{8.1.3 Interaction with Closures  . . 43}

Consider a simple counter:
\lsts{src/8/1/3_1.rkt}{rkt}
Every time it is invoked, it produces the next integer:
\lst{src/8/1/3_1.log}
Why does this work? It’s because the box is created only once, and bound to n,
and then closed over. All subsequent mutations affect the same box. In contrast,
swapping two lines makes a big difference:
\lsts{src/8/1/3_2.rkt}{rkt}
Observe:
\lst{src/8/1/3_2.log}
In this case, a new box is allocated on every invocation of the function, so the
answer each time is the same (despite the mutation inside the procedure). Our
implementation of boxes should be certain to preserve this distinction.

The examples above hint at an implementation necessity. Clearly, whatever the
environment closes over in new-loc must refer to the same box each time. Yet
something also needs to make sure that the value in that box is different each
time! Look at it more carefully: it must be lexically the same, but dynamically
different. This distinction will be at the heart of our implementation.

\secrel{8.1.4 Understanding the Interpretation of Boxes . . 44}

Let’s begin by reproducing our current interpreter:
\lst{src/8/1/4_1.rkt}
Because we’ve introduced a new kind of value, the box, we have to update the set
of values:
\lst{src/8/1/4_2.rkt}
Two of these cases should be easy. When we’re given a box expression, we simply
evaluate it and return it wrapped in a boxV:
\lst{src/8/1/4_3.rkt}
Similarly, extracting a value from a box is easy:
\lst{src/8/1/4_4.rkt}

By now, you should be constructing a healthy set of test cases to make sure
these behave as you’d expect.

Of course, we haven’t done any hard work yet. All the interesting behavior is,
presumably, hidden in the treatment of setboxC. It may therefore surprise you
that we’re going to look at seqC first instead (and you’ll see why we included
it in the core).

Let’s take the most natural implementation of a sequence of two instructions:
\lst{src/8/1/4_5.rkt}

That is, we evaluate the first term, then the second, and return the result of
the second.

You should immediately spot something troubling. We bound the result of
evaluating the first term, but didn’t subsequently do anything with it. That’s
okay: presumably the first term contained a mutation expression of some sort,
and its value is uninteresting (indeed, note that set-box! returns a void
value). Thus, another implementation might be this:
\lst{src/8/1/4_6.rkt}

Not only is this slightly dissatisfying in that it just uses the analogous
Racket sequencing construct, it still can’t possibly be right! This can only
work only if the result of the mutation is being stored somewhere. But because
our interpreter only computes values, and does not perform any mutation itself,
any mutations in (interp b1 env) are completely lost. This is obviously not what
we want.

\input{8_1_5_envhelp}
\secrel{8.1.6 Introducing the Store  . 48}
\secrel{8.1.7 Interpreting Boxes  . . 49}
\secrel{8.1.8 The Bigger Picture  . . 54}
\secup

\secrel{8.2 Variables   . . 57}\label{8_2_vars}

Now that we’ve got structure mutation worked out, let’s consider the other case:
variable mutation.

\secdown
\secrel{8.2.1 Terminology   . . 57}

First, our choice of terms. We’ve insisted on using the word “identifier” before because
we wanted to reserve “variable” for what we’re about to study. In Java, when we say
(assuming x is locally bound, e.g., as a method parameter)
\lsts{src/8/2/1.rkt}{java}
we’re asking to change the value of x. After the first assignment, the value of
x is 1; after the second one, it’s 3. Thus, the value of x varies over the
course of the execution of the method.

Now, we also use the term “variable” in mathematics to refer to function
parameters. For instance, in f(y) = y + 3 we say that y is a “variable”. That is
called a variable because it varies across invocations; however, within each
invocation, it has the same value in its scope. Our identifiers until now have
corresponded to this notion of a variable. In contrast, programming variables
can vary even within each invocation, like the Java x above.
\note{If the identifier was bound to a box, then it remained bound to the same
box value. It’s the content of the box that changed, not which box the
identifier was bound to.}

Henceforth, we will use variable when we mean an identifier whose value can
change within its scope, and identifier when this cannot happen. If in doubt, we
might play it safe and use “variable”; if the difference doesn’t really matter,
we might use either one. It is less important to get caught up in these specific
terms than to understand that they represent a distinction that matters \ref{}.

\secrel{8.2.2 Syntax   . . 57}

\secrel{8.2.3 Interpreting Variables  . 58}

\secup

\secrel{8.3 The Design of Stateful Language Operations  . . 59}

Though most programming languages include one or both kinds of state we have
studied, their admission should not be regarded as a trivial or foregone matter.
On the one hand, state brings some vital benefits:

\begin{itemize}
  \item 
State provides a form of modularity. As our very interpreter demonstrates,
without explicit stateful operations, to achieve the same effect:
\begin{itemize}
  \item 
We would need to add explicit parameters and return values that pass the
equivalent of the store around.
  \item
These changes would have to be made to all procedures that may be involved in a
communication path between producers and consumers of state.
\end{itemize}

Thus, a different way to think of state in a programming language is that it is
an implicit parameter already passed to and returned from all procedures,
without imposing that burden on the programmer. This enables procedures to
communicate “at a distance” without all the intermediaries having to be aware of
the communication.

  \item
State makes it possible to construct dynamic, cyclic data structures, or at
least to do so in a relatively straightforward manner \ref{sec9}.

  \item
State gives procedures memory, such as new-loc above. If a procedure could not
remember things for itself, the callers would need to perform the remembering on
its behalf, employing the moral equivalent of store-passing. This is not only
unwieldy, it creates the potential for a caller to interfere with the memory for
its own nefarious purposes (e.g., a caller might purposely send back an old
store, thereby obtaining a reference already granted to some other party,
through which it might launch a correctness or security attack).

\end{itemize}

On the other hand, state imposes real costs on programmers as well as on
programs that process programs (such as compilers). One is “aliasing”, which we
discuss later \ref{}. Another is “referential transparency”, which too I
hope to return to \ref{}. Finally, we have described above how state provides a
form of modularity. However, this same description could be viewed as that of a
back-channel of communication that the intermediaries did not know and could not
monitor. In some (especially security and distributed system) settings, such
back-channels can lead to collusion, and can hence be extremely dangerous and
undesirable.

Because there is no optimal answer, it is probably wise to include mutation
operators but to carefully delinate them. In Standard ML, for instance, there is
no variable mutation, because it is considered unnecessary. Instead, the
language has the equivalent of boxes (called refs). One can easily simulate
variables using boxes (e.g., see new-loc and consider how it would be written
with variables instead), so no expressive power is lost, though it does create
more potential for aliasing than variables alone would have (aliasing
\ref{aliasing}) if the boxes are not used carefully.

In return, however, developers obtain expressive types: every data structure is
considered immutable unless it contains a ref, and the presence of a ref is a
warning to both developers and programs (such as compilers) that the underlying
value may keep changing. Thus, for instance, if b is a box, a developer should
be aware that replacing all instances of (unbox b) with v, where v is bound to
(unbox b), is unwise: the former always fetches the current value in the box,
while the latter may be referring to an older content. (Conversely, if the
developer wants the value at a certain point in time, oblivious to future
mutations to the box, they should be sure to retrieve and bind it rather than
always use unbox.)

\secrel{8.4 Parameter Passing   . 60}

\secup
