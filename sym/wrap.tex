\secrel{\ru{Классы-обертки}}

Можно обратить внимание, что определяя класс-обертку для каждого нового
типа\note{в нашем случае класс \class{Num} входит в ядро языка, но мы можем
рассмотреть на его примере расширение ядра новыми типами, например если вы
захотите добавить \term{комплексные числа} или битовые строки}, мы
переопределяем:
\begin{description}
\item[тип \var{val}]\ \\
в случае числа мы использовали примитивный тип \class{float}, но в общем случае
это может быть любой сложный объект: файл, соединение в БД, окно GUI,
спецификация формата файла или протокола,\ldots
\item[функцию \class{string} $\rightarrow$ \var{val}]
\item[обратную функцию \var{val}$\rightarrow$\class{string}]
\item[функцию создающую объект-обертку] из объекта обертываемого типа
\end{description}

