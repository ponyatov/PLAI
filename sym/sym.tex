\secrel{$\oplus$ \class{Sym}: \ru{универсальный} [sym]bol \ru{тип
данных}}\secdown

Прежде чем мы продолжим с парсером, нам нужно уточнить, как мы
будем хранить результаты разбора. Выше \ref{allparsing}\ в качестве хранилища
\termdef{дерева разбора}{дерево разбора} используется традиционное для
\lisp-мира представление в виде списков, где \emph{первым элементом идет вершина
дерева}, а в следующих элементах находятся поддеревья.

Но если мы внимательно рассмотрим синтаксис современных main\-stream языков
программирования\note{и текстовых форматов данных}, и учтем что нашей
целью является создание метаязыка для трансформации программ, более
удобным и правильным кажется использование не списков, а \termdef{атрибутных
деревьев}{атрибутное дерево}.

Вместо класса \class{ArithC} нам нужно дерево классов, наследованных от одного
\term{виртуального} базового класса \class{Sym}. Это требование вытекает из
жесткой типизации и отсутствия в \cpp\ поддержки \term{гетерогенных} структур
данных, способных хранить в себе \emph{разнотипные} элементы. Если мы
используем набор типов\note{которые будут хранить в себе элементы нашего дерева
разбора}, наследованных от одного виртуального класса, мы сможем оперировать ими
через указатели на базовый класс \class{Sym*}, и воспользоваться динамическими
хранилищами из библиотеки \cpp\ STL: \class{vector<Sym*>} и
\class{map<string,Sym*>}.

\lstxl{hpp.hpp}{sym/head.hpp}{C++}
\lstxl{cpp.cpp}{sym/head.cpp}{C++}

\clearpage
Любой элемент данных должен:
\begin{description}
\item[хранить свое значение и идентифицировать свой тип]\ \\\emph{универсальным
представлением любых данных является строка}:
\lstxl{hpp.hpp}{sym/tagval.hpp}{C++}

В \cpp\ есть средства идентификации класса по указателю на экземпляр (RTTI,
\fn{typeid()}), и по крайней мере тип тэга должен быть \class{static
Sym*}\note{указывать на экземпляр \class{Clazz::Sym *symbol, *number,
*string,..} т.е. на элемент данных типа ``класс'', существующий внутри нашей
DLR; собственно на текущий момент нам от него нужно только название класса} или
хотя бы \class{static string}, но для максимального \emph{упрощения кода} мы
используем просто строку.

\item[конструктор]\ \\создает элемент данных по паре тэг:значение <T:V>
\lstxl{hpp.hpp}{sym/constv.hpp}{C++}
\item[лексемы] (=\term{токены} =\term{терминалы})\ \\должны иметь конструктор из
строки, выделенной лексером
\lstxl{hpp.hpp}{sym/constoc.hpp}{C++}
\lstxl{cpp.cpp}{sym/constv.cpp}{C++}
\clearpage
\item[базовый класс]\ \\должен иметь \emph{по крайней мере одну} виртуальную
функцию. Классически для этого рекомендуется использовать виртуальный
деструктор. Позже \ref{symgc} при реализации управления памятью мы так и
сделаем, а пока воспользуемся следущим требованием:
\item[выводить себя в текстовом представлении]\ \\для отладки или трансляции
\lstxl{hpp.hpp}{sym/dump.hpp}{C++}
\lstxl{cpp.cpp}{sym/dump.cpp}{C++}

\item[хранить в себе вложенные элементы]\ \\\emph{самое важное для наших целей
требование}. 

\item[1) с доступом по целочисленному индексу] (аналог массивов)\ \\
простейший случай, типичное решение для реализации деревьев с \emph{любым}
количеством ветвей, для реализации применим C++ шаблонный класс STL
\class{vector<Sym*>}. Каждый \class{Sym}-объект может выступать
в роли \term{массива с доступом на целому индексу}.

\lstxl{hpp.hpp}{sym/nest.hpp}{C++}
\lstxl{cpp.cpp}{sym/nest.cpp}{C++}

\item[2) с доступом по имени]\ \\необходимо для реализации таблиц
символов\note{подробно ср\'{е}ды и применение lookup таблиц рассмотрено в
\ref{env}}, адресации полей классов, хранения других атрибутов\note{например
поле \var{doc} для документирования аналогично \var{\_\_doc\_\_} в \py}, в общем
реализации \term{атрибутного} дерева.

\lstxl{hpp.hpp}{sym/lookup.hpp}{C++}

\item[выводить себя в текстовом виде в виде дерева]\ \\так как предполается что
мы работаем исключительно с атрибутными деревьями, необходимо наличие средства
просмотра не только содержимого, но и структуры любого элемента данных.

Мы должны иметь возможность визуально определять вложенность, а в идеале\ ---
иметь возможность копировать полученный \termdef{дамп}{дамп} в исходный код в
виде, который распознает парсер\note{сейчас это потребует значительного
усложнения синтаксиса и кода парсера, поэтому реализацию
\termdef{bootstrap}{bootstrap}-дампа пока отложим}.

\lstxl{hpp.hpp}{sym/dumptree.hpp}{C++}

\begin{tabular}{l l}
\fn{head()} & вывод заголовка \\
\fn{pad(int)} & отбивка табуляциями каждой строки дампа,\\
& определяется вложенностью \var{depth}\\
\fn{dump(depth++)} & \emph{рекурсивный} дамп дерева \\
\end{tabular}

\lstxl{cpp.cpp}{sym/dumptree.cpp}{C++}

Итак, у нас есть универсальный базовый тип для представления атрибутных
деревьев. Но для практического применения желательно добавить пару 
производных типов для представления чисел и операторов.
Через наследование \class{Sym} мы можем определять новые элементы данных,
отличающиеся особенным поведением\note{например \class{оператор}ы могут уметь
вычислять математические функции от своих вложенных аргументов}, или
\term{обертывать} классы\note{элементы GUI, интерфейсы СУБД, сетевые
соединения,\ldots} и произвольные \emph{библиотеки}, не связанные с ядром DLR.   

\end{description}

\secrel{\class{Num}: [num]bers \ru{числа}}

\secup