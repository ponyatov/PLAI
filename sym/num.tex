\secrel{\class{Num}: [num]bers \ru{числа}}

Базовый класс \class{Sym} хорошо подходит для хранения строк и идентификаторов,
числа также можно хранить в виде строк, но лучше определить отдельный
\term{класс-обертку} заменив тип \var{val} на \class{float}:

\lstxl{hpp.hpp}{sym/num.hpp}{C++}
\lstxl{cpp.cpp}{sym/num.cpp}{C++}
\begin{tabular}{l l}
\class{float} \var{val} & обернутый объект \\
\fn{Num(string)} & токен-конструктор (используется в лексере) \\
\fn{Num(float)} & конструктор обертки (будет использоваться при\\&упаковке
результатов вычислений в интерпретаторе \ref{cppinterp})\\
\fn{head()} & так как у нас изменился тип хранимого значения,\\&необходимо
переопределить функцию \var{val} $\rightarrow$ \class{string}\\ 
\end{tabular}

\lstxl{lpp.lpp}{parse/num.lpp}{C++}
