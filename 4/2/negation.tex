\secrel{Extension: Unary Negation\\
\ru{Расширение: унарный минус}}

Now let’s consider another extension, which is a little more interesting: unary
negation.
\ru{Теперь давайте рассмотрим другое расширение, которое является немного более
интересным: унарный минус.}
This forces you to do a little more work in the parser because, depending on
your surface syntax, you may need to look ahead to determine whether you’re in
the unary or binary case.
\ru{Оно заставит вас сделать немного больше работы в парсере, потому что, в
зависимости от вашего frontend синтаксиса, вам требуется посмотреть вперед для
определения, находитесь ли вы в унарном или бинарном случае.}
But that’s not even the interesting part\,!
\ru{Но это даже не самое интересное\,!}

There are many ways we can desugar unary negation.
\ru{Существует много способов, которыми мы можем сделать обессахаривание
унарного минуса.}
We can define it naturally as \ru{Мы можем определить его естественно как}
$-b=0-b$, or we could abstract over the desugaring of binary subtraction with
this expansion: \ru{или мы можем построить абстракцию используя обессахаривание
бинарного вычитания через это расширение:} $-b=0+ -1 \times b$.

\DoNow{
Which one do you prefer\,? Why\,?\\
\ru{Что вы предпочтете\,? Почему\,?}
}

It’s tempting to pick the first expansion, because it’s much simpler.
\ru{Очень заманчиво выбрать первое расширение, потому что оно намного проще.}
Imagine we’ve extended the \class{ArithS} datatype with a representation of
unary negation:
\ru{Представьте что мы расширили тип данных \class{ArithS} представлением
унарного минуса:}
\lst{4/p18_2.rkt}
Now the implementation in \class{desugar} is straightforward:
\ru{Теперь реализация \class{desugar} проста:}
\lst{4/p18_3.rkt}
Let’s make sure the types match up.
\ru{Давайте удостоверимся что типы совпадают.}
Observe that $e$ is a \class{ArithS} term, so it is valid to use as an argument
to \class{bminusS}, and the entire term can legally be passed to \term{desugar}.
\ru{Заметим что $e$ терм класса \class{ArithS}, так что правомерно его
использовать как аргумент \class{bminusS}, и все выражение может быть корректно
послано в \term{обессахариватель}.}
It is therefore important to \emph{not} desugar $e$ but rather embed it directly
in the generated term.
\ru{Поэтому важно \emph{не обессахаривать} $e$, а встроить его напрямую в
сгенерированное выражение.}
This embedding of an input term in another one and recursively calling desugar
is a common pattern in desugaring tools;
\ru{Это встраивание входного терма в другой и рекурсивный вызов
обессахирвателя\ --- типовой шаблон при использовании средств обессахаривания;}
it is called a \termdef{macro}{macro}
\ru{он называется \termdef{макро}{макро}}
(specifically, the ``macro'' here is this definition of \class{uminusS}
\ru{конкретно в этом случае \term{макро} является определение \class{uminusS}}).

However, there are two problems with the definition above:
\ru{Тем не менее, существует две проблемы, связанные с приведенным выше
определением:}
\begin{enumerate}

\item
The first is that the recursion is \termdef{generative}{generative
recursion}\note{If you haven’t heard of \term{generative recursion} before, read
the section on it in \emph{How to Design Programs}. Essentially, in generative
recursion the sub-problem is a computed function of the input, rather than a
structural piece of it. This is an especially simple case of generative
recursion, because the ``function'' is simple: it’s just the \class{bminusS}
constructor}, which forces us to take extra care. 
\ru{Во первых, рекурсия является \termdef{генеративной}{генеративная
рекурсия}\note{\ru{Если вы еще не слышали о \term{генеративной рекурсии},
прочитайте раздел о ней в \emph{How to Design Programs}. По сути, при
генеративной рекурсии подпроблемой является \emph{вычислимая функция} программно
синтезированная из входных данных, а не рекурсивный вызов функция от структурной
части этих данных. В нашем случае рассматривается простой вариант генеративной
рекурсии, потому что ``функция'' проста: это просто конструктор
\class{bminusS}}}, что заставляет нас быть особенно осторожными.}
We might be tempted to fix this by using a different rewrite:
\ru{Мы могли бы попытаться исправить это через следующее исправление:}
\lst{4/p18_4.rkt}
which does indeed eliminate the generativity.
\ru{которое действительно устраняет генеративность.}

\DoNow{
Unfortunately, this desugaring transformation won’t work at all\,!
Do you see why\,? If you don’t, try to run it.\\
\ru{К сожалению, эта обессахаривающая трансформация вообще не работает\,!
Вы видите почему\,? Если нет, попробуйте это запустить.}
}

\item
The second is that we are implicitly depending on exactly what \class{bminusS}
means;
\ru{Во вторых, мы находимся в неявной зависимости от то, что точно обозначает
\class{bminusS};}
if its meaning changes, so will that of \class{uminusS}, even if we don’t want
it to.
\ru{если его \term{семантика} изменится, о же случиться и с \class{uminusS},
даже если мы этого и не хотим.}
In contrast, defining a functional abstraction that consumes two terms and
generates one representing the addition of the first to $-1$ times the second,
\ru{В противоположность этому, определение функциональной абстрации, которая
получает два терма, и генерирует один, представляющий сложение первого терма со
вторым, умноженным на $-1$,}
and using this to define the desugaring of both
\class{uminusS} and \class{bminusS}, is a little more faulttolerant.
\ru{и использование ее для определения обессахаривания для обоих \class{uminusS}
и \class{bminusS}, является немного более отказоустойчивой.}

You might say that the meaning of subtraction is never going to change, so why
bother\,?
\ru{Вы можете сказать что значение вычитания никогда не изменится, так что
почему надо беспокоиться\,?}
Yes and no.
\ru{Да и нет.}
Yes, it’s \emph{meaning} is unlikely to change;
\ru{Да, его \emph{значение} вряд ли изменится;}
but no, its \emph{implementation} might.
\ru{но нет, его \emph{реализация} может.}
For instance, the developer may decide to log all uses of binary subtraction.
\ru{Например разработчик может решить логировать все использования двухместного
вычитания.}
In the macro expansion, all uses of unary negation would also get logged, but
they would not in the second expansion.
\ru{В макроподстановке все использования унарного минуса также будут
залогрованы, но не будут во второй подстановке.}

\end{enumerate}

Fortunately, in this particular case we have a much simpler option, which is to
define
\ru{К счастью, в этом конкретном случае мы имеем гораздо более простой вариант,
как можно определить}
$-b=-1\times b$.
This expansion works with the primitives we have, and follows structural
recursion.
\ru{Это раскрытие макроса работает с примитивами, которые у нас есть, и следует
структурной рекурсии.}
The reason we took the above detour, however, is to alert you to these problems,
and warn that you might not always be so fortunate.
\ru{Причиной, по которой мы отклонились выше, стала необходимость предупредить
вас об этих проблемах, и о том что вам не всегда будет так везти.}
