\secrel{8.1.3 Interaction with Closures  . . 43}

Consider a simple counter:
\lsts{src/8/1/3_1.rkt}{rkt}
Every time it is invoked, it produces the next integer:
\lst{src/8/1/3_1.log}
Why does this work? It’s because the box is created only once, and bound to n,
and then closed over. All subsequent mutations affect the same box. In contrast,
swapping two lines makes a big difference:
\lsts{src/8/1/3_2.rkt}{rkt}
Observe:
\lst{src/8/1/3_2.log}
In this case, a new box is allocated on every invocation of the function, so the
answer each time is the same (despite the mutation inside the procedure). Our
implementation of boxes should be certain to preserve this distinction.

The examples above hint at an implementation necessity. Clearly, whatever the
environment closes over in new-loc must refer to the same box each time. Yet
something also needs to make sure that the value in that box is different each
time! Look at it more carefully: it must be lexically the same, but dynamically
different. This distinction will be at the heart of our implementation.
