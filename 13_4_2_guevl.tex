\secrel{13.4.2 Guarding Evaluation  . 97}

We said above that this expands as we expect. Or does it? Let’s try the
following example:
\lsts{src/13/4/2/1.rkt}{rkt}
Observe that or returns the actual value of the first “truthy” value, so the
developer can use it in further computations. Therefore, this returns the value
of init. What do we expect it to be? Naturally, because we’ve negated the value
of init once, we expect it to be \#t. But evaluating it produces \#f!
\note{This problem is not an artifact of set!.
If instead of internal mutation we had, say, printed output, the printing would
have occurred twice.}

To understand why, we have to examine the expanded code. It is this:
\lsts{src/13/4/2/2.rkt}{rkt}
Aha! Because we’ve written the output pattern as
\lsts{src/13/4/2/3.rkt}{rkt}
This looked entirely benign when we first wrote it, but it illustrates a very
important principle when writing macros (or indeed any other program
transformation systems): do not copy code! In our setting, a syntactic variable
should never be repeated; if you need to repeat it in a way that might cause
multiple execution of that code, make sure you have considered the consequences
of this. Alternatively, if you meant to work with the value of the expression,
bind it once and use the bound identifier’s name subsequently. This is easy to
demonstrate:
\lsts{src/13/4/2/4.rkt}{rkt}
This pattern of introducing a binding creates a new potential problem: you may
end up evaluating expressions that weren’t necessary. In fact, it creates a
second, even more subtle one: even if it going to be evaluated, you may evaluate
it in the wrong context! Therefore, you have to reason carefully about whether
an expression will be evaluated, and if so, evaluate it once in just the right
place, then store that value for subsequent use.

When we repeat our previous example, that contained the set!, with my-or-4, we
see that the result is \#t, as we would have hoped.
