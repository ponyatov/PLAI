\secrel{17.1.1 A Lazy Application Example 196}

The lazy alternative has a distinguished history (for instance, this is what the
true - calculus uses), but returned to the fore from programming experiments
considering what might happen if certain operators did not evaluate the
arguments at application but only when the value was needed. For instance,
consider the definition
\lsts{17/1/1/1.rkt}{rkt}
In ordinary Racket, this is clearly ill-defined: ones has not yet been defined
(on the left) when we try to evaluate it (on the right), so this results in an
error. If, however, we do not try to evaluate it until we actually need it, by
that time the definition is well-formed. Because each rest obtains another ones,
this produces an infinite list.

We’ve glossed over a lot that needs explaining. Does the ones in the rest
position of the cons evaluate to a copy of that expression, or to the result of
the very same expression itself? In other words, have we simply created an
infinitely unfolding list, or have we created an actually cyclic one?

This depends in good part on whether or not our language has mutation. If it
does, then perhaps we can modify each of the cells of the resulting list, which
means we can observe the difference between the two implementations above: in
the unfolded version mutating one first will not affect another, but in the
cyclic one, changing one will affect them all. Therefore, in a language with
mutation, we might argue that this should represent a lazy unfolding, but not an
actual cyclic datum.

Keep this discussion in mind. We cannot resolve it right now; rather, let us
examine lazy evaluation a little more, then return to this question \ref{}.
