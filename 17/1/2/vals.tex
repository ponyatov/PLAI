\secrel{17.1.2 What Are Values? . 197}

If we return to our core higher-order function interpreter \ref{}, we recall
that we have two kinds of values: numbers and closures. If we want to support
lazy evaluation instead, we need to ask what happens at function application.
What exactly are we passing?

This seems obvious enough: in a lazy application semantics, we need to pass
expressions. But a moment’s thought shows that this can be problematic.
Expressions contain identifier names, and we don’t want them to be accidentally
bound.
\note{And now these truly will be identifiers, not variables, as we will see
\ref{}.}

For instance, suppose we have
\lsts{17/1/2/1.rkt}{rkt}
and apply it as follows:
\lsts{17/1/2/2.rkt}{rkt}

\DoNow{
What should this produce?
}
Clearly, we should get an error reporting x as not being bound.

Now let’s trace it. The first application creates a closure where x is bound to
3. If we now bind y to (+ x 4), this results in the expression (+ x (+ x 4)) in
an environment where x is bound. As a result we get the answer 10, not an error.

\DoNow{
Have we made a subtle assumption above?
}

Yes we have: we’ve assumed that + evaluates arguments and returns numeric
answers. Perhaps + also behaves lazily; we will study this issue in a moment.
Nevertheless, the central point remains: if we are not careful, this erroneous
expression will produce some kind of valid answer, not an error.

In case you think this is entirely a problem with erroneous programs, and can
hence be treated specially (e.g., first scan the program source for free
identifiers), here is another use of the same f:
\lsts{17/1/2/3.rkt}{rkt}

\DoNow{
What should this produce?
}
We would expect this to produce the result of (+ 3 5) (probably 8). However, if
we substitute x inside the arithmetic expression, we would get (+ 3 3) instead.

This latter example holds the key to our solution. In the latter example, the
problem ostensibly arises only when we use environments; if instead we use
substitution, x in the application is substituted as soon as we encounter the
let, and the result is what we expect. In fact, note that the same argument
holds earlier: if we had used substitution, the very occurrence of x would have
signaled an error. In short, we have to make sure our environment-based
implementation matches what substitution would have done.
Doesn’t that sound familiar!

In other words, the solution is to bundle the argument expression with its
environment: i.e., create a closure. This closure has no parameters, so it is
effectively a thunk. We could use existing functions to represent these thunks,
but our instinct should tell us that it is better to use different data
representations for logically different purposes: closV for user-created
closures, and something else for internally-created ones. Indeed, as we will
see, it will have been wise to keep them separate because there is one place
where it is critical we can tell them apart.
\note{Indeed, this demonstrates that functions have two uses: to substitute
names with values, and also to defer substitution. let is the former without the
latter; thunks are the latter without the former. We have already established
that the former is valuable in its own right; this section shows that the same
is true of the latter.}

To conclude this discussion, here is our new set of values:
\lsts{17/1/2/4.rkt}{rkt}
The first two variants are exactly the same; the third is new, and as we
discussed, is effectively a parameter-less procedure, as its type suggests.
