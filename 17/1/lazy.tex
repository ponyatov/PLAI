\secrel{17.1 Lazy Application 196}
\secdown

Let’s start by considering when parameters are reduced to values. That is, do we
substitute formal parameters with the value of the actual parameter, or with the
actual parameter expression itself? If we define
\lsts{17/1/1.rkt}{rkt}
and invoke it as
\lsts{17/1/2.rkt}{rkt}
does that reduce to
\lsts{17/1/3.rkt}{rkt}
or to
\lsts{17/1/4.rkt}{rkt}
? The former is called eager application, while the latter is lazy. Of course we
don’t want to return to defining interpreters by substitution, but it is always
useful to think of substitution as a design principle.
\note{Some people also use the term strict for the former. A more arcane
terminology is applicative-order evaluation for the former and normal-order
evaluation for the latter. Or, call-by-value for the former and call-by-name or
call-by-need for the latter. The last two terms—by-name versus by-need—actually
represent a technical distinction we will see below.
This concludes our name-dump.}

\secrel{17.1.1 A Lazy Application Example . . . . . . . . . . . . . . . . . . 196}

\secrel{17.1.2 What Are Values? . . . . . . . . . . . . . . . . . . . . . . . 197}

\secrel{17.1.3 What Causes Evaluation? . 198}

Let us now return to discussing arithmetic expressions. On evaluating (+ 1 2), a
lazy application interpreter could return any number of things, including
(suspendV (+ 1 2) mt-env). In this way suspended computation could cascade on
suspended computation, and in the limiting case every program would return
immediately with an “answer”: the thunk representing the suspension of its
computation.
\note{It is legitimate to write mt-env here because even if the (+ 1 2)
expression was written in a non-empty environment, it has no free identifiers,
so it doesn’t need any of the environment’s bindings.}

Clearly, something must force a suspension to be lifted. (Lifting a suspension
means, of course, evaluating its body in the stored environment.) Those
expression positions that undo suspensions are called strictness points. The
most obvious strictness point is the interactive environment’s printer, because
a user clearly would not use such an environment if they did not wish to see
answers. We will embody the act of lifting suspension in the procedure strict:
\lsts{17/1/3/1.rkt}{rkt}
where the returned Value is guaranteed to not be a suspendV. We can imagine the
printer as wrapping strict around the result of evaluating the program, to
obtain a value to print.

\DoNow{
What impact would using closures to represent suspended computation have had?
}

The definition of strict above depends crucially on being able to distinguish
deferred computations—which are internally-constructed closures—from
user-defined closures. Had we conflated the two, then we would have to guess
what to do with zero-argument closures. If we fail to further process them, we
might incorrectly get an error (e.g., + might get a thunk rather than the
numeric value residing inside it). If we do process it further, we might
accidentally force a user-defined thunk prematurely. In short, we need a flag on
thunks telling us whether they are internal or user-defined. For clarity, our
interpreter uses a separate variant.

Let us now return to the interaction between strict and the interpreter.
Unfortunately, as we have defined things, this will cause an infinite loop. The
act of trying to interpret an addition creates a suspension, which strict tries
to undo by forcing the interpreter to interpret an addition, which.... Clearly,
therefore, we cannot have every expression simply suspend its computation;
instead, we will limit suspension to applications. This suffices to give us the
rich power of laziness, without making the language absurd.

\secrel{17.1.4 An Interpreter . . . . . . . . . . . . . . . . . . . . . . . . . . 199}

\secrel{17.1.5 Laziness and Mutation . . . . . . . . . . . . . . . . . . . . . 201}

\secrel{17.1.6 Caching Computation . . . . . . . . . . . . . . . . . . . . . 201}

\secup
