\secrel{17.1.4 An Interpreter 199}

As usual, we will define the interpreter in cases.
\lsts{17/1/4/1.rkt}{rkt}

Numbers are easy: they are already values, so there is no point needlessly
suspending them:
\lsts{17/1/4/2.rkt}{rkt}

Closures, similarly, remain the same:
\lsts{17/1/4/3.rkt}{rkt}

Identifiers should just return whatever they are bound to:
\lsts{17/1/4/4.rkt}{rkt}

The arguments of arithmetic expressions are usually defined as strictness
points, because otherwise we would simply have to implement the actual
arithmetic elsewhere:
\lsts{17/1/4/5.rkt}{rkt}

Finally, we have application. Here, instead of evaluating the argument position,
we suspend it. The function position has to be a strictness point, however,
otherwise we wouldn’t know what function to apply and hence how to continue the
computation:
\lsts{17/1/4/6.rkt}{rkt}

And that’s it! By adding a new kind of answer, inserting a few calls to strict,
and replacing interp with suspendV in the argument position of application, we
have turned our eager application interpreter into one with lazy application.
Yet this small change has such enormous impact on the programs we write! For a
more thorough examination of this impact, study Haskell or the \#lang lazy
language in Racket.

\Exercise{
If we instead replace the identifier case with (strict (lookup n env)) (i.e.,
wrapped strict around the result of looking up an identifier), what impact would
it have on the language? Consider richer languages with data structures, etc.
}

\Exercise{
Construct programs that produce different results in a lazy evaluation than an
eager evaluation (i.e., the same program text with different answers in the two
cases). Try to make the differences interesting, i.e., beyond whether one
returns a suspendV while the other doesn’t. For instance, does one terminate or
produce an error while the other one doesn’t?
}

\Exercise{
Instrument both interpreters to count the number of steps they take to return
answers. For programs that produce the same answer under both evaluation
strategies, does one strategy always take more steps than the other?
}
