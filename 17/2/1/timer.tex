\secrel{17.2.1 Motivating Example: A Timer . 202}

Suppose we were trying to implement a timer that measures elapsed time. Ideally,
we would like to write a program such as this:
\lsts{17/2/1/1.rkt}{rkt}
In JavaScript, we might write:
\lsts{17/2/1/2.java}{java}
On most machines this Racket expression, or the value of elapsed in JavaScript,
will evaluate to 0 or some other very small number. This is because these
programs represent one measure of the elapsed time: that at the second
invocation of the procedure that gets the current time. This gives us an
instanteous time split, but not an actual timer.

In most languages, to build an actual timer, we would have to create an instance
of some sort of timer object, and install a callback. Every time the clock
ticks, the timer object—representing the operating system—invokes the callback.
The callback is then responsible for updating values in the rest of the system,
and hopefully doing so globally and consistently. However, it cannot do so by
returning values, because it would return to the operating system, which is
agnostic to and does not care about our application; therefore, the callback is
forced to perform its action through mutation. In JavaScript, for instance:
\lsts{17/2/1/3.java}{java}
assuming we have an HTML page with an id named curTime, and that the onload or
other callback invokes startTimer.

One alternative to this spaghetti code is for the application program to
repeatedly poll the operating system for the current time. However:
\begin{itemize}
  \item 
Calling too frequently wastes resources, while calling too infrequently results
in incorrect answers. However, to call at just the right resolution, we would
need a timer signal in the first place!
  \item 
While it may be possible to create such a polling loop for regular events such
as timers, it is impossible to do so accurately for unpredictable behaviors such
as user input (whose frequency cannot, in general, be predicted).
  \item 
On top of all this, writing this loop pollutes the program’s structure and
forces the developer to sustain this extra burden.
\end{itemize}

The callback-based solution, however, demonstrates an inversion of control.
Instead of the application program calling the operating system, the operating
system has now been charged with calling (into) the application program. The
reactive behavior that should have been deeply nested, inside the display
expression, has instead been brought to the top-level, and its value drives the
other computations. The fundamental cause for this is that the world is in
control, not the program, so external stimuli determine when and how the program
should next run, not intrinsic program expressions.
