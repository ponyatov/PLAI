\secrel{17.2.3 The Alternative: Reactive Languages . 204}

Consider the FrTime (pronounced “Father Time”) language in DrRacket. If we run
this expression at the interactions window, we still get 0 or some other very
small non-negative number:
\note{In DrRacket v5.3, you must select the language from the Language menu;
writing \#lang frtime will not provide the interesting interactions window
behavior.}
\lsts{17/2/3/1.rkt}{rkt}
In fact, we can try several other expressions and see that FrTime seems to have
exactly like traditional Racket.

However, it also binds a few additional identifiers. For instance, it provides a
value bound to seconds. If we type this into the interaction prompt, we get
something very interesting! First we see 1353030630, then a second later
1353030631, another second later 1353030632, and so on. This kind of value is
called a behavior: a value that changes over time. Except we haven’t written any
callbacks or other code to keep it current.

A behavior can be used in computations. For instance, we can write (- seconds
seconds), and this always evaluates to 0. Here are some more expressions to try
at the interaction prompt:
\lsts{17/2/3/2.rkt}{rkt}
As you can see, being a behavior is “sticky”: if any sub-expression is a
behavior, so is its enclosing expression.

Thanks to this evaluation model, every time seconds updates the entire
application happens afresh: as a result, even though we have written seemingly
simple expressions without any explicit loop-like control, the program still
“loops”. In other words, having explored an application semantics where
arguments are evaluated once, then another where they may be evaluated zero
times, now we have one where they are evaluated as many times as necessary, and
the entire corresponding function with them. As a consequence, reactive values
that are “inside” an expression no longer brought “outisde”; rather, they can
reside nested inside expressions, giving programmers a more natural means of
expression. This style of evaluation is called dataflow or functional reactive
programming.
\note{Historically, dataflow has tended to refer to languages with first-order
functions, whereas functional reactive languages support higher-order functions
too. See the \href{http://www.flapjax-lang.org/}{Flapjax Web site}.}

FrTime implements what we call transparent reactivity, whereby the programmer
can inject a reactive behavior anywhere in a program’s evaluation without
needing to make any syntactic changes to its context. This has the virtue of
making it easy to inject reactivity into existing programs, but it can make the
evaluation and cost model more complex for programmers. In other languages,
programmers can instead explicitly introduce behavior through appropriate
primitives, trading convenience for greater predictability. FrTime’s sister
language, Flapjax, an extension of JavaScript, provides both modes.
