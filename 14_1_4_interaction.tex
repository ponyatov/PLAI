\secrel{14.1.4 Interaction with State  . 107}

The fundamental difference between cookies and hidden fields is that all pages
share the same cookie, but each page has its own hidden fields.

First, let’s consider a sequence of interactions with the existing program that
uses read-number/suspend (at both interaction points). It looks like this:
\lsts{src/14/1/4/1.rkt}{rkt}
Thus, resuming with label 2 appears to represent adding 3 to the given argument
(i.e., form field value). To be sure,
\lsts{src/14/1/4/2.rkt}{rkt}
So far, so good. Now suppose we use label 1 again:
\lsts{src/14/1/4/3.rkt}{rkt}
Observe that this new program execution needs to be resumed by using label 3,
not 1.
Indeed,
\lsts{src/14/1/4/4.rkt}{rkt}
But we ought to ask, what happens if we reuse label 2?

\DoNow{
Try (resume 2 10).
}

Doing this is tantamount to resuming the old computation. We therefore expect it
produce the same answer as before:
\lsts{src/14/1/4/5.rkt}{rkt}

Now let’s create a stateful implementation. We can simulate this by observing
that each closure has its own environment, but all closures share the same
mutable state. We can simulate this using our existing read-number/suspend by
making sure we don’t rely on the closure behavior of lambda, i.e., by not having
any free identifiers in the body.
\lsts{src/14/1/4/6.rkt}{rkt}

\Exercise{
What do we expect for the same sequence as before?
}

\DoNow{
What happens?
}

Initially, nothing seems different:
\lsts{src/14/1/4/7.rkt}{rkt}
When we reuse the initial computation, we indeed get a new resumption label:
\lsts{src/14/1/4/8.rkt}{rkt}
which, when used, computes what we’d expect:
\lsts{src/14/1/4/9.rkt}{rkt}
Now we come to the critical step:
\lsts{src/14/1/4/10.rkt}{rkt}
It is unsurprising that the two resumptions of label 2 would produce different
answers, given that they rely on mutable state. The reason it’s problematic is
because of what happens when we translate the same behavior to the Web.

Imagine visiting a hotel reservation Web site and searching for hotels in a
city. In return, you are shown a list of hotels and the label 1. You explore one
of them in a new tab or window; this produces information on that hotel, and
label 2 to make the reservation. You decide, however, to return to the hotel
listing and explore another hotel in a fresh tab or window. This produces the
second hotel’s information, with label 3 to reserve at that hotel. You decide,
however, to choose the first hotel, return to the first hotel’s page, and choose
its reservation button—i.e., submit label 2. Which hotel did you expect to be
booked into? Though you expected a reservation at the first hotel, on most
travel sites, this will either reserve at the second hotel—i.e., the one you
last viewed, but not the one on the page whose reservation button you clicked—
or produce an error. This is because of the pervasive use of cookies on Web
sites, a practice encouraged by most Web APIs.
