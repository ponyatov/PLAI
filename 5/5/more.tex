\secrel{Oh Wait, There's More\,!\\
\ru{Подождите, есть кое-что еще\,!}}\label{waitmore}

Earlier, we gave the following type to \ru{Ранее мы описали следующий тип для}
\verb|subst|:
\lsts{5/5/1.rkt}{rkt}
Sticking to surface syntax for brevity \ru{Придерживаясь поверхностного
синтаксиса для краткости}, suppose we apply \ru{предположим мы применяем}
\verb|double| to \ru{к} \verb|(+ 1 2)|. This would substitute \ru{В результате
будет выполнена подстановка} \verb|(+ 1 2)| for each \ru{для каждого} \verb|x|,
resulting in the following expression \ru{в результате будет использовано
следующее выражение}\ ---\\\verb|(+ (+ 1 2) (+ 1 2))|\ --- for interpretation
\ru{для интерпретации}. Is this ne\-ces\-sa\-ri\-ly what we want \ru{Является ли
это тем, что мы хотим}\,?

When you learned algebra in school \ru{Когда вы изучали алгебру в школе}, you
may have been taught to do this differently \ru{вас возможно учили делать это по
другому}: first reduce the argument to an answer \ru{сначала редуцировать
аргумент до ответа} (in this case \ru{в нашем случае}, 3), then substitute the
answer for the parameter \ru{а только потом подставить ответ в параметре}. This
notion of substitution \ru{Эта концепция подстановки} might have the following
type instead \ru{может иметь следующий тип}:
\lsts{5/5/2.rkt}{rkt}
Careful now \ru{Теперь осторожно}: we can’t put raw numbers inside expressions
\ru{мы не можем использовать числа сами по себе внутри выражений}, so we’d have
to constantly wrap the number \ru{так что мы должны постоянно оборачивать
число} in an invocation of \ru{в экземпляр} \verb|numC|. Thus, it would make
sense for \ru{Таким образом, для} \verb|subst| to have a helper \ru{имело бы
смысл создать вспомогательную функцию} that it invokes after wrapping the first
parameter \ru{которая вызывается после оборачивания первого параметра}.
(In fact \ru{Фактически}, our existing \ru{наша существующая} \verb|subst|
would be a perfectly good candidate \ru{является идеальным кандидатом}:
because it accepts any \ru{она принимает на вход любой} \verb|ExprC| in the
first parameter \ru{в первом параметре}, it will certainly work just fine with a
\ru{так что она идеально сработает и для} numC.) \note{In fact, we don’t even
have substitution quite right\,! The version of substitution we have doesn’t
scale past this language due to a subtle problem known as \termdef{name
capture}{name capture}. Fixing substitution is complex, subtle, and an exciting
intellectual endeavor, but it’s not the direction I want to go in here. We’ll
instead sidestep this problem in this book. If you’re interested, however, read
about the \term{lambda calculus}, which provides the tools for defining
substitution correctly.}
\note{\ru{На самом деле, у нас даже нет подстановки в ее изначальном значении\,!
Версия подстановки, которую мы используем, не масштабируется для этого языка
из-за тонкой проблемы, известной как \termdef{захват имени}{захват имени}.
Исправление подстановки сложная, тонкая и захватывающая интеллектуальная
деятельность, но в этой книге я не хочу идти этой дорогой. Если вам тем не менее
интересно, прочитайте про \term{лябмда-исчисление}, которое предоставляет
средства корректного определения подстановки.}}

\Exercise{
Modify your interpreter to substitute names with answers, not expressions.\\
\ru{Модифицируйте ваш интерпретатор для подстановки имен результатом вычисления,
а не выражения.}
}

We’ve actually stumbled \ru{Мы на самом деле споткнулись} on a profound
distinction in programming languages \ru{на глубоком различии в языках
программирования}. The act of evaluating arguments before substituting them in
functions is called \ru{Процесс вычисления аргументов перед их подстановкой в
функциях называется} \termdef{eager application}{eager application} \ru{жадное
применение}{жадное применение}, while that of deferring evaluation is called
\ru{в то время как отложенное вычисление называется} \termdef{lazy
evaluation}{lazy evaluation} \ru{ленивые вычисления}{ленивые вычисления}\ ---
and has some variations \ru{и имеет несколько вариантов}. For now, we will
actually prefer the eager semantics \ru{Пока мы ограничимся жадной семантикой},
because this is what most mainstream languages adopt \ru{потому что ее
использует большинство широко используемых языков}. Later \ru{Позже} \ref{}, we
will return to talking about \ru{мы вернемся к обсуждению} the lazy application
semantics and its implications \ru{семантики ленивых вычислений и ее
применениям}.
