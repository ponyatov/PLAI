\secrel{The Interpreter, Resumed\\
\ru{Возвращаемся к интерпретатору}}\label{interpresumed}

Phew \ru{Фу}\,! Now that we’ve completed the definition of substitution
\ru{Теперь, когда мы завершили определение подстановки} (or so we think \ru{или
мы так думаем}), let’s complete the interpreter \ru{давайте доделаем
интерпретатор}. Substitution was a heavyweight step \ru{Подстановка была сложным
шагом}, but it also does much of the work \ru{но она также выполняет б\'{о}льшую
часть работы} involved in applying a function \ru{по применению функции}. It is
tempting to write \ru{Заманчиво написать}
\lsts{5/4/1.rkt}{rkt}
Tempting, but wrong \ru{Заманчиво, но неправильно}.

\DoNow{
Do you see why \ru{Вы видите почему}\,?
}

Reason from the types \ru{Дело в типах}. What does the interpreter return
\ru{Что должен возвращать интерпретатор}\,? \emph{Numbers} \ru{Числа}. What does
substitution return \ru{Что возвращает подстановка}\,? Oh, that’s right,
\emph{expressions} \ru{О, правильно, выражения}\,! For instance \ru{Например},
when we substituted in the body of \ru{когда мы подставили тело} \verb|double|,
we got back the representation of \ru{мы получили представление} \verb|(+ 5 5)|.
This is not a valid answer for the interpreter \ru{Это неправильное поведение
для интерпретатора}. Instead, it must be reduced to an answer \ru{Вместо этого
это выражение должно быть редуцировано до ответа}. That, of course, is precisely
what the interpreter does \ru{Естественно, вот что точно должен сделать
интерпретатор}:
\lsts{5/4/2.rkt}{rkt}
Okay, that leaves only one case: \termdef{identifiers}{identifier} \ru{Итак,
остается только один случай: \termdef{идентификаторы}{идентификатор}}. What
could possibly be complicated about them \ru{Что может быть сложного}\,? They
should be just about as simple as numbers \ru{Они должны быть так же просто
реализуемы, как числа}\,! And yet we’ve put them off to the very end \ru{И все
же мы оставили их на самый конец}, suggesting something subtle or complex is
afoot \ru{предполагая что там есть что-то тонкое и сложное}.

\DoNow{
Work through some examples to understand \ru{Пройдите через несколько примеров,
чтобы понять} what the interpreter should do in the identifier case \ru{что
интерпретатор должен делать в случае идентификатора}.
}

Let’s suppose we had defined \ru{Предположим что мы определили} \verb|double| as
follows \ru{следующим образом}:
\lsts{5/4/3.rkt}{rkt}
When we substitute \ru{Когда мы подставляем} \verb|5| for \ru{вместо} \verb|x|,
this produces the expression \ru{это создает выражение} \verb|(+ 5 y)|.
So far so good \ru{Все идет нормально}, but what is left to substitute \ru{но
что осталось для подстановки} \verb|y|\,? As a matter of fact \ru{Собственно
говоря}, it should be clear from the very outset \ru{должно быть ясно с самого
начала} that this definition of \ru{что это определение} \verb|double| is
\emph{erroneous} \ru{является \emph{ошибочным}}. The identifier
\ru{Идентификатор} \verb|y| is said to be \termdef{free}{free identifier}
\ru{нужно назвать \termdef{свободным}{свободный идентификатор}}, an adjective
\ru{прилагательное} that in this setting has negative connotations \ru{которое в
этой ситуации имеет негативный оттенок}.

In other words \ru{Другими словами}, the interpreter should never confront an
identifier \ru{интерпретатор никогда не должен сопоставлять такой
идентификатор}. All identifiers ought to be parameters \ru{Все идентификаторы
являющиеся параметрами} that have already been substituted \ru{которые уже были
замещены} (known as \termdef{bound}{bound identifier} identifiers \ru{известные
как \termdef{связанные}{связанный идентификатор} идентификаторы}\ --- here, a
positive connotation \ru{здесь в положительном смысле}) before the interpreter
ever sees them \ru{до того как интерпретатор их увидел}. As a result \ru{В
результате}, there is only one possible response given an identifier \ru{есть
только один положительный ответ для данного идентификатора}:
\lsts{5/4/4.rkt}{rkt}
And that’s it \ru{Вот и все},!

Finally \ru{В итоге}, to complete our interpreter \ru{для завершения нашего
интерпретатора}, we should define \ru{мы должны определить} \verb|get-fundef|:
\lsts{5/4/5.rkt}{rkt}
