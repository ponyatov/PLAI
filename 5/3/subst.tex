\secrel{Substitution\\\ru{Подстановка}}

Substitution is the act of replacing a name (in this case, that of the formal
parameter) in an expression (in this case, the body of the function) with another expression (in this case, the actual parameter). Let’s define its type:
\lsts{5/3/1.rkt}{rkt}
It helps to also give its parameters informative names:
\lsts{5/3/2.rkt}{rkt}
The first argument is what we want to replace the name with; the second is for
what name we want to perform substitution; and the third is in which expression
we want to do it.
\DoNow{
Suppose we want to substitute 3 for the identifier x in the bodies of the three
example functions above. What should it produce\,?
}
In double, this should produce (+ 3 3); in quadruple, it should produce (double
(double 3)); and in const5, it should produce 5 (i.e., no substitution happens
because there are no instances of x in the body).
\note{A common mistake is to assume that the result of substituting, e.g., 3
for x in double is (define (double x) (+ 3 3)). This is incorrect. We only
substitute at the point when we apply the function, at which point the
function’s invocation is replaced by its body. The header enables us to find the
function and ascertain the name of its parameter; but only its body remains for
evaluation. Examine how substitution is used to notice the type error that would
result from returning a function definition.}

These examples already tell us what to do in almost all the cases. Given a
number, there’s nothing to substitute. If it’s an identifier, we haven’t seen an
example with a different identifier but you’ve guessed what should happen: it
stays unchanged. In the other cases, descend into the sub-expressions,
performing substitution.

Before we turn this into code, there’s an important case to consider. Suppose
the name we are substituting happens to be the name of a function. Then what
should happen\,?
\DoNow{
What, indeed, should happen?
}
There are many ways to approach this question. One is from a design perspective:
function names live in their own “world”, distinct from ordinary program
identifiers. Some languages (such as C and Common Lisp, in slightly different
ways) take this perspective, and partition identifiers into different namespaces
depending on how they are used. In other languages, there is no such
distinction; indeed, we will examine such languages soon \ref{}.

For now, we will take a pragmatic viewpoint. Because expressions evaluate to
numbers, that means a function name could turn into a number. However, numbers
cannot name functions, only symbols can. Therefore, it makes no sense to
substitute in that position, and we should leave the function name unmolested
irrespective of its relationship to the variable being substituted. (Thus, a
function could have a parameter named x as well as refer to another function
called x, and these would be kept distinct.)

Now we’ve made all our decisions, and we can provide the body code:
\lsts{5/3/3.rkt}{rkt}

\Exercise{
Observe that, whereas in the numC case the interpreter returned n, substitution
returns in (i.e., the original expression, equivalent at that point to writing
(numC n). Why\,?
}
