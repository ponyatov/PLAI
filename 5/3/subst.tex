\secrel{Substitution\\\ru{Подстановка}}

\termdef{Substitution}{substitution} is the act of replacing a name
\ru{\termdef{Подстановка}{подстановка} это процесс замены имени}
(in this case, that of the formal parameter
\ru{в этом конкретном случае, это имя формального параметра})
in an expression \ru{в выражении}
(in this case, the \termdef{body}{function body} of the function
\ru{в этом случае это \termdef{тело функции}{тело функции}})
with another expression
\ru{на другое выражение}
(in this case, the actual parameter
\ru{значение пареметра}).
Let’s define its type
\ru{Давайте определим его тип}:
\lst{5/3/1.rkt}
It helps to also give its parameters informative names
\ru{Это поможет нам также дать его параметрам информативные имена}:
\lsts{5/3/2.rkt}{rkt}
The first argument is \ru{Первый аргумент} \term{what} we want to replace the
name with \ru{на что мы хотим заменить имя};
the second \ru{второй} \term{is} for what name we want to perform substitution
\ru{для какого имени мы хотим выполнить подстановку};
and the third is \ru{и третий это } \term{in} which expression we want to
do it \ru{в каком выражения мы хотим сделать это}.
\DoNow{
Suppose we want to substitute \ru{Предположим что мы хотим подставить} $3$ for
the identifier \ru{вместо идентификатора} $x$ in the bodies of the three example
functions above \ru{в телах трех примеров функций выше}.
What should it produce \ru{Что должно получиться в результате}\,? } In \ru{В}
\verb|double|, this should produce \ru{должно получиться} \verb|(+ 3 3)|; in
\ru{в} \verb|quadruple|, it should produce \ru{результат}
\verb|(double (double 3))|; \clearpage and in \ru{и в} \verb|const5|, it should
produce \ru{должно получиться} \verb|5| (i.e., no substitution happens \ru{т.е.
никакой подстановки не происходит} because there are no instances of \ru{потому
что нет вхождений} \verb|x| in the body \ru{в теле функции}).
\note{A common mistake is to assume that the result of substituting, e.g., 3 for
x in double is (define (double x) (+ 3 3)). This is incorrect. We only
substitute \emph{at the point when we apply the function}, at which point the
\termdef{function’s invocation}{function invocation} is replaced by its body.
The \termdef{header}{function header} enables us to find the function and
ascertain the name of its parameter; but \emph{only its body remains for
evaluation}. Examine how substitution is used to notice the type error that
would result from returning a function definition.}
\note{\ru{Распространенной ошибкой является предположение того что результат
подстановки, например} 3 \ru{для} x \ru{в} double \ru{это} (define (double x) (+
3 3)). \ru{Это неправильно. Мы выполняем подстановку только \emph{в точке где мы
применяем функцию}, в том месте где \termdef{вызов функции}{вызов функции}
заменяется на ее тело. \termdef{Заголовок}{заголовок функции} позволяет нам
найти функцию и установить имя ее параметра; но \emph{только ее тело остается
для вычисления}. Рассмотрите внимательно как происходит подстановка, чтобы
запомнить типичную ошибку которая может быть результатом возврата определения
функции.}}

\clearpage
These examples already tell us what to do in almost all the cases. \ru{Эти
примеры показывают нам что делать в большинстве случаев.} Given a number,
there’s nothing to substitute. \ru{Для числа не нужна подстановка.} If it’s an
identifier \ru{Если это идентификатор}, we haven’t seen an example with a
\emph{different} identifier \ru{мы еще не видели пример с \emph{другим}
идентификатором}, but you’ve guessed what should happen \ru{но вы предполагаете
что должно произойти}: it stays unchanged \ru{он остается неизменным}. In the
other cases \ru{В других случаях}, descend into the sub-expressions \ru{обходим
подвыражения}, performing substitution \ru{выполняя подстановку}.

Before we turn this into code \ru{Перед тем как превратить эти правила в код},
there’s an important case to consider \ru{необходимо рассмотреть важный случай}.
Suppose the name we are substituting happens to be the name of a function
\ru{Предположим что подставляемое имя оказывается именем функции}. Then what
should happen \ru{Что должно при этом произойти}\,?
\DoNow{
What, indeed, should happen \ru{И шо-таки должно быть}\,?
}
There are many ways to approach this question \ru{Есть много подходов к этому
вопросу}. One is from a design perspective \ru{Один подход с точки зрения
дизайна}: function names live in their own ``world'' \ru{имена функций живут в
их собственном ``мире''}, distinct from ordinary program identifiers \ru{и
отличаются от обычных идентификаторов в прогремме}. Some languages \ru{Некоторые
языки} (such as C and Common Lisp \ru{такие как \ci\ и Common \lisp}, in
slightly different ways \ru{немного различными способами}) take this perspective
\ru{идут этим путем}, and partition identifiers into different
\termdef{namespaces}{namespace} \ru{разделяют идентификаторы в различные \termdef{пространства имен}{пространство имен}}
depending on how they are used \ru{в зависимости от того как они используются}.
In other languages \ru{В других языках}, there is no such distinction \ru{такого
разделения нет}; indeed, we will examine such languages soon \ru{на самом деле
мы рассмотрим такие языки в ближайшее время} \ref{}.

For now, we will take a pragmatic viewpoint \ru{Теперь примем практическую точку
зрения}. Because expressions evaluate to numbers \ru{Так как выражения
вычисляются в числа}, that means a function name could turn into a number
\ru{это значит что имя функции в итоге превращается в число}. However, numbers
cannot name functions \ru{Тем не менее, числа не могут именовать функции}, only
symbols can \ru{это могут делать только символы}. Therefore, it makes no sense
to substitute in that position \ru{Так что нет смысла делать подстановку в этой
позиции}, and we should leave the function name unmolested \ru{и мы должны
оставить имя функции нетронутым} irrespective of its relationship to the
variable being substituted \ru{независимо от ее отношения к подставляемой
переменной}. (Thus, a function could have a parameter named \ru{Таким образом
функция может иметь параметр названный} x as well as refer to another
\emph{function} called \ru{одновременно со ссылкой на другую \emph{функцию} с
именем} x, and these would be kept distinct \ru{и они будут различимы}.)

Now we’ve made all our decisions \ru{Теперь мы приняли все наши решения}, and we
can provide the body code \ru{и можем предоставить body-код}:
\lsts{5/3/3.rkt}{rkt}

\Exercise{
Observe that \ru{Заметим что}, whereas in the \ru{в случае} numC case the
interpreter returned \ru{интерпретатор возвращает} n, substitution returns
\ru{подстановка возвращает} in (i.e., the original expression \ru{т.е.
оригинальное выражение}, equivalent at that point to writing \ru{в этом месте
эквивалентное} (numC n). Why \ru{Зачем}\,?
}
