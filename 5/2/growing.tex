\secrel{Growing the Interpreter\\
\ru{Выращиваем интерпретатор}}

Now we’re ready to tackle the interpreter proper.
\ru{Теперь мы готовы взяться за работающий интерпретатор.}
First, let’s remind ourselves of what it needs to consume.
\ru{Во первых, давайте вспомним что ему нужно на входе.}
Previously, it consumed only an expression to evaluate.
\ru{Ранее он получал только выражение для вычисления.}
Now it also needs to take a list of function definitions
\ru{Теперь ему также нужен список определений функций}
:
\lsts{5/2/1.rkt}{rkt}
Let’s revisit our old interpreter \ru{Давайте вернемся к нашему старому
интерпретатору} \ref{firstinterp}. In the case of numbers \ru{Что касается
чисел}, clearly we still return the number as the answer \ru{ясно что мы все еще
возвращаем число как результат}. In the addition and multiplication case \ru{В
ветках сложения и умножения}, we still need to recur \ru{нам все
еще нужна рекурсия} (because the sub-expressions might be complex \ru{так как
подвыражение может быть сложным}), but which set of function definitions do we
use \ru{но какой набор определений функций мы используем}\ ? Because the act of
evaluating an expression \ru{Так как процесс вычисления выражения} neither adds
nor removes function \termdef{definitions}{function definition} \ru{не добавляет
и не удаляет \termdef{определения}{определение функции}}, the set of definitions
remains the same \ru{набор определений остается тем же самым}, and should just be passed along unchanged in
the recursive calls \ru{и только должен передаваться неизменным между
рекурсивными вызовами}.
\lsts{5/2/2.rkt}{rkt}
Now let’s tackle application \ru{Теперь давайте поковыряем приложение}. First we
have to look up the function definition \ru{Сначала нам надо найти определение
функции}, for which we’ll assume we have a helper function of this type
available \ru{причем мы предполагаем что у нас есть для этого вспомогательная
функция}:
\lsts{5/2/3.rkt}{rkt}
Assuming we find a function of the given name \ru{Предполагая что мы нашли
функцию с заданным именем}, we need to evaluate its body \ru{нам нужно вычислить
ее тело}. However, remember what we said about identifiers and parameters
\ru{Однако помните что мы говорили об идентификаторах и параметрах}\,? We must
“search-and-replace” \ru{Мы должны сделать ``поиск-и-замену''}, a process you
have seen before in school algebra \ru{процесс который вы уже видели в
школьной алгебре} called \termdef{substitution}{substitution} \ru{который
называется \termdef{подстановка}{подстановка}}. This is sufficiently important
that \ru{Существенно что} we should talk first about substitution \ru{мы должны
сначала поговорить о подстановке} before returning to the interpreter \ru{перед
тем как вернуться к интерпретатору}\ \ref{interpresumed}.
