\secrel{6.2 Interpreting with Environments  27}

Now we can tackle the interpreter. One case is easy, but we should revisit all
the others:
\lsts{src/6/6_2_1.rkt}{rkt}
The arithmetic operations are easiest. Recall that before, the interpreter
recurred without performing any new substitutions. As a result, there are no new
deferred substitutions to perform either, which means the environment does not
change:
\lsts{src/6/6_2_2.rkt}{rkt}

Now let’s handle identifiers. Clearly, encountering an identifier is no longer
an error: this was the very motivation for this change. Instead, we must look up
its value in the directory:
\lsts{src/6/6_2_3.rkt}{rkt}

\DoNow{
Implement lookup.
}

Finally, application. Observe that in the substitution interpreter, the only
case that caused new substitutions to occur was application. Therefore, this
should be the case that constructs bindings. Let’s first extract the function
definition, just as before:
\lsts{src/6/6_2_4.rkt}{rkt}
Previously, we substituted, then interpreted. Because we have no substitution
step, we can proceed with interpretation, so long as we record the deferral of
substitution.
\lsts{src/6/6_2_5.rkt}{rkt}
That is, the set of function definitions remains unchanged; we’re interpreting
the body of the function, as before; but we have to do it in an environment that
binds the formal parameter. Let’s now define that binding process:
\lsts{src/6/6_2_6.rkt}{rkt}
the name being bound is the formal parameter (the same name that was substituted
for, before). It is bound to the result of interpreting the argument (because
we’ve decided on an eager application semantics). And finally, this extends the
environment we already have. Type-checking this helps to make sure we got all
the little pieces right.

Once we have a definition for lookup, we’d have a full interpreter. So here’s
one:
\lsts{src/6/6_2_7.rkt}{rkt}
Observe that looking up a free identifier still produces an error, but it has
moved from the interpreter—which is by itself unable to determine whether or not
an identifier is free—to lookup, which determines this based on the content of
the environment.

Now we have a full interpreter. You should of course test it make sure it works
as you’d expect. For instance, these tests pass:
\lsts{src/6/6_2_8.rkt}{rkt}
So we’re done, right?

\DoNow{
Spot the bug.
}
