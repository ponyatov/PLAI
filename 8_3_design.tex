\secrel{8.3 The Design of Stateful Language Operations  . . 59}

Though most programming languages include one or both kinds of state we have
studied, their admission should not be regarded as a trivial or foregone matter.
On the one hand, state brings some vital benefits:

\begin{itemize}
  \item 
State provides a form of modularity. As our very interpreter demonstrates,
without explicit stateful operations, to achieve the same effect:
\begin{itemize}
  \item 
We would need to add explicit parameters and return values that pass the
equivalent of the store around.
  \item
These changes would have to be made to all procedures that may be involved in a
communication path between producers and consumers of state.
\end{itemize}

Thus, a different way to think of state in a programming language is that it is
an implicit parameter already passed to and returned from all procedures,
without imposing that burden on the programmer. This enables procedures to
communicate “at a distance” without all the intermediaries having to be aware of
the communication.

  \item
State makes it possible to construct dynamic, cyclic data structures, or at
least to do so in a relatively straightforward manner \ref{sec9}.

  \item
State gives procedures memory, such as new-loc above. If a procedure could not
remember things for itself, the callers would need to perform the remembering on
its behalf, employing the moral equivalent of store-passing. This is not only
unwieldy, it creates the potential for a caller to interfere with the memory for
its own nefarious purposes (e.g., a caller might purposely send back an old
store, thereby obtaining a reference already granted to some other party,
through which it might launch a correctness or security attack).

\end{itemize}

On the other hand, state imposes real costs on programmers as well as on
programs that process programs (such as compilers). One is “aliasing”, which we
discuss later \ref{}. Another is “referential transparency”, which too I
hope to return to \ref{}. Finally, we have described above how state provides a
form of modularity. However, this same description could be viewed as that of a
back-channel of communication that the intermediaries did not know and could not
monitor. In some (especially security and distributed system) settings, such
back-channels can lead to collusion, and can hence be extremely dangerous and
undesirable.

Because there is no optimal answer, it is probably wise to include mutation
operators but to carefully delinate them. In Standard ML, for instance, there is
no variable mutation, because it is considered unnecessary. Instead, the
language has the equivalent of boxes (called refs). One can easily simulate
variables using boxes (e.g., see new-loc and consider how it would be written
with variables instead), so no expressive power is lost, though it does create
more potential for aliasing than variables alone would have (aliasing
\ref{aliasing}) if the boxes are not used carefully.

In return, however, developers obtain expressive types: every data structure is
considered immutable unless it contains a ref, and the presence of a ref is a
warning to both developers and programs (such as compilers) that the underlying
value may keep changing. Thus, for instance, if b is a box, a developer should
be aware that replacing all instances of (unbox b) with v, where v is bound to
(unbox b), is unwise: the former always fetches the current value in the box,
while the latter may be referring to an older content. (Conversely, if the
developer wants the value at a certain point in time, oblivious to future
mutations to the box, they should be sure to retrieve and bind it rather than
always use unbox.)
