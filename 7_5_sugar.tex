\secrel{7.5 Sugaring Over Anonymity  . . 39}

Now let’s get back to the idea of naming functions, which has evident value for
program understanding. Observe that we do have a way of naming things: by
passing them to functions, where they acquire a local name (that of the formal
parameter). Anywhere within that function’s body, we can refer to that entity
using the formal parameter name.

Therefore, we can take a collection of function definitions and name them using
other...functions. For instance, the Racket code
\lsts{src/7/5/1.rkt}{rkt}
could first be rewritten as the equivalent
\lsts{src/7/5/2.rkt}{rkt}
We can of course just inline the definition of double, but to preserve the name,
we could write this as:
\lsts{src/7/5/3.rkt}{rkt}
Indeed, this pattern—which we will pronounce as “left-left-lambda”—is a local
naming mechanism. It is so useful that in Racket, it has its own special syntax:
\lsts{src/7/5/4.rkt}{rkt}
where let can be defined by desugaring as shown above.

Here’s a more complex example:
\lsts{src/7/5/5.rkt}{rkt}
This could be rewritten as
\lsts{src/7/5/6.rkt}{rkt}
which works just as we’d expect; but if we change the order, it no longer works—
\lsts{src/7/5/7.rkt}{rkt}
—because quadruple can’t “see” double. so we see that top-level binding is
different from local binding: essentially, the top-level has an “infinite
scope”. This is the source of both its power and problems.

There is another, subtler, problem: it has to do with recursion. Consider the
simplest infinite loop:
\lsts{src/7/5/8.rkt}{rkt}
Let’s convert it to let:
\lsts{src/7/5/9.rkt}{rkt}
Seems fine, right? Rewrite in terms of lambda:
\lsts{src/7/5/10.rkt}{rkt}
Clearly, the loop-forever on the last line isn’t bound!

This is another feature we get “for free” from the top-level. To eliminate this
magical force, we need to understand recursion explicitly, which we will do soon
\ref{}.