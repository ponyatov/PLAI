\secrel{Adding Functions to the Language\\
\ru{Добавление функций в язык}}\secdown

Let’s start turning this into a real programming language.
\ru{Теперь давайте превратим нашу поделку в настоящий язык программирования.}
We could add intermediate features such as conditionals,
\ru{Мы можем добавить вспомогательные фичи типа условных конструкций,}
but to do almost anything interesting
\ru{но чтобы сделать что-то интересное}
we’re going to need functions or their moral equivalent, 
\ru{нам нужны \termdef{функции}{функция} или их эквивалент,}
so let’s get to it.
\ru{так что начнем.}

\Exercise{
Add conditionals to your language.
\ru{Добавьте условные конструкции в ваш язык.}
You can either add boolean datatypes
\ru{Вы можете добавить булевый тип данных}
or, if you want to do something quicker,
\ru{или, если вы хотите сделать это побыстрее,}
add a conditional that treats \emph{0} as \term{false}
\ru{добавьте условие что \emph{0} считается \term{ложью},}
and \emph{everything else} as \term{true}.
\ru{ и не-ноль\ --- \term{истиной}.}

What are the important test cases you should write\,?\\
\ru{Какие варианты тестов вы должны написать\,?}
}

Imagine, therefore, that we’re modeling a system like Dr\racket.
\ru{Представьте что мы моделируюем систему типа Dr\racket.}
The developer defines functions in the definitions window, 
\ru{Разработчик определяет функции в окне определений,}
and uses them in the interactions window.
\ru{и использует их к интерактивном окне.}
For now, let’s assume all definitions go in the definitions window only
\ru{Предполагается что все определения вводятя \emph{только} в окне определений}
(we’ll relax this soon \ru{позже мы ослабим это требование} \ref{}),
and all expressions in the interactions window only.
\ru{а все выражения только в интерактивном окне.}
Thus, running a program simply loads definitions.
\ru{Таким образом, запуск программы просто загружает определения.}
Because our interpreter corresponds to the interactions window prompt,
\ru{Так как наш интерпретатор соответствует строке ввода интерактивного окна,}
we’ll therefore assume it is supplied with a set of definitions.
\ru{мы также предполагаем что интерпретатор укомплектован набором этих
определений.}
\note{A \emph{set} of definitions suggests \emph{no ordering}, which means,
presumably, any definition can refer to any other. That’s what I intend here,
but when you are designing your own language, be sure to think about this.}
\note{\ru{\emph{Набор} определений предполагает \emph{отсутствие
упорядочивания}, то есть любое определение может ссылаться на любое другое. Вот
что я здесь предполагаю сделать, но когда вы разрабатываете ваш собственный
язык, подумайте об этом.}}
\clearpage

\secrel{Defining Data Representations\\\ru{Представление определения}}

To keep things simple, let’s just consider functions of one argument.
\ru{Для упрощения будем рассматривать только функции от одного
аргумента.\note{\ru{подход имеет право на жизнь\ --- в языке Ы несколько
параметров передаются в контейнере-списке в сочетании с оператором @, не говоря
о \termdef{карринге}{карринг}, когда применение функции к аргументу \#1
возвращает анонимную лямбда-\emph{функцию}, применямую к аргументу
\#2 и т.д.: $f(x,y,z) \sim
(f(x)_{\lambda_x}(y))_{\lambda_{x,y}}(z)$}}} Here are some
\racket\ examples:
\ru{Вот несколько примеров на \racket:}
\lsts{src/5/p19_1.rkt}{rkt}

\Exercise{
When a function has multiple arguments, what simple but important criterion
governs the names of those arguments\,?\\
\ru{Если функция имеет несколько аргументов, какой простой но важный
критерий определяет имена этих аргументов\,?} }

What are the parts of a \termdef{function definition}{function definition}\,?
\ru{Из каких частей состоит \termdef{определение функции}{определение
функции}\,?}
It has a name \ru{Это имя}
(above, \verb|double|, \verb|quadruple|, and \verb|const5|),
which we’ll represent as a symbol
\ru{которые мы будем представлять символом}
('double, etc.);
its \termdef{formal parameter}{formal parameter}
\ru{ее \termdef{формальный параметр}{формальный параметр}}
or \termdef{argument}{argument}
\ru{или \termdef{аргумент}{аргумент}}
has a name \ru{имеющий имя} (e.g., x),
which too we can model as a symbol
\ru{который мы тоже смоделируем как символ} ('x);
and it has a \termdef{body}{function body}. \ru{и ее \termdef{тело}{тело
функции}.}
We’ll determine the body’s representation in stages,
\ru{Мы определим представление тела пошагово,}
but let’s start to lay out a datatype for function definitions:
\ru{но для начала выделим тип данных для всех определений функций:}
\lstx{FunDefC(ore)}{src/5/p20_2.rkt}{rkt}

What is the body\,? \ru{Что такое \term{тело функции}\,?}
Clearly, it has the form of an arithmetic expression,
\ru{Очевидно, оно имеет форму арифметичского выражения,}
and sometimes it can even be represented using the existing
\ru{и иногда оно даже может быть представлено используя существующий} 
\verb|ArithC(ore)| language \ru{язык}:
for instance, the body of \ru{например, тело} \verb|const5|
can be represented as \ru{может быть представлено как } \verb|(numC 5)|.
But representing the body of \ru{Но представление тела}
\verb|double| requires something more: \ru{требует кое-чего еще:}
not just addition (which we have), \ru{не только сложение (которое у нас есть),} 
but also \ru{но также и} ''x''.
You are probably used to calling this a \termdef{variable}{variable},
\ru{Вы возможно уже назвали это \termdef{переменной}{переменная}} 
but we will \emph{not use} that term for now.
\ru{но мы пока \emph{не будем использовать} этот термин.} 
Instead, we will call it an \termdef{identifier}{identifier}.\note{I promise
we’ll return to this issue of nomenclature later \ref{}.}
\ru{Вместо этого мы назовем его
\termdef{идентификатор}{идентификатор}.\note{\ru{Я вам обещаю, что мы
вернемся к этому вопросу номенклатуры позже \ref{}.}}}

\DoNow{Anything else\,?\\\ru{Что-то еще\,?}}

Finally, let’s look at the body of
\ru{Давайте посмотрим на тело} \verb|quadruple|.
It has yet another new construct:
\ru{Оно имеет другой новый конструкт:}
a function \termdef{application}{function application}.
\ru{\termdef{применение}{применение функции} функции.}
Be very careful to distinguish between a function \term{definition},
\ru{Будьте очень осторожны, и отличайте \term{определение} функции,} 
which describes what the function is,
\ru{которое описывает что есть некоторая функция,} 
and an \term{application}, which uses it.
\ru{от \term{применения}, т.е. ее использования.}
These are uses. \ru{Это виды использования.}
The \term{argument} (or \term{actual parameter})
\ru{\term{Аргумент} (или \term{фактический параметр})}
in the inner application of \ru{во внутреннем применении}
\verb|double| is \verb|x|;
the argument in the outer application is \ru{аргумент во внешнем применении}
\verb|(double x)|.
Thus, the argument can be any complex expression.
\ru{Таким образом, аргумент может быть любым сложным выражением.}

Let’s commit all this to a crisp datatype.
\ru{Давайте применим все это к нашему сверкающему типу данных.}
Clearly we’re extending what we had before 
\ru{Расширяем то что у нас уже было раньше}
(because we still want all of arithmetic
\ru{потому что нам все еще нужна вся арифметика}).
We’ll give a new name to our datatype
\ru{Мы дадим новое имя нашему типу данных}
to signify that it’s growing up:
\ru{чтобы отметить что он расширяется:}
\lstx{ExprC(ore)}{src/5/p20_3.rkt}{rkt}

Identifiers are closely related to formal parameters.
\ru{Идентификаторы близко связаны с формальными параметрами.}
When we apply a function by giving it a value for its parameter,
\ru{когда при применяем функцию придавая значение ее параметру,} 
we are in effect asking it
\ru{фактически мы просим функцию} 
to replace all instances of that formal parameter in the body
\ru{заменить все вхождения этого формального параметра в ее теле}\ ---
i.e., the identifiers with the same name as the formal parameter
\ru{т.е., идентификаторы с тем же именем что и формальный параметр}
--- with that value. \ru{этим значением.}
To simplify this process of search-and-replace,
\ru{Для упрощения этого процесса поиска/замены,}
we might as well use the same datatype to represent both.
\ru{мы могли бы так же использовать один и тот же тип данных для представления
обоих.}
We’ve already chosen symbols to represent formal parameters, so:
\ru{Мы уже выбрали символы для представления формальных параметров, так что:}
\note{Observe that we are being coy about a few issues: what kind of ''value''
\ref{} and when to replace \ref{}.}
\note{\ru{Заметим что были скромны в нескольких вопросах: какого рода
``значения'' \ref{} и когда их заменять \ref{}}}
\lsts{src/5/p21_1.rkt}{rkt}

Finally, applications.
\ru{В заключение, применения.}
They have two parts: \ru{Они имеют две части:} 
the function’s name, \ru{имя функции,}
and its argument. \ru{и ее аргумент.}
We’ve already agreed that the argument can be any full-fledged expression
\ru{мы уже убедились, что аргумент может быть любым полноценным выражением}
(including identifiers and other applications
\ru{включая идентификаторы и другие применения}).
As for the function name,
\ru{что касается имени функции,}
it again makes sense to use the same datatype
\ru{снова имеет смысл использовать тот же тип данных,}
as we did when giving the function its name in a function definition.
\ru{как мы сделали когда давали функции имя в ее определении.}
Thus: \ru{Так что:}
\lsts{src/5/p21_2.rkt}{rkt}
identifying which function to apply, and providing its argument.
\ru{определяет какая функция применяется, и предоставляет ее аргумент.}

Using these definitions,
\ru{используя эти определения,}
it’s instructive to write out the representations of the examples
we defined above:
\ru{полезно выписать представления примеров которые мы рассматривалии выше:}
\lsts{src/5/p21_3.rkt}{rkt}
We also need to choose a representation for a set of function definitions. 
\ru{Нам также необходимо выбрать представление для набора определений функций.}
It’s convenient to represent these by a list.
\ru{Удобно определить из через списки.}

\secrel{Growing the Interpreter\\\ru{Рост интерпретатора}}

Now we’re ready to tackle the interpreter proper. First, let’s remind ourselves
of what it needs to consume. Previously, it consumed only an expression to
evaluate. Now it also needs to take a list of function definitions:
\lsts{src/5/5_2_1.rkt}{rkt}
Let’s revisit our old interpreter \ref{firstinterp}. In the case of numbers,
clearly we still return the number as the answer. In the addition and
multiplication case, we still need to recur (because the sub-expressions might
be complex), but which set of function definitions do we use? Because the act of
evaluating an expression neither adds nor removes function \emph{definitions},
the set of definitions remains the same, and should just be passed along
unchanged in the recursive calls.
\lsts{src/5/5_2_2.rkt}{rkt}
Now let’s tackle application. First we have to look up the function definition,
for which we’ll assume we have a helper function of this type available:
\lsts{src/5/5_2_3.rkt}{rkt}
Assuming we find a function of the given name, we need to evaluate its body.
However, remember what we said about identifiers and parameters\,? We must
“search-and-replace”, a process you have seen before in school algebra called
\termdef{substitution}{substitution}. This is sufficiently important that we
should talk first about substitution before returning to the interpreter
\ref{interpresumed}.

\secrel{5.3 Substitution   . 22}
\secrel{5.4 The Interpreter, Resumed   23}\label{interpresumed}
\secrel{5.5 Oh Wait, There’s More!   . 25}
\secup
