\secrel{Adding Functions to the Language\\
\ru{Добавление функций в язык}}\secdown

Let’s start turning this into a real programming language.
\ru{Теперь давайте превратим нашу поделку в настоящий язык программирования.}
We could add intermediate features such as conditionals,
\ru{Мы можем добавить вспомогательные фичи типа условных конструкций,}
but to do almost anything interesting
\ru{но чтобы сделать что-то интересное}
we’re going to need functions or their moral equivalent, 
\ru{нам нужны \termdef{функции}{функция} или их эквивалент,}
so let’s get to it.
\ru{так что начнем.}

\Exercise{
Add conditionals to your language.
\ru{Добавьте условные конструкции в ваш язык.}
You can either add boolean datatypes
\ru{Вы можете добавить булевый тип данных}
or, if you want to do something quicker,
\ru{или, если вы хотите сделать это побыстрее,}
add a conditional that treats \emph{0} as \term{false}
\ru{добавьте условие что \emph{0} считается \term{ложью},}
and \emph{everything else} as \term{true}.
\ru{ и не-ноль\ --- \term{истиной}.}

What are the important test cases you should write\,?\\
\ru{Какие варианты тестов вы должны написать\,?}
}

Imagine, therefore, that we’re modeling a system like Dr\racket.
\ru{Представьте что мы моделируюем систему типа Dr\racket.}
The developer defines functions in the definitions window, 
\ru{Разработчик определяет функции в окне определений,}
and uses them in the interactions window.
\ru{и использует их к интерактивном окне.}
For now, let’s assume all definitions go in the definitions window only
\ru{Предполагается что все определения вводятя \emph{только} в окне определений}
(we’ll relax this soon \ru{позже мы ослабим это требование} \ref{}),
and all expressions in the interactions window only.
\ru{а все выражения только в интерактивном окне.}
Thus, running a program simply loads definitions.
\ru{Таким образом, запуск программы просто загружает определения.}
Because our interpreter corresponds to the interactions window prompt,
\ru{Так как наш интерпретатор соответствует строке ввода интерактивного окна,}
we’ll therefore assume it is supplied with a set of definitions.
\ru{мы также предполагаем что интерпретатор укомплектован набором этих
определений.}
\note{A \emph{set} of definitions suggests \emph{no ordering}, which means,
presumably, any definition can refer to any other. That’s what I intend here,
but when you are designing your own language, be sure to think about this.}
\note{\ru{\emph{Набор} определений предполагает \emph{отсутствие
упорядочивания}, то есть любое определение может ссылаться на любое другое. Вот
что я здесь предполагаю сделать, но когда вы разрабатываете ваш собственный
язык, подумайте об этом.}}
\clearpage

\secrel{5.1 Defining Data Representations  19}

\secrel{Growing the Interpreter\\\ru{Рост интерпретатора}}

Now we’re ready to tackle the interpreter proper. First, let’s remind ourselves
of what it needs to consume. Previously, it consumed only an expression to
evaluate. Now it also needs to take a list of function definitions:
\lsts{src/5/5_2_1.rkt}{rkt}
Let’s revisit our old interpreter \ref{firstinterp}. In the case of numbers,
clearly we still return the number as the answer. In the addition and
multiplication case, we still need to recur (because the sub-expressions might
be complex), but which set of function definitions do we use? Because the act of
evaluating an expression neither adds nor removes function \emph{definitions},
the set of definitions remains the same, and should just be passed along
unchanged in the recursive calls.
\lsts{src/5/5_2_2.rkt}{rkt}
Now let’s tackle application. First we have to look up the function definition,
for which we’ll assume we have a helper function of this type available:
\lsts{src/5/5_2_3.rkt}{rkt}
Assuming we find a function of the given name, we need to evaluate its body.
However, remember what we said about identifiers and parameters\,? We must
“search-and-replace”, a process you have seen before in school algebra called
\termdef{substitution}{substitution}. This is sufficiently important that we
should talk first about substitution before returning to the interpreter
\ref{interpresumed}.

\input{5_3_subst}
\input{5_4_resumed}
\input{5_5_more}
\secup
