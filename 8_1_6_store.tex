\secrel{8.1.6 Introducing the Store  . 48}

The preceding discussion tells us that we need two repositories to accompany the
expression, not one. One of them, the environment, continues to be responsible
for maintaining lexical scope. But the environment cannot directly map
identifiers to their value, because the value might change. Instead, something
else needs to be responsible for maintaining the dynamic state of mutated boxes.
This latter data structure is called the store.

Like the environment, the store is a partial map. Its domain could be any
abstract set of names, but it is natural to think of these as numbers, meant to
stand for memory locations. This is because the store in the semantics maps
directly onto (abstracted) physical memory in the machine, which is
traditionally addressed by numbers. Thus the environment maps names to
locations, and the store maps locations to values:
\lsts{src/8/1/6_1.rkt}{rkt}
We’ll also equip ourselves with a function to look up values in the store, just
as we already have one for the environment (which now returns locations
instead):
\lsts{src/8/1/6_2.rkt}{rkt}

With this, we can refine our notion of values to the correct one:
\lsts{src/8/1/6_3.rkt}{rkt}

\Exercise{
Fill in the bodies of lookup and fetch.
}
