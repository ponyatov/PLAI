\secrel{6.4 Scope   . 30}

The broken environment interpreter above implements what is known as dynamic
scope. This means the environment accumulates bindings as the program executes.
As a result, whether an identifier is even bound depends on the history of
program execution. We should regard this unambiguously as a flaw of programming
language design. It adversely affects all tools that read and process programs:
compilers, IDEs, and humans.

In contrast, substitution—and environments, done correctly—give us lexical scope
or static scope. “Lexical” in this context means “as determined from the source program”, while “static” in computer science means “without running the program”, so these are appealing to the same intuition. When we examine an identifier, we want to know two things: (1) Is it bound? (2) If so, where? By “where” we mean: if there are multiple bindings for the same name, which one governs this identifier? Put differently, which one’s substitution will give a value to this identifier? In general, these questions cannot be answered statically in a dynamically-scoped language: so your IDE, for instance, 
cannot overlay arrows to show you this information (as DrRacket does).
Thus, even though the rules of scope become more complex as the space of names
becomes richer (e.g., objects, threads, etc.), we should always strive to preserve the spirit of static 
scoping.
\note{A different way to think about it is that in a dynamically-scoped
language, the answer to these questions is the same for all identifiers, and it
simply refers to the dynamic environment. In other words, it provides no useful
information.}

\secdown
\secrel{6.4.1 How Bad Is It?   . 30}

You might look at our running example and wonder whether we’re creating a
tempest in a teapot. In return, you should consider two situations:
\begin{enumerate}

\item To understand the binding structure of your program, you may need to look
at the whole program. No matter how much you’ve decomposed your program into
small, understandable fragments, it doesn’t matter if you have a free identifier
anywhere.

\item Understanding the binding structure is not only a function of the size of
the program but also of the complexity of its control flow. Imagine an
interactive program with numerous callbacks; you’d have to track through every
one of them, too, to know which binding governs an identifier.

\end{enumerate}

Need a little more of a nudge? Let’s replace the expression of our example
program with this one:
\lsts{src/6/6_4_1.rkt}{rkt}

Suppose moon-visible? is a function that presumably evaluates to false on
new-moon nights, and true at other times. Then, this program will evaluate to an
answer except on new-moon nights, when it will fail with an unbound identifier
error.

\Exercise{
What happens on cloudy nights?
}

\secrel{6.4.2 The Top-Level Scope  . 31}

Matters become more complex when we contemplate top-level definitions in many
languages. For instance, some versions of Scheme (which is a paragon of lexical scoping) allow you to write this:
\lsts{src/6/6_4_2.rkt}{rkt}
which seems to pretty clearly suggest where the y in the body of f will come
from, except:
\lsts{src/6/6_4_3.rkt}{rkt}
is legal and (f 10) produces 12. Wait, you might think, always take the last
one! But:
\lsts{src/6/6_4_4.rkt}{rkt}

Here, z is bound to the first value of y whereas the inner y is bound to the
second value. There is actually a valid explanation of this behavior in terms of
lexical scope, but it can become convoluted, and perhaps a more sensible option is to
prevent such redefinition. Racket does precisely this, thereby offering the
convenience of a top-level without its pain.
\note{Most “scripting” languages exhibit similar problems. As a result, on the
Web you will find enormous confusion about whether a certain language is
statically- or dynamically-scoped, when in fact readers are comparing behavior
inside functions (often static) against the top-level (usually dynamic).
Beware!}

\secup
