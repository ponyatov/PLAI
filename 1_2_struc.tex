\secrel{The Structure of This Book \ru{Структура книги}}

Unlike some other textbooks, this one does not follow a top-down narrative.
\ru{В отличие от большинства учебников, эта книга не следует подходу ``сверху
вниз''.}
Rather it has the flow of a conversation, with backtracking.
\ru{Скорее она имеет форму повествования с возвратами к предыдущим темам.}
We will often build up programs incrementally, just as a pair of programmers
would.
\ru{Мы часто будет строить программы инкрементно, так же как мы бы это делали в
парном программировании.}
We will include mistakes, not because I don’t know the answer, but because
\emph{this is the best way for you to learn}.
\ru{Наш код будет включать ошибки, не потому что мы не знаем правильный ответ,
но потому что \emph{это лучший способ научить вас}, углубляясь от поверхностного
в детали}.
Including mistakes makes it impossible for you to read passively:
\ru{Включение намеренных ошибок делает невозможным для вас читать материал
пассивно:}
you must instead engage with the material,
\ru{вы должны взаимодействовать с ним,}
because you can never be sure of the veracity of what you’re reading.
\ru{потому что вы никогда не можете быть уверены в правильности того что вы
читаете.}

At the end, you’ll always get to the right answer.
\ru{В конце концов вы всегда получите правильный ответ.}
However, this non-linear path is more frustrating in the short term
\ru{Тем не меннее, это нелинейное повествование немного раздражающе в
краткосрочной перспективе}
(you will often be tempted to say,
\ru{у вас всегда будет соблазн сказать}
"Just tell me the answer, already\,!
\ru{Скажите же мне наконец ответ\,!}"),
and it makes the book a poor reference guide
\ru{и это также делает эту книгу плохим справочником}
(you can’t open up to a random page and be sure what it says is correct
\ru{вы не можете открыть произвольную страницу и быть уверенным что на ней
написана правда}).
However, that feeling of frustration is the sensation of learning.
\ru{Тем не менее, это чувство разочарования\ --- ощущение обучения.}
I don’t know of a way around it.
\ru{Я не знаю другого способа.}

\bigskip
At various points you will encounter this:
\ru{В некоторых местах вы встретите следующие выделения:}

\Exercise{
This is an exercise. Do try it.\\
\ru{Это упражнение. Попробуйте это сделать.}
}

This is a traditional textbook exercise.
\ru{Это традиционное для учебников упражнение.}
It’s something you need to do on your own.
\ru{Это то, что вам нужно сделать по своему усмотрению.}
If you’re using this book as part of a course,
\ru{Если вы используете эту книгу как часть курса,}
this may very well have been assigned as homework.
\ru{это упражнение хорошо задавать как домашнюю работу.}
In contrast, you will also find exercise-like questions that look like this:
\ru{В противоположность этому вы также можете найти подобные вопросы, выделенные
как}

\DoNow{
There’s an activity here\,! Do you see it\,?\\
\ru{Здесь предполагаются немедленные действия\,! Вы видите это\,?}
}

When you get to one of these, \emph{stop}.
\ru{Когда вы доберетесь до одного из этих блоков, \emph{остановитесь}.}
Read, think, and formulate an answer before you proceed.
\ru{Прочитайте, подумайте, сформулируйте ответ перед тем как продолжить чтение.}
You must do this because this is actually an \emph{exercise},
\ru{Вы должны сделать это потому что это действительно \emph{упражнение},}
but the answer is already in the book
\ru{но ответ уже есть в книге}\ ---
most often in the text immediately following
\ru{чаще всего в тексте непосредственно после упражнения}
(i.e., in the part you’re reading right now
\ru{т.е. в части которую вы сейчас читаете})\ ---
or is something you can determine for yourself by running a program.
\ru{или это что-то, что вы можете получить самостоятельно, запустив программу.}
If you just read on,
\ru{Если вы просто продолжите читать,}
you’ll see the answer without having thought about it
\ru{то вы увидете ответ без его обдумывания}
(or not see it at all, if the instructions are to run a program
\ru{или не увидите его вообще, если это инструкции по запуску программы}),
so you will get to neither
\ru{так что вы ни}
(a) test your knowledge, nor \ru{проверите свои знания, ни}
(b) improve your intuitions. \ru{улучшите свое понимание.}
In other words, these are additional, explicit attempts to encourage active
learning.
\ru{Другими словами, это дополнительные, явные попытки стимулировать ваше
активное обучение.}
Ultimately, however, I can only encourage it;
\ru{В конце концов, я могу только поощрять вас работать;}
it’s up to you to practice it.
\ru{решение применять это или нет остается за вами.}
