\secrel{Growing the Language\\
\ru{Расширение языка}}

We’ve picked a very restricted first language, so there are many ways we can
grow it.
\ru{Мы бвырали очень ограниченный первый язык, так что есть много способов,
которыми мы можем расширить его.}
Some, such as representing data structures and functions, will clearly force us
to add new features to the interpreter itself
\ru{Некоторые, такие как представление структур данных и функций, явно заставят
нас добавить новые возможности с самому интерпретатору} (
assuming we don’t want to use G\"odel numbering
\ru{предполагая что бы не собираемся использовать гёделевскую нумерацию}
).

Others, such as adding more of arithmetic itself, can be done without disturbing
the core language and hence its interpreter.
\ru{Что-то другое, типа добавление арифметики, может быть сделано без изменения
ядра языка и соответственно его интерпретатора.}
We’ll examine this next \ru{Продолжение следует} (section \ru{секция}
\ref{sec4}).
