\secrel{Representing Arithmetic\\
\ru{Представление арифметики}}\label{sec31}

Let’s first agree on how we will represent arithmetic expressions.
\ru{Давайте сначала договориться о том, как мы будем представлять арифметические
выражения.}
Let’s say we want to support only two operations\ --- addition and
multiplication\ --- in addition to primitive numbers.
\ru{Допустим, мы хотим поддерживать только две простые операции\ --- сложение и
умножение\ --- в дополнение к примитивным числам.}
We need to represent arithmetic \termdef{expressions}{expression}.
\ru{Нам необходимо представление для арифметических
\termdef{выражений}{выражение}}.
What are the rules that govern nesting of arithmetic expressions\,?
\ru{Какие правила регулируют вложенность арифметических выражений\,?} 
We’re actually free to nest any expression inside another.
\ru{На самом мы свободно можем владывать любое выражение внутрь любого другого.}

\DoNow{
Why did we not include division\,?
\ru{Почему мы не включили деление\,?}
\\
What impact does it have on the remarks above\,?
\ru{Какое влияние это имеет на замечания выше\,?}
}

We’ve ignored division
\ru{Мы игнорировали деление,}
because it forces us into a discussion of what
expressions we might consider legal:
\ru{потому что оно вовлекает нас в дикуссию о том,
какие выражения мы можем считать правильными:}
clearly the representation of $1/2$ ought to be legal;
\ru{ясно что представление $1/2$ должно быть легальным;} 
the representation of $1/0$ is much more debatable;
\ru{представление $1/0$ спорно;}
and that of \ru{и что-то типа} $1/(1-1)$ seems even more controversial.
\ru{кажется гораздо более противоречивым.}
 
We’d like to sidestep this controversy for now and return to it later
\ru{Мы хотели бы обойти сейчас это противоречие, в вернуться к нему позже}
\ref{}.

Thus, we want a representation for numbers and arbitrarily nestable
addition and multiplication.
\ru{Таким образом нам нужно представление для чисел и произвольно вложенных
сложений и умножений.} 
Here’s one we can use \ru{Вот то что мы можем использовать}
(used in \ru{использовано в} \ref{arithc} ):
\lstx{ArithC}{2/p12_1.rkt}{rkt}
