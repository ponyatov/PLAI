\secrel{Did You Notice\,?\\
\ru{Вы заметили\,?}}

I just slipped something by you:
\ru{Я только что утаил что-то от вас:}
\DoNow{
What is the ``meaning'' of addition and multiplication in this new language\,?
\ru{Каков ``смысл'' сложения и умножения в этом новом языке\,?}
}

That’s a pretty abstract question, isn’t it.
\ru{Это достаточно абстрактный вопрос, не так ли.} 
Let’s make it concrete.
\ru{Давайте его конкретизируем.} 
There are many kinds of addition in computer science:
\ru{В информатике существует множество видов сложения:}

\begin{itemize}
  \item 
First of all, there’s many different kinds of \termdef{numbers}{number}:
\ru{Прежде всего, существует множество видов \termdef{чисел}{число}:}
fixed-width \ru{фиксированной длины} (e.g., 32- bit \ru{например 32-битные})
integers \ru{целые}, signed fixed-width \ru{знаковые фиксированной длины} (e.g.,
31-bits plus a sign-bit \ru{например 31-битные плюс бит знака}) integers
\ru{целые}, arbitrary precision integers \ru{целые числа произвольной точности};
in some languages, rationals \ru{в некоторых языках\ --- натуральные дроби};
various formats of fixed- and floating-point numbers
\ru{различные форматы чисел с фиксированной и плавающей точкой}; in some
languages, complex numbers \ru{в некоторых языках комплектные числа}; and so on
\ru{и так далее}.
After the numbers have been chosen, addition may support only some combinations
of them.
\ru{После того как были выбраны определенные виды чисел, сложение может
поддерживать только некоторые их комбинации.}
  \item 
In addition, some languages permit the addition of datatypes such as matrices.
\ru{В дополнение, некоторые языки поддерживают сложение таких типов данных, как
матрицы.}
  \item 
Furthermore, many languages support ``addition'' of strings
\ru{Кроме того, многие языки поддерживают ``сложение'' строк} (
we use scare-quotes because we don’t really mean the mathematical concept of
addition, but rather the operation performed by an operator with the syntax +
\ru{Мы используем кавычки, так как предполагаем не математическую идею
сложения, а операцию, выполняемую оператором с синтаксисом +} ). 
In some languages this always means concatenation;
\ru{В некоторых языках это всегда значит конкатенацию строк;} 
in some others, it can result in numeric results (or numbers stored in strings).
\ru{в некоторых долбанутых языках иногда может получиться численный
результат (или числа хранимые в строках).}
\end{itemize}

These are all different meanings for addition. 
\ru{Все это является смыслом сложения.}
\termdef{Semantics}{semantics}
is the mapping of \emph{syntax} (e.g., +) to \emph{meaning} (e.g., some or all
of the above).
\ru{\termdef{Семантика}{семантика} это отображение \emph{синтаксиса} (например
+) на \emph{смысл} (например что-то или все из вышеперечисленного).}

This brings us to our first game of:
\ru{Это подводит нас к первоначальной игре:}
\begin{description}\item[\emph{Which of these is the same\,?} \ru{Что из
этого дает одинаковый результат\,?}]\
\\
\begin{itemize}[nosep]
  \item 1 + 2
  \item 1 + 2
  \item '1' + '2'
  \item '1' + '2'
\end{itemize}
\end{description}

Now return to the question above.
\ru{Теперь возвращаемся к предыдущему вопросу.} 
What semantics do we have\,?
\ru{Какую семантику мы имеем\,?}
We’ve adopted whatever semantics \racket\ provides, because we map + to
\racket’s +. 
\ru{Мы приняли ту же семантику, которую предоставляет \racket, потому что мы
отоборазили наш + на + в \racket е.}
In fact that’s not even quite true:
\ru{На самом деле это даже не совсем верно:} 
\racket\ may, for all we know, also enable +
to apply to strings, so we’ve chosen the restriction of \racket’s semantics to
numbers\note{though in fact \racket’s + doesn’t tolerate strings}.
\ru{\racket\ может, как все мы знаем, также использовать + к строкам, так что
мы выбрали ограничение семантики \racket а только для чисел\note{\ru{хотя на
самом деле \racket ский + нетолерантен к строкам}}.}

If we wanted a different semantics, we’d have to implement it explicitly.
\ru{Если мы хотим другую семантику, мы должны реализовать ее в явном виде.}

\Exercise{
What all would you have to change so that the number had signed- 32-bit
arithmetic\,?
\\
\ru{Что мы должны изменить, чтобы числа поддерживали знаковую 32-битную
арифметику\,?}
}

In general, we have to be careful about too readily borrowing from the host
language.
\ru{В общем, мы должны быть осторожными с заимствованиями из языка-носителя.}
We’ll return to this topic later \ru{Мы вернемся к этой теме позже} \ref{}.
