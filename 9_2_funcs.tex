\secrel{9.2 Recursive Functions   64}

In a shift in terminology, a recursive function is not a reference to a same
kind of function but rather to the same function itself. It’s useful to first
ensure we’ve first extended our language with conditionals (even of the kind
that only check for 0, as described earlier: section 5), so we can write
non-trivial programs that terminate.

Let’s now try to write a recursive factorial:
\lsts{src/9/2/1.rkt}{rkt}
But this doesn’t work at all! The inner fact gives an unbound identifier error,
just as in our cyclic datum example.

It is no surprise that we should encounter the same error, because it has the
same cause. Our traditional binding mechanism does not automatically make
function definitions cyclic (indeed, in some early programming languages, they
were not: misguidedly, recursion was considered a special feature). Instead, if
we want recursion—i.e., for a function definition to cyclically refer to
itself—we must implement it by hand.
\note{Because you typically write top-level definitions, you don’t encounter
this issue. At the top-level, every binding is implicitly a variable or a box.
As a result, the pattern below is more-or-less automatically put in place for
you. This is why, when you want a recursive local binding, you must use letrec
or local, not let.}

The means to do so is now clear: the problem is the same one we diagnosed
before, so we can reuse the same solution. We again have to follow a three-step
process: first create a placeholder, then refer to the placeholder where we want
the cyclic reference, and finally mutate the placeholder before use. Thus:
\lsts{src/9/2/2.rkt}{rkt}
In fact, we don’t even need fact-fun: I’ve used that binding just for clarity.
Observe that because it isn’t recursive, and we have identifiers rather than
variables, its use can simply be substituted with its value:
\lsts{src/9/2/3.rkt}{rkt}
There is the small nuisance of having to repeatedly unbox fact. In a language
with variables, this would be even more seamless:
\note{ Indeed, one use for variables is that they simplify the desguaring of the
above pattern, instead of requiring every use of a cyclically-bound identifier
to be unboxed. On the other hand, with a little extra effort the desugaring
process could take care of doing the unboxing, too.}
\lsts{src/9/2/4.rkt}{rkt}
