\secrel{$\oplus$ [D]ynamic [L]anguage [R]untime/core in \cpp\\
\ru{$\oplus$ Реализация ядра динамического языка на \cpp}}\secdown

Параллельно с основным текстом оригинальной книги были добавлены разделы по
реализации динамического языка на паре mainstream языков; прежде всего это
комплект flex/bison/\cpp.

Диалект \lisp а, использованный в книге, хорош для описания логики. Но для
практического применения больше подходит реализация в виде
\termdef{встраиваемого интерпретатора}{встраиваемый интерпретатор}\ --- вы
можете взять свою готовую программу, и добавить в нее скриптовый движок,
расширяющий возможности (расширение пользователем, реализация языковых фич
недоступных в \cpp, сборка мусора,\ldots). $DrRacket$ слишком тяжел по ресурсам
для встраивания, и хотелось бы иметь образец реализации на более низкоуровневом
языке.

% \bigskip
% 
% Показаны реализации двух языков:
% \begin{enumerate}[nosep]
%   \item \bi\ питоно-подобный инфиксный язык для быстрого напиcания скриптов в
%   привычном синтаксисе, и
%   \item \hm\ экспериментальный гомоиконичный язык для
%   мета-про\-грам\-ми\-ро\-ва\-ния.
% \end{enumerate}

\secrel{$\oplus$ Lexical program skeleton \& [\make] tool usage\\
\ru{$\oplus$ Структура проекта ``лексической'' программы}}

\begin{tabular}{l l l}
src.src & & исходный код на нашем языке \\
log.log & & лог выполнения интерпретатора \\
&&(обычно используется пакетный режим) \\
ypp.ypp & \bison & синтаксический парсер \\
&&(грамматика и код построения AST) \\
lpp.lpp & \flex & лексический анализатор \\
&&(выделение языковых элементов \term{regexp}ами:\\
&& идентификаторы, числа, строки, операторы)\\
hpp.hpp & \cpp & хедеры и определения типов \\
cpp.cpp & \cpp & код ядра интерпретатора \\
Makefile & \make & скрипт сборки проекта \\
bat.bat & \gvim & запуск \gvim\ для \win \\
.gitignore & \git & маски временных и производных файлов \\
\end{tabular}

\noindent\fig{}{tmp/1_4_files.pdf}{width=\textwidth,height=\textheight}

Сборка проекта выполняется типовой утилитой \make\note{mingw32-make под
\win}\ :
\lstxl{Makefile}{tmp/mk.mk}{mk}

Используется два вида блоков:

\begin{description}
\item[присвоение переменной]
\begin{verbatim}
VAR = значение
\end{verbatim}
или модификация: \emph{добавление} значения
\verb|VAR += значение|
\item[обновление файлов] выполняется если файл-\term{источник} \emph{новее}
файла-\term{цели}\\
\begin{verbatim}
target1 [target2 ...] <colon> [ source1 source2 ... ]
[ <tab><command1> ]
[ <tab><command2> ]
... 
\end{verbatim}
\end{description}

\begin{tabular}{l l}
target & файл-цель (можеть быть несколько) \\
source & файл-источник\\&(иногда источников нет, например для цели
\verb|make all|)
\\
<colon> & двоеточие \\
<tab> & \emph{обязательный символ табуляции}\\&замена на пробелы не
допускается\\
\verb|$(...)| & переменная \\ 
\verb|C| & \cpp\ файлы \\ 
\verb|H| & хедеры \\ 
\verb|L| & дополнительные библиотеки (пустая переменная) \\
\end{tabular}

\paragraph{В командах} могут использоваться следующие переменные-подстановки:\\

\begin{tabular}{l l}
\verb|$@| & имя цели \\
\verb|$<| & имя \emph{первого} источника \\
\end{tabular}

\paragraph{стандартные переменные}\ \\

\begin{tabular}{l l}
\verb|CXX = g++| & компилятор GNU \cpp\\
& (для \win\ испольуется пакет \textbf{mingw32})\\
\verb|CXXFLAGS| & флаги компилятора \cpp\\
\hline
\verb|CC = gcc| & компилятор GNU \ci\\
\verb|CFLAGS| & флаги компилятора \ci\\
\hline
\verb|LD = ld| & линкер \\
\verb|LDFLAGS| & флаги линкера\\
\hline
\verb|AS = as| & ассемблер \\
\end{tabular}

\bigskip
При запуске утилиты \verb|[mingw32-]make| из командной строки можно
жестко \emph{переопределять значения переменных}:
\verb|mingw32-make TAIL=-n7|\\
позволяет уменьшить вывод лога до 7 последних строк\note{ограничение окна IDE на
старом маленьком мониторе}

\bigskip
Временные и производные файлы не включаем в проект:
\lstt{.gitignore}{tmp/git.ignore}

Для запуска \gvim\ под \win\ используем простой батник, открывающий в табах
исходный код, лог и файлы DLR-ядра. \file{Makefile}\ исключен, так как он
шаблонный, и при работе с кодом ядра не изменяется.
\lstt{bat.bat}{cpp/bat.bat}
 
\secup
