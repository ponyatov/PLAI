\secrel{\ru{Реализация динамического языка на \cpp}}\secdown

Параллельно с основным текстом оригинальной книги были добавлены разделы по
реализации динамического языка на паре mainstream языков; прежде всего это
комплект flex/bison/\cpp.

Диалект \lisp а, использованный в книге, хорош для описания логики. Но для
практического применения больше подходит реализация в виде
\termdef{встраиваемого интерпретатора}{встраиваемый интерпретатор}\ --- вы
можете взять свою готовую программу, и добавить в нее скриптовый движок,
расширяющий возможности (расширение пользователем, реализация языковых фич
недоступных в \cpp, сборка мусора,\ldots). $DrRacket$ слишком тяжел по ресурсам
для встраивания, и хотелось бы иметь образец реализации на более низкоуровневом
языке.

% \bigskip
% 
% Показаны реализации двух языков:
% \begin{enumerate}[nosep]
%   \item \bi\ питоно-подобный инфиксный язык для быстрого напиcания скриптов в
%   привычном синтаксисе, и
%   \item \hm\ экспериментальный гомоиконичный язык для
%   мета-про\-грам\-ми\-ро\-ва\-ния.
% \end{enumerate}

\secrel{\ru{Структура проекта ``лексической'' программы}}

\begin{tabular}{l l l}
src.src & & исходный код нашего языка \\
log.log & & лог выполнения интерпретатора \\
&&(обычно используется пакетный режим) \\
ypp.ypp & \bison & синтаксический парсер \\
&&(грамматика и код построения AST) \\
lpp.lpp & \flex & лексический анализатор \\&&\\
hpp.hpp & \cpp & хедеры и определения типов \\
cpp.cpp & \cpp & код ядра интерпретатора \\
Makefile & \make & скрипт сборки проекта \\
bat.bat & \gvim & запуск \gvim\ для \win \\
.gitignore & \git & маски игнорируемых файлов \\&&(временные и производные)\\
\end{tabular}

\noindent\fig{}{tmp/1_4_files.pdf}{width=\textwidth}

Сборка проекта выполняется типовой утилитой \make\note{mingw32-make под
\win}\ и управляется файлом
\lstxl{Makefile}{1/4/mk.mk}{mk}

\secup
