\secdown\secrel{\ru{Реализация динамического языка на \cpp}}

Параллельно с основным текстом оригинальной книги были добавлены разделы по
реализации динамического языка на паре mainstream языков; прежде всего это
комплект flex/bison/\cpp.

Диалект \lisp а, использованный в книге, хорош для описания логики. Но для
практического применения больше подходит реализация в виде
\termdef{встраиваемого интерпретатора}{встраиваемый интерпретатор}\ --- вы
можете взять свою готовую программу, и добавить в нее скриптовый движок,
расширяющий возможности (расширение пользователем, реализация языковых фич
недоступных в \cpp, сборка мусора,\ldots). $DrRacket$ слишком тяжел по ресурсам
для встраивания, и хотелось бы иметь образец реализации на более низкоуровневом
языке.

\bigskip

Показаны реализации двух языков:
\begin{enumerate}[nosep]
  \item \bi\ питоно-подобный инфиксный язык для быстрого напиcания скриптов в
  привычном синтаксисе, и
  \item \hm\ экспериментальный гомоиконичный язык для
  мета-про\-грам\-ми\-ро\-ва\-ния.
\end{enumerate}

\secup