\secdown\secrel{\ru{$\oplus$ От переводчика}}

Я наткнулся на эту книгу в поисках информации по реализации динамических языков.
Первоначально целью был только (подстрочный) перевод, чтобы разобраться в теме
самому, и заодно заполнить дыру в отечественной учебной литературе по реализации
языков программирования.
% Кто не согласен с дырой\ --- назовите хотя бы пару mainstream \ru{ходовых}
% языков программирования, созданных в России\note{Рефал и 1С.Васик}.

Потом мне захотелось добавить кое-что из своих находок, а поскольку лицензия
оригинальной книги\note{Creative Commons
\href{https://creativecommons.org/licenses/by-nc-sa/3.0/us/}{BY-NC-SA} 3.0 US}
позволяет ShareAlike \ru{модификации}, я добавил свою реализацию
DLR\note{[D]ynamic [L]anguage [R]untime} и интерпретатора\note{\ru{мне наиболее
интересно направление мета-/трансформаторного программирования, и обработка
текстовых форматов данных}}\ на flex/bison/\cpp. Сразу предупрежу что
правильность моего кода была принесена в жертву его простоте, поэтому не
удивляйтесь диким расходам и утечкам памяти, и топорному \cpp\ кун-фу.

\secup