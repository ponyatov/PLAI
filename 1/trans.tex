\secdown\secrel{\ru{$\oplus$ От переводчика}}

Я наткнулся на эту книгу в поисках информации по реализации динамических языков.
Первоначально целью был только (подстрочный) перевод, чтобы разобраться в теме
самому, и заодно заполнить дыру в отечественной учебной литературе по реализации
языков программирования.
% Кто не согласен с дырой\ --- назовите хотя бы пару mainstream \ru{ходовых}
% языков программирования, созданных в России\note{Рефал и 1С.Васик}.

Потом мне захотелось добавить кое-что из своих находок, лицензия
оригинальной книги\note{Creative Commons
\href{https://creativecommons.org/licenses/by-nc-sa/3.0/us/}{BY-NC-SA} 3.0 US}
позволяет ShareAlike \ru{модификации}, но \emph{автор ограничил меня только
переводом}. Во втором томе я добавил свою реализацию DLR\note{[D]ynamic
[L]anguage [R]untime\ --- \term{ядра} реализации динамического языка}
% , включает парсер синтаксиса, управление памятью, реализациюалгоритмов
% оптимизации, компиляции,\ldots}
и интерпретатора на ходовых языках программирования:
\begin{itemize}[nosep]
  \item 
\flex/\bison/\cpp. Сразу предупрежу что правильность моего кода была принесена в
жертву его простоте, поэтому не удивляйтесь диким расходам и утечкам памяти, и
топорному \cpp\ кун-фу.
\item \py
\item \java
\item ECMAscript (для браузерных приложений)
\end{itemize}

\bigskip
Также мне интересны направления, которым в дополнительном томе были посвящены
отдельные разделы:
\begin{itemize}[nosep]
  \item мета-/трансформаторное программирование, в частности
  \item трансляторы и компиляторы;
  \item обработка текстовых и бинарных форматов данных;
  \item \term{инженерия знаний};
  \item элементы искуственного интеллекта;
  \item САПР;
  \item планирование и логистика.
\end{itemize}

\secup