\secrel{8.1.7 Interpreting Boxes  . . 49}

Now we have something that the environment can return, updated, reflecting
mutations during the evaluation of the expression, without having to change the
environment in any way. Because a function can return only one value, let’s
define a data structure to hold the new result from the interpreter:
\lsts{src/8/1/7_1.rkt}{rkt}
Thus the interpreter’s type becomes:
\lsts{src/8/1/7_2.rkt}{rkt}

The easiest one to dispatch is numbers. Remember that we have to return the
store reflecting all mutations that happened while evaluating the given
expression. Because a number is a constant, no mutations could have happened, so
the returned store is the same as the one passed in:
\lsts{src/8/1/7_3.rkt}{rkt}

A similar argument applies to closure creation; observe that we are speaking of
the creation, not use, of closures:
\lsts{src/8/1/7_4.rkt}{rkt}

Identifiers are almost as straightforward, though if you are simplistic, you’ll
get a type error that will alert you that to obtain a value, you must now look
up both in the environment and in the store:
\lsts{src/8/1/7_5.rkt}{rkt}

Notice how lookup and fetch compose to produce the same result that lookup
alone produced before.

Now things get interesting.

Let’s take sequencing. Clearly, we need to interpret the two terms:
\lsts{src/8/1/7_6.rkt}{rkt}
Oh, but wait. The whole point was to evaluate the second term in the store
returned by the first one—otherwise there would have been no point to all these
changes. Therefore, instead we must evaluate the first term, capture the
resulting store, and use it to evaluate the second. (Evaluating the first term
also yields its value, but sequencing ignores this value and assumes the first
time was run purely for its potential mutations.) We will write this in a
stylized manner:
\lsts{src/8/1/7_7.rkt}{rkt}

This says to (interp b1 env sto); name the resulting value and store v-b1 and
s-b1, respectively; and evaluate the second term in the store from the first:
(interp b2 env s-b1). The result will be the value and store returned by the
second term, which is what we expect. The fact that the first term’s effect is
only on the store can be read from the code because, though we bind v-b1, we
never subsequently use it.
\DoNow{
Spend a moment contemplating the code above. You’ll soon need to adjust
your eyes to read this pattern fluently.
}

Now let’s move on to the binary arithmetic primitives. These are similar to
sequencing in that they have two sub-terms, but in this case we really do care
about the value from each branch. As usual, we’ll look at only plusC since multC
is virtually identical.
\lsts{src/8/1/7_8.rkt}{rkt}

Observe that we’ve unfolded the sequencing pattern out another level, so we can
hold on to both results and supply them to num+.

Here’s an important distinction. When we evaluate a term, we usually use the
same environment for all its sub-terms in accordance with the scoping rules of
the language. The environment thus flows in a recursive-descent pattern. In
contrast, the store is threaded: rather than using the same store in all
branches, we take the store from one branch and pass it on to the next, and take
the result and send it back out. This pattern is called store-passing style.

Now the penny drops. We see that store-passing style is our secret ingredient:
it enables the environment to preserve lexical scope while still giving a
binding structure that can reflect changes. Our intution told us that the
environment had to somehow participate in obtaining different results for the
same expression, and we can now see how it does: not directly, by itself
changing, but indirectly, by referring to the store, which updates. Now we only
need to see how the store itself “changes”.

Let’s begin with boxing. To store a value in a box, we have to first allocate a
new place in the store where its value will reside. The value corresponding to a
box will then remember this location, for use in box mutation.
\lsts{src/8/1/7_9.rkt}{rkt}

\DoNow{
Observe that we have relied above on new-loc, which is itself implemented in
terms of boxes! This is outright cheating. How would you modify the interpreter
so that we no longer need an mutating implementation of new-loc?
}

To eliminate this style of new-loc, the simplest option would be to add yet
another parameter to and return value from the interpreter, which represents the
largest address used so far. Every operation that allocates in the store would
return an incremented address, while all others would return it unchanged. In
other words, this is precisely another application of the store-passing pattern.
Writing the interpreter this way would make it extremely unwieldy and might
obscure the more important use of store-passing for the store itself, which is
why we have not done so. However, it is important to make sure that we can:
that’s what tells us that we are not reliant on boxes to add boxes to the
language.

Now that boxes are recording the location in memory, getting the value
corresponding to them is easy.
\lsts{src/8/1/7_10.rkt}{rkt}

It’s the same pattern we saw before, where we have to use fetch to obtain the
actual value residing at that location. Note that we are relying on Racket to
halt with an error if the underlying value isn’t actually a boxV; otherwise it
would be dangerous to not check, since this would be tantamount to dereferencing
arbitrary memory (as C programs can, sometimes with disastrous consequences).

Let’s now see how to update the value held in a box. First we have to evaluate
the box expression to obtain a box, and the value expression to obtain the new
value to store in it. The box’s value is going to be a boxV holding a location.

In principle, we want to “change”, or override, the value at that location in
the store. We can do this in two ways.
\begin{enumerate}[nosep]
  \item 
One is to traverse the store, find the old binding for that location, and
replace it with the new one, copying all the other store bindings unchanged.
  \item 
The other, lazier, option is to simply extend the store with a new binding for
that location, which works provided we always obtain the most recent binding for
a location (which is how lookup works in the environment, so fetch presumably
also does in the store).
\end{enumerate}

The code below is written to be independent of these options:
\lsts{src/8/1/7_11.rkt}{rkt}

However, because we’ve implemented override-store as cons above, we’ve actually
taken the lazier (and slightly riskier, because of its dependence on the
implementation of fetch) option.
\Exercise{
Implement the other version of store alteration, whereby we update an existing
binding and thereby avoid multiple bindings for a location in the store.
}
\Exercise{
When we look for a location to override the value stored at it, can the location
fail to be present? If so, write a program that demonstrates this. If not,
explain what invariant of the interpreter prevents this from happening.
}

Alright, we’re now done with everything other than application! Most of
application should already be familiar: evaluate the function position, evaluate
the argument position, interpret the closure body in an extension of the
closure’s environment...but how do stores interact with this?
\lsts{src/8/1/7_12.rkt}{rkt}

Let’s start by thinking about extending the closure environment. The name we’re
extending it with is obviously the name of the function’s formal parameter. But
what location do we bind it to? To avoid any confusion with already-used
locations (a confusion we will explicitly introduce later! [REF]), let’s just
allocate a new location. This location is used in the environment, and the value
of the argument resides at this location in the store:
\lsts{src/8/1/7_13.rkt}{rkt}

Because we have not said the function parameter is mutable, there is no real
need to have implemented procedure calls this way. We could instead have
followed the same strategy as before. Indeed, observe that the mutability of
this location will never be used: only setboxC changes what’s in an existing
store location (the override- store above is technically a store
initialization), and then only when they are referred to by boxVs, but no box is being allocated above. However, we have chosen to implement application this way
for uniformity, and to reduce the number of cases we’d have to handle.
\note{You could call this the useless app store.}

\Exercise{
It’s a useful exercise to try to limit the use of store locations only to boxes.
How many changes would you need to make?
}
