\secrel{6.3 Deferring Correctly   29}

Here’s another test:
\lsts{src/6/6_3_1.rkt}{rkt}
In our interpreter, this evaluates to 7. Should it?

Translated into Racket, this test corresponds to the following two definitions
and expression:
\lsts{src/6/6_3_2.rkt}{rkt}
What should this produce? (f1 3) substitutes x with 3 in the body of f1, which
then invokes (f2 4). But notably, in f2, the identifier x is not bound! Sure
enough, Racket will produce an error.

In fact, so will our substitution-based interpreter!

Why does the substitution process result in an error? It’s because, when we
replace the representation of x with the representation of 3 in the
representation of f1, we do so in f1 only.
(Obviously: x is f1’s parameter; even if another function had a parameter named
x, that’s a different x.) Thus, when we get to evaluating the body of f2, its x
hasn’t been substituted, resulting in the error.
\note{This “the representation of” is getting a little annoying, isn’t it?
Therefore, I’ll stop saying that, but do make sure you understand why I had to
say it. It’s an important bit of pedantry.}

What went wrong when we switched to environments? Watch carefully: this is
subtle. We can focus on applications, because only they affect the environment.
When we substituted the formal for the value of the actual, we did so by
extending the current environment. In terms of our example, we asked the
interpreter to substitute not only f2’s substitution in f2’s body, but also the
current ones (those for the caller, f1), and indeed all past ones as well. That
is, the environment only grows; it never shrinks.

Because we agreed that environments are only an alternate implementation
strategy for substitution—and in particular, that the language’s meaning should
not change—we have to alter the interpreter. Concretely, we should not ask it to
carry around all past deferred substitution requests, but instead make it start
afresh for every new function, just as the substitution-based interpreter does.
This is an easy change:
\lsts{src/6/6_3_3.rkt}{rkt}

Now we have truly reproduced the behavior of the substitution interpreter.
\note{In case you’re wondering how to write a test case that catches errors,
look up test/exn.}
