\secrel{10.1.9 Dynamic Dispatch  . . 74}

Finally, we should make sure our objects can handle a characteristic attribute
of object systems, which is the ability to invoke a method without the caller
having to know or decide which object will handle the invocation. Suppose we
have a binary tree data structure, where a tree consists of either empty nodes
or leaves that hold a value. In traditional functions, we are forced to
implement the equivalent some form of conditional—either a cond or a type-case
or pattern-match or other moral equivalent—that exhaustively lists and selects
between the different kinds of trees. If the definition of a tree grows to
include new kinds of trees, each of these code fragments must be modified.
Dynamic dispatch solves this problem by making that conditional branch disappear
from the user’s program and instead be handled by the method selection code
built into the language. The key feature that this provides is an extensible
conditional. This is one dimension of the extensibility that objects provide.
\note{This property—which appears to make systems more black-box extensible
because one part of the system can grow without the other part needing to be
modified to accommodate those changes—is often hailed as a key benefit of
object-orientation.
While this is indeed an advantage objects have over functions, there is a dual
advantage that functions have over objects, and indeed many object programmers
end up contorting their code—using the Visitor pattern—to make it look more like
a function-based organization. Read
\href{http://cs.brown.edu/~sk/Publications/Papers/Published/kff-synth-fp-oo/}{Synthesizing
Object-Oriented and Functional Design to Promote Re-Use} for a running example
that will lay out the problem in its full glory. Try to solve it in your
favorite language, and see the
\href{http://www.cs.utah.edu/plt/publications/icfp98-ff/paper.shtml}{\racket\
solution}.}

Let’s now defined our two kinds of tree objects:
\lsts{10/1/9/1.rkt}{rkt}

With these, we can make a concrete tree:
\lsts{10/1/9/2.rkt}{rkt}

And finally, test it:
\lsts{10/1/9/3.rkt}{rkt}

Observe that both in the test case and in the add method of node, there is a
reference to 'add without checking whether the recipient is a mt or node.
Instead, the run-time system extracts the recipient’s add method and invokes it.
This missing conditional in the user’s program is the essence of dynamic
dispatch.
