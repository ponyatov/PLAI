\secrel{10.1.8 Objects with Self-Reference  72}

Until now, our objects have simply been packages of named functions: functions
with multiple named entry-points, if you will. We’ve seen that many of the
features considered important in object systems are actually simple patterns
over functions and scope, and have indeed been used—without names assigned to
them—for decades by programmers armed with lambda.

One characteristic that actually distinguishes object systems is that each
object is automatically equipped with a reference to the same object, often
called self or this. Can we implement this easily?
\note{I prefer this slightly dry way of putting it to the anthropomorphic “knows
about itself” terminology often adopted by object advocates.
Indeed, note that we have gotten this far into object system properties without
ever needing to resort to anthropomorphism.}

\secdown

\secrel{Self-Reference Using Mutation}

Yes, we can, because we have seen just this very pattern when we implemented
recursion; we’ll just generalize it now to refer not just to the same box or
function but to the same object.
\lsts{src/10/1/8/1.rkt}{rkt}
Observe that this is precisely the recursion pattern (section 9.2), adapted
slightly. We’ve tested it having first send a method to its own second. Sure
enough, this produces the expected answer:
\lsts{src/10/1/8/2.rkt}{rkt}

\secrel{Self-Reference Without Mutation}

If you studied how to implement recursion without mutation, you’ll notice that
the same solution applies here, too. Observe:
\lsts{src/10/1/8/3.rkt}{rkt}
Each method now takes self as an argument. That means method invocation must be
modified to follow this new pattern:
\lsts{src/10/1/8/4.rkt}{rkt}
That is, when invoking a method on o, we must pass o as a parameter to the
method. Obviously, this approach is dangerous because we can potentially pass a
different object as the “self”. Exposing this to the developer is therefore
probably a bad idea; if this implementation technique is used, it should only be
done in desugaring.
\note{Nevertheless, Python exposes just this in its surface syntax. While this
tribute to the Y-combinator is touching, perhaps the resultant brittleness was
unnecessary.}

\secup