\secrel{10.1.1 Objects in the Core  . . 68}

Therefore, starting from the language with first-class functions, let’s define
this very simple notion of objects by adding it to the core language. We clearly
have to extend our notion of values:
\lsts{src/10/1/1/1.rkt}{rkt}
We’ll extend the expression grammar to support literal object construction
expressions:
\note{Observe that this is already a design decision. In some languages, like
JavaScript, a developer can write literal objects: a notion so popular that a
subset of the syntax for it in JavaScript has become a Web standard, JSON. In
other languages, like Java, objects can only be created by invoking a
constructor on a class. We can simulate both by assuming that to model the
latter kind of language, we must write object literals only in special positions
following a stylized convention, as we do when desugaring below.}
\lsts{src/10/1/1/2.rkt}{rkt}
Evaluating such an object expression is easy: we just evaluate each of its
expression positions:
\lsts{src/10/1/1/3.rkt}{rkt}
Unfortunately, we can’t actually use an object, because we have no way of
obtaining its content. For that reason, we could add an operation to extract
members:
\lsts{src/10/1/1/4.rkt}{rkt}
whose behavior is intuitive:
\lsts{src/10/1/1/5.rkt}{rkt}

\Exercise{
Implement
\lsts{src/10/1/1/6.rkt}{rkt}
where the second argument is expected to be a objV.
}

In principle, msgC can be used to obtain any kind of member but for simplicity,
we need only assume that we have functions. To use them, we must apply them to
values. This is cumbersome to write in the concrete syntax, so let’s assume
desugaring has taken care of it for us: the concrete syntax for message
invocation includes both the name of the message to fetch and its argument
expression,
\lsts{src/10/1/1/7.rkt}{rkt}
and this desguars into msgC composed with application:
\lsts{src/10/1/1/8.rkt}{rkt}

With this we have a full first language with objects. For instance, here is an
object definition and invocation:
\lsts{src/10/1/1/9.rkt}{rkt}
and this evaluates to \verb|(numV 4)|.
