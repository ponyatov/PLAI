\secrel{10.3.5 Mixins and Traits   79}

Let’s return to our “blobs”.

When we write a class in Java, what are we really defining between the opening
and closing braces? It is not the entire class: that depends on the parent that
it extends, and so on recursively. Rather, what we define inside the braces is a
class extension. It only becomes a full-blown class because we also identify the
parent class in the same place.

Naturally, we should ask: Why? Why not separate the act of defining an extension
from applying the extension to a base class? That is, suppose instead of
\lsts{10/3/5/1.rkt}{rkt}
we instead write:
\lsts{10/3/5/2.rkt}{rkt}
and separately
\lsts{10/3/5/3.rkt}{rkt}
where B is some already-defined class.

Thusfar, it looks like we’ve just gone to great lengths to obtain what we had
before. However, the function-application-like syntax is meant to be suggestive:
we can “apply” this extension to several different base classes. Thus:
\lsts{10/3/5/4.rkt}{rkt}
and so on. What we have done by separating the definition of E from that of the
class it extends is to liberate class extensions from the tyranny of the fixed
base class. We have a name for these extensions: they’re called mixins.
\note{The term “mixin” originated in Common Lisp, where it was a particular
pattern of using multiple inheritance.
Lipstick on a pig.}

Mixins make class definition more compositional. They provide many of the
benefits of multiple-inheritance (reusing multiple fragments of functionality)
but within the aegis of a single-inheritance language (i.e., no complicated
rules about lookup order). Observe that when desugaring, it’s actually quite
easy to add mixins to the language. A mixin is primarily a “function over
classes’;. Because we have already determined how to desugar classes, and our
target language for desugaring also has functions, and classes desugar to
expressions that can be nested inside functions, it becomes almost trivial to
implement a simple model of mixins.
\note{This is a case where the greater generality of the target language of
desugaring can lead us to a better construct, if we reflect it back into the
source language.}

In a typed language, a good design for mixins can actually improve
object-oriented programming practice. Suppose we’re defining a mixin-based
version of Java. If a mixin is effectively a class-to-class function, what is
the “type” of this “function”? Clearly, mixin ought to use interfaces to
describe what it expects and provides. Java already enables (but does not
require) the latter, but it does not enable the former: a class (extension)
extends another class—with all its members visible to the extension— not its
interface. That means it obtains all of the parent’s behavior, not a
specification thereof. In turn, if the parent changes, the class might break.

In a mixin language, we can instead write
\lsts{10/3/5/5.rkt}{rkt}
where I is an interface. Then M can only be applied to a class that satisfies
the interface I, and in turn the language can ensure that only members specified
in I are visible in M. This follows one of the important principles of good
software design.
\note{“Program to an interface, not an implementation.” —Design Patterns}

A good design for mixins can go even further. A class can only be used once in
an inheritance chain, by definition (if a class eventually referred back to
itself, there would be a cycle in the inheritance chain, causing potential
infinite loops). In contrast, when we compose functions, we have no qualms about
using the same function twice (e.g.: \verb|(map ... (filter ... (map ...))))|.
Is there value to using a mixin twice?
\note{There certainly is! See sections 3 and 4 of
\href{http://www.cs.brown.edu/~sk/Publications/Papers/Published/fkf-classes-mixins/}{Classes
and Mixins}.}

Mixins solve an important problem that arises in the design of libraries.
Suppose we have a dozen different features which can be combined in different
ways. How many classes should we provide? Furthermore, not all of these can be
combined with each other. It is obviously impractical to generate the entire
combinatorial explosion of classes. It would be better if the devleoper could
pick and choose the features they care about, with some mechanism to prevent
unreasonable combinations. This is precisely the problem that mixins solve: they
provide the class extensions, which the developers can combine, in an
interface-preserving way, to create just the classes they need.
\note{Mixins are used extensively in the Racket GUI library.
For instance, color:text-mixin consumes basic text editor interfaces and
implements the colored text editor interface. The latter is iself a basic text
editor interface, so additional basic text mixins can be applied to the result.}

\Exercise{
How does your favorite object-oriented library solve this problem?
}

Mixins do have one limitation: they enforce a linearity of composition. This
strictness is sometimes misplaced, because it puts a burden on programmers that
may not be necessary. A generalization of mixins called traits says that instead
of extending a single mixin, we can extend a set of them. Of course, the moment
we extend more than one, we must again contend with potential name-clashes. Thus
traits must be equipped with mechanisms for resolving name clashes, often in the
form of some name-combination algebra. Traits thus offer a nice complement to
mixins, enabling programmers to choose the mechanism that best fits their needs.
As a result, Racket provides both mixins and traits.
