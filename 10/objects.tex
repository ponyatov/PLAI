\secrel{10 Objects 67}\secdown

When a language admits functions as values, it provides developers the most
natural way to represent a unit of computation. Suppose a developer wants to
parameterize some function f. Any language lets f be parameterized by passive
data, such as numbers and strings. But it is often attractive to parameterize it
over active data: a datum that can compute an answer, perhaps in response to
some information. Furthermore, the function passed to f can—assuming
lexically-scoped functions—refer to data from the caller without those data
having to be revealed to f, thus providing a foundation for security and
privacy. Thus, lexically-scoped functions are central to the design of many
secure programming techniques.

While a function is a splendid thing, it suffers from excessive terseness.
Sometimes we might want multiple functions to all close over to the same shared
data; the sharing especially matters if some of the functions mutate it and
expect the others to see the result of those mutations. In such cases, it
becomes unwieldly to send just a single function as a parameter; it is more
useful to send a group of functions. The recipient then needs a way to choose
between the different functions in the group. This grouping of functions, and
the means to select one from the group, is the essence of an object.
We are therefore perfectly placed to study objects having covered functions
(section \ref{sec7}) and mutation (section \ref{sec8})—and, it will emerge,
recursion (section \ref{sec9}).
\note{
I cannot hope to do justice to the enormous space of object systems.
Please read \href{http://users.dcc.uchile.cl/~etanter/ooplai/}{Object-Oriented
Programming Languages: Application and Interpretation} by Éric Tanter, which
goes into more detail and covers topics ignored here.
}

Let’s add this notion of objects to our language. Then we’ll flesh it out and
grow it, and explore the many dimensions in the design space of objects. We’ll
first show how to add objects to the core language, but because we’ll want to
prototype many different ideas quickly, we’ll soon shift to a desguaring-based
strategy. Which one you use depends on whether you think understanding them is
critical to understanding the essence of your language. One way to measure this
is how complex your desguaring strategy becomes, and whether by adding some key
core language enhancements, you can greatly reduce the complexity of desugaring.

\secrel{10.1 Objects Without Inheritance  . 67}

The simplest notion of an object—pretty much the only thing everyone who talks about
objects agrees about—is that an object is
\begin{itemize}[nosep]
  \item a value, that
  \item maps names to
  \item stuff: either other values or “methods”.
\end{itemize}
From a minimalist perspective, methods seem to be just functions, and since we
already have those in the language, we can put aside this distinction.
\note{We’re about to find out that “methods” are awfully close to functions but
differ in important ways in how they’re called and/or what’s bound in them.}

\secdown
% \secrel{10.1.1 Objects in the Core  . . 68}

Therefore, starting from the language with first-class functions, let’s define
this very simple notion of objects by adding it to the core language. We clearly
have to extend our notion of values:
\lsts{10/1/1/1.rkt}{rkt}
We’ll extend the expression grammar to support literal object construction
expressions:
\note{Observe that this is already a design decision. In some languages, like
JavaScript, a developer can write literal objects: a notion so popular that a
subset of the syntax for it in JavaScript has become a Web standard, JSON. In
other languages, like Java, objects can only be created by invoking a
constructor on a class. We can simulate both by assuming that to model the
latter kind of language, we must write object literals only in special positions
following a stylized convention, as we do when desugaring below.}
\lsts{10/1/1/2.rkt}{rkt}
Evaluating such an object expression is easy: we just evaluate each of its
expression positions:
\lsts{10/1/1/3.rkt}{rkt}
Unfortunately, we can’t actually use an object, because we have no way of
obtaining its content. For that reason, we could add an operation to extract
members:
\lsts{10/1/1/4.rkt}{rkt}
whose behavior is intuitive:
\lsts{10/1/1/5.rkt}{rkt}

\Exercise{
Implement
\lsts{10/1/1/6.rkt}{rkt}
where the second argument is expected to be a objV.
}

In principle, msgC can be used to obtain any kind of member but for simplicity,
we need only assume that we have functions. To use them, we must apply them to
values. This is cumbersome to write in the concrete syntax, so let’s assume
desugaring has taken care of it for us: the concrete syntax for message
invocation includes both the name of the message to fetch and its argument
expression,
\lsts{10/1/1/7.rkt}{rkt}
and this desguars into msgC composed with application:
\lsts{10/1/1/8.rkt}{rkt}

With this we have a full first language with objects. For instance, here is an
object definition and invocation:
\lsts{10/1/1/9.rkt}{rkt}
and this evaluates to \verb|(numV 4)|.

% \secrel{10.1.2 Objects by Desugaring  69}

While defining objects in the core language may be worthwhile, it’s an unwieldy
way to go about studying them. Instead, we’ll use Racket to represent objects,
sticking to the parts of the language we already know how to implement in our
interpreter. That is, we’ll assume that we are looking at the output of
desugaring. (For this reason, we’ll also stick to stylized code, potentially
writing unnecessary expressions on the grounds that this is what a simple
program generator would produce.)

\begin{framed}
Alert: All the code that follows will be in \verb|#lang plai|, not in the typed
language.
\end{framed}

\Exercise{
Why \#lang plai? What problems do you encounter when you try to type the
following code? Are some of them amenable to easy fixes, such as introducing a
new datatype and applying it consistently? How about if we make simplifications
for the purposes of modeling, such as assuming methods have only one argument?
Or are some of them less tractable?
}

% \secrel{10.1.3 Objects as Named Collections  . . 69}

Let’s begin by reproducing the object language we had above. An object is just a
value that dispatches on a given name. For simplicity, we’ll use lambda to
represent the object and case to implement the dispatching.
\note{Observe that basic objects are a generalization of lambda to have multiple
“entry-points”. Conversely, a lambda is an object with just one entry-point, so
it doesn’t need a “method name” to disambiguate.}
\lsts{10/1/3/1.rkt}{rkt}

This is the same object we defined earlier, and we use its method in the same
way:
\lsts{10/1/3/2.rkt}{rkt}

Of course, writing method invocations with these nested function calls is
unwieldy (and is about to become even more so), so we’d be best off equipping
ourselves with a convenient syntax for invoking methods—the same one we saw
earlier (msgS), but here we can simply define it as a function:
\note{We’ve taken advantage of Racket’s variable-arity syntax: . a says “bind
all the remaining—zero or more—arguments to a list named a”. apply “splices” in
such lists of arguments to call functions.}
\lsts{10/1/3/3.rkt}{rkt}
This enables us to rewrite our test:
\lsts{10/1/3/4.rkt}{rkt}

\DoNow{
Something very important changed when we switched to the desguaring
strategy. Do you see what it is?
}

Recall the syntax definition we had earlier:
\lsts{10/1/3/5.rkt}{rkt}
The “name” position of a message was very explicitly a symbol. That is, the
developer had to write the literal name of the symbol there. In our desugared
version, the name position is just an expression that must evaluate to a symbol;
for instance, one could have written
\lsts{10/1/3/6.rkt}{rkt}
This is a general problem with desugaring: the target language may allow
expressions that have no counterpart in the source, and hence cannot be mapped
back to it. Fortunately we don’t often need to perform this inverse mapping,
though it does arise in some debugging and program comprehension tools. More
subtly, however, we must ensure that the target language does not produce values
that have no corresponding equivalent in the source.

Now that we have basic objects, let’s start adding the kinds of features we’ve
come to expect from most object systems.
% \secrel{10.1.4 Constructors   . . 70}

A constructor is simply a function that is invoked at object construction time.
We currently lack such a function. by turning an object from a literal into a
function that takes constructor parameters, we achieve this effect:
\lsts{10/1/4/1.rkt}{rkt}
In the first example, we pass 5 as the constructor’s argument, so adding 3
yields 8. The second is similar, and shows that the two invocations of the
constructors don’t interfere with one another.

% \secrel{10.1.5 State   71}

Many people believe that objects primarily exist to encapsulate state. We
certainly haven’t lost that ability. If we desugar to a language with variables
(we could equivalently use boxes, in return for a slight desugaring overhead),
we can easily have multiple methods mutate common state, such as a constructor
argument:
\note{Alan Kay, who won a Turing Award for inventing Smalltalk and modern object
technology, disagrees. In
\href{http://www.smalltalk.org/smalltalk/TheEarlyHistoryOfSmalltalk_Abstract.html}{The
Early History of Smalltalk}, he says, “[t]he small scale [motivation for OOP]
was to find a more flexible version of assignment, and then to try to eliminate
it altogether”. He adds, “It is unfortunate that much of what is called
‘object-oriented programming’ today is simply old style programming with fancier
constructs. Many programs are loaded with ‘assignment-style’ operations now done
by more expensive attached procedures.”}
\lsts{10/1/5/1.rkt}{rkt}
For instance, we can test a sequence of operations:
\lsts{10/1/5/2.rkt}{rkt}
and also notice that mutating one object doesn’t affect another:
\lsts{10/1/5/3.rkt}{rkt}

% \secrel{10.1.6 Private Members   71}

Another common object language feature is private members: ones that are visible
only inside the object, not outside it. These may seem like an additional
feature we need to implement, but we already have the necessary mechanism in the
form of locallyscoped, lexically-bound variables:
\note{Except that, in Java, instances of other classes of the same type are
privy to “private” members. Otherwise, you would simply never be able to
implement an Abstract Data Type.}
\lsts{10/1/6/1.rkt}{rkt}
The desugaring above provides no means for accessing count, and lexical scoping
ensures that it remains hidden to the world.

% \secrel{10.1.7 Static Members   . 72}

Another feature often valuable to users of objects is static members: those that
are common to all instances of the “same” type of object. This, however, is
merely a lexically-scoped identifier (making it private) that lives outside the
constructor (making it common to all uses of the constructor):
\note{We use quotes because there are many notions of sameness for objects. And
then some.}
\lsts{10/1/7/1.rkt}{rkt}
We’ve written the counter increment where the “constructor” for this object
would go, though it could just as well be manipulated inside the methods.

To test it, we should make multiple objects and ensure they each affect the
global count:
\lsts{10/1/7/2.rkt}{rkt}

% \secrel{10.1.8 Objects with Self-Reference  72}

Until now, our objects have simply been packages of named functions: functions
with multiple named entry-points, if you will. We’ve seen that many of the
features considered important in object systems are actually simple patterns
over functions and scope, and have indeed been used—without names assigned to
them—for decades by programmers armed with lambda.

One characteristic that actually distinguishes object systems is that each
object is automatically equipped with a reference to the same object, often
called self or this. Can we implement this easily?
\note{I prefer this slightly dry way of putting it to the anthropomorphic “knows
about itself” terminology often adopted by object advocates.
Indeed, note that we have gotten this far into object system properties without
ever needing to resort to anthropomorphism.}

\secdown

\secrel{Self-Reference Using Mutation}

Yes, we can, because we have seen just this very pattern when we implemented
recursion; we’ll just generalize it now to refer not just to the same box or
function but to the same object.
\lsts{10/1/8/1.rkt}{rkt}
Observe that this is precisely the recursion pattern (section 9.2), adapted
slightly. We’ve tested it having first send a method to its own second. Sure
enough, this produces the expected answer:
\lsts{10/1/8/2.rkt}{rkt}

\secrel{Self-Reference Without Mutation}

If you studied how to implement recursion without mutation, you’ll notice that
the same solution applies here, too. Observe:
\lsts{10/1/8/3.rkt}{rkt}
Each method now takes self as an argument. That means method invocation must be
modified to follow this new pattern:
\lsts{10/1/8/4.rkt}{rkt}
That is, when invoking a method on o, we must pass o as a parameter to the
method. Obviously, this approach is dangerous because we can potentially pass a
different object as the “self”. Exposing this to the developer is therefore
probably a bad idea; if this implementation technique is used, it should only be
done in desugaring.
\note{Nevertheless, Python exposes just this in its surface syntax. While this
tribute to the Y-combinator is touching, perhaps the resultant brittleness was
unnecessary.}

\secup
% \secrel{10.1.9 Dynamic Dispatch  . . 74}

Finally, we should make sure our objects can handle a characteristic attribute
of object systems, which is the ability to invoke a method without the caller
having to know or decide which object will handle the invocation. Suppose we
have a binary tree data structure, where a tree consists of either empty nodes
or leaves that hold a value. In traditional functions, we are forced to
implement the equivalent some form of conditional—either a cond or a type-case
or pattern-match or other moral equivalent—that exhaustively lists and selects
between the different kinds of trees. If the definition of a tree grows to
include new kinds of trees, each of these code fragments must be modified.
Dynamic dispatch solves this problem by making that conditional branch disappear
from the user’s program and instead be handled by the method selection code
built into the language. The key feature that this provides is an extensible
conditional. This is one dimension of the extensibility that objects provide.
\note{This property—which appears to make systems more black-box extensible
because one part of the system can grow without the other part needing to be
modified to accommodate those changes—is often hailed as a key benefit of
object-orientation.
While this is indeed an advantage objects have over functions, there is a dual
advantage that functions have over objects, and indeed many object programmers
end up contorting their code—using the Visitor pattern—to make it look more like
a function-based organization. Read
\href{http://cs.brown.edu/~sk/Publications/Papers/Published/kff-synth-fp-oo/}{Synthesizing
Object-Oriented and Functional Design to Promote Re-Use} for a running example
that will lay out the problem in its full glory. Try to solve it in your
favorite language, and see the
\href{http://www.cs.utah.edu/plt/publications/icfp98-ff/paper.shtml}{\racket\
solution}.}

Let’s now defined our two kinds of tree objects:
\lsts{10/1/9/1.rkt}{rkt}

With these, we can make a concrete tree:
\lsts{10/1/9/2.rkt}{rkt}

And finally, test it:
\lsts{10/1/9/3.rkt}{rkt}

Observe that both in the test case and in the add method of node, there is a
reference to 'add without checking whether the recipient is a mt or node.
Instead, the run-time system extracts the recipient’s add method and invokes it.
This missing conditional in the user’s program is the essence of dynamic
dispatch.

\secup

\secrel{10.2 Member Access Design Space  75}

We already have two orthogonal dimensions when it comes to the treatment of member
names. One dimension is whether the name is provided statically or computed, and the
other is whether the set of names is fixed or variable:
\begin{tabular}{|p{0.3\linewidth}|p{0.3\linewidth}|p{0.3\linewidth}|}
\hline
& Name is Static & Name is Computed \\
\hline
Fixed Set & As in base Java. & As in Java \\
of Members && with reflection \\ &&to compute the name.\\
\hline
Variable Set& Difficult to envision & Most
scripting \\
of Members & (what use & languages. \\ & would it be?). &\\ 
\hline
\end{tabular}
Only one case does not quite make sense: if we force the developer to specify
the member name in the source file explicitly, then no new members would be
accessible (and some accesses to previously-existing, but deleted, members would
fail). All other points in this design space have, however, been explored by
languages.

The lower-right quadrant corresponds closely with languages that use hash-tables
to represent objects. Then the name is simply the index into the hash-table.
Some languages carry this to an extreme and use the same representation even for
numeric indices, thereby (for instance) conflating objects with dictionaries and
even arrays. Even when the object only handles “member names”, this style of
object creates significant difficulty for type-checking \ref{}\ and is hence not
automatically desirable.

Therefore, in the rest of this section, we will stick with “traditional” objects
that have a fixed set of names and even static member name references (the
top-left quadrant). Even then, we will find there is much, much more to study.

\secrel{10.3 What (Goes In) Else?   . . 75}

Until now, our case statements have not had an else clause. One reason to do so
would be if we had a variable set of members in an object, though that is
probably better handled through a different representation than a conditional: a
hash-table, for instance, as we’ve discussed above. In contrast, if an object’s
set of members is fixed, desugaring to a conditional works well for the purpose
of illustration (because it emphasizes the fixed nature of the set of member
names, which a hash table leaves open to interpretation—and also error). There
is, however, another reason for an else clause, which is to “chain” control to
another, parent, object. This is called inheritance.

Let’s return to our model of desugared objects above. To implement inheritance,
the object must be given “something” to which it can delegate method invocations
that it does not recognize. A great deal will depend on what that “something”
is.

One answer could be that it is simply another object.
\lsts{10/3/1.rkt}{rkt}

Due to our representation of objects, this application effectively searches for
the method in the parent object (and, presumably, recursively in its parents).
If a method matching the name is found, it returns through this chain to the
original call in msg that sought the method. If none is found, the final object
presumably signals a “message not found” error.

\Exercise{
Observe that the application (parent-object m) is like “half a msg”, just like
an l-value was “half a value lookup” \ref{}. Is there any connection?
}

Let’s try this by extending our trees to implement another method, size. We’ll
write an “extension” (you may be tempted to say “sub-class”, but hold off for
now!) for each node and mt to implement the size method. We intend these to
extend the existing definitions of node and mt, so we’ll use the extension
pattern described above.
\note{We’re not editing the existing definitions because that is supposed to be
the whole point of object inheritance: to reuse code in a black-box fashion.
This also means different parties, who do not know one another, can each extend
the same base code. If they had to edit the base, first they have to find out
about each other, and in addition, one might dislike the edits of the other.
Inheritance is meant to sidestep these issues entirely.}

\secdown
\secrel{10.3.1 Classes   . . 76}

Immediately we see a difficulty. Is this the constructor pattern?
\lsts{10/3/1/1.rkt}{rkt}
That suggests that the parent is at the “same level” as the object’s constructor
fields. That seems reasonable, in that once all these parameters are given, the
object is “fully defined”. However, we also still have
\lsts{10/3/1/2.rkt}{rkt}
Are we going to write all the parameters twice? (Whenever we write something
twice, we should worry that we may not do so consistently, thereby inducing
subtle errors.) Here’s an alternative: node/size can construct the instance of
node that is its parent. That is, node/size’s parent parameter is not the parent
object but rather the parent’s object maker.
\lsts{10/3/1/3.rkt}{rkt}
\lsts{10/3/1/4.rkt}{rkt}
Then the object constructor must remember to pass the parent-object maker on
every invocation:
\lsts{10/3/1/5.rkt}{rkt}
Obviously, this is something we might simplify with appropriate syntactic sugar.
We can confirm that both the old and new tests still work:
\lsts{10/3/1/6.rkt}{rkt}

\Exercise{
Rewrite this block of code using self-application instead of mutation.
}

What we have done is capture the essence of a class. Each function parameterized
over a parent is...well, it’s a bit tricky, really. Let’s call it a blob for
now. A blob corresponds to what a Java programmer defines when they write a
class:
\lsts{10/3/1/7.rkt}{rkt}

\DoNow{
So why are we going out of the way to not call it a “class”?
}

When a developer invokes a Java class’s constructor, it in effect constructs
objects all the way up the inheritance chain (in practice, a compiler might
optimize this to require only one constructor invocation and one object
allocation). These are private copies of the objects corresponding to the parent
classes (private, that is, up to the presence of static members). There is,
however, a question of how much of these objects is visible. Java chooses
that—unlike in our implementation above—only one method of a given name (and
signature) remains, no matter how many there might have been on the inheritance
chain, whereas every field remains in the result, and can be accessed by
casting. The latter makes some sense because each field presumably has
invariants governing it, so keeping them separate (and hence all present) is
wise. In contrast, it is easy to imagine an implementation that also makes all
the methods available, not only the ones lowest (i.e., most refined) in the
inheritance hierarchy. Many scripting languages take the latter approach.

\Exercise{
The code above is fundamentally broken. The self reference is to the same
syntactic object, whereas it needs to refer to the most-refined object:
this is known as open recursion. Modify the object representations so that self
always refers to the most refined version of the object. Hint: You will find the
self-application method (section 10.1.8.2) of recursion handy
\note{This demonstrates the other form of extensibility we get from traditional
objects: extensible recursion.}
}
\secrel{10.3.2 Prototypes   78}

In our description above, we’ve supplied each class with a description of its
parent class. Object construction then makes instances of each as it goes up the
inheritance chain. There is another way to think of the parent: not as a class
to be instantiated but, instead, directly as an object itself. Then all children
with the same parent would observe the very same object, which means changes to
it from one child object would be visible to another child. The shared parent
object is known as a prototype.
\note{The archetypal prototype-based language is
\href{http://www.selflanguage.org/}{Self}. Though you may have read that
languages like JavaScript are “based on” Self, there is value to studying the
idea from its source, especially because Self presents these ideas in their
purest form.}

Some language designers have argued that prototypes are more primitive than
classes in that, with other basic mechanisms such as functions, one can recover
classes from prototypes—but not the other way around. That is essentially what
we have done above: each “class” function contains inside it an object
description, so a class is an object-returning-function. Had we exposed these
are two different operations and chosen to inherit directly an object, we would
have something akin to prototypes.

\Exercise{
Modify the inheritance pattern above to implement a Self-like, prototypebased
language, instead of a class-based language. Because classes provide each object
with distinct copies of their parent objects, a prototypelanguage might provide
a clone operation to simplify creation of the operation that simulates classes
atop prototypes.
}
\secrel{10.3.3 Multiple Inheritance  . 78}

Now you might ask, why is there only one fall-through option? It’s easy to
generalize this to there being many, which leads naturally to multiple
inheritance. In effect, we have multiple objects to which we can chain the
lookup, which of course raises the question of what order in which we should do
so. It would be bad enough if the ascendants were arranged in a tree, because
even a tree does not have a canonical order of traversal: take just
breadth-first and depth-first traversal, for instance (each of which has
compelling uses). Worse, suppose a blob A extends B and C; but now suppose B and
C each extend D. Now we have to confront this question: will there be one or two
\note{This infamous situation is called diamond inheritance. If you choose to
include multiple inheritance in your language you can lose yourself for days in
design decisions on this. Because it is highly unlikely you will find a
canonical answer, your pain}
D objects in the instance of A? Having only one saves space and might interact
better with our expectations, but then, will we visit this object once or twice?
Visiting it twice should not make any difference, so it seems unnecessary. But
visiting it once means the behavior of one of B or C might change. And so on. As
a result, virtually every multiple-inheritance language is accompanied by a
subtle algorithm merely to define the lookup order.

Multiple inheritance is only attractive until you’ve thought it through.

\secrel{10.3.4 Super-Duper!   . . 79}

Many languages have a notion of super-invocations, i.e., the ability to invoke a
method or access a field higher up in the inheritance chain. This includes doing
so at the point of object construction, where there is often a requirement that
all constructors be invoked, to make sure the object is properly defined.
\note{Note that I say “the” and “chain”. When we switch to multiple inheritance,
these concepts are replaced with something much more complex.}

We have become so accustomed to thinking of these calls as going “up” the chain
that we may have forgotten to ask whether this is the most natural direction.
Keep in mind that constructors and methods are expected to enforce invariants.
Whom should we trust more: the super-class or the sub-class? One argument would
say that the sub-class is most refined, so it has the most global view of the
object. Conversely, each super-class has a vested interest in protecting its
invariants against violation by ignorant sub-classes.

These are two fundamentally opposed views of what inheritance means. Going up
the chain means we view the extension as replacing the parent. Going down the
chain means we view the extension as refining the parent. Because we normally
associate sub-classing with refinement, why do our languages choose the “wrong”
order of calling? Some languages have, therefore, explored invocation in the
downward direction by default.
\note{
\href{http://www.daimi.au.dk/~eernst/gbeta/}{gbeta} is a modern programming
language that supports inner, as well as many other interesting features.
It is also interesting to consider
\href{http://www.cs.utah.edu/plt/publications/oopsla04-gff.pdf}{combining both
directions}.
}
\secrel{10.3.5 Mixins and Traits   79}

Let’s return to our “blobs”.

When we write a class in Java, what are we really defining between the opening
and closing braces? It is not the entire class: that depends on the parent that
it extends, and so on recursively. Rather, what we define inside the braces is a
class extension. It only becomes a full-blown class because we also identify the
parent class in the same place.

Naturally, we should ask: Why? Why not separate the act of defining an extension
from applying the extension to a base class? That is, suppose instead of
\lsts{10/3/5/1.rkt}{rkt}
we instead write:
\lsts{10/3/5/2.rkt}{rkt}
and separately
\lsts{10/3/5/3.rkt}{rkt}
where B is some already-defined class.

Thusfar, it looks like we’ve just gone to great lengths to obtain what we had
before. However, the function-application-like syntax is meant to be suggestive:
we can “apply” this extension to several different base classes. Thus:
\lsts{10/3/5/4.rkt}{rkt}
and so on. What we have done by separating the definition of E from that of the
class it extends is to liberate class extensions from the tyranny of the fixed
base class. We have a name for these extensions: they’re called mixins.
\note{The term “mixin” originated in Common Lisp, where it was a particular
pattern of using multiple inheritance.
Lipstick on a pig.}

Mixins make class definition more compositional. They provide many of the
benefits of multiple-inheritance (reusing multiple fragments of functionality)
but within the aegis of a single-inheritance language (i.e., no complicated
rules about lookup order). Observe that when desugaring, it’s actually quite
easy to add mixins to the language. A mixin is primarily a “function over
classes’;. Because we have already determined how to desugar classes, and our
target language for desugaring also has functions, and classes desugar to
expressions that can be nested inside functions, it becomes almost trivial to
implement a simple model of mixins.
\note{This is a case where the greater generality of the target language of
desugaring can lead us to a better construct, if we reflect it back into the
source language.}

In a typed language, a good design for mixins can actually improve
object-oriented programming practice. Suppose we’re defining a mixin-based
version of Java. If a mixin is effectively a class-to-class function, what is
the “type” of this “function”? Clearly, mixin ought to use interfaces to
describe what it expects and provides. Java already enables (but does not
require) the latter, but it does not enable the former: a class (extension)
extends another class—with all its members visible to the extension— not its
interface. That means it obtains all of the parent’s behavior, not a
specification thereof. In turn, if the parent changes, the class might break.

In a mixin language, we can instead write
\lsts{10/3/5/5.rkt}{rkt}
where I is an interface. Then M can only be applied to a class that satisfies
the interface I, and in turn the language can ensure that only members specified
in I are visible in M. This follows one of the important principles of good
software design.
\note{“Program to an interface, not an implementation.” —Design Patterns}

A good design for mixins can go even further. A class can only be used once in
an inheritance chain, by definition (if a class eventually referred back to
itself, there would be a cycle in the inheritance chain, causing potential
infinite loops). In contrast, when we compose functions, we have no qualms about
using the same function twice (e.g.: \verb|(map ... (filter ... (map ...))))|.
Is there value to using a mixin twice?
\note{There certainly is! See sections 3 and 4 of
\href{http://www.cs.brown.edu/~sk/Publications/Papers/Published/fkf-classes-mixins/}{Classes
and Mixins}.}

Mixins solve an important problem that arises in the design of libraries.
Suppose we have a dozen different features which can be combined in different
ways. How many classes should we provide? Furthermore, not all of these can be
combined with each other. It is obviously impractical to generate the entire
combinatorial explosion of classes. It would be better if the devleoper could
pick and choose the features they care about, with some mechanism to prevent
unreasonable combinations. This is precisely the problem that mixins solve: they
provide the class extensions, which the developers can combine, in an
interface-preserving way, to create just the classes they need.
\note{Mixins are used extensively in the Racket GUI library.
For instance, color:text-mixin consumes basic text editor interfaces and
implements the colored text editor interface. The latter is iself a basic text
editor interface, so additional basic text mixins can be applied to the result.}

\Exercise{
How does your favorite object-oriented library solve this problem?
}

Mixins do have one limitation: they enforce a linearity of composition. This
strictness is sometimes misplaced, because it puts a burden on programmers that
may not be necessary. A generalization of mixins called traits says that instead
of extending a single mixin, we can extend a set of them. Of course, the moment
we extend more than one, we must again contend with potential name-clashes. Thus
traits must be equipped with mechanisms for resolving name clashes, often in the
form of some name-combination algebra. Traits thus offer a nice complement to
mixins, enabling programmers to choose the mechanism that best fits their needs.
As a result, Racket provides both mixins and traits.

\secup

\secup
