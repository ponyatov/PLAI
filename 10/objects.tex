\secrel{10 Objects 67}\secdown

When a language admits functions as values, it provides developers the most
natural way to represent a unit of computation. Suppose a developer wants to
parameterize some function f. Any language lets f be parameterized by passive
data, such as numbers and strings. But it is often attractive to parameterize it
over active data: a datum that can compute an answer, perhaps in response to
some information. Furthermore, the function passed to f can—assuming
lexically-scoped functions—refer to data from the caller without those data
having to be revealed to f, thus providing a foundation for security and
privacy. Thus, lexically-scoped functions are central to the design of many
secure programming techniques.

While a function is a splendid thing, it suffers from excessive terseness.
Sometimes we might want multiple functions to all close over to the same shared
data; the sharing especially matters if some of the functions mutate it and
expect the others to see the result of those mutations. In such cases, it
becomes unwieldly to send just a single function as a parameter; it is more
useful to send a group of functions. The recipient then needs a way to choose
between the different functions in the group. This grouping of functions, and
the means to select one from the group, is the essence of an object.
We are therefore perfectly placed to study objects having covered functions
(section \ref{sec7}) and mutation (section \ref{sec8})—and, it will emerge,
recursion (section \ref{sec9}).
\note{
I cannot hope to do justice to the enormous space of object systems.
Please read \href{http://users.dcc.uchile.cl/~etanter/ooplai/}{Object-Oriented
Programming Languages: Application and Interpretation} by Éric Tanter, which
goes into more detail and covers topics ignored here.
}

Let’s add this notion of objects to our language. Then we’ll flesh it out and
grow it, and explore the many dimensions in the design space of objects. We’ll
first show how to add objects to the core language, but because we’ll want to
prototype many different ideas quickly, we’ll soon shift to a desguaring-based
strategy. Which one you use depends on whether you think understanding them is
critical to understanding the essence of your language. One way to measure this
is how complex your desguaring strategy becomes, and whether by adding some key
core language enhancements, you can greatly reduce the complexity of desugaring.

\secrel{10.1 Objects Without Inheritance  . 67}

The simplest notion of an object—pretty much the only thing everyone who talks about
objects agrees about—is that an object is
\begin{itemize}[nosep]
  \item a value, that
  \item maps names to
  \item stuff: either other values or “methods”.
\end{itemize}
From a minimalist perspective, methods seem to be just functions, and since we
already have those in the language, we can put aside this distinction.
\note{We’re about to find out that “methods” are awfully close to functions but
differ in important ways in how they’re called and/or what’s bound in them.}

\secdown
\secrel{10.1.1 Objects in the Core  . . 68}

Therefore, starting from the language with first-class functions, let’s define
this very simple notion of objects by adding it to the core language. We clearly
have to extend our notion of values:
\lsts{src/10/1/1/1.rkt}{rkt}
We’ll extend the expression grammar to support literal object construction
expressions:
\note{Observe that this is already a design decision. In some languages, like
JavaScript, a developer can write literal objects: a notion so popular that a
subset of the syntax for it in JavaScript has become a Web standard, JSON. In
other languages, like Java, objects can only be created by invoking a
constructor on a class. We can simulate both by assuming that to model the
latter kind of language, we must write object literals only in special positions
following a stylized convention, as we do when desugaring below.}
\lsts{src/10/1/1/2.rkt}{rkt}
Evaluating such an object expression is easy: we just evaluate each of its
expression positions:
\lsts{src/10/1/1/3.rkt}{rkt}
Unfortunately, we can’t actually use an object, because we have no way of
obtaining its content. For that reason, we could add an operation to extract
members:
\lsts{src/10/1/1/4.rkt}{rkt}
whose behavior is intuitive:
\lsts{src/10/1/1/5.rkt}{rkt}

\Exercise{
Implement
\lsts{src/10/1/1/6.rkt}{rkt}
where the second argument is expected to be a objV.
}

In principle, msgC can be used to obtain any kind of member but for simplicity,
we need only assume that we have functions. To use them, we must apply them to
values. This is cumbersome to write in the concrete syntax, so let’s assume
desugaring has taken care of it for us: the concrete syntax for message
invocation includes both the name of the message to fetch and its argument
expression,
\lsts{src/10/1/1/7.rkt}{rkt}
and this desguars into msgC composed with application:
\lsts{src/10/1/1/8.rkt}{rkt}

With this we have a full first language with objects. For instance, here is an
object definition and invocation:
\lsts{src/10/1/1/9.rkt}{rkt}
and this evaluates to \verb|(numV 4)|.

\secrel{10.1.2 Objects by Desugaring  69}

\secrel{10.1.3 Objects as Named Collections  . . 69}

Let’s begin by reproducing the object language we had above. An object is just a
value that dispatches on a given name. For simplicity, we’ll use lambda to
represent the object and case to implement the dispatching.
\note{Observe that basic objects are a generalization of lambda to have multiple
“entry-points”. Conversely, a lambda is an object with just one entry-point, so
it doesn’t need a “method name” to disambiguate.}
\lsts{src/10/1/3/1.rkt}{rkt}

This is the same object we defined earlier, and we use its method in the same
way:
\lsts{src/10/1/3/2.rkt}{rkt}

Of course, writing method invocations with these nested function calls is
unwieldy (and is about to become even more so), so we’d be best off equipping
ourselves with a convenient syntax for invoking methods—the same one we saw
earlier (msgS), but here we can simply define it as a function:
\note{We’ve taken advantage of Racket’s variable-arity syntax: . a says “bind
all the remaining—zero or more—arguments to a list named a”. apply “splices” in
such lists of arguments to call functions.}
\lsts{src/10/1/3/3.rkt}{rkt}
This enables us to rewrite our test:
\lsts{src/10/1/3/4.rkt}{rkt}

\DoNow{
Something very important changed when we switched to the desguaring
strategy. Do you see what it is?
}

Recall the syntax definition we had earlier:
\lsts{src/10/1/3/5.rkt}{rkt}
The “name” position of a message was very explicitly a symbol. That is, the
developer had to write the literal name of the symbol there. In our desugared
version, the name position is just an expression that must evaluate to a symbol;
for instance, one could have written
\lsts{src/10/1/3/6.rkt}{rkt}
This is a general problem with desugaring: the target language may allow
expressions that have no counterpart in the source, and hence cannot be mapped
back to it. Fortunately we don’t often need to perform this inverse mapping,
though it does arise in some debugging and program comprehension tools. More
subtly, however, we must ensure that the target language does not produce values
that have no corresponding equivalent in the source.

Now that we have basic objects, let’s start adding the kinds of features we’ve
come to expect from most object systems.
\secrel{10.1.4 Constructors   . . 70}

\secrel{10.1.5 State   71}

Many people believe that objects primarily exist to encapsulate state. We
certainly haven’t lost that ability. If we desugar to a language with variables
(we could equivalently use boxes, in return for a slight desugaring overhead),
we can easily have multiple methods mutate common state, such as a constructor
argument:
\note{Alan Kay, who won a Turing Award for inventing Smalltalk and modern object
technology, disagrees. In
\href{http://www.smalltalk.org/smalltalk/TheEarlyHistoryOfSmalltalk_Abstract.html}{The
Early History of Smalltalk}, he says, “[t]he small scale [motivation for OOP]
was to find a more flexible version of assignment, and then to try to eliminate
it altogether”. He adds, “It is unfortunate that much of what is called
‘object-oriented programming’ today is simply old style programming with fancier
constructs. Many programs are loaded with ‘assignment-style’ operations now done
by more expensive attached procedures.”}
\lsts{src/10/1/5/1.rkt}{rkt}
For instance, we can test a sequence of operations:
\lsts{src/10/1/5/2.rkt}{rkt}
and also notice that mutating one object doesn’t affect another:
\lsts{src/10/1/5/3.rkt}{rkt}

\secrel{10.1.6 Private Members   71}

\secrel{10.1.7 Static Members   . 72}

\secrel{10.1.8 Objects with Self-Reference  72}

Until now, our objects have simply been packages of named functions: functions
with multiple named entry-points, if you will. We’ve seen that many of the
features considered important in object systems are actually simple patterns
over functions and scope, and have indeed been used—without names assigned to
them—for decades by programmers armed with lambda.

One characteristic that actually distinguishes object systems is that each
object is automatically equipped with a reference to the same object, often
called self or this. Can we implement this easily?
\note{I prefer this slightly dry way of putting it to the anthropomorphic “knows
about itself” terminology often adopted by object advocates.
Indeed, note that we have gotten this far into object system properties without
ever needing to resort to anthropomorphism.}

\secdown

\secrel{Self-Reference Using Mutation}

Yes, we can, because we have seen just this very pattern when we implemented
recursion; we’ll just generalize it now to refer not just to the same box or
function but to the same object.
\lsts{src/10/1/8/1.rkt}{rkt}
Observe that this is precisely the recursion pattern (section 9.2), adapted
slightly. We’ve tested it having first send a method to its own second. Sure
enough, this produces the expected answer:
\lsts{src/10/1/8/2.rkt}{rkt}

\secrel{Self-Reference Without Mutation}

If you studied how to implement recursion without mutation, you’ll notice that
the same solution applies here, too. Observe:
\lsts{src/10/1/8/3.rkt}{rkt}
Each method now takes self as an argument. That means method invocation must be
modified to follow this new pattern:
\lsts{src/10/1/8/4.rkt}{rkt}
That is, when invoking a method on o, we must pass o as a parameter to the
method. Obviously, this approach is dangerous because we can potentially pass a
different object as the “self”. Exposing this to the developer is therefore
probably a bad idea; if this implementation technique is used, it should only be
done in desugaring.
\note{Nevertheless, Python exposes just this in its surface syntax. While this
tribute to the Y-combinator is touching, perhaps the resultant brittleness was
unnecessary.}

\secup
\secrel{10.1.9 Dynamic Dispatch  . . 74}

\secup

\secrel{10.2 Member Access Design Space  75}

We already have two orthogonal dimensions when it comes to the treatment of member
names. One dimension is whether the name is provided statically or computed, and the
other is whether the set of names is fixed or variable:
\begin{tabular}{|p{0.3\linewidth}|p{0.3\linewidth}|p{0.3\linewidth}|}
\hline
& Name is Static & Name is Computed \\
\hline
Fixed Set & As in base Java. & As in Java \\
of Members && with reflection \\ &&to compute the name.\\
\hline
Variable Set& Difficult to envision & Most
scripting \\
of Members & (what use & languages. \\ & would it be?). &\\ 
\hline
\end{tabular}
Only one case does not quite make sense: if we force the developer to specify
the member name in the source file explicitly, then no new members would be
accessible (and some accesses to previously-existing, but deleted, members would
fail). All other points in this design space have, however, been explored by
languages.

The lower-right quadrant corresponds closely with languages that use hash-tables
to represent objects. Then the name is simply the index into the hash-table.
Some languages carry this to an extreme and use the same representation even for
numeric indices, thereby (for instance) conflating objects with dictionaries and
even arrays. Even when the object only handles “member names”, this style of
object creates significant difficulty for type-checking \ref{}\ and is hence not
automatically desirable.

Therefore, in the rest of this section, we will stick with “traditional” objects
that have a fixed set of names and even static member name references (the
top-left quadrant). Even then, we will find there is much, much more to study.

\secrel{10.3 What (Goes In) Else?   . . 75}
\secdown
\secrel{10.3.1 Classes   . . 76}
\secrel{10.3.2 Prototypes   78}
\secrel{10.3.3 Multiple Inheritance  . 78}
\secrel{10.3.4 Super-Duper!   . . 79}
\secrel{10.3.5 Mixins and Traits   79}
\secup

\secup
