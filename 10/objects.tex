\secrel{10 Objects 67}\secdown

When a language admits functions as values, it provides developers the most
natural way to represent a unit of computation. Suppose a developer wants to
parameterize some function f. Any language lets f be parameterized by passive
data, such as numbers and strings. But it is often attractive to parameterize it
over active data: a datum that can compute an answer, perhaps in response to
some information. Furthermore, the function passed to f can—assuming
lexically-scoped functions—refer to data from the caller without those data
having to be revealed to f, thus providing a foundation for security and
privacy. Thus, lexically-scoped functions are central to the design of many
secure programming techniques.

While a function is a splendid thing, it suffers from excessive terseness.
Sometimes we might want multiple functions to all close over to the same shared
data; the sharing especially matters if some of the functions mutate it and
expect the others to see the result of those mutations. In such cases, it
becomes unwieldly to send just a single function as a parameter; it is more
useful to send a group of functions. The recipient then needs a way to choose
between the different functions in the group. This grouping of functions, and
the means to select one from the group, is the essence of an object.
We are therefore perfectly placed to study objects having covered functions
(section \ref{sec7}) and mutation (section \ref{sec8})—and, it will emerge,
recursion (section \ref{sec9}).
\note{
I cannot hope to do justice to the enormous space of object systems.
Please read \href{http://users.dcc.uchile.cl/~etanter/ooplai/}{Object-Oriented
Programming Languages: Application and Interpretation} by Éric Tanter, which
goes into more detail and covers topics ignored here.
}

Let’s add this notion of objects to our language. Then we’ll flesh it out and
grow it, and explore the many dimensions in the design space of objects. We’ll
first show how to add objects to the core language, but because we’ll want to
prototype many different ideas quickly, we’ll soon shift to a desguaring-based
strategy. Which one you use depends on whether you think understanding them is
critical to understanding the essence of your language. One way to measure this
is how complex your desguaring strategy becomes, and whether by adding some key
core language enhancements, you can greatly reduce the complexity of desugaring.

\secrel{10.1 Objects Without Inheritance  . 67}

The simplest notion of an object—pretty much the only thing everyone who talks about
objects agrees about—is that an object is
\begin{itemize}[nosep]
  \item a value, that
  \item maps names to
  \item stuff: either other values or “methods”.
\end{itemize}
From a minimalist perspective, methods seem to be just functions, and since we
already have those in the language, we can put aside this distinction.
\note{We’re about to find out that “methods” are awfully close to functions but
differ in important ways in how they’re called and/or what’s bound in them.}

\secdown
% \secrel{14.2.3 Implementation in the Core  115}

Now that we’ve seen how CPS can be implemented through desguaring, we should ask
whether it can be put in the core instead.

Recall that we’ve said that CPS applies to all programs. We have one program we
are especially interested in: the interpreter. Sure enough, we can apply the CPS
transformation to it, making available what are effectively the same
continuations.

First, we’ll find it convenient to use a procedural representation of closures
\ref{}. We’ll have the interpreter take an extra argument, which consumes values
(those given to the continuation) and eventually returns them:
\lsts{14/2/3/1.rkt}{rkt}
In the easy cases, instead of returning a value we need to simply pass it to the
continuation argument:
\lsts{14/2/3/2.rkt}{rkt}
(Note that multC is handled entirely analogous to plusC.)

Let’s start with the easy case, plusC. First we interpret the left
sub-expression. The continuation for this evaluation interprets the right
sub-expression. The continuation for that adds the result. What should happen to
the result of addition? In interp, it was returned to whichever computation
caused the plusC to be interpreted. Now, remember, we no longer return values;
instead we pass them to the continuation:
\lsts{14/2/3/3.rkt}{rkt}
\Exercise{
Implement the code for multC.
}
This leaves the two difficult, and related, pieces.

In an application, we again have to interpret the two sub-expressions, and then
apply the resulting closure to the argument. But we’ve already agreed that every
application needs a continuation argument. Therefore, we have to update our
definition of a value:
\lsts{14/2/3/4.rkt}{rkt}

Now we have to decide what continuation to pass. In an application, it’s the
continuation given to the interpreter:
\lsts{14/2/3/5.rkt}{rkt}

Finally, the lamC case. We have to create a closV using a lambda, as before.
However, this procedure needs to take two arguments: the actual value of the
argument, and the continuation of the application. The critical question is,
what is this latter value?

We have essentially two choices. k represents the static continuation: the one
active at the point of closure construction. However, what we want is the
continuation at the point of closure invocation: the dynamic continuation.
\lsts{14/2/3/6.rkt}{rkt}

To test this revised interpreter, we need to invoke interp/k with some kind of
initial continuation value. This needs to be a procedure that represents nothing
remaining in the computation. A natural representation for this is the identity
function:
\lsts{14/2/3/7.rkt}{rkt}
To signify that this is strictly a top-level interface to interp/k, we’ve
dropped the environment parameter and pass the empty environment automatically.
If we want to be especially sure we haven’t accidentally used this procedure
recursively, we could insert a call to error at its end to prevent it from
returning and its return value being used.

% \input{10/1/2/desugar}
% \secrel{10.1.3 Objects as Named Collections  . . 69}

Let’s begin by reproducing the object language we had above. An object is just a
value that dispatches on a given name. For simplicity, we’ll use lambda to
represent the object and case to implement the dispatching.
\note{Observe that basic objects are a generalization of lambda to have multiple
“entry-points”. Conversely, a lambda is an object with just one entry-point, so
it doesn’t need a “method name” to disambiguate.}
\lsts{10/1/3/1.rkt}{rkt}

This is the same object we defined earlier, and we use its method in the same
way:
\lsts{10/1/3/2.rkt}{rkt}

Of course, writing method invocations with these nested function calls is
unwieldy (and is about to become even more so), so we’d be best off equipping
ourselves with a convenient syntax for invoking methods—the same one we saw
earlier (msgS), but here we can simply define it as a function:
\note{We’ve taken advantage of Racket’s variable-arity syntax: . a says “bind
all the remaining—zero or more—arguments to a list named a”. apply “splices” in
such lists of arguments to call functions.}
\lsts{10/1/3/3.rkt}{rkt}
This enables us to rewrite our test:
\lsts{10/1/3/4.rkt}{rkt}

\DoNow{
Something very important changed when we switched to the desguaring
strategy. Do you see what it is?
}

Recall the syntax definition we had earlier:
\lsts{10/1/3/5.rkt}{rkt}
The “name” position of a message was very explicitly a symbol. That is, the
developer had to write the literal name of the symbol there. In our desugared
version, the name position is just an expression that must evaluate to a symbol;
for instance, one could have written
\lsts{10/1/3/6.rkt}{rkt}
This is a general problem with desugaring: the target language may allow
expressions that have no counterpart in the source, and hence cannot be mapped
back to it. Fortunately we don’t often need to perform this inverse mapping,
though it does arise in some debugging and program comprehension tools. More
subtly, however, we must ensure that the target language does not produce values
that have no corresponding equivalent in the source.

Now that we have basic objects, let’s start adding the kinds of features we’ve
come to expect from most object systems.
% \secrel{10.1.4 Constructors   . . 70}

A constructor is simply a function that is invoked at object construction time.
We currently lack such a function. by turning an object from a literal into a
function that takes constructor parameters, we achieve this effect:
\lsts{10/1/4/1.rkt}{rkt}
In the first example, we pass 5 as the constructor’s argument, so adding 3
yields 8. The second is similar, and shows that the two invocations of the
constructors don’t interfere with one another.

% \secrel{10.1.5 State   71}

Many people believe that objects primarily exist to encapsulate state. We
certainly haven’t lost that ability. If we desugar to a language with variables
(we could equivalently use boxes, in return for a slight desugaring overhead),
we can easily have multiple methods mutate common state, such as a constructor
argument:
\note{Alan Kay, who won a Turing Award for inventing Smalltalk and modern object
technology, disagrees. In
\href{http://www.smalltalk.org/smalltalk/TheEarlyHistoryOfSmalltalk_Abstract.html}{The
Early History of Smalltalk}, he says, “[t]he small scale [motivation for OOP]
was to find a more flexible version of assignment, and then to try to eliminate
it altogether”. He adds, “It is unfortunate that much of what is called
‘object-oriented programming’ today is simply old style programming with fancier
constructs. Many programs are loaded with ‘assignment-style’ operations now done
by more expensive attached procedures.”}
\lsts{10/1/5/1.rkt}{rkt}
For instance, we can test a sequence of operations:
\lsts{10/1/5/2.rkt}{rkt}
and also notice that mutating one object doesn’t affect another:
\lsts{10/1/5/3.rkt}{rkt}

% \secrel{10.1.6 Private Members   71}

Another common object language feature is private members: ones that are visible
only inside the object, not outside it. These may seem like an additional
feature we need to implement, but we already have the necessary mechanism in the
form of locallyscoped, lexically-bound variables:
\note{Except that, in Java, instances of other classes of the same type are
privy to “private” members. Otherwise, you would simply never be able to
implement an Abstract Data Type.}
\lsts{10/1/6/1.rkt}{rkt}
The desugaring above provides no means for accessing count, and lexical scoping
ensures that it remains hidden to the world.

% \input{10/1/7/static}
% \input{10/1/8/selfref}
% \secrel{10.1.9 Dynamic Dispatch  . . 74}

Finally, we should make sure our objects can handle a characteristic attribute
of object systems, which is the ability to invoke a method without the caller
having to know or decide which object will handle the invocation. Suppose we
have a binary tree data structure, where a tree consists of either empty nodes
or leaves that hold a value. In traditional functions, we are forced to
implement the equivalent some form of conditional—either a cond or a type-case
or pattern-match or other moral equivalent—that exhaustively lists and selects
between the different kinds of trees. If the definition of a tree grows to
include new kinds of trees, each of these code fragments must be modified.
Dynamic dispatch solves this problem by making that conditional branch disappear
from the user’s program and instead be handled by the method selection code
built into the language. The key feature that this provides is an extensible
conditional. This is one dimension of the extensibility that objects provide.
\note{This property—which appears to make systems more black-box extensible
because one part of the system can grow without the other part needing to be
modified to accommodate those changes—is often hailed as a key benefit of
object-orientation.
While this is indeed an advantage objects have over functions, there is a dual
advantage that functions have over objects, and indeed many object programmers
end up contorting their code—using the Visitor pattern—to make it look more like
a function-based organization. Read
\href{http://cs.brown.edu/~sk/Publications/Papers/Published/kff-synth-fp-oo/}{Synthesizing
Object-Oriented and Functional Design to Promote Re-Use} for a running example
that will lay out the problem in its full glory. Try to solve it in your
favorite language, and see the
\href{http://www.cs.utah.edu/plt/publications/icfp98-ff/paper.shtml}{\racket\
solution}.}

Let’s now defined our two kinds of tree objects:
\lsts{10/1/9/1.rkt}{rkt}

With these, we can make a concrete tree:
\lsts{10/1/9/2.rkt}{rkt}

And finally, test it:
\lsts{10/1/9/3.rkt}{rkt}

Observe that both in the test case and in the add method of node, there is a
reference to 'add without checking whether the recipient is a mt or node.
Instead, the run-time system extracts the recipient’s add method and invokes it.
This missing conditional in the user’s program is the essence of dynamic
dispatch.

\secup

\input{10/2/member}
\secrel{10.3 What (Goes In) Else?   . . 75}

Until now, our case statements have not had an else clause. One reason to do so
would be if we had a variable set of members in an object, though that is
probably better handled through a different representation than a conditional: a
hash-table, for instance, as we’ve discussed above. In contrast, if an object’s
set of members is fixed, desugaring to a conditional works well for the purpose
of illustration (because it emphasizes the fixed nature of the set of member
names, which a hash table leaves open to interpretation—and also error). There
is, however, another reason for an else clause, which is to “chain” control to
another, parent, object. This is called inheritance.

Let’s return to our model of desugared objects above. To implement inheritance,
the object must be given “something” to which it can delegate method invocations
that it does not recognize. A great deal will depend on what that “something”
is.

One answer could be that it is simply another object.
\lsts{10/3/1.rkt}{rkt}

Due to our representation of objects, this application effectively searches for
the method in the parent object (and, presumably, recursively in its parents).
If a method matching the name is found, it returns through this chain to the
original call in msg that sought the method. If none is found, the final object
presumably signals a “message not found” error.

\Exercise{
Observe that the application (parent-object m) is like “half a msg”, just like
an l-value was “half a value lookup” \ref{}. Is there any connection?
}

Let’s try this by extending our trees to implement another method, size. We’ll
write an “extension” (you may be tempted to say “sub-class”, but hold off for
now!) for each node and mt to implement the size method. We intend these to
extend the existing definitions of node and mt, so we’ll use the extension
pattern described above.
\note{We’re not editing the existing definitions because that is supposed to be
the whole point of object inheritance: to reuse code in a black-box fashion.
This also means different parties, who do not know one another, can each extend
the same base code. If they had to edit the base, first they have to find out
about each other, and in addition, one might dislike the edits of the other.
Inheritance is meant to sidestep these issues entirely.}

\secdown
\input{10/3/1/classes}
\input{10/3/2/proto}
\input{10/3/3/multi}
\input{10/3/4/super}
\secrel{10.3.5 Mixins and Traits   79}

Let’s return to our “blobs”.

When we write a class in Java, what are we really defining between the opening
and closing braces? It is not the entire class: that depends on the parent that
it extends, and so on recursively. Rather, what we define inside the braces is a
class extension. It only becomes a full-blown class because we also identify the
parent class in the same place.

Naturally, we should ask: Why? Why not separate the act of defining an extension
from applying the extension to a base class? That is, suppose instead of
\lsts{10/3/5/1.rkt}{rkt}
we instead write:
\lsts{10/3/5/2.rkt}{rkt}
and separately
\lsts{10/3/5/3.rkt}{rkt}
where B is some already-defined class.

Thusfar, it looks like we’ve just gone to great lengths to obtain what we had
before. However, the function-application-like syntax is meant to be suggestive:
we can “apply” this extension to several different base classes. Thus:
\lsts{10/3/5/4.rkt}{rkt}
and so on. What we have done by separating the definition of E from that of the
class it extends is to liberate class extensions from the tyranny of the fixed
base class. We have a name for these extensions: they’re called mixins.
\note{The term “mixin” originated in Common Lisp, where it was a particular
pattern of using multiple inheritance.
Lipstick on a pig.}

Mixins make class definition more compositional. They provide many of the
benefits of multiple-inheritance (reusing multiple fragments of functionality)
but within the aegis of a single-inheritance language (i.e., no complicated
rules about lookup order). Observe that when desugaring, it’s actually quite
easy to add mixins to the language. A mixin is primarily a “function over
classes’;. Because we have already determined how to desugar classes, and our
target language for desugaring also has functions, and classes desugar to
expressions that can be nested inside functions, it becomes almost trivial to
implement a simple model of mixins.
\note{This is a case where the greater generality of the target language of
desugaring can lead us to a better construct, if we reflect it back into the
source language.}

In a typed language, a good design for mixins can actually improve
object-oriented programming practice. Suppose we’re defining a mixin-based
version of Java. If a mixin is effectively a class-to-class function, what is
the “type” of this “function”? Clearly, mixin ought to use interfaces to
describe what it expects and provides. Java already enables (but does not
require) the latter, but it does not enable the former: a class (extension)
extends another class—with all its members visible to the extension— not its
interface. That means it obtains all of the parent’s behavior, not a
specification thereof. In turn, if the parent changes, the class might break.

In a mixin language, we can instead write
\lsts{10/3/5/5.rkt}{rkt}
where I is an interface. Then M can only be applied to a class that satisfies
the interface I, and in turn the language can ensure that only members specified
in I are visible in M. This follows one of the important principles of good
software design.
\note{“Program to an interface, not an implementation.” —Design Patterns}

A good design for mixins can go even further. A class can only be used once in
an inheritance chain, by definition (if a class eventually referred back to
itself, there would be a cycle in the inheritance chain, causing potential
infinite loops). In contrast, when we compose functions, we have no qualms about
using the same function twice (e.g.: \verb|(map ... (filter ... (map ...))))|.
Is there value to using a mixin twice?
\note{There certainly is! See sections 3 and 4 of
\href{http://www.cs.brown.edu/~sk/Publications/Papers/Published/fkf-classes-mixins/}{Classes
and Mixins}.}

Mixins solve an important problem that arises in the design of libraries.
Suppose we have a dozen different features which can be combined in different
ways. How many classes should we provide? Furthermore, not all of these can be
combined with each other. It is obviously impractical to generate the entire
combinatorial explosion of classes. It would be better if the devleoper could
pick and choose the features they care about, with some mechanism to prevent
unreasonable combinations. This is precisely the problem that mixins solve: they
provide the class extensions, which the developers can combine, in an
interface-preserving way, to create just the classes they need.
\note{Mixins are used extensively in the Racket GUI library.
For instance, color:text-mixin consumes basic text editor interfaces and
implements the colored text editor interface. The latter is iself a basic text
editor interface, so additional basic text mixins can be applied to the result.}

\Exercise{
How does your favorite object-oriented library solve this problem?
}

Mixins do have one limitation: they enforce a linearity of composition. This
strictness is sometimes misplaced, because it puts a burden on programmers that
may not be necessary. A generalization of mixins called traits says that instead
of extending a single mixin, we can extend a set of them. Of course, the moment
we extend more than one, we must again contend with potential name-clashes. Thus
traits must be equipped with mechanisms for resolving name clashes, often in the
form of some name-combination algebra. Traits thus offer a nice complement to
mixins, enabling programmers to choose the mechanism that best fits their needs.
As a result, Racket provides both mixins and traits.

\secup

\secup
