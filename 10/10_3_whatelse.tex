\secrel{10.3 What (Goes In) Else?   . . 75}

Until now, our case statements have not had an else clause. One reason to do so
would be if we had a variable set of members in an object, though that is
probably better handled through a different representation than a conditional: a
hash-table, for instance, as we’ve discussed above. In contrast, if an object’s
set of members is fixed, desugaring to a conditional works well for the purpose
of illustration (because it emphasizes the fixed nature of the set of member
names, which a hash table leaves open to interpretation—and also error). There
is, however, another reason for an else clause, which is to “chain” control to
another, parent, object. This is called inheritance.

Let’s return to our model of desugared objects above. To implement inheritance,
the object must be given “something” to which it can delegate method invocations
that it does not recognize. A great deal will depend on what that “something”
is.

One answer could be that it is simply another object.
\lsts{src/10/3/1.rkt}{rkt}

Due to our representation of objects, this application effectively searches for
the method in the parent object (and, presumably, recursively in its parents).
If a method matching the name is found, it returns through this chain to the
original call in msg that sought the method. If none is found, the final object
presumably signals a “message not found” error.

\Exercise{
Observe that the application (parent-object m) is like “half a msg”, just like
an l-value was “half a value lookup” \ref{}. Is there any connection?
}

Let’s try this by extending our trees to implement another method, size. We’ll
write an “extension” (you may be tempted to say “sub-class”, but hold off for
now!) for each node and mt to implement the size method. We intend these to
extend the existing definitions of node and mt, so we’ll use the extension
pattern described above.
\note{We’re not editing the existing definitions because that is supposed to be
the whole point of object inheritance: to reuse code in a black-box fashion.
This also means different parties, who do not know one another, can each extend
the same base code. If they had to edit the base, first they have to find out
about each other, and in addition, one might dislike the edits of the other.
Inheritance is meant to sidestep these issues entirely.}

\secdown
\secrel{10.3.1 Classes   . . 76}

\secrel{10.3.2 Prototypes   78}

In our description above, we’ve supplied each class with a description of its
parent class. Object construction then makes instances of each as it goes up the
inheritance chain. There is another way to think of the parent: not as a class
to be instantiated but, instead, directly as an object itself. Then all children
with the same parent would observe the very same object, which means changes to
it from one child object would be visible to another child. The shared parent
object is known as a prototype.
\note{The archetypal prototype-based language is
\href{http://www.selflanguage.org/}{Self}. Though you may have read that
languages like JavaScript are “based on” Self, there is value to studying the
idea from its source, especially because Self presents these ideas in their
purest form.}

Some language designers have argued that prototypes are more primitive than
classes in that, with other basic mechanisms such as functions, one can recover
classes from prototypes—but not the other way around. That is essentially what
we have done above: each “class” function contains inside it an object
description, so a class is an object-returning-function. Had we exposed these
are two different operations and chosen to inherit directly an object, we would
have something akin to prototypes.

\Exercise{
Modify the inheritance pattern above to implement a Self-like, prototypebased
language, instead of a class-based language. Because classes provide each object
with distinct copies of their parent objects, a prototypelanguage might provide
a clone operation to simplify creation of the operation that simulates classes
atop prototypes.
}
\secrel{10.3.3 Multiple Inheritance  . 78}

\secrel{10.3.4 Super-Duper!   . . 79}

Many languages have a notion of super-invocations, i.e., the ability to invoke a
method or access a field higher up in the inheritance chain. This includes doing
so at the point of object construction, where there is often a requirement that
all constructors be invoked, to make sure the object is properly defined.
\note{Note that I say “the” and “chain”. When we switch to multiple inheritance,
these concepts are replaced with something much more complex.}

We have become so accustomed to thinking of these calls as going “up” the chain
that we may have forgotten to ask whether this is the most natural direction.
Keep in mind that constructors and methods are expected to enforce invariants.
Whom should we trust more: the super-class or the sub-class? One argument would
say that the sub-class is most refined, so it has the most global view of the
object. Conversely, each super-class has a vested interest in protecting its
invariants against violation by ignorant sub-classes.

These are two fundamentally opposed views of what inheritance means. Going up
the chain means we view the extension as replacing the parent. Going down the
chain means we view the extension as refining the parent. Because we normally
associate sub-classing with refinement, why do our languages choose the “wrong”
order of calling? Some languages have, therefore, explored invocation in the
downward direction by default.
\note{
\href{http://www.daimi.au.dk/~eernst/gbeta/}{gbeta} is a modern programming
language that supports inner, as well as many other interesting features.
It is also interesting to consider
\href{http://www.cs.utah.edu/plt/publications/oopsla04-gff.pdf}{combining both
directions}.
}
\secrel{10.3.5 Mixins and Traits   79}

Let’s return to our “blobs”.

When we write a class in Java, what are we really defining between the opening
and closing braces? It is not the entire class: that depends on the parent that
it extends, and so on recursively. Rather, what we define inside the braces is a
class extension. It only becomes a full-blown class because we also identify the
parent class in the same place.

Naturally, we should ask: Why? Why not separate the act of defining an extension
from applying the extension to a base class? That is, suppose instead of
\lsts{src/10/3/5/1.rkt}{rkt}
we instead write:
\lsts{src/10/3/5/2.rkt}{rkt}
and separately
\lsts{src/10/3/5/3.rkt}{rkt}
where B is some already-defined class.

Thusfar, it looks like we’ve just gone to great lengths to obtain what we had
before. However, the function-application-like syntax is meant to be suggestive:
we can “apply” this extension to several different base classes. Thus:
\lsts{src/10/3/5/4.rkt}{rkt}
and so on. What we have done by separating the definition of E from that of the
class it extends is to liberate class extensions from the tyranny of the fixed
base class. We have a name for these extensions: they’re called mixins.
\note{The term “mixin” originated in Common Lisp, where it was a particular
pattern of using multiple inheritance.
Lipstick on a pig.}

Mixins make class definition more compositional. They provide many of the
benefits of multiple-inheritance (reusing multiple fragments of functionality)
but within the aegis of a single-inheritance language (i.e., no complicated
rules about lookup order). Observe that when desugaring, it’s actually quite
easy to add mixins to the language. A mixin is primarily a “function over
classes’;. Because we have already determined how to desugar classes, and our
target language for desugaring also has functions, and classes desugar to
expressions that can be nested inside functions, it becomes almost trivial to
implement a simple model of mixins.
\note{This is a case where the greater generality of the target language of
desugaring can lead us to a better construct, if we reflect it back into the
source language.}

In a typed language, a good design for mixins can actually improve
object-oriented programming practice. Suppose we’re defining a mixin-based
version of Java. If a mixin is effectively a class-to-class function, what is
the “type” of this “function”? Clearly, mixin ought to use interfaces to
describe what it expects and provides. Java already enables (but does not
require) the latter, but it does not enable the former: a class (extension)
extends another class—with all its members visible to the extension— not its
interface. That means it obtains all of the parent’s behavior, not a
specification thereof. In turn, if the parent changes, the class might break.

In a mixin language, we can instead write
\lsts{src/10/3/5/5.rkt}{rkt}
where I is an interface. Then M can only be applied to a class that satisfies
the interface I, and in turn the language can ensure that only members specified
in I are visible in M. This follows one of the important principles of good
software design.
\note{“Program to an interface, not an implementation.” —Design Patterns}

A good design for mixins can go even further. A class can only be used once in
an inheritance chain, by definition (if a class eventually referred back to
itself, there would be a cycle in the inheritance chain, causing potential
infinite loops). In contrast, when we compose functions, we have no qualms about
using the same function twice (e.g.: \verb|(map ... (filter ... (map ...))))|.
Is there value to using a mixin twice?
\note{There certainly is! See sections 3 and 4 of
\href{http://www.cs.brown.edu/~sk/Publications/Papers/Published/fkf-classes-mixins/}{Classes
and Mixins}.}

Mixins solve an important problem that arises in the design of libraries.
Suppose we have a dozen different features which can be combined in different
ways. How many classes should we provide? Furthermore, not all of these can be
combined with each other. It is obviously impractical to generate the entire
combinatorial explosion of classes. It would be better if the devleoper could
pick and choose the features they care about, with some mechanism to prevent
unreasonable combinations. This is precisely the problem that mixins solve: they
provide the class extensions, which the developers can combine, in an
interface-preserving way, to create just the classes they need.
\note{Mixins are used extensively in the Racket GUI library.
For instance, color:text-mixin consumes basic text editor interfaces and
implements the colored text editor interface. The latter is iself a basic text
editor interface, so additional basic text mixins can be applied to the result.}

\Exercise{
How does your favorite object-oriented library solve this problem?
}

Mixins do have one limitation: they enforce a linearity of composition. This
strictness is sometimes misplaced, because it puts a burden on programmers that
may not be necessary. A generalization of mixins called traits says that instead
of extending a single mixin, we can extend a set of them. Of course, the moment
we extend more than one, we must again contend with potential name-clashes. Thus
traits must be equipped with mechanisms for resolving name clashes, often in the
form of some name-combination algebra. Traits thus offer a nice complement to
mixins, enabling programmers to choose the mechanism that best fits their needs.
As a result, Racket provides both mixins and traits.

\secup
