\secrel{8.4 Parameter Passing   . 60}

In our current implementation, on every function call, we allocate a fresh
location in the store for the parameter. This means the following program
\lsts{src/8/4/1.rkt}{rkt}
evaluates to 5, not 3. That is because the value of the formal parameter x is
held at a different location than that of the actual parameter y, so the
mutation affects the location of x, leaving y unscathed.

Now suppose, instead, that application behaved as follows. When the actual
parameter is a variable, and hence has a location in memory, instead of
allocating a new location for the value, it simply passes along the existing one
for the variable. Now the formal parameter is referring to the same store
location as the actual: i.e., they are variable aliases. Thus any mutation on
the formal will leak back out into the calling context; the above program would
evaluate to 3 rather than 5. These is called a call-by-reference
parameter-passing strategy.
\note{Instead, our interpreter implements call-by-value, and this is the same
strategy followed by languages like Java. This causes confusion because when the
value is itself mutable, changes made to the value in the callee are observed by
the caller. However, that is simply an artifact of mutable values, not of the
calling strategy. Please avoid this confusion!}

For some years, this power was considered a good idea. It was useful because
programmers could write abstractions such as swap, which swaps the value of two
variables in the caller. However, the disadvantages greatly outweigh the
advantages:
\begin{itemize}
  \item 
A careless programmer can alias a variable in the caller and modify it without
realizing they have done so, and the caller may not even realize this has
happened until some obscure condition triggers it.
  \item 
Some people thought this was necessary for efficiency: they assumed the
alternative was to copy large data structures. However, call-by-value is
compatible with passing just the address of the data structure. You only need
make a copy if (a) the data structure is mutable, (b) you do not want the caller
to be able to mutate it, and (c) the language does not itself provide
immutability annotations or other mechanisms.
  \item 
It can force non-uniform and hence non-modular reasoning. For instance, suppose
we have the procedure:
\lsts{src/8/4/2.rkt}{rkt}
If the language were to permit by-reference parameter passing, then the
programmer cannot locally—i.e., just from the above code—determine what the
value of x will be in the ellipses.
\end{itemize}

At the very least, then, if the language is going to permit by-reference
parameters, it should let the caller determine whether to pass the
reference—i.e., let the callee share the memory address of the caller’s
variable—or not. However, even this option is not quite as attractive as it may
sound, because now the callee faces a symmetric problem, not knowing whether its
parameters are aliased or not. In traditional, sequential programs this is less
of a concern, but if the procedure is reentrant, the callee faces precisely the
same predicaments.

At some point, therefore, we should consider whether any of this fuss is
worthwhile. Instead, callers who want the callee to perform a mutation could
simply send a boxed value to the callee. The box signals that the caller
accepts—indeed, invites—the callee to perform a mutation, and the caller can
extract the value when it’s done. This does obviate the ability to write a
simple swapper, but that’s a small price to pay for genuine software engineering
concerns.
