\secrel{10.3 What (Goes In) Else?   . . 75}

Until now, our case statements have not had an else clause. One reason to do so
would be if we had a variable set of members in an object, though that is
probably better handled through a different representation than a conditional: a
hash-table, for instance, as we’ve discussed above. In contrast, if an object’s
set of members is fixed, desugaring to a conditional works well for the purpose
of illustration (because it emphasizes the fixed nature of the set of member
names, which a hash table leaves open to interpretation—and also error). There
is, however, another reason for an else clause, which is to “chain” control to
another, parent, object. This is called inheritance.

Let’s return to our model of desugared objects above. To implement inheritance,
the object must be given “something” to which it can delegate method invocations
that it does not recognize. A great deal will depend on what that “something”
is.

One answer could be that it is simply another object.
\lsts{src/10/3/1.rkt}{rkt}

Due to our representation of objects, this application effectively searches for
the method in the parent object (and, presumably, recursively in its parents).
If a method matching the name is found, it returns through this chain to the
original call in msg that sought the method. If none is found, the final object
presumably signals a “message not found” error.

\Exercise{
Observe that the application (parent-object m) is like “half a msg”, just like
an l-value was “half a value lookup” \ref{}. Is there any connection?
}

Let’s try this by extending our trees to implement another method, size. We’ll
write an “extension” (you may be tempted to say “sub-class”, but hold off for
now!) for each node and mt to implement the size method. We intend these to
extend the existing definitions of node and mt, so we’ll use the extension
pattern described above.
\note{We’re not editing the existing definitions because that is supposed to be
the whole point of object inheritance: to reuse code in a black-box fashion.
This also means different parties, who do not know one another, can each extend
the same base code. If they had to edit the base, first they have to find out
about each other, and in addition, one might dislike the edits of the other.
Inheritance is meant to sidestep these issues entirely.}

\secdown
\secrel{10.3.1 Classes   . . 76}
\secrel{10.3.2 Prototypes   78}
\secrel{10.3.3 Multiple Inheritance  . 78}
\secrel{10.3.4 Super-Duper!   . . 79}
\secrel{10.3.5 Mixins and Traits   79}
\secup
