\secrel{2 Everything (We Will Say) About Parsing
\ru{Все (что мы будем говорить) о разборе}}\secdown

Parsing is the act of turning an input character stream into a more structured,
internal representation.
\ru{\termdef{Парсинг}{парсинг} или \termdef{разбор}{синтаксический разбор}\
--- процесс превращения входного потока одиночных символов в более
структурированное внутреннее представление\note{программы или данных, заданных в
текстовой синтаксической форме}.}
A common internal representation is as a tree, which programs can recursively
process.
\ru{Обычно используется внутреннее представление в виде дерева, которое может
быть обработано программой рекурсивно.}

For instance, given the stream
\ru{Например, для входного потока символов\note{включая пробелы, табуляции и
концы строк}}
\begin{verbatim}
23 + 5 - 6
\end{verbatim}
we might want a tree representing addition whose left (L) node represents the
number 23 and whose right (R) node represents subtraction of 6 from 5.
\ru{мы хотим получить деревянное представление сложения, в котором левая (L)
ветвь содержит число 23, а правая (R)\ --- вложенное представление вычитания 6
из 5.}
A parser is responsible for performing this transformation.
\ru{Парсер отвечает за выполнение такой транформации.}

\noindent
\begin{tabular}{c c}
\noindent\includegraphics[height=0.6\textheight]{tmp/2_p10_R.pdf}
&
\noindent\includegraphics[height=0.6\textheight]{tmp/2_p10_L.pdf}
\\
\emph{Право}ассоциативный разбор
&
(*) \emph{Лево}ассоциативный разбор
\\
\end{tabular}\bigskip

Parsing is a large, complex problem that is far from solved due to the
difficulties of ambiguity.
\ru{Парсинг\ --- большая проблема информатики, сложность которой
заключается в трудностях неоднозначности.}
For instance, an alternate parse tree (*) for the above input expression might
put subtraction at the top and addition below it.
\ru{Например, существует альтернативное (*) \termdef{дерево разбора}{дерево
разбора} для того же входного выражения, в котором мы можем поместить
вычитание на вершину дерева, а сложение будет вложенным поддеревом.}
We might also want to consider whether this addition operation is commutative
and hence whether the order of arguments can be switched.
\ru{Нас также может интересовать, является ли это сложение коммутативной
операцией, то есть можем ли мы изменить порядок аргументов\note{например для
оптимизации кода}.}
Everything only gets much, much worse when we get to full-fledged programming
languages (to say nothing of natural languages).
\ru{Все становится намного хуже по мере того, как мы пробираемся в сторону
полноценных языков программирования (даже не говоря о натуральных языках).}

\secrel{A Lightweight, Built-In First Half of a Parser
\ru{Легковесная встроенная часть парсера}}

These problems make parsing a worthy topic in its own right, and entire books,
tools, and courses are devoted to it.
\ru{Эти проблемы делают разбор достойной темой саму по себе, и ей посвящены
целые книги, утилиты и учебные курсы, например первая глава ``Книги Дракона''
\cite{dragon}.}
However, from our perspective parsing is mostly a distraction, because we want
to study the parts of programming languages that are not parsing.
\ru{Однако, с нашей точки зрения тема парсинга является сильным отвлечением,
так как есть другие более достойные темы, касающиеся реализации языков
программирования.}
We will therefore exploit a handy feature of Racket to manage the transformation
of input streams into trees: 
\ru{Поэтому мы сделаем финт ушами, и будем использовать встроенную фичу
\racket а для получения готовых деревьев разбора из входного потока: функцию}
\verb|read|.
\verb|read| is tied to the parenthetical form of the language, in that it parses
fully (and hence unambiguously) parenthesized terms into a built-in tree form.
\verb|read| \ru{привязана к скобочной форме языка, и полностью (и следовательно
однозначно) разбирает скобочные выражения во встроенное представление\ ---
дерево.}
For instance, running \ru{Например применение} \verb|(read)| on the
parenthesized form of the above input \ru{к следующему входному потоку символов
(включающему скобки)}\ ---
\begin{verbatim}
(+ 23 (- 5 6))
\end{verbatim}
--- will produce a list, whose first element is the symbol \ru{создаст список, в
котором первым элементом будет символ} \verb|'+|, second element is the number
\ru{вторым элементом число} 23, and third element is a list \ru{и третий
элемент список}: this list’s first element is the
symbol \ru{в котором первым элемент будет} \verb|'-|, second element is the
number \ru{второй элемент число} 5, and third element is the number \ru{и
третий элемент число} 6.

\fig{\ru{Дерево разбора для} (+ 23 (- 5 6))}{tmp/2_1.pdf}{height=0.7\textheight}

\lst{src/2/p10_1.rkt}

\secrel{A Convenient Shortcut \ru{Удобный трюк}}

As you know you need to test your programs extensively, which is hard to do when you
must manually type terms in over and over again.
\ru{Как вы знаете, нужно тщательно тестировать свои программы, что особенно
сложно, если вам нужно снова и снова вводить выражения вручную.}
Fortunately, as you might expect, the parenthetical syntax is integrated deeply
into \racket\ through the mechanism of quotation.
\ru{К счатью, как и следовало ожидать, скобочный синтаксис глубоко
интегрирован в \racket\ через механизм \termdef{квотирования}{квотирование}.}
That is, \ru{это то самое выражение} \verb|'<expr>|\ --—
which you saw a moment ago in the above example
\ru{которое вы видели только что при выполнении предыдущего примера}\ --- 
acts as if you had run \ru{действует так же, как если бы вы запустили}
\verb|(read)| and typed \ru{и ввели} <expr> at the prompt \ru{в текстовом поле
ввода} (and, of course, evaluates to the value the (read) would have
\ru{и конечно же вычисляется в то же значение, что дает} \verb|(read)|).

\secrel{Types for Parsing \ru{Типы для разбора}}

Actually, I’ve lied a little.
\ru{На самом деле, я немного соврал.}
I said that \ru{я сказал что} \verb|(read)|\ --- or equivalently, using
quotation \ru{или использование квотирования, что эквивалентно}\ --- will
produce a \emph{list}, etc. \ru{создаст \emph{список}, блаблабла.}
That’s true in regular \racket, but in $Typed PLAI$
\ru{Это так для оригинального \racket, но в $Typed PLAI$} the type it
returns a distinct type called an \ru{возвращается специальный тип, который
называется} \termdef{s-expression}{s-expression}
\ru{s-выражение\index{s-выражение}}, written in $Typed PLAI$ as
\ru{который в $Typed PLAI$ записывается как}
\verb|s-expression|:
\begin{verbatim}
> (read)
- s-expression
[type in (+ 23 (- 5 6))]
'(+ 23 (- 5 6))
\end{verbatim}
\racket\ has a very rich language of s-expressions
\ru{имеет очень богатый язык на s-выражениях}
(it even has notation to represent cyclic structures
\ru{он даже имеет нотацию для представления циклических структур}), 
but we will use only the simple fragment of it.
\ru{но мы будем использовать только простейшую часть этого синтаксиса.}

In the typed language, an s-expression is treated distinctly from the other
types, such as numbers and lists.
\ru{В типизированном языке s-выражения обрабатыватся обособленно от других
типов, таких как числа и списки.}
\begin{framed}
Underneath, an s-expression is a large
recursive datatype that consists of all the base printable values—numbers,
strings, symbols, and so on—and printable collections (lists, vectors, etc.) of
s-expressions.
\ru{Далее s-выражение рассматривается как большой рекурсивный тип данных,
который содержит все базовые отображаемые (представимые в тексте) значения\
--- числа, строки, символы и т.д.\ --- и коллекции (списки, вектора и т.д.)
других s-выражений}.
\end{framed}
As a result, base types like numbers, symbols, and strings are
\emph{both} their own type and an instance of s-expression.
\ru{В результате такие базовые типы как числа, символы и строки, могут
\emph{одновременно} являться как собственным типом (число,..), так и
экземпляром s-выражениея}.
Typing such data can be fairly problematic, as we will discuss later
\ru{Типизация таких данных может быть очень проблематична, детальнее мы обсудим
это позже} \ref{}.

$Typed PLAI$ takes a simple approach.
\ru{$Typed PLAI$ применяет более простой подход.}
When written on their own, values like numbers are of those respective types.
\ru{Когда значания простых типов, типа чисел, написаны сами по себе, они
являются собственными типами (число).}
But when written inside a complex
s-expression—in particular, as created by read or quotation—they have type
s-expression.
\ru{Но когда они включены в состав сложного s-выражения\ --- в частности,
созданы через (read) или квотацию\ --- они имеют тип s-выражения.}
You have to then cast them to their native types.
\ru{Вы должны привести их к нативному типу.}
For instance \ru{Например}:
\lstl{src/2/p11_1.rkt}
This is similar to the casting that a Java programmer would have to insert.
\ru{Это похоже на явное приведение типов, которое должен вставить программист
на \java.}
We will study casting itself later \ru{Мы обсудим само приведение типов позже}
\ref{}.

Observe that the first element of the list is still not treated by the
type checker as a symbol:
\ru{Отметим, что первый элемент списка все еще не распознается контролером
типов как символ:}
a list-shaped s-expression is a list of s-expressions.
\ru{списко-образное s-выражение является списком s-выражений.}
Thus \ru{Таким образом},
\lst{src/2/p11_2.rkt}
whereas again, casting does the trick:
\ru{и снова приведение типов решает проблему :}
\lst{src/2/p11_3.rkt}
The need to cast s-expressions is a bit of a nuisance,
\ru{Необходимость приведения s-выражений немного геморна,}
but some complexity is unavoidable because of what we’re trying to accomplish:
\ru{но некоторая сложность неизбежна из-за того что мы пытаемся достичь:}
to convert an \emph{untyped input} stream into a \emph{typed output} stream
\ru{преобразование \emph{нетипизированного} входного потока в
\emph{типизированный} выходной поток}
through robustly typed means.
\ru{через средства робастной типизации.}
Somehow we have to make explicit our assumptions about that input stream.
\ru{Каким-то образом мы должны делать явные предположения об этом входном
потоке.}

Fortunately we will use s-expressions only in our parser, and our goal is to
\emph{get away from parsing as quickly as possible\,!}
\ru{К счастью, мы будем использовать s-выражения только в нашем парсере, и наша
цель состоит в том, чтобы \emph{уйти от разбора как можно быстрее\,!}}
Indeed, if anything this should be inducement to get away even quicker.
\ru{В самом деле, все эти заморочки являются побуждением сделать этот уход еще
быстрее.}

\secrel{Completing the Parser \ru{Заканчиваем с парсером}}\label{sec2_4}

In principle, we can think of \ru{В принципе, мы можем думать о} \verb|read|\
as a complete parser \ru{как о законченном парсере}.
However, its output is generic \ru{Тем не менее, его вывод все еще сырой}:
it represents the token structure without offering any comment on its intent.
\ru{он содержит структуру токенов не предлагая каких-либо комментариев об их
назначении.}
We would instead prefer to have a representation
\ru{Вместо этого мы предпочли бы иметь представление,}
that tells us something about the \emph{intended meaning} of the terms in our
language,
\ru{которое говорит нам что-то о \emph{предполагаемом значении} термов нашего
языка,}
just as we wrote at the very beginning: “representing addition”, “represents a
number”, and so on.
\ru{так же как мы писали в самом начале: ``представление сложения'',
``представление числа'' и так далее.}

To do this, we must first introduce a datatype
\ru{Чтобы сделать это, мы сначала введем тип данных,}
that captures this representation.
\ru{который зафиксирует это представление.}
We will separately discuss \ru{Мы
отдельно рассмотрим} (section \ru{в разделе} \ref{sec31}) how and why we obtained this
datatype \ru{как и зачем мы применяем этот тип}, but for now let’s say it’s
given to us \ru{но сейчас пока будем считать, что он нам задан}:
\lstx{ArithC.rkt}{src/2/p12_1.rkt}{rkt}\label{arithc}
We now need a function that will convert s-expressions into instances of this
datatype.
\ru{Теперь нам нужна функция, которая преобразует s-выражение в структуру из
экземпляров этого типа.}
This is the other half of our parser \ru{Это вторая половина нашего парсера}:
\lstxl{ArithC.rkt}{src/2/p12_2.rkt}{rkt}

Thus\note{typing in \racket\ console \emph{after program run}} \ru{Таким
образом\note{\ru{введя выражение в \racket-консоли \emph{после выполнения
программы}}}}
\lst{src/2/v1.rkt}
\lstx{ArithC.rkt}{src/2/p12_3.rkt}{rkt}
\lst{src/2/v2.rkt}

Congratulations\,! \ru{Мои поздравления\,!}
You have just completed your first representation of a program.
\ru{Вы только что завершили ваше первое представление программы.}
From now on we can focus entirely on programs
\ru{С этого момента мы можем полностью сосредоточиться на программах,}
represented as recursive trees,
\ru{представленных в виде рекурсивных деревьев,}
ignoring the vagaries of surface syntax
\ru{не обращая внимания на капризы наносного синтаксиса}
and how to get them into the tree form.
\ru{и процесс получения из него дерева разбора.}
We’re finally ready to start studying programming languages\,!
\ru{Мы, наконец, готовы приступить к изучению языков программирования\,!}

\Exercise{
What happens if you forget to quote the argument to the 
\ru{Что случиться, если вы забудете заквотить аргумент вызова}
parser\,?
Why\,? \ru{Почему\,?}
}
\secrel{Coda \ru{Кода}}

\racket’s syntax, which it inherits from Scheme and \lisp, is controversial.
\ru{Синтаксис \racket а, который он наследует от Scheme и Lisp, спорен.}
Observe, however, something deeply valuable that we get from it.
\ru{Заметим, однако, что мы получаем от него нечто глубоко ценное.} 
While parsing traditional languages can be very complex,
\ru{В то время как парсинг традиционных языков может быть очень сложным,}
parsing this syntax is virtually trivial.
\ru{разбор этого синтаксиса практически тривиален.}
Given a sequence of tokens corresponding to the input,
\ru{Для заданной последовательности лексем, соответствующих входному потоку,}
it is absolutely straightforward to turn paren\-the\-sized sequences into
s-expressions;
\ru{абсолютно тривиально превратить скобочные последовательности в s-выражения;}
it is equally straightforward (as we see above) to turn sexpressions into proper
syntax trees.
\ru{столь же просто (как мы видим выше) преобразовать s-выражения в правильные
синтаксические деревья.}
I like to call such two-level languages \term{bicameral}, in loose analogy to
government legislative houses:
\ru{Мне нравится называть такие двухуровневые языки \term{двухпалатными}, в
свободной аналогии к государственным законодательным учреждениям:}
the lower-level does rudimentary well-formedness checking, while the upper-level
does deeper validity checking.
\ru{нижний уровень делает рудиментарную проверку правильности оформления, в то
время как верхний уровень выполняет глубокую проверку валидности.}
(We haven’t done any of the latter yet, but we will
\ru{Мы еще не делали последнего, но мы будем}
\ref{}.)

The virtues of this syntax are thus manifold.
\ru{Достоинства этого синтаксиса, таким образом, многообразны.}
The amount of code it requires is small, and can easily be embedded in many
contexts.
\ru{Объем кода, который он требует, очень мал, и может быть встроен во многих
контекстах.}
By integrating the syntax into the language, it becomes easy for programs to
manipulate representations of programs (as we will see more of in \ref{}).
\ru{Интеграция синтаксиса в язык делает простой программную манипуляцию
представлением программ (как мы увидим в \ref{}).}
It’s therefore no surprise that even though many Lisp-based syntaxes have had
wildly different semantics, they all share this syntactic legacy.
\ru{Поэтому неудивительно, что множество основанных на \lisp е синтаксисов,
имеющих дико разную семантику, все равно разделяют это общее синтаксическое
наследие.}

Of course, we could just use XML instead.
\ru{Конечно, мы могли бы использовать XML.}
That would be much better.
\ru{Это было бы намного лучше.}
Or JSON.
\ru{Или JSON.}
Because that wouldn’t be anything like an s-expression at all.
\ru{Потому что все равно это в итоге было бы тем же s-выражением.}

\secup
