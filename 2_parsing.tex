\secrel{Everything (We Will Say) About Parsing
\ru{Все (что мы будем говорить) о разборе}}\secdown
\clearpage

Parsing is the act of turning an input character stream into a more structured,
internal representation.
\ru{\termdef{Парсинг}{парсинг} или \termdef{разбор}{синтаксический разбор}\
--- процесс превращения входного потока одиночных символов в более
структурированное внутреннее представление\note{программы или данных, заданных в
текстовой синтаксической форме}.}
A common internal representation is as a tree, which programs can recursively
process.
\ru{Обычно используется внутреннее представление в виде дерева, которое может
быть обработано программой рекурсивно.}

For instance, given the stream
\ru{Например, для входного потока символов\note{включая пробелы, табуляции и
концы строк}}
\begin{verbatim}
23 + 5 - 6
\end{verbatim}
we might want a tree representing addition whose left (L) node represents the
number 23 and whose right (R) node represents subtraction of 6 from 5.
\ru{мы хотим получить деревянное представление сложения, в котором левая (L)
ветвь содержит число 23, а правая (R)\ --- вложенное представление вычитания 6
из 5.}
A parser is responsible for performing this transformation.
\ru{Парсер отвечает за выполнение такой транформации.}

\noindent
\begin{tabular}{c c}
\noindent\includegraphics[height=0.8\textheight]{tmp/2_p10_R.pdf}
&
\noindent\includegraphics[height=0.8\textheight]{tmp/2_p10_L.pdf}
\\
\emph{Право}ассоциативный разбор
&
(*) \emph{Лево}ассоциативный разбор
\\
\end{tabular}\bigskip

Parsing is a large, complex problem that is far from solved due to the
difficulties of ambiguity.
\ru{Парсинг\ --- большая проблема информатики, сложность которой
заключается в трудностях неоднозначности.}
For instance, an alternate parse tree (*) for the above input expression might
put subtraction at the top and addition below it.
\ru{Например, существует альтернативное (*) \termdef{дерево разбора}{дерево
разбора} для того же входного выражения, в котором мы можем поместить
вычитание на вершину дерева, а сложение будет вложенным поддеревом.}
We might also want to consider whether this addition operation is commutative
and hence whether the order of arguments can be switched.
\ru{Нас также может интересовать, является ли это сложение коммутативной
операцией, то есть можем ли мы изменить порядок аргументов\note{например для
оптимизации кода}.}
Everything only gets much, much worse when we get to full-fledged programming
languages (to say nothing of natural languages).
\ru{Все становится намного хуже по мере того, как мы пробираемся в сторону
полноценных языков программирования (даже не говоря о натуральных языках).}

\secrel{2.1 A Lightweight, Built-In First Half of a Parser . . . . . . . . . . . . . 10}

\input{2_2_shortcut}
\input{2_3_types}
\secrel{Completing the Parser \ru{Заканчиваем с парсером}}\label{sec2_4}

In principle, we can think of \ru{В принципе, мы можем думать о} \verb|read|\
as a complete parser \ru{как о законченном парсере}.
However, its output is generic \ru{Тем не менее, его вывод все еще сырой}:
it represents the token structure without offering any comment on its intent.
\ru{он содержит структуру токенов не предлагая каких-либо комментариев об их
назначении.}
We would instead prefer to have a representation that tells us something about
the \emph{intended meaning} of the terms in our language, just as we wrote at
the very beginning: “representing addition”, “represents a number”, and so on.
\ru{Вместо этого мы предпочли бы иметь представление, которое говорит нам
что-то о \emph{предполагаемом значении} термов нашего языка, так же как мы
писали в самом начале: ``представление сложения'', ``представление числа'' и
так далее.}

To do this, we must first introduce a datatype that captures this
representation. \ru{Чтобы сделать это, мы сначала введем тип данных, который
зафиксирует это представление.} We will separately discuss \ru{Мы отдельно
рассмотрим} (section \ru{в разделе} \ref{sec31}) how and why we obtained this
datatype \ru{как и зачем мы применяем этот тип}, but for now let’s say it’s
given to us \ru{но сейчас пока будем считать, что он нам задан}:
\lstx{ArithC.rkt}{src/2/p12_1.rkt}{rkt}
We now need a function that will convert s-expressions into instances of this
datatype.
\ru{Теперь нам нужна функция, которая преобразует s-выражение в структуру из
экземпляров этого типа.}
This is the other half of our parser \ru{Это вторая половина нашего парсера}:
\lstx{ArithC.rkt}{src/2/p12_2.rkt}{rkt}

Thus\note{typing in \racket\ console \emph{after program run}} \ru{Таким
образом\note{\ru{введя выражение в \racket-консоли \emph{после выполнения
программы}}}}
\begin{verbatim}
> (parse '(+ (* 1 2) (+ 2 3)))
- ArithC
(plusC
    (multC (numC 1) (numC 2))
    (plusC (numC 2) (numC 3)))
\end{verbatim}
\lstx{ArithC.rkt}{src/2/p12_3.rkt}{rkt}
\begin{verbatim}
(good
  (parse '(+ (* 1 2) (+ 2 3)))
  (plusC (multC (numC 1) (numC 2)) (plusC (numC 2) (numC 3)))
  (plusC (multC (numC 1) (numC 2)) (plusC (numC 2) (numC 3)))
  "at line 26")
\end{verbatim}

Congratulations\,! \ru{Мои поздравления\,!}
You have just completed your first representation of a program.
\ru{Вы только что завершили ваше первое представление программы.}
From now on we can focus entirely on programs represented as recursive trees,
ignoring the vagaries of surface syntax and how to get them into the tree form.
\ru{С этого момента мы можем полностью сосредоточиться на программах,
представленных в виде рекурсивных деревьев, не обращая внимания на капризы
наносного синтаксиса и процесс получения из него дерева разбора.}
We’re finally ready to start studying programming languages\,!
\ru{Мы, наконец, готовы приступить к изучению языков программирования\,!}

\Exercise{
What happens if you forget to quote the argument to the 
\ru{Что случиться, если вы забудете заквотить аргумент вызова}
parser\,?
Why\,? \ru{Почему\,?}
}
\input{2_5_coda}
\secup
