\secrel{The Language of This Book \ru{Язык программирования используемый в
книге}}

Язык программирования используемый в книге\ --- 
\href{http://www.racket-lang.org/}{\racket}. Аналогично операционным системам,
\racket-система является исполняющей средой для целого ряда языков
программирования, так что \emph{вы должны указать \racket у на каком языке 
вы программируете}. Например, в Unix вы указываете в строку типа

\begin{verbatim}
#!/bin/sh
\end{verbatim}

в первой строке shell-скрипта; вы указываете веб-браузеру
тип документа, добавляя заголовок

\begin{verbatim}
<!DOCTYPE HTML PUBLIC "-//W3C//DTD HTML 4.01//EN" ...>
\end{verbatim}

Аналогично, \racket\ требует от вас указать какой язык вы будете использовать.
Диалекты языков \racket\ имеют тот же скобочный синтаксис, что и сам \racket, но
другую семантику; ту же семантику но другой синтаксис; или и то и то.
Так что каждая программа, которую может выполнять \racket-система, начинается со
строки \#lang за которой следует имя диалекта языка: по умолчанию, это
оригинальный \racket\ (указыватся как \verb|racket|). В этой книге мы почти
всегда будем использовать диалект\note{В DrRacket v.6.6, выберите меню
\menu{Язык > Выбрать язык\ldots > Start your program with \#lang to specify the
desired dialect}.}

\begin{verbatim}
#lang plai-typed
\end{verbatim}

Когда мы будем отклоняться от этого правила, это будет указано особо, так что
если не указано иное, добавляйте заголовок \verb|#lang plai-typed| в начало
каждого файла программы (предполагается что я тоже это сделал)\note{В DrRacket
v.6.6 требуется установить расширение plai-typed:\\\menu{Файл>Install
package\ldots>Package
Source:>\url{github://github.com/mflatt/plai-typed/master}>Install>\ldots>Закрыть}}.

The \termdef{Typed PLAI}{Typed PLAI}\ language differs from traditional \racket\
most importantly by being statically typed.
\ru{Язык \term{Typed PLAI}\ отличается от традиционного \racket\ в основном
\emph{статической типизацией}.}
It also gives you some useful new constructs:
\ru{Он также дает вам некоторые новые полезные конструкции:}
\verb|define-type| \ru{определение-типа}, \verb|type-case| \ru{выбор-по-типу},
and \verb|test|\note{There are additional commands for controlling the output
of testing, for instance. \ru{Также существуют дополнительные команды для
управления выводом тестов.} Be sure to read the documentation for the language.
\ru{Обязательно прочитайте документацию для языка.}
In DrRacket v.6.6, go to \menu{Help>Help Desk}, and in the Help Desk search bar, type
\menu{plai-typed}. \ru{В DrRacket v.6.6 идите в меню \menu{Help>Help
Desk}, и в поле поиска \menu{Help Desk} введите \menu{plai-typed}.}}
Here’s an example of each in use.
\ru{Вот примеры использования каждого из них.} 
We can introduce new datatypes
\ru{Вы можем создавать новые типы данных\note{запустить программу можно нажав
\keys{Ctrl+R}}}:

\lst{src/1/p8_1.rkt}

You can roughly think of this as analogous to the following in Java:
\ru{Вы можете примерно понять идею в терминах языка \java:}
an abstract
class \term{абстрактный класс} \verb|MisspelledAnimal| and two concrete
sub-classes \ru{и два конкретизирующих класса}\ \verb|caml| \ru{верблюд} and
\verb|yacc| \ru{якк}, each of which has one numeric constructor argument named
\ru{каждый из которых имеет конструктор с числовым аргументом}
\verb|humps| \ru{горбы} and \verb|height| \ru{высота}, respectively
\ru{соответственно}.

In this language, we construct instances as follows:
\ru{На этом языке мы строим экземпляры классов следующим образом:}

\lst{src/1/p8_2.rkt}

As the name suggests \ru{Как следует из названия,}, \verb|define-type| creates a
type of the given name \ru{создает тип с заданным именем}.
We can use
this when, for instance, binding the above instances to names:
\ru{Мы можем это использовать например при связывании эксземпляров с именами:}

\lst{src/1/p8_3.rkt}

In fact you don’t need these particular type declarations, because \term{Typed
PLAI} will infer types for you here and in many other cases.
\ru{Фактически вам не нужны эти частные определения типов, так как \term{Typed
PLAI} в этом и других случаях будет сам делать для вас \term{вывод типов}.}
Thus you could just as well have written
\ru{Так что вы можете написать короче}

\lst{src/1/p8_4.rkt}

\noindent
but we prefer to write explicit type declarations as a matter of both discipline
and comprehensibility when we return to programs later.
\ru{но мы предпочтем писать полные объявления типов с точки зрения как
дисциплины, так и усвояемости, когда мы вернемся к программам позже.}

The type names can even be used recursively, as we will see repeatedly in this
book (for instance, section \ref{sec2_4}).
\ru{Имена типов даже могут быть использованы рекурсивно, как мы увидим
несколько позже в этой книге (например в разделе \ref{sec2_4}).}

The language provides a pattern-matcher for use when writing expressions, such
as a function’s body:
\ru{Язык предоставляет pattern-matcher для использования при написании
выражений, таких как тело функции:}

\lst{src/1/p9_1.rkt}

\noindent
In the expression \ru{Например в выражении} (>= humps 2), for instance,
\verb|humps| is the name given to whatever value was given as the argument to
the constructor \ru{имя humps соответствует любому значению, данному как
аргумент для конструктора} \verb|caml|.

Finally, you should write test cases, ideally before you’ve defined your
function, but also afterwards to protect against accidental changes:
\ru{И наконец, вы должны написать тесты, в идеале до того как вы ее реализовали,
или хотя бы после, чтобы защититься от внезапных несоответствий в ее поведении
при внесении изменений в код:}

\lst{src/1/p9_2.rkt}

\noindent
When you run the above program, the language will give you verbose output telling
you both tests passed.
\ru{При запуске тестов язык даст вам подробный отчет, что оба теста успешно
пройдены.}
Read the documentation to learn how to suppress most of these messages.
\ru{Прочитайте документацию, чтобы узнать как подавить вывод большей части
этих сообщений.}

Here’s something important that is obscured above.
\ru{Вот еще кое-что важное, что было неясно выше.}
We’ve used the same name,
\ru{Мы использовали одно и то же имя,}
humps (and height), in both the datatype definition and in the fields of the
patternmatch.
\ru{и в определении типа данных, и в полях объекта при проверке совпадения
шаблонов.}
This is absolutely unnecessary because the two are related by position, not
name.
\ru{Это совершенно необязательно, так как каждая пара связана по позиции, а не
по имени.}
Thus, we could have as well written the function as
\ru{Так что мы могли бы также написать эту функцию как}

\lst{src/1/p9_3.rkt}

\noindent
Because each h is only visible in the case branch in which it is introduced, the
two hs do not in fact clash.
\ru{Так как каждый h виден только в той case-секции, где он используется,
два h фактически не сталкиваются.}
You can therefore use convention and readability to
dictate your choices.
\ru{Таким образом вы можете использовать соглашения по оформлению кода для
улучшения читаемости, и диктовать свой выбор.}
In general, it makes sense to provide a long and
descriptive name when defining the datatype\note{because you probably won’t use
that name again}, but shorter names in the type-case because you’re
likely to use use those names one or more times.
\ru{В общем, имеет смысл использовать длинные описательные имена при определении
типа данных\note{\ru{потому что вы возможно больше никогда не будете
использовать это имя снова}}, и короткие имена в type-case, где они обычно
используются несколько раз.}

I did just say you’re unlikely to use the field descriptors introduced in the
datatype definition, but you can.
\ru{Также я хочу упомянуть декрипторы полей класса, которые вы возможно
захотите использовать.}
The language provides \term{selectors} to extract fields without the need for
pattern-matching: e.g., caml-humps.
\ru{Язык предоставляет \term{селекторы} для получения значений полей без
необходимости использовать pattern-matching, например caml-humps.}
Sometimes,
it’s much easier to use the selector directly rather than go through the
pattern-matcher.
\ru{Иногда намного проще использовать селектор, чем возиться с мэтчингом
шаблонов.}
It often isn’t, as when defining  above, but just to be
clear, let’s write it without pattern-matching:
\ru{Часто это не так, как в случае определения good?, но для ясности давайте
напишем без pattern-matching:}

\lst{src/1/p9_4.rkt}

\DoNow{
What happens if you mis-apply functions to the wrong kinds of values\,?
\ru{Что произойдет, если вы ошибочно примените функции к неправильным типам
значений\,?}

For instance, what if you give the caml constructor a string\,?
\ru{Например, что если вы дадите конструктору caml строковый аргумент\,?}

What if you send a number into each version of good? above\,?
\ru{Что произойдет если вы пошлете число к каждой версии good? описанных
выше\,?} }
