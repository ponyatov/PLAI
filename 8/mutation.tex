\secrel{8 Mutation: Structures and Variables 41}\secdown

It’s time for another

\bigskip
\textbf{Which of these is the same?}
\begin{itemize}
  \item \verb|f = 3|
  \item \verb|o.f = 3|
  \item \verb|f = 3|
\end{itemize}

Assuming all three are in Java, the first and third could behave exactly like
each other or exactly like the second: it all depends on whether f is a local
identifier (such as a parameter) or a field of the object (i.e., the code is
really this.f = 3).

In either case, we are asking the evaluator to permanently change the value
bound to f. This has important implications for other observers. Until now, for
a given set of inputs, a computation always returned the same value. Now, the
answer depends on when it was invoked: above, it depends on whether it was
invoked before or after the value of f was changed. The introduction of time has
profound effects on reasoning about programs.

However, there are really two quite different notions of change buried in the
uniform syntax above. Changing the value of a field (o.f = 3 or this.f = 3) is
extremely different from changing that of an identifier (f = 3 where f is bound
inside the method, not by the object). We will explore these in turn. We’ll
tackle fields below, and return to identifiers in section \ref{8_2_vars}.

\secrel{8.1 Mutable Structures   41}
\secdown
\secrel{8.1.1 A Simple Model of Mutable Structures  41}

Objects are a generalization of structures, as we will soon see \ref{}.
Therefore, fields in objects are a generalization of fields in structures and to
understand mutation, it is mostly (but not entirely! \ref{}) sufficient to
understand mutable objects. To be even more reductionist, we don’t need a
structure to have many fields: a single one will suffice. We call this a box. In
Racket, boxes support just three operations:
\lsts{8/1/1/1.rkt}{rkt}
Thus, box takes a value and wraps it in a mutable container. unbox extracts the
current value inside the container. Finally, set-box! changes the value in the
container, and in a typed language, the new value is expected to be
type-consistent with what was there before. You can thus think of a box as
equivalent to a Java container class with parameterized type, which has a single
member field with a getter and setter: box is the constructor, unbox is the
getter, and set-box! is the setter. (Because there is only one field, its name
is irrelevant.)
\lsts{8/1/1/2.rkt}{rkt}

Because we must sometimes mutate in groups (e.g., removing money from one bank
account and depositing it in another), it is useful to be able to sequence a
group of mutable operations. In Racket, begin lets you write a sequence of
operations; it evaluates them in order and returns the value of the last one.

\Exercise{
Define begin by desugaring into let (and hence into lambda).
}

Even though it is possible to eliminate begin as syntactic sugar, it will prove
extremely useful for understanding how mutation works. Therefore, we will add a
simple, two-term version of sequencing to the core.

\secrel{8.1.2 Scaffolding   42}

First, let’s extend our core language datatype:
\note{This is an excellent illustration of the non-canonical nature of
desguaring. We’ve chosen to add to the core a construct that is certainly not
necessary. If our goal was to shrink the size of the interpreter— perhaps at
some cost to the size of the input program—we would not make this choice. But
our goal in this book is to study pedagogic interpreters, so we choose a larger
language because it is more instructive.}
\lsts{8/1/2/1.rkt}{rkt}
Observe that in a setboxC expression, both the box position and its new value
are expressions. The latter is unsurprising, but the former might be. It means
we can write programs such as this in corresponding Racket:
\lsts{8/1/2/2.rkt}{rkt}
This evaluates to a list of boxes, the first containing 1 and the second 2.
Observe that the first argument to the first set-box! instruction was (first l),
i.e., an expression that evaluated to a box, rather than just a literal box or
an identifier. This is precisely analogous to languages like Java, where one can
(taking some type liberties) write
\note{Your output may look like '(\#\&1 \#\&2). The \#\& notation is Racket’s
abbreviated syntactic prefix for “box”.}
\lsts{8/1/2/3.rkt}{rkt}
Observe that l.get(0) is a compound expression being used to find the
appropriate box, and evaluates to the box object on which set is invoked.

For convenience, we will assume that we have implemented desguaring to provide
us with (a) let and (b) if necessary, more than two terms in a sequence (which
can be desugared into nested sequences). We will also sometimes write
expressions in the original Racket syntax, both for brevity (because the core
language terms can grow quite large and unwieldy) and so that you can run these
same terms in Racket and observe what answers they produce. As this implies, we
are taking the behavior in Racket—which is similar to the behavior in just about
every mainstream language with mutable objects and structures—as the reference
behavior.

\secrel{8.1.3 Interaction with Closures  . . 43}

Consider a simple counter:
\lsts{8/1/3/1.rkt}{rkt}
Every time it is invoked, it produces the next integer:
\lst{8/1/3/1.log}
Why does this work? It’s because the box is created only once, and bound to n,
and then closed over. All subsequent mutations affect the same box. In contrast,
swapping two lines makes a big difference:
\lsts{8/1/3/2.rkt}{rkt}
Observe:
\lst{8/1/3/2.log}
In this case, a new box is allocated on every invocation of the function, so the
answer each time is the same (despite the mutation inside the procedure). Our
implementation of boxes should be certain to preserve this distinction.

The examples above hint at an implementation necessity. Clearly, whatever the
environment closes over in new-loc must refer to the same box each time. Yet
something also needs to make sure that the value in that box is different each
time! Look at it more carefully: it must be lexically the same, but dynamically
different. This distinction will be at the heart of our implementation.

\secrel{8.1.4 Understanding the Interpretation of Boxes . . 44}

Let’s begin by reproducing our current interpreter:
\lst{8/1/4/1.rkt}
Because we’ve introduced a new kind of value, the box, we have to update the set
of values:
\lst{8/1/4/2.rkt}
Two of these cases should be easy. When we’re given a box expression, we simply
evaluate it and return it wrapped in a boxV:
\lst{8/1/4/3.rkt}
Similarly, extracting a value from a box is easy:
\lst{8/1/4/4.rkt}

By now, you should be constructing a healthy set of test cases to make sure
these behave as you’d expect.

Of course, we haven’t done any hard work yet. All the interesting behavior is,
presumably, hidden in the treatment of setboxC. It may therefore surprise you
that we’re going to look at seqC first instead (and you’ll see why we included
it in the core).

Let’s take the most natural implementation of a sequence of two instructions:
\lst{8/1/4/5.rkt}

That is, we evaluate the first term, then the second, and return the result of
the second.

You should immediately spot something troubling. We bound the result of
evaluating the first term, but didn’t subsequently do anything with it. That’s
okay: presumably the first term contained a mutation expression of some sort,
and its value is uninteresting (indeed, note that set-box! returns a void
value). Thus, another implementation might be this:
\lst{8/1/4/6.rkt}

Not only is this slightly dissatisfying in that it just uses the analogous
Racket sequencing construct, it still can’t possibly be right! This can only
work only if the result of the mutation is being stored somewhere. But because
our interpreter only computes values, and does not perform any mutation itself,
any mutations in (interp b1 env) are completely lost. This is obviously not what
we want.

\secrel{8.1.5 Can the Environment Help?  46}

Here is another example that can help:
\lst{8/1/5/1.rkt}
In Racket, this evaluates to 2.

\Exercise{
Represent this expression in ExprC.
}

Let’s consider the evaluation of the inner sequence. In both cases, the
expression (the representation of (set-box! ...)) is exactly identical. Yet
something is changing underneath, because these cause the value of the box to go
from 0 to 2! We can “see” this even more clearly if instead we evaluate
\lst{8/1/5/2.rkt}
which evaluates to 3. Here, the two calls to interp in the rule for addition are
sending exactly the same textual expression in both cases. Yet somehow the
effects from the left branch of the addition are being felt in the right branch,
and we must rule out spukhafte Fernwirkung.

If the interpreter is being given precisely the same expression, how can it
possibly avoid producing precisely the same answer? The most obvious way is if
the interpreter’s other parameter, the environment were somehow different. As of
now the exact same environment is sent to both both branches of the sequence and
both arms of the addition, so our interpreter—which produces the same output
every time on a given input—cannot possibly produce the answers we want.

Here is what we know so far:
\begin{enumerate}[nosep]
  \item 
We must somehow make sure the interpreter is fed different arguments on calls
that are expected to potentially produce different results.
  \item 
We must return from the interpreter some record of the mutations made when
evaluating its argument expression.
\end{enumerate}

Because the expression is what it is, the first point suggests that we might try
to use the environment to reflect the differences between invocations. In turn,
the second point suggests that each invocation of the interpreter should also
return the environment, so it can be passed to the next invocation. Roughly,
then, the type of the interpreter might become:
\lst{8/1/5/3.rkt}
That is, the interpreter consumes an expression and environment; it evaluates in
that environment, updating it as it proceeds; when the expression is done
evaluating, the interpreter returns the answer (as it did before), along with an
updated environment, which in turn is sent to the next invocation of the
interpreter. And the treatment of setboxC would somehow impact the environment
to reflect the mutation.

Before we dive into the implementation, however, we should consider the
consequences of such a change. The environment already serves an important
purpose: it holds deferred substitutions. In that respect, it already has a
precise semantics—given by substitution—and we must be careful to not alter
that. One consequence of its tie to substitution is that it is also the
repository of lexical scope information. If we were to allow the extended
environment escape from one branch of addition and be used in the other, for
instance, consider the impact on the equivalent of the following program:
\lst{8/1/5/4.rkt}
It should be evident that this program has an error: b in the right branch of
the addition is unbound (the scope of the b in the left branch ends with the
closing of the let—if this is not evident, desugar the above expression to use
functions). But the extended environment at the end of interpreting the let
clearly has b bound in it.
\Exercise{
Work out the above problem in detail and make sure you understand it.
}

You could try various other related proposals, but they are likely to all have
similar failings. For instance, you may decide that, because the problem has to
do with additional bindings in the environment, you will instead remove all
added bindings in the returned environment. Sounds attractive? Did you remember
we have closures?
\Exercise{
Consider the representation of the following program:
\lst{8/1/5/5.rkt}
What problems does this example cause?
}

Rather, we should note that while the constraints described above are all valid,
the solution we proposed is not the only one. What we require are the two
conditions enumerated above; observe that neither one actually requires the
environment to be the responsible agent. Indeed, it is quite evident that the
environment cannot be the principal agent.

\secrel{8.1.6 Introducing the Store  . 48}

The preceding discussion tells us that we need two repositories to accompany the
expression, not one. One of them, the environment, continues to be responsible
for maintaining lexical scope. But the environment cannot directly map
identifiers to their value, because the value might change. Instead, something
else needs to be responsible for maintaining the dynamic state of mutated boxes.
This latter data structure is called the store.

Like the environment, the store is a partial map. Its domain could be any
abstract set of names, but it is natural to think of these as numbers, meant to
stand for memory locations. This is because the store in the semantics maps
directly onto (abstracted) physical memory in the machine, which is
traditionally addressed by numbers. Thus the environment maps names to
locations, and the store maps locations to values:
\lsts{8/1/6/1.rkt}{rkt}
We’ll also equip ourselves with a function to look up values in the store, just
as we already have one for the environment (which now returns locations
instead):
\lsts{8/1/6/2.rkt}{rkt}

With this, we can refine our notion of values to the correct one:
\lsts{8/1/6/3.rkt}{rkt}

\Exercise{
Fill in the bodies of lookup and fetch.
}

\secrel{8.1.7 Interpreting Boxes  . . 49}

Now we have something that the environment can return, updated, reflecting
mutations during the evaluation of the expression, without having to change the
environment in any way. Because a function can return only one value, let’s
define a data structure to hold the new result from the interpreter:
\lsts{8/1/7/1.rkt}{rkt}
Thus the interpreter’s type becomes:
\lsts{8/1/7/2.rkt}{rkt}

The easiest one to dispatch is numbers. Remember that we have to return the
store reflecting all mutations that happened while evaluating the given
expression. Because a number is a constant, no mutations could have happened, so
the returned store is the same as the one passed in:
\lsts{8/1/7/3.rkt}{rkt}

A similar argument applies to closure creation; observe that we are speaking of
the creation, not use, of closures:
\lsts{8/1/7/4.rkt}{rkt}

Identifiers are almost as straightforward, though if you are simplistic, you’ll
get a type error that will alert you that to obtain a value, you must now look
up both in the environment and in the store:
\lsts{8/1/7/5.rkt}{rkt}

Notice how lookup and fetch compose to produce the same result that lookup
alone produced before.

Now things get interesting.

Let’s take sequencing. Clearly, we need to interpret the two terms:
\lsts{8/1/7/6.rkt}{rkt}
Oh, but wait. The whole point was to evaluate the second term in the store
returned by the first one—otherwise there would have been no point to all these
changes. Therefore, instead we must evaluate the first term, capture the
resulting store, and use it to evaluate the second. (Evaluating the first term
also yields its value, but sequencing ignores this value and assumes the first
time was run purely for its potential mutations.) We will write this in a
stylized manner:
\lsts{8/1/7/7.rkt}{rkt}

This says to (interp b1 env sto); name the resulting value and store v-b1 and
s-b1, respectively; and evaluate the second term in the store from the first:
(interp b2 env s-b1). The result will be the value and store returned by the
second term, which is what we expect. The fact that the first term’s effect is
only on the store can be read from the code because, though we bind v-b1, we
never subsequently use it.
\DoNow{
Spend a moment contemplating the code above. You’ll soon need to adjust
your eyes to read this pattern fluently.
}

Now let’s move on to the binary arithmetic primitives. These are similar to
sequencing in that they have two sub-terms, but in this case we really do care
about the value from each branch. As usual, we’ll look at only plusC since multC
is virtually identical.
\lsts{8/1/7/8.rkt}{rkt}

Observe that we’ve unfolded the sequencing pattern out another level, so we can
hold on to both results and supply them to num+.

Here’s an important distinction. When we evaluate a term, we usually use the
same environment for all its sub-terms in accordance with the scoping rules of
the language. The environment thus flows in a recursive-descent pattern. In
contrast, the store is threaded: rather than using the same store in all
branches, we take the store from one branch and pass it on to the next, and take
the result and send it back out. This pattern is called store-passing style.

Now the penny drops. We see that store-passing style is our secret ingredient:
it enables the environment to preserve lexical scope while still giving a
binding structure that can reflect changes. Our intution told us that the
environment had to somehow participate in obtaining different results for the
same expression, and we can now see how it does: not directly, by itself
changing, but indirectly, by referring to the store, which updates. Now we only
need to see how the store itself “changes”.

Let’s begin with boxing. To store a value in a box, we have to first allocate a
new place in the store where its value will reside. The value corresponding to a
box will then remember this location, for use in box mutation.
\lsts{8/1/7/9.rkt}{rkt}

\DoNow{
Observe that we have relied above on new-loc, which is itself implemented in
terms of boxes! This is outright cheating. How would you modify the interpreter
so that we no longer need an mutating implementation of new-loc?
}

To eliminate this style of new-loc, the simplest option would be to add yet
another parameter to and return value from the interpreter, which represents the
largest address used so far. Every operation that allocates in the store would
return an incremented address, while all others would return it unchanged. In
other words, this is precisely another application of the store-passing pattern.
Writing the interpreter this way would make it extremely unwieldy and might
obscure the more important use of store-passing for the store itself, which is
why we have not done so. However, it is important to make sure that we can:
that’s what tells us that we are not reliant on boxes to add boxes to the
language.

Now that boxes are recording the location in memory, getting the value
corresponding to them is easy.
\lsts{8/1/7/10.rkt}{rkt}

It’s the same pattern we saw before, where we have to use fetch to obtain the
actual value residing at that location. Note that we are relying on Racket to
halt with an error if the underlying value isn’t actually a boxV; otherwise it
would be dangerous to not check, since this would be tantamount to dereferencing
arbitrary memory (as C programs can, sometimes with disastrous consequences).

Let’s now see how to update the value held in a box. First we have to evaluate
the box expression to obtain a box, and the value expression to obtain the new
value to store in it. The box’s value is going to be a boxV holding a location.

In principle, we want to “change”, or override, the value at that location in
the store. We can do this in two ways.
\begin{enumerate}[nosep]
  \item 
One is to traverse the store, find the old binding for that location, and
replace it with the new one, copying all the other store bindings unchanged.
  \item 
The other, lazier, option is to simply extend the store with a new binding for
that location, which works provided we always obtain the most recent binding for
a location (which is how lookup works in the environment, so fetch presumably
also does in the store).
\end{enumerate}

The code below is written to be independent of these options:
\lsts{8/1/7/11.rkt}{rkt}

However, because we’ve implemented override-store as cons above, we’ve actually
taken the lazier (and slightly riskier, because of its dependence on the
implementation of fetch) option.
\Exercise{
Implement the other version of store alteration, whereby we update an existing
binding and thereby avoid multiple bindings for a location in the store.
}
\Exercise{
When we look for a location to override the value stored at it, can the location
fail to be present? If so, write a program that demonstrates this. If not,
explain what invariant of the interpreter prevents this from happening.
}

Alright, we’re now done with everything other than application! Most of
application should already be familiar: evaluate the function position, evaluate
the argument position, interpret the closure body in an extension of the
closure’s environment...but how do stores interact with this?
\lsts{8/1/7/12.rkt}{rkt}

Let’s start by thinking about extending the closure environment. The name we’re
extending it with is obviously the name of the function’s formal parameter. But
what location do we bind it to? To avoid any confusion with already-used
locations (a confusion we will explicitly introduce later! [REF]), let’s just
allocate a new location. This location is used in the environment, and the value
of the argument resides at this location in the store:
\lsts{8/1/7/13.rkt}{rkt}

Because we have not said the function parameter is mutable, there is no real
need to have implemented procedure calls this way. We could instead have
followed the same strategy as before. Indeed, observe that the mutability of
this location will never be used: only setboxC changes what’s in an existing
store location (the override- store above is technically a store
initialization), and then only when they are referred to by boxVs, but no box is
being allocated above. However, we have chosen to implement application this way
for uniformity, and to reduce the number of cases we’d have to handle.
\note{You could call this the useless app store.}

\Exercise{
It’s a useful exercise to try to limit the use of store locations only to boxes.
How many changes would you need to make?
}

\secrel{8.1.8 The Bigger Picture  . . 54}

Even though we’ve finished the implementation, there are still many subtleties
and insights to discuss.

\begin{enumerate}
  \item 
Implicit in our implementation is a subtle and important decision: the order of
evaluation. For instance, why did we not implement addition thus?  
\lsts{8/1/8/1.rkt}{rkt}
It would have been perfectly consistent to do so. Similarly, embodied in the
pattern of store-passing is the decision to evaluate the function position
before the argument. Observe that:
\begin{itemize}
  \item 
Previously, we delegated such decisions to the underlying language
implementation. Now, store-passing has forced us to sequentialize the
computation, and hence make this decision ourselves (whether we realized it or
not).
  \item
Even more importantly, this decision is now a semantic one. Before there were
mutations, one branch of an addition, for instance, could not affect the value
produced by the other branch. Because each branch can have mutations that impact
the value of the other, we must choose some order so that programmers can
predict what their program is going to do! Being forced to write a store-passing
interpreter has made this clear.
\note{The only effect they could have was halting with an error or failing to
terminate—which, to be sure, are certainly observable effects, but at a much
more gross level. A program would not terminate with two different answers
depending on the order of evaluation.}
\end{itemize}

  \item
Observe that in the application rule, we are passing along the dynamic store,
i.e., the one resulting from evaluating both function and argument. This is
precisely the opposite of what we said to do with the environment. This
distinction is critical. The store is, in effect, “dynamically scoped”, in that
it reflects the history of the computation, not its lexical shape. Because we
are already using the term “scope” to refer to the bindings of identifiers,
however, it would be confusing to say “dynamically scoped” to refer to the
store. Instead, we simply say that it is persistent.

Languages sometimes dangerously conflate these two. In C, for instance, values
bound to local identifiers are allocated (by default) on the stack. However, the
stack matches the environment, and hence disappears upon completion of the call.
If the call, however, returned references to any of these values, these
references are now pointing to unused or even overridden memory: a genuine
source of serious errors in C programs. The problem is that the values
themselves persist; it is only the identifiers that refer to them that have
lexical scope.

  \item
We have already discussed how there are two strategies for overriding the store:
to simply extend it (and rely on fetch to extract the newest one) or to
“searchand- replace”. The latter strategy has the virtue of not holding on to
useless store bindings that will can never be obtained again.

However, this does not cover all the wasted memory. Over time, we cease to be
able to access some boxes entirely: e.g., if they are bound to only one
identifier, and that identifier is no longer in scope. These locations are
called garbage. Thinking more conceptually, garbage locations are those whose
elimination does not have any impact on the value produced by a program. There
are many strategies for identifying and reclaiming garbage locations, usually
called garbage collection \ref{}.
  
  \item
It’s very important to evaluate every expression position and thread the store
that results from it. Consider, for instance, this implementation of unboxC:
\lsts{8/1/8/2.rkt}{rkt}
Did you notice? We fetched the location from sto, not s-a. But sto reflects
mutations up to but before the evaluation of the unboxC expression, not any
within it. Could there possibly be any? Mais oui!
\lsts{8/1/8/3.rkt}{rkt}
With the incorrect code above, this would evaluate to 0 rather than 1.

  \item
Here’s another, similar, error:
\lsts{8/1/8/4.rkt}{rkt}

How do we break this? Well, we’re returning the old store, the one before any
mutations in the unboxC happened. Thus, we just need the outside context to
depend on one of them.
\lsts{8/1/8/5.rkt}{rkt}

This should evaluate to 2, but because the store being returned is one where b’s
location is bound to the representation of 0, the result is 1.

If we combined both bugs above—i.e., using sto twice in the last line instead of
s-a twice—this expression would evaluate to 0 rather than 2.

\Exercise{
Go through the interpreter; replace every reference to an updated store
with a reference to one before update; make sure your test cases catch
all the introduced errors!
}

  \item
Observe that these uses of “old” stores enable us to perform a kind of time
travel: because mutation introduces a notion of time, these enable us to go back
in time to when the mutation had not yet occurred. This sounds both interesting
and perverse; does it have any use?

It does! Imagine that instead of directly mutating the store, we introduce the
idea of a journal of intended updates to the store. The journal flows in a
threaded manner just like the real store itself. Some instruction creates a new
journal; after that, all lookups first check the journal, and only if the
journal cannot find a binding for a location is it looked for in the actual
store. There are two other new instructions: one to discard the journal (i.e.,
perform time travel), and the other to commit it (i.e., all of its edits get
applied to the real store).

This is the essence of software transactional memory. Each thread maintains its
own journal. Thus, one thread does not see the edits made by the other before
committing (because each thread sees only its own journal and the global store,
but not the journals of other threads). At the same time, each thread gets its
own consistent view of the world (it sees edits it made, because these are
recorded in the journal). If the transaction ends successfully, all threads
atomically see the updated global store. If the transaction aborts, the
discarded journal takes with it all changes and the state of the thread reverts
(modulo global changes committed by other threads).

Software transactional memory offers one of the most sensible approaches to
tackling the difficulties of multi-threaded programming, if we insist on
programming with shared mutable state. Because most computers have only one
global store, however, maintaining the journals can be expensive, and much
effort goes into optimizing them. As an alternative, some hardware architectures
have begun to provide direct support for transactional memory by making the
creation, maintenance, and commitment of journals as efficient as using the
global store, removing one important barrier to the adoption of this idea.

\Exercise{
Augment the language with the journal features of software transactional
memory journal.
}

\end{enumerate}

\Exercise{
An alternate implementation strategy is to have the environment map names to
boxed Values. We don’t do it here because it: (a) would be cheating, (b)
wouldn’t tell us how to implement the same feature in a language without boxes,
(c) doesn’t necessarily carry over to other mutation operations, and (d) most of
all, doesn’t really give us insight into what is happening here.

It is nevertheless useful to understand, not least because you may find it a
useful strategy to adopt when implementing your own language. Therefore, alter
the implementation to obey this strategy. Do you still need store-passing style?
Why or why not?
}

\secup

\secrel{8.2 Variables   . . 57}\label{8_2_vars}

Now that we’ve got structure mutation worked out, let’s consider the other case:
variable mutation.

\secdown
\secrel{8.2.1 Terminology   . . 57}

First, our choice of terms. We’ve insisted on using the word “identifier” before
because we wanted to reserve “variable” for what we’re about to study. In Java,
when we say (assuming x is locally bound, e.g., as a method parameter)
\lsts{8/2/1/1.rkt}{java}
we’re asking to change the value of x. After the first assignment, the value of
x is 1; after the second one, it’s 3. Thus, the value of x varies over the
course of the execution of the method.

Now, we also use the term “variable” in mathematics to refer to function
parameters. For instance, in f(y) = y + 3 we say that y is a “variable”. That is
called a variable because it varies across invocations; however, within each
invocation, it has the same value in its scope. Our identifiers until now have
corresponded to this notion of a variable. In contrast, programming variables
can vary even within each invocation, like the Java x above.
\note{If the identifier was bound to a box, then it remained bound to the same
box value. It’s the content of the box that changed, not which box the
identifier was bound to.}

Henceforth, we will use variable when we mean an identifier whose value can
change within its scope, and identifier when this cannot happen. If in doubt, we
might play it safe and use “variable”; if the difference doesn’t really matter,
we might use either one. It is less important to get caught up in these specific
terms than to understand that they represent a distinction that matters \ref{}.

\secrel{8.2.2 Syntax   . . 57}

Whereas other languages overload the mutation syntax (= or :=), in Racket they
are kept distinct: set! is used to mutate variables. This forces Racket
programmers to confront the distinction we introduced at the beginning of
section 8. We will, of course, sidestep these syntactic issues in our core
language by using different constructs for boxes and for variables.

The first thing to note about variable mutation is that, although it too has two subterms
like box mutation (setboxC), its syntax is fundamentally different. To understand
why, let’s return to our Java fragment:
\lsts{8/2/2/2.rkt}{java}
In this setting, we cannot write an arbitrary expression in place of x: we must
literally write the name of the identifier itself. That is because, if it were
an expression position, then we could evaluate it, yielding a value: for
instance, if x were previously bound to 1, this would be tantamout to writing
the following statement:
\lsts{8/2/2/3.rkt}{java}
But this is, of course, nonsensical! We can’t assign a new value to 1, and
indeed 1 is pretty much the definition of immutable. Thus, what we instead want
is to find where x is in the store, and change the value held over there.

Here’s another way to see this. Suppose the local variable o were bound to some
String object; let’s call this object s. Say we write
\lsts{8/2/2/4.rkt}{java}
Are we trying to change s in any way? Certainly not: this statement intends to
leave s alone. It only wants to change the value that o is referring to, so that
subsequent references evaluate to this new string object instead.

\secrel{8.2.3 Interpreting Variables  . 58}

We’ll start by reflecting this in our syntax:
\lsts{8/2/3/5.rkt}{rkt}
Observe that we’ve jettisoned the box operations, but kept sequencing because
it’s handy around mutation. Importantly, we’ve now added the setC case, and its
first subterm is not an expression but the literal name of a variable. We’ve
also renamed idC to varC.

Because we’ve gotten rid of boxes, we can also get rid of the special box
values. When the only kind of mutation you have is variables, you don’t need new
kinds of values.
\lsts{8/2/3/6.rkt}{rkt}

As you might imagine, to support variables we need the same store-passing style
that we’ve seen before (section 8.1.7), and for the same reasons. What differs
is in precisely how we use it. Because sequencing is interpreted in just the
same way (observe that the code for it does not depend on boxes versus
variables), that leaves us just the variable mutation case to handle.

First, we might as well evaluate the value expression and obtain the updated
store:
\lsts{8/2/3/7.rkt}{rkt}

What now? Remember we just said that we don’t want to fully evaluate the
variable, because that would just give the value it is bound to. Instead, we
want to know which memory location it corresponds to, and update what is stored
at that memory location; this latter part is just the same thing we did when
mutating boxes:
\lsts{8/2/3/8.rkt}{rkt}

The very interesting new pattern we have here is this. When we added boxes, in
the idC case, we looked up an identifier in the environment, and immediately
fetched the value at that location from the store; the composition yielded a
value, just as it used to before we added stores. Now, however, we have a new
pattern: looking up an identifier in the environment without subsequently
fetching its value from the store. The result of invoking just lookup is
traditionally called an l-value, for “left-handside (of an assignment) value”.
This is a fancy way of saying “memory address”, and stands in contast to the
actual values that the store yields: observe that it does not directly
correspond to anything in the type Value.

And we’re done! We did all the hard work when we implemented store-passing style
(and also in that application allocated new locations for variables).


\secup

\secrel{8.3 The Design of Stateful Language Operations  . . 59}

Though most programming languages include one or both kinds of state we have
studied, their admission should not be regarded as a trivial or foregone matter.
On the one hand, state brings some vital benefits:

\begin{itemize}
  \item 
State provides a form of modularity. As our very interpreter demonstrates,
without explicit stateful operations, to achieve the same effect:
\begin{itemize}
  \item 
We would need to add explicit parameters and return values that pass the
equivalent of the store around.
  \item
These changes would have to be made to all procedures that may be involved in a
communication path between producers and consumers of state.
\end{itemize}

Thus, a different way to think of state in a programming language is that it is
an implicit parameter already passed to and returned from all procedures,
without imposing that burden on the programmer. This enables procedures to
communicate “at a distance” without all the intermediaries having to be aware of
the communication.

  \item
State makes it possible to construct dynamic, cyclic data structures, or at
least to do so in a relatively straightforward manner \ref{sec9}.

  \item
State gives procedures memory, such as new-loc above. If a procedure could not
remember things for itself, the callers would need to perform the remembering on
its behalf, employing the moral equivalent of store-passing. This is not only
unwieldy, it creates the potential for a caller to interfere with the memory for
its own nefarious purposes (e.g., a caller might purposely send back an old
store, thereby obtaining a reference already granted to some other party,
through which it might launch a correctness or security attack).

\end{itemize}

On the other hand, state imposes real costs on programmers as well as on
programs that process programs (such as compilers). One is “aliasing”, which we
discuss later \ref{}. Another is “referential transparency”, which too I
hope to return to \ref{}. Finally, we have described above how state provides a
form of modularity. However, this same description could be viewed as that of a
back-channel of communication that the intermediaries did not know and could not
monitor. In some (especially security and distributed system) settings, such
back-channels can lead to collusion, and can hence be extremely dangerous and
undesirable.

Because there is no optimal answer, it is probably wise to include mutation
operators but to carefully delinate them. In Standard ML, for instance, there is
no variable mutation, because it is considered unnecessary. Instead, the
language has the equivalent of boxes (called refs). One can easily simulate
variables using boxes (e.g., see new-loc and consider how it would be written
with variables instead), so no expressive power is lost, though it does create
more potential for aliasing than variables alone would have (aliasing
\ref{aliasing}) if the boxes are not used carefully.

In return, however, developers obtain expressive types: every data structure is
considered immutable unless it contains a ref, and the presence of a ref is a
warning to both developers and programs (such as compilers) that the underlying
value may keep changing. Thus, for instance, if b is a box, a developer should
be aware that replacing all instances of (unbox b) with v, where v is bound to
(unbox b), is unwise: the former always fetches the current value in the box,
while the latter may be referring to an older content. (Conversely, if the
developer wants the value at a certain point in time, oblivious to future
mutations to the box, they should be sure to retrieve and bind it rather than
always use unbox.)

\secrel{8.4 Parameter Passing   . 60}

In our current implementation, on every function call, we allocate a fresh
location in the store for the parameter. This means the following program
\lsts{8/4/1.rkt}{rkt}
evaluates to 5, not 3. That is because the value of the formal parameter x is
held at a different location than that of the actual parameter y, so the
mutation affects the location of x, leaving y unscathed.

Now suppose, instead, that application behaved as follows. When the actual
parameter is a variable, and hence has a location in memory, instead of
allocating a new location for the value, it simply passes along the existing one
for the variable. Now the formal parameter is referring to the same store
location as the actual: i.e., they are variable aliases. Thus any mutation on
the formal will leak back out into the calling context; the above program would
evaluate to 3 rather than 5. These is called a call-by-reference
parameter-passing strategy.
\note{Instead, our interpreter implements call-by-value, and this is the same
strategy followed by languages like Java. This causes confusion because when the
value is itself mutable, changes made to the value in the callee are observed by
the caller. However, that is simply an artifact of mutable values, not of the
calling strategy. Please avoid this confusion!}

For some years, this power was considered a good idea. It was useful because
programmers could write abstractions such as swap, which swaps the value of two
variables in the caller. However, the disadvantages greatly outweigh the
advantages:
\begin{itemize}
  \item 
A careless programmer can alias a variable in the caller and modify it without
realizing they have done so, and the caller may not even realize this has
happened until some obscure condition triggers it.
  \item 
Some people thought this was necessary for efficiency: they assumed the
alternative was to copy large data structures. However, call-by-value is
compatible with passing just the address of the data structure. You only need
make a copy if (a) the data structure is mutable, (b) you do not want the caller
to be able to mutate it, and (c) the language does not itself provide
immutability annotations or other mechanisms.
  \item 
It can force non-uniform and hence non-modular reasoning. For instance, suppose
we have the procedure:
\lsts{8/4/2.rkt}{rkt}
If the language were to permit by-reference parameter passing, then the
programmer cannot locally—i.e., just from the above code—determine what the
value of x will be in the ellipses.
\end{itemize}

At the very least, then, if the language is going to permit by-reference
parameters, it should let the caller determine whether to pass the
reference—i.e., let the callee share the memory address of the caller’s
variable—or not. However, even this option is not quite as attractive as it may
sound, because now the callee faces a symmetric problem, not knowing whether its
parameters are aliased or not. In traditional, sequential programs this is less
of a concern, but if the procedure is reentrant, the callee faces precisely the
same predicaments.

At some point, therefore, we should consider whether any of this fuss is
worthwhile. Instead, callers who want the callee to perform a mutation could
simply send a boxed value to the callee. The box signals that the caller
accepts—indeed, invites—the callee to perform a mutation, and the caller can
extract the value when it’s done. This does obviate the ability to write a
simple swapper, but that’s a small price to pay for genuine software engineering
concerns.

\secup
