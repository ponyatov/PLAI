\secrel{8.2.2 Syntax   . . 57}

Whereas other languages overload the mutation syntax (= or :=), in Racket they
are kept distinct: set! is used to mutate variables. This forces Racket
programmers to confront the distinction we introduced at the beginning of
section 8. We will, of course, sidestep these syntactic issues in our core
language by using different constructs for boxes and for variables.

The first thing to note about variable mutation is that, although it too has two subterms
like box mutation (setboxC), its syntax is fundamentally different. To understand
why, let’s return to our Java fragment:
\lsts{8/2/2/2.rkt}{java}
In this setting, we cannot write an arbitrary expression in place of x: we must
literally write the name of the identifier itself. That is because, if it were
an expression position, then we could evaluate it, yielding a value: for
instance, if x were previously bound to 1, this would be tantamout to writing
the following statement:
\lsts{8/2/2/3.rkt}{java}
But this is, of course, nonsensical! We can’t assign a new value to 1, and
indeed 1 is pretty much the definition of immutable. Thus, what we instead want
is to find where x is in the store, and change the value held over there.

Here’s another way to see this. Suppose the local variable o were bound to some
String object; let’s call this object s. Say we write
\lsts{8/2/2/4.rkt}{java}
Are we trying to change s in any way? Certainly not: this statement intends to
leave s alone. It only wants to change the value that o is referring to, so that
subsequent references evaluate to this new string object instead.
