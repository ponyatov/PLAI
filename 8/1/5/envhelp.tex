\secrel{8.1.5 Can the Environment Help?  46}

Here is another example that can help:
\lst{8/1/5/1.rkt}
In Racket, this evaluates to 2.

\Exercise{
Represent this expression in ExprC.
}

Let’s consider the evaluation of the inner sequence. In both cases, the
expression (the representation of (set-box! ...)) is exactly identical. Yet
something is changing underneath, because these cause the value of the box to go
from 0 to 2! We can “see” this even more clearly if instead we evaluate
\lst{8/1/5/2.rkt}
which evaluates to 3. Here, the two calls to interp in the rule for addition are
sending exactly the same textual expression in both cases. Yet somehow the
effects from the left branch of the addition are being felt in the right branch,
and we must rule out spukhafte Fernwirkung.

If the interpreter is being given precisely the same expression, how can it
possibly avoid producing precisely the same answer? The most obvious way is if
the interpreter’s other parameter, the environment were somehow different. As of
now the exact same environment is sent to both both branches of the sequence and
both arms of the addition, so our interpreter—which produces the same output
every time on a given input—cannot possibly produce the answers we want.

Here is what we know so far:
\begin{enumerate}[nosep]
  \item 
We must somehow make sure the interpreter is fed different arguments on calls
that are expected to potentially produce different results.
  \item 
We must return from the interpreter some record of the mutations made when
evaluating its argument expression.
\end{enumerate}

Because the expression is what it is, the first point suggests that we might try
to use the environment to reflect the differences between invocations. In turn,
the second point suggests that each invocation of the interpreter should also
return the environment, so it can be passed to the next invocation. Roughly,
then, the type of the interpreter might become:
\lst{8/1/5/3.rkt}
That is, the interpreter consumes an expression and environment; it evaluates in
that environment, updating it as it proceeds; when the expression is done
evaluating, the interpreter returns the answer (as it did before), along with an
updated environment, which in turn is sent to the next invocation of the
interpreter. And the treatment of setboxC would somehow impact the environment
to reflect the mutation.

Before we dive into the implementation, however, we should consider the
consequences of such a change. The environment already serves an important
purpose: it holds deferred substitutions. In that respect, it already has a
precise semantics—given by substitution—and we must be careful to not alter
that. One consequence of its tie to substitution is that it is also the
repository of lexical scope information. If we were to allow the extended
environment escape from one branch of addition and be used in the other, for
instance, consider the impact on the equivalent of the following program:
\lst{8/1/5/4.rkt}
It should be evident that this program has an error: b in the right branch of
the addition is unbound (the scope of the b in the left branch ends with the
closing of the let—if this is not evident, desugar the above expression to use
functions). But the extended environment at the end of interpreting the let
clearly has b bound in it.
\Exercise{
Work out the above problem in detail and make sure you understand it.
}

You could try various other related proposals, but they are likely to all have
similar failings. For instance, you may decide that, because the problem has to
do with additional bindings in the environment, you will instead remove all
added bindings in the returned environment. Sounds attractive? Did you remember
we have closures?
\Exercise{
Consider the representation of the following program:
\lst{8/1/5/5.rkt}
What problems does this example cause?
}

Rather, we should note that while the constraints described above are all valid,
the solution we proposed is not the only one. What we require are the two
conditions enumerated above; observe that neither one actually requires the
environment to be the responsible agent. Indeed, it is quite evident that the
environment cannot be the principal agent.
