\secrel{8.1.4 Understanding the Interpretation of Boxes . . 44}

Let’s begin by reproducing our current interpreter:
\lst{8/1/4/1.rkt}
Because we’ve introduced a new kind of value, the box, we have to update the set
of values:
\lst{8/1/4/2.rkt}
Two of these cases should be easy. When we’re given a box expression, we simply
evaluate it and return it wrapped in a boxV:
\lst{8/1/4/3.rkt}
Similarly, extracting a value from a box is easy:
\lst{8/1/4/4.rkt}

By now, you should be constructing a healthy set of test cases to make sure
these behave as you’d expect.

Of course, we haven’t done any hard work yet. All the interesting behavior is,
presumably, hidden in the treatment of setboxC. It may therefore surprise you
that we’re going to look at seqC first instead (and you’ll see why we included
it in the core).

Let’s take the most natural implementation of a sequence of two instructions:
\lst{8/1/4/5.rkt}

That is, we evaluate the first term, then the second, and return the result of
the second.

You should immediately spot something troubling. We bound the result of
evaluating the first term, but didn’t subsequently do anything with it. That’s
okay: presumably the first term contained a mutation expression of some sort,
and its value is uninteresting (indeed, note that set-box! returns a void
value). Thus, another implementation might be this:
\lst{8/1/4/6.rkt}

Not only is this slightly dissatisfying in that it just uses the analogous
Racket sequencing construct, it still can’t possibly be right! This can only
work only if the result of the mutation is being stored somewhere. But because
our interpreter only computes values, and does not perform any mutation itself,
any mutations in (interp b1 env) are completely lost. This is obviously not what
we want.
