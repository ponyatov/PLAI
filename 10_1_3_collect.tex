\secrel{10.1.3 Objects as Named Collections  . . 69}

Let’s begin by reproducing the object language we had above. An object is just a
value that dispatches on a given name. For simplicity, we’ll use lambda to
represent the object and case to implement the dispatching.
\note{Observe that basic objects are a generalization of lambda to have multiple
“entry-points”. Conversely, a lambda is an object with just one entry-point, so
it doesn’t need a “method name” to disambiguate.}
\lsts{src/10/1/3/1.rkt}{rkt}

This is the same object we defined earlier, and we use its method in the same
way:
\lsts{src/10/1/3/2.rkt}{rkt}

Of course, writing method invocations with these nested function calls is
unwieldy (and is about to become even more so), so we’d be best off equipping
ourselves with a convenient syntax for invoking methods—the same one we saw
earlier (msgS), but here we can simply define it as a function:
\note{We’ve taken advantage of Racket’s variable-arity syntax: . a says “bind
all the remaining—zero or more—arguments to a list named a”. apply “splices” in
such lists of arguments to call functions.}
\lsts{src/10/1/3/3.rkt}{rkt}
This enables us to rewrite our test:
\lsts{src/10/1/3/4.rkt}{rkt}

\DoNow{
Something very important changed when we switched to the desguaring
strategy. Do you see what it is?
}

Recall the syntax definition we had earlier:
\lsts{src/10/1/3/5.rkt}{rkt}
The “name” position of a message was very explicitly a symbol. That is, the
developer had to write the literal name of the symbol there. In our desugared
version, the name position is just an expression that must evaluate to a symbol;
for instance, one could have written
\lsts{src/10/1/3/6.rkt}{rkt}
This is a general problem with desugaring: the target language may allow
expressions that have no counterpart in the source, and hence cannot be mapped
back to it. Fortunately we don’t often need to perform this inverse mapping,
though it does arise in some debugging and program comprehension tools. More
subtly, however, we must ensure that the target language does not produce values
that have no corresponding equivalent in the source.

Now that we have basic objects, let’s start adding the kinds of features we’ve
come to expect from most object systems.