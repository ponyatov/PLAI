\secrel{13.4.1 A First Attempt   . 95}

Let’s try a first version of or:
\lsts{13/4/1/1.rkt}{rkt}
It says that we can provide any number of sub-terms (more on this in a moment).
Expansion rewrites this into a conditional that tests the first sub-term; if
this is a true value it returns that value (more on this in a moment!),
otherwise it is the disjunction of the remaining terms.

Let’s try this on a simple example. We would expect this to evaluate to true,
but instead:
\lsts{13/4/1/2.rkt}{rkt}
What happened? This expression turned into
\lsts{13/4/1/3.rkt}{rkt}
which in turn expanded into
\lsts{13/4/1/4.rkt}{rkt}
for which there is no definition. That’s because the pattern e0 e1 ... means one
or more sub-terms, but we ignored the case when there are zero. What happens
when there are no sub-terms? The identity for disjunction is falsehood.

\Exercise{
Why is \#f the right default?
}

By filling it in below, we illustrate a macro that has more than one rule. Macro
rules are matched sequentially, so we should be sure to put the most specific
rules first, lest they get overridden by more general ones (though in this
particular case, the two rules are non-overlapping). This yields our improved
macro:
\lsts{13/4/1/5.rkt}{rkt}
which now expands as we expect. In what follows, we will find it useful to have
a special case when there is only a single sub-term:
\lsts{13/4/1/6.rkt}{rkt}
This keeps the output of expansion more concise, which we will find useful
below.
\note{Observe that in this version of the macro, the patterns are not disjoint:
the third (one-or-more sub-terms) subsumes the second (one sub-term).
Therefore, it is essential that the second rule not swap with the third.}

