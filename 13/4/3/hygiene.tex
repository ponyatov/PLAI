\secrel{13.4.3 Hygiene   . . 98}

Hopefully now you’re nervous about something else.
\DoNow{
What?
}

Consider the macro \verb|(let ([v #t]) (my-or-4 #f v))|. What would we expect
this to compute? Naturally, \#t: the first branch is \#f but the second is v,
which is bound to \#t. But let’s look at the expansion:
\lsts{13/4/3/1.rkt}{rkt}
This expression, when run directly, evaluates to \verb|#f|. However,
\verb|(let ([v #t]) (my- or-4 #f v))| evaluates to \verb|#t|. In other words,
the macro seems to magically produce the right value: the names of identifiers
chosen in the macro seem to be independent of those introduced by the macro!
This is unsurprising when it happens in a function; the macro expander enjoys a
property called hygiene that gives it the same property.

One way to think about hygiene is that it effectively automatically renames all
bound identifiers. That is, it’s as if the program expands as follows:
\lsts{13/4/3/2.rkt}{rkt}
turns into
\lsts{13/4/3/3.rkt}{rkt}
(notice the consistent renaming of v to v1), which turns into
\lsts{13/4/3/4.rkt}{rkt}
which, after renaming, becomes
\lsts{13/4/3/5.rkt}{rkt}
when expansion terminates. Observe that each of the programs above, if run
directly, will produce the correct answer.
