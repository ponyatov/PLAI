\secrel{13.1 A First Example   . . 91}

Remember that in \ref{}\ we added let as syntactic sugar over lambda. The
pattern we followed was this:
\note{DrRacket has a very useful tool called the Macro Stepper, which shows the
step-by-step expansion of programs. You should try all the examples in this
chapter using the Macro Stepper. For now, however, you should run them in \#lang
plai rather than \#lang plai-typed.}
\lsts{src/13/1/1.rkt}{rkt}
is transformed into
\lsts{src/13/1/2.rkt}{rkt}

\DoNow{
If this doesn’t sound familiar, now would be a good time to refresh your
memory of why this works.
}

The simplest way of describing this transformation would be to state it
directly: to write, somehow,
\lsts{src/13/1/3.rkt}{rkt}
In fact, this is almost precisely what Racket enables you to do.
\note{We’ll use the name my-let instead of let Because the latter is already
defined in Racket.}
\lsts{src/13/1/4.rkt}{rkt}
syntax-rules tells Racket that whenever it sees an expression with my-let-1
immediately after the opening parenthesis, it should check that it follows the
pattern (my- let-1 (var val) body). The var, val and body are syntactic
variables: they are variables that stand for bodies of code. In particular, they
match whatever s-expression is in that position. If the expression matches the
pattern, then the syntactic variables are bound to the corresponding parts of
the expression, and become available for use in the right-hand side.
\note{You may have noticed some additional syntax, such as (). We’ll explain
this later.}

The right-hand side—in this case, ((lambda (var) body) val)—is the output
generated. Each of the syntactic variables are replaced with the corresponding
parts of the input using our old friend, substitution. This substitution is
utterly simplistic: it makes no attempt to. Thus, if we were to try using it
with
\lsts{src/13/1/5.rkt}{rkt}
Racket would not initially complain that 3 is provided in an identifier
position; rather, it would let the identifier percolate through, desugaring this
into
\lsts{src/13/1/6.rkt}{rkt}
which in turn produces an error:
\lst{src/13/1/6.err}

This immediately tells us that the desugaring process is straightforward in its
function: it doesn’t attempt to guess or be clever, but instead simply rewrites
while substituting. The output is an expression that is again subject to
desugaring.

As a matter of terminology, this simple form of expression-rewriting is often
called a macro, as we mentioned earlier in \ref{}. Traditionally this form of
desugaring is called macro expansion, though this term is misleading because the
output of desugaring can be smaller than the input (though it is usually
larger).

Of course, in Racket, a let can bind multiple identifiers, not just one. If we
were to write this informally, say on a board, we might write something like
\verb|(let ([var val] ...) body) -> ((lambda (var ...) body) val ...)| with the
\ldots meaning “zero or more”, and the intent being that the var ... in the
output should correspond to the sequence of vars in the input. Again, this is
almost precisely Racket syntax:
\lsts{src/13/1/7.rkt}{rkt}

Observe the power of the \ldots\ notation: the sequence of “pairs” in the input
is turned into a pair of sequences in the output; put differently, Racket
“unzips” the input sequence. Conversely, this same notation can be used to zip
together sequences.
