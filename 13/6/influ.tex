\secrel{13.6 Influence on Compiler Design  101}

The use of macros in a language’s definition has an impact on all tools,
especially compilers. As a working example, consider let. let has the virtue
that it can be compiled efficiently, by just extending the current environment.
In contrast, the expansion of let into function application results in a much
more expensive operation: the creation of a closure and its application to the
argument, achieving effectively the same result but at the cost of more time
(and often space).

This would seem to be an argument against using the macro. However, a smart
compiler recognizes that this pattern occurs often, and instead internally
effectively converts left-left-lambda \ref{}\ back into the equivalent of let.
This has two advantages. First, it means the language designer can freely use
macros to obtain a smaller core language, rather than having to trade that off
against the execution cost.

It has a second, much subtler, advantage. Because the compiler recognizes this
pattern, other macros can also exploit it and obtain the same optimization; they
don’t need to contort their output to insert let terms if the left-left-lambda
pattern occurs naturally, as they would have to do otherwise. For instance, the
left-left-lambda pattern occurs naturally when writing certain kinds of
pattern-matchers, but it would take an extra step to convert this into a let in
the expansion—which is no longer necessary.
