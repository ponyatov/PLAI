\secrel{13.5 Identifier Capture   . 99}

Hygienic macros address a routine and important pain that creators of syntactic
sugar confront. On rare instances, however, a developer wants to intentionally
break hygiene. Returning to objects, consider this input program:
\lsts{src/13/5/1.rkt}{rkt}
What does the macro look like? Here’s an obvious candidate:
\lsts{src/13/5/2.rkt}{rkt}
Unfortunately, this macro produces the following error:
\lst{src/13/5/2.err}
which is referring to the self in the body of the method bound to first.

\Exercise{
Work through the hygienic expansion process to understand why error is the
expected outcome.
}

Before we jump to the richer macro, let’s consider a variant of the input term
that makes the binding explicit:
\lsts{src/13/5/3.rkt}{rkt}
The corresponding macro is a small variation on what we had before:
\lsts{src/13/5/4.rkt}{rkt}
This macro expands without difficulty.

\Exercise{
Work through the expansion of this version and see what’s different.
}

This offers a critical insight: had the identifier that goes in the binding
position been written by the macro user, there would have been no problem.
Therefore, we want to be able to pretend that the introduced identifier was
written by the user. The function datum->syntax converts the s-expression in its
second argument; its first argument is which syntax to pretend it was a part of
(in our case, the original macro use, which is bound to x). To introduce the
result into the environment used for expansion, we use with-syntax to bind it in
that environment:
\lsts{src/13/5/5.rkt}{rkt}

With this, we can go back to having self be implicit:
\lsts{src/13/5/6.rkt}{rkt}
