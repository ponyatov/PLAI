\secrel{13.3 Guards   . 95}

Now we can return to the problem that originally motivated the introduction of
syntax- case: ensuring that the binding position of a my-let-3 is syntactically
an identifier. For this, you need to know one new feature of syntax-case: each
rewriting rule can have two parts (as above), or three. If there are three
present, the middle one is treated as a guard: a predicate that must evaluate to
true for expansion to proceed rather than signal a syntax error. Especially
useful in this context is the predicate identifier?, which determines whether a
syntax object is syntactically an identifier (or variable).

\DoNow{
Write the guard and rewrite the rule to incorporate it.
}

Hopefully you stumbled on a subtlety: the argument to identifier? is of type
syntax. It needs to refer to the actual fragment of syntax bound to var. Recall
that var is bound in the syntax space, and \#' substitutes identifiers bound
there. Therefore, the correct way to write the guard is:
\lsts{13/3/1.rkt}{rkt}
With this information, we can now write the entire rule:
\lsts{13/3/2.rkt}{rkt}

\DoNow{
Now that you have a guarded rule definition, try to use the macro with a
non-identifier in the binding position and see what happens.
}
