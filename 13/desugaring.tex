\secrel{13 Desugaring as a Language Feature 91}\secdown

We have thus far extensively discussed and relied upon desugaring, but our
current desguaring mechanism have been weak. We have actually used desugaring in
two different ways. One, we have used it to shrink the language: to take a large
language and distill it down to its core \ref{}. But we have also used it to
grow the language: to take an existing language and add new features to it
\ref{}. This just shows that desugaring is a tremendously useful feature to
have. Indeed, it is so useful that we might ask two questions:
\begin{itemize}
  \item 
Because we create languages to simplify the creation of common tasks, what would
a language designed to support desugaring look like? Note that by “look” we
don’t mean only syntax but also its key behavioral properties.
  \item 
Given that general-purpose languages are often used as a target for desugaring,
why don’t they offer desugaring capabilities in the language itself ? For
instance, this might mean extending a base language with the additional language
that is the response to the previous question.
\end{itemize}
We are going to explore the answer to both questions simultaneously, by studying
the facilities provided for this by \racket.

\secrel{13.1 A First Example   . . 91}

Remember that in \ref{}\ we added let as syntactic sugar over lambda. The
pattern we followed was this:
\note{DrRacket has a very useful tool called the Macro Stepper, which shows the
step-by-step expansion of programs. You should try all the examples in this
chapter using the Macro Stepper. For now, however, you should run them in \#lang
plai rather than \#lang plai-typed.}
\lsts{13/1/1.rkt}{rkt}
is transformed into
\lsts{13/1/2.rkt}{rkt}

\DoNow{
If this doesn’t sound familiar, now would be a good time to refresh your
memory of why this works.
}

The simplest way of describing this transformation would be to state it
directly: to write, somehow,
\lsts{13/1/3.rkt}{rkt}
In fact, this is almost precisely what Racket enables you to do.
\note{We’ll use the name my-let instead of let Because the latter is already
defined in Racket.}
\lsts{13/1/4.rkt}{rkt}
syntax-rules tells Racket that whenever it sees an expression with my-let-1
immediately after the opening parenthesis, it should check that it follows the
pattern (my- let-1 (var val) body). The var, val and body are syntactic
variables: they are variables that stand for bodies of code. In particular, they
match whatever s-expression is in that position. If the expression matches the
pattern, then the syntactic variables are bound to the corresponding parts of
the expression, and become available for use in the right-hand side.
\note{You may have noticed some additional syntax, such as (). We’ll explain
this later.}

The right-hand side—in this case, ((lambda (var) body) val)—is the output
generated. Each of the syntactic variables are replaced with the corresponding
parts of the input using our old friend, substitution. This substitution is
utterly simplistic: it makes no attempt to. Thus, if we were to try using it
with
\lsts{13/1/5.rkt}{rkt}
Racket would not initially complain that 3 is provided in an identifier
position; rather, it would let the identifier percolate through, desugaring this
into
\lsts{13/1/6.rkt}{rkt}
which in turn produces an error:
\lst{13/1/6.err}

This immediately tells us that the desugaring process is straightforward in its
function: it doesn’t attempt to guess or be clever, but instead simply rewrites
while substituting. The output is an expression that is again subject to
desugaring.

As a matter of terminology, this simple form of expression-rewriting is often
called a macro, as we mentioned earlier in \ref{}. Traditionally this form of
desugaring is called macro expansion, though this term is misleading because the
output of desugaring can be smaller than the input (though it is usually
larger).

Of course, in Racket, a let can bind multiple identifiers, not just one. If we
were to write this informally, say on a board, we might write something like
\verb|(let ([var val] ...) body) -> ((lambda (var ...) body) val ...)| with the
\ldots meaning “zero or more”, and the intent being that the var ... in the
output should correspond to the sequence of vars in the input. Again, this is
almost precisely Racket syntax:
\lsts{13/1/7.rkt}{rkt}

Observe the power of the \ldots\ notation: the sequence of “pairs” in the input
is turned into a pair of sequences in the output; put differently, Racket
“unzips” the input sequence. Conversely, this same notation can be used to zip
together sequences.

\secrel{13.2 Syntax Transformers as Functions  . 93}

Earlier we saw that my-let-1 does not even attempt to ensure that the syntax in
the identifier position is truly (i.e., syntactically) an identifier. We cannot
remedy that with the syntax-rules mechanism, but we can with a much more
powerful mechanism called syntax-case. Because syntax-case exhibits many other
useful features as well, we’ll introduce it and then grow it gradually.

The first thing to understand is that a macro is actually a function. It is not,
however, a function from regular run-time values to other run-time values, but
rather a function from syntax to syntax. These functions execute in a world
whose purpose is to create the program to execute. Observe that we’re talking
about the program to execute: the actual execution of the program may only occur
much later (or never at all). This point is actually extremely clear when we
examine desugaring, which is very explicitly a function from (one kind of)
syntax to (another kind of) syntax. This is perhaps obscured above in two ways:
\begin{itemize}
  \item 
The notation of syntax-rules, with no explicit parameter name or other “function
header”, may not make clear that it is a functional transformation (though the
rewriting rule format does allude to this fact).
  \item 
In desugaring, we specify one atomic function for the entire process. Here, we
are actually writing several little functions, one for each kind of new
syntactic construct (such as my-let-1), and these pieces are woven together by
an invisible function that controls the overall rewriting process. (As a
concrete example, it is not inherently clear that the output of a macro is
expanded further—though a simple example immediately demonstrates that this is
indeed the case.)
\end{itemize}

\Exercise{
Write one or more macros to confirm that the output of a macro is expanded
further.
}

There is one more subtlety. Because the form of a macro looks rather like Racket
code, it is not immediately clear that it “lives in another world”. In the
abstract, it may be helpful to imagine that the macro definitions are actually
written in an entirely different language that processes only syntax. This
simplicity is, however, misleading. In practice, program transformers—also
called compilers—are full-blown programs, too, and need all the power of
ordinary programs. This would have necessitated the creation of a parallel
language purely for processing programs. This would be wasteful and pointless;
therefore, Racket instead endows syntax-transforming programs with the full
power of Racket itself.

With that prelude, let’s now introduce syntax-case. We’ll begin by simply
rewriting my-let-1 (under the name my-let-3) using this new notation. First, we
have to write a header for the definition; notice already the explicit
parameter:
\lsts{13/2/1.rkt}{rkt}

This binds x to the entire (my-let-3 ...) expression.

As you might imagine, define-syntax simply tells Racket you’re about to define a
new macro. It does not pick precisely how you want to implement it, leaving you
free to use any mechanism that’s convenient. Earlier we used syntax-rules; now
we’re going to use syntax-case. In particular, syntax-case needs to explicitly
be given access to the expression to pattern-match:
\lsts{13/2/2.rkt}{rkt}

Now we’re ready to express the rewrite we wanted. Previously a rewriting rule
had two parts: the structure of the input and the corresponding output. The same
holds here. The first (matching the input) is the same as before, but the second
(the output) is a little different:
\lsts{13/2/3.rkt}{rkt}

Observe the crucial extra characters: \#'. Let’s examine what that means.

In syntax-rules, the entire output part simply specifies the structure of the
output. In contrast, because syntax-case is laying bare the functional nature of
transformation, the output part is in fact an arbitrary expression that may
perform any computations it wishes. It must simply evaluate to a piece of
syntax.

Syntax is actually a distinct datatype. As with any distinct dataype, it has its
own rules for construction. Concretely, we construct syntax values by writing
\#'; the following s-expression is treated as a syntax value. (In case you were
wondering, the x bound in the macro definition above is also of this datatype.)

The syntax constructor, \#', enjoys a special property. Inside the output part
of the macro, all syntax variables in the input are automatically bound, and
replaced on occurrence. As a result, when the expander encounters var in the
output, say, it replaces var with the corresponding part of the input expression.

\DoNow{
Remove the \#' and try using the above macro definition. What happens?
}

So far, syntax-case merely appears to be a more complicated form of syntax-
rules: perhaps slightly better in that it more cleanly delineates the functional
nature of expansion, and the type of output, but otherwise simply more unwieldy.
As we will see, however, it also offers significant power.

\Exercise{
syntax-rules can actually be expressed as a macro over syntax-case.
Define it.
}

\secrel{13.3 Guards   . 95}

Now we can return to the problem that originally motivated the introduction of
syntax- case: ensuring that the binding position of a my-let-3 is syntactically
an identifier. For this, you need to know one new feature of syntax-case: each
rewriting rule can have two parts (as above), or three. If there are three
present, the middle one is treated as a guard: a predicate that must evaluate to
true for expansion to proceed rather than signal a syntax error. Especially
useful in this context is the predicate identifier?, which determines whether a
syntax object is syntactically an identifier (or variable).

\DoNow{
Write the guard and rewrite the rule to incorporate it.
}

Hopefully you stumbled on a subtlety: the argument to identifier? is of type
syntax. It needs to refer to the actual fragment of syntax bound to var. Recall
that var is bound in the syntax space, and \#' substitutes identifiers bound
there. Therefore, the correct way to write the guard is:
\lsts{13/3/1.rkt}{rkt}
With this information, we can now write the entire rule:
\lsts{13/3/2.rkt}{rkt}

\DoNow{
Now that you have a guarded rule definition, try to use the macro with a
non-identifier in the binding position and see what happens.
}

\secrel{13.4 Or: A Simple Macro with Many Features  95}\secdown

Consider or, which implements disjunction. It is natural, with prefix syntax, to
allow or to have an arbitrary number of sub-terms. We expand or into nested
conditionals that determine the truth of the expression.

\secrel{13.4.1 A First Attempt   . 95}

Let’s try a first version of or:
\lsts{13/4/1/1.rkt}{rkt}
It says that we can provide any number of sub-terms (more on this in a moment).
Expansion rewrites this into a conditional that tests the first sub-term; if
this is a true value it returns that value (more on this in a moment!),
otherwise it is the disjunction of the remaining terms.

Let’s try this on a simple example. We would expect this to evaluate to true,
but instead:
\lsts{13/4/1/2.rkt}{rkt}
What happened? This expression turned into
\lsts{13/4/1/3.rkt}{rkt}
which in turn expanded into
\lsts{13/4/1/4.rkt}{rkt}
for which there is no definition. That’s because the pattern e0 e1 ... means one
or more sub-terms, but we ignored the case when there are zero. What happens
when there are no sub-terms? The identity for disjunction is falsehood.

\Exercise{
Why is \#f the right default?
}

By filling it in below, we illustrate a macro that has more than one rule. Macro
rules are matched sequentially, so we should be sure to put the most specific
rules first, lest they get overridden by more general ones (though in this
particular case, the two rules are non-overlapping). This yields our improved
macro:
\lsts{13/4/1/5.rkt}{rkt}
which now expands as we expect. In what follows, we will find it useful to have
a special case when there is only a single sub-term:
\lsts{13/4/1/6.rkt}{rkt}
This keeps the output of expansion more concise, which we will find useful
below.
\note{Observe that in this version of the macro, the patterns are not disjoint:
the third (one-or-more sub-terms) subsumes the second (one sub-term).
Therefore, it is essential that the second rule not swap with the third.}


\secrel{13.4.2 Guarding Evaluation  . 97}

We said above that this expands as we expect. Or does it? Let’s try the
following example:
\lsts{13/4/2/1.rkt}{rkt}
Observe that or returns the actual value of the first “truthy” value, so the
developer can use it in further computations. Therefore, this returns the value
of init. What do we expect it to be? Naturally, because we’ve negated the value
of init once, we expect it to be \#t. But evaluating it produces \#f!
\note{This problem is not an artifact of set!.
If instead of internal mutation we had, say, printed output, the printing would
have occurred twice.}

To understand why, we have to examine the expanded code. It is this:
\lsts{13/4/2/2.rkt}{rkt}
Aha! Because we’ve written the output pattern as
\lsts{13/4/2/3.rkt}{rkt}
This looked entirely benign when we first wrote it, but it illustrates a very
important principle when writing macros (or indeed any other program
transformation systems): do not copy code! In our setting, a syntactic variable
should never be repeated; if you need to repeat it in a way that might cause
multiple execution of that code, make sure you have considered the consequences
of this. Alternatively, if you meant to work with the value of the expression,
bind it once and use the bound identifier’s name subsequently. This is easy to
demonstrate:
\lsts{13/4/2/4.rkt}{rkt}
This pattern of introducing a binding creates a new potential problem: you may
end up evaluating expressions that weren’t necessary. In fact, it creates a
second, even more subtle one: even if it going to be evaluated, you may evaluate
it in the wrong context! Therefore, you have to reason carefully about whether
an expression will be evaluated, and if so, evaluate it once in just the right
place, then store that value for subsequent use.

When we repeat our previous example, that contained the set!, with my-or-4, we
see that the result is \#t, as we would have hoped.

\secrel{13.4.3 Hygiene   . . 98}

Hopefully now you’re nervous about something else.
\DoNow{
What?
}

Consider the macro \verb|(let ([v #t]) (my-or-4 #f v))|. What would we expect
this to compute? Naturally, \#t: the first branch is \#f but the second is v,
which is bound to \#t. But let’s look at the expansion:
\lsts{13/4/3/1.rkt}{rkt}
This expression, when run directly, evaluates to \verb|#f|. However,
\verb|(let ([v #t]) (my- or-4 #f v))| evaluates to \verb|#t|. In other words,
the macro seems to magically produce the right value: the names of identifiers
chosen in the macro seem to be independent of those introduced by the macro!
This is unsurprising when it happens in a function; the macro expander enjoys a
property called hygiene that gives it the same property.

One way to think about hygiene is that it effectively automatically renames all
bound identifiers. That is, it’s as if the program expands as follows:
\lsts{13/4/3/2.rkt}{rkt}
turns into
\lsts{13/4/3/3.rkt}{rkt}
(notice the consistent renaming of v to v1), which turns into
\lsts{13/4/3/4.rkt}{rkt}
which, after renaming, becomes
\lsts{13/4/3/5.rkt}{rkt}
when expansion terminates. Observe that each of the programs above, if run
directly, will produce the correct answer.

\secup

\secrel{13.5 Identifier Capture   . 99}

Hygienic macros address a routine and important pain that creators of syntactic
sugar confront. On rare instances, however, a developer wants to intentionally
break hygiene. Returning to objects, consider this input program:
\lsts{13/5/1.rkt}{rkt}
What does the macro look like? Here’s an obvious candidate:
\lsts{13/5/2.rkt}{rkt}
Unfortunately, this macro produces the following error:
\lst{13/5/2.err}
which is referring to the self in the body of the method bound to first.

\Exercise{
Work through the hygienic expansion process to understand why error is the
expected outcome.
}

Before we jump to the richer macro, let’s consider a variant of the input term
that makes the binding explicit:
\lsts{13/5/3.rkt}{rkt}
The corresponding macro is a small variation on what we had before:
\lsts{13/5/4.rkt}{rkt}
This macro expands without difficulty.

\Exercise{
Work through the expansion of this version and see what’s different.
}

This offers a critical insight: had the identifier that goes in the binding
position been written by the macro user, there would have been no problem.
Therefore, we want to be able to pretend that the introduced identifier was
written by the user. The function datum->syntax converts the s-expression in its
second argument; its first argument is which syntax to pretend it was a part of
(in our case, the original macro use, which is bound to x). To introduce the
result into the environment used for expansion, we use with-syntax to bind it in
that environment:
\lsts{13/5/5.rkt}{rkt}

With this, we can go back to having self be implicit:
\lsts{13/5/6.rkt}{rkt}

\secrel{13.6 Influence on Compiler Design  101}

The use of macros in a language’s definition has an impact on all tools,
especially compilers. As a working example, consider let. let has the virtue
that it can be compiled efficiently, by just extending the current environment.
In contrast, the expansion of let into function application results in a much
more expensive operation: the creation of a closure and its application to the
argument, achieving effectively the same result but at the cost of more time
(and often space).

This would seem to be an argument against using the macro. However, a smart
compiler recognizes that this pattern occurs often, and instead internally
effectively converts left-left-lambda \ref{}\ back into the equivalent of let.
This has two advantages. First, it means the language designer can freely use
macros to obtain a smaller core language, rather than having to trade that off
against the execution cost.

It has a second, much subtler, advantage. Because the compiler recognizes this
pattern, other macros can also exploit it and obtain the same optimization; they
don’t need to contort their output to insert let terms if the left-left-lambda
pattern occurs naturally, as they would have to do otherwise. For instance, the
left-left-lambda pattern occurs naturally when writing certain kinds of
pattern-matchers, but it would take an extra step to convert this into a let in
the expansion—which is no longer necessary.

\secrel{13.7 Desugaring in Other Languages  . . 101}

Many modern languages define operations via desguaring, not only Racket. In
Python, for instance, iterating using for is simply a syntactic pattern. A
developer who writes for x in o is
\begin{itemize}
  \item 
introducing a new identifier (call it i—but be sure to not capture any other i
the programmer has already defined, i.e., bind i hygienically!),
  \item 
binding it to an iterator obtained from o, and
  \item 
creating a (potentially) infinite while loop that repeatedly invokes the .next
method of i until the iterator raises the StopIteration exception.
\end{itemize}
There are many such patterns in modern programming languages.

\secup
