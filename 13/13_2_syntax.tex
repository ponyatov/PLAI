\secrel{13.2 Syntax Transformers as Functions  . 93}

Earlier we saw that my-let-1 does not even attempt to ensure that the syntax in
the identifier position is truly (i.e., syntactically) an identifier. We cannot
remedy that with the syntax-rules mechanism, but we can with a much more
powerful mechanism called syntax-case. Because syntax-case exhibits many other
useful features as well, we’ll introduce it and then grow it gradually.

The first thing to understand is that a macro is actually a function. It is not,
however, a function from regular run-time values to other run-time values, but
rather a function from syntax to syntax. These functions execute in a world
whose purpose is to create the program to execute. Observe that we’re talking
about the program to execute: the actual execution of the program may only occur
much later (or never at all). This point is actually extremely clear when we
examine desugaring, which is very explicitly a function from (one kind of)
syntax to (another kind of) syntax. This is perhaps obscured above in two ways:
\begin{itemize}
  \item 
The notation of syntax-rules, with no explicit parameter name or other “function
header”, may not make clear that it is a functional transformation (though the
rewriting rule format does allude to this fact).
  \item 
In desugaring, we specify one atomic function for the entire process. Here, we
are actually writing several little functions, one for each kind of new
syntactic construct (such as my-let-1), and these pieces are woven together by
an invisible function that controls the overall rewriting process. (As a
concrete example, it is not inherently clear that the output of a macro is
expanded further—though a simple example immediately demonstrates that this is
indeed the case.)
\end{itemize}

\Exercise{
Write one or more macros to confirm that the output of a macro is expanded
further.
}

There is one more subtlety. Because the form of a macro looks rather like Racket
code, it is not immediately clear that it “lives in another world”. In the
abstract, it may be helpful to imagine that the macro definitions are actually
written in an entirely different language that processes only syntax. This
simplicity is, however, misleading. In practice, program transformers—also
called compilers—are full-blown programs, too, and need all the power of
ordinary programs. This would have necessitated the creation of a parallel
language purely for processing programs. This would be wasteful and pointless;
therefore, Racket instead endows syntax-transforming programs with the full
power of Racket itself.

With that prelude, let’s now introduce syntax-case. We’ll begin by simply
rewriting my-let-1 (under the name my-let-3) using this new notation. First, we
have to write a header for the definition; notice already the explicit
parameter:
\lsts{src/13/2/1.rkt}{rkt}

This binds x to the entire (my-let-3 ...) expression.

As you might imagine, define-syntax simply tells Racket you’re about to define a
new macro. It does not pick precisely how you want to implement it, leaving you
free to use any mechanism that’s convenient. Earlier we used syntax-rules; now
we’re going to use syntax-case. In particular, syntax-case needs to explicitly
be given access to the expression to pattern-match:
\lsts{src/13/2/2.rkt}{rkt}

Now we’re ready to express the rewrite we wanted. Previously a rewriting rule
had two parts: the structure of the input and the corresponding output. The same
holds here. The first (matching the input) is the same as before, but the second
(the output) is a little different:
\lsts{src/13/2/3.rkt}{rkt}

Observe the crucial extra characters: \#'. Let’s examine what that means.

In syntax-rules, the entire output part simply specifies the structure of the
output. In contrast, because syntax-case is laying bare the functional nature of
transformation, the output part is in fact an arbitrary expression that may
perform any computations it wishes. It must simply evaluate to a piece of
syntax.

Syntax is actually a distinct datatype. As with any distinct dataype, it has its
own rules for construction. Concretely, we construct syntax values by writing
\#'; the following s-expression is treated as a syntax value. (In case you were
wondering, the x bound in the macro definition above is also of this datatype.)

The syntax constructor, \#', enjoys a special property. Inside the output part
of the macro, all syntax variables in the input are automatically bound, and
replaced on occurrence. As a result, when the expander encounters var in the
output, say, it replaces var with the corresponding part of the input expression.

\DoNow{
Remove the \#' and try using the above macro definition. What happens?
}

So far, syntax-case merely appears to be a more complicated form of syntax-
rules: perhaps slightly better in that it more cleanly delineates the functional
nature of expansion, and the type of output, but otherwise simply more unwieldy.
As we will see, however, it also offers significant power.

\Exercise{
syntax-rules can actually be expressed as a macro over syntax-case.
Define it.
}
