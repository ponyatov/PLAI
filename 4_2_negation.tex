\secrel{Extension: Unary Negation \ru{Расширение: унарное отрицание}}

Now let’s consider another extension, which is a little more interesting: unary
negation. This forces you to do a little more work in the parser because,
depending on your surface syntax, you may need to look ahead to determine
whether you’re in the unary or binary case. But that’s not even the interesting
part\,!

There are many ways we can desugar unary negation. We can define it naturally as
$-b=0-b$, or we could abstract over the desugaring of binary
subtraction with this expansion: $-b=0+ -1 \times b$.

\DoNow{
Which one do you prefer? Why?
}

It’s tempting to pick the first expansion, because it’s much simpler. Imagine
we’ve extended the \class{ArithS} datatype with a representation of unary
negation:
\lst{src/4/p18_2.rkt}
Now the implementation in \term{desugar} is straightforward:
\lst{src/4/p18_3.rkt}
Let’s make sure the types match up. Observe that $e$ is a \class{ArithS} term,
so it is valid to use as an argument to \class{bminusS}, and the entire term can
legally be passed to \term{desugar}.
It is therefore important to \emph{not} desugar $e$ but rather embed it directly
in the generated term. This embedding of an input term in another one and
recursively calling desugar is a common pattern in desugaring tools; it is
called a \termdef{macro}{macro} (specifically, the ``macro'' here is this
definition of \class{uminusS}).

However, there are two problems with the definition above:
\begin{enumerate}

\item
The first is that the recursion is \termdef{generative}{generative
recursion}\note{If you haven’t heard of \term{generative recursion} before, read
the section on it in \emph{How to Design Programs}. Essentially, in generative
recursion the sub-problem is a computed function of the input, rather than a
structural piece of it. This is an especially simple case of generative
recursion, because the ``function'' is simple: it’s just the \class{bminusS}
constructor}, which forces us to take extra care. We might be tempted to fix
this by using a different rewrite:
\lst{src/4/p18_4.rkt}
which does indeed eliminate the generativity.

\DoNow{
Unfortunately, this desugaring transformation won’t work at all\,!
Do you see why\,? If you don’t, try to run it.
}

\item
The second is that we are implicitly depending on exactly what \class{bminusS}
means; if its meaning changes, so will that of \class{uminusS}, even if we don’t
want it to. In contrast, defining a functional abstraction that consumes two
terms and generates one representing the addition of the first to $-1$ times the
second, and using this to define the desugaring of both \class{uminusS} and
\class{bminusS}, is a little more faulttolerant.

You might say that the meaning of subtraction is never going to change, so why
bother\,? Yes and no. Yes, it’s \emph{meaning} is unlikely to change; but no,
its \emph{implementation} might. For instance, the developer may decide to log
all uses of binary subtraction. In the macro expansion, all uses of unary
negation would also get logged, but they would not in the second expansion.

\end{enumerate}

Fortunately, in this particular case we have a much simpler option, which is to
define $-b=-1\times b$. This expansion works with the primitives we have, and
follows structural recursion. The reason we took the above detour, however, is
to alert you to these problems, and warn that you might not always be so
fortunate.
