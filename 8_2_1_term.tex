\secrel{8.2.1 Terminology   . . 57}

First, our choice of terms. We’ve insisted on using the word “identifier” before because
we wanted to reserve “variable” for what we’re about to study. In Java, when we say
(assuming x is locally bound, e.g., as a method parameter)
\lsts{src/8/2/1.rkt}{java}
we’re asking to change the value of x. After the first assignment, the value of
x is 1; after the second one, it’s 3. Thus, the value of x varies over the
course of the execution of the method.

Now, we also use the term “variable” in mathematics to refer to function
parameters. For instance, in f(y) = y + 3 we say that y is a “variable”. That is
called a variable because it varies across invocations; however, within each
invocation, it has the same value in its scope. Our identifiers until now have
corresponded to this notion of a variable. In contrast, programming variables
can vary even within each invocation, like the Java x above.
\note{If the identifier was bound to a box, then it remained bound to the same
box value. It’s the content of the box that changed, not which box the
identifier was bound to.}

Henceforth, we will use variable when we mean an identifier whose value can
change within its scope, and identifier when this cannot happen. If in doubt, we
might play it safe and use “variable”; if the difference doesn’t really matter,
we might use either one. It is less important to get caught up in these specific
terms than to understand that they represent a distinction that matters \ref{}.
