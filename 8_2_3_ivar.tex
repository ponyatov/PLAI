\secrel{8.2.3 Interpreting Variables  . 58}

We’ll start by reflecting this in our syntax:
\lsts{src/8/2/5.rkt}{rkt}
Observe that we’ve jettisoned the box operations, but kept sequencing because
it’s handy around mutation. Importantly, we’ve now added the setC case, and its
first subterm is not an expression but the literal name of a variable. We’ve
also renamed idC to varC.

Because we’ve gotten rid of boxes, we can also get rid of the special box
values. When the only kind of mutation you have is variables, you don’t need new
kinds of values.
\lsts{src/8/2/6.rkt}{rkt}

As you might imagine, to support variables we need the same store-passing style
that we’ve seen before (section 8.1.7), and for the same reasons. What differs
is in precisely how we use it. Because sequencing is interpreted in just the
same way (observe that the code for it does not depend on boxes versus
variables), that leaves us just the variable mutation case to handle.

First, we might as well evaluate the value expression and obtain the updated
store:
\lsts{src/8/2/7.rkt}{rkt}

What now? Remember we just said that we don’t want to fully evaluate the
variable, because that would just give the value it is bound to. Instead, we
want to know which memory location it corresponds to, and update what is stored
at that memory location; this latter part is just the same thing we did when
mutating boxes:
\lsts{src/8/2/8.rkt}{rkt}

The very interesting new pattern we have here is this. When we added boxes, in
the idC case, we looked up an identifier in the environment, and immediately
fetched the value at that location from the store; the composition yielded a
value, just as it used to before we added stores. Now, however, we have a new
pattern: looking up an identifier in the environment without subsequently
fetching its value from the store. The result of invoking just lookup is
traditionally called an l-value, for “left-handside (of an assignment) value”.
This is a fancy way of saying “memory address”, and stands in contast to the
actual values that the store yields: observe that it does not directly
correspond to anything in the type Value.

And we’re done! We did all the hard work when we implemented store-passing style
(and also in that application allocated new locations for variables).

