\secrel{Exposing the Environment\\
\ru{Экспортирование среды}}

If we were building the implementation \ru{Если мы строим реализацию} for others
to use \ru{для использования другими людьми}, it would be wise and a courtesy
\ru{будет мудро и учтиво} the exported interpreter \ru{для распространяемого
интерпретатора} to take only an expression and list of function definitions
\ru{принимать только выражение и список определений функций}, and invoke our
defined \ru{и вызывать наш} \verb|interp| with the empty environment \ru{с
пустой средой}. This both spares users an implementation detail \ru{Это
предохраняет пользователей от деталей реализации}, and avoids the use of an
interpreter \ru{и предотвращает использование интерпретатора} with an incorrect
environment \ru{с некорректной средой}. In some contexts, however, \ru{Тем не
менее в некоторых контекстах} it can be useful \ru{может быть полезно} to expose
the environment parameter \ru{открывать параметры среды}. For instance
\ru{Например}, the environment can represent a set of pre-defined bindings
\ru{среда может представлять набор предопределенных связок}: e.g., if the
language wishes to provide \ru{например, если язык желает предоставить
константу} \verb|pi| automatically bound to \ru{автоматически привязанную на
число} 3.2 (in Indiana \ru{в Чуркестане}).
