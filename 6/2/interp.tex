\secrel{Interpreting with Environments\\
\ru{Интерпретация со средами}}

Now we can tackle the interpreter \ru{Теперь мы можем реализовать
интерпретатор}. One case is easy \ru{Одна ветка простая}, but we should revisit
all the others \ru{но другие нам нужно будет обсудить отдельно}:
\lsts{6/2/1.rkt}{rkt}
The arithmetic operations are easiest \ru{Арифметические опреации проще всего}.
Recall that before \ru{Как мы сказали ранее}, the interpreter recurred
\ru{интерпретатор вызывается рекурсивно} without performing any new
substitutions \ru{без выполнения новых подстановок}. As a result \ru{в
результате}, there are no new deferred substitutions to perform either \ru{нет
никаких отсроченных подстановок которые нужно выполнить}, which means the
environment does not change \ru{что значит что среда не меняется}:
\lsts{6/2/2.rkt}{rkt}

Now let’s handle identifiers \ru{Теперь обработаем идентификаторы}. Clearly
\ru{Ясно что}, encountering an identifier is no longer an error
\ru{неопределенный идентификатор больше не является ошибкой}:
this was the very motivation for this change \ru{это было одной из причин
изменения принципа интерпретации}. Instead, we must look up its value in the
directory \ru{Вместо этого мы должны выполнить поиск его значения в каталоге}:
\lsts{6/2/3.rkt}{rkt}

\DoNow{
Implement lookup.\\\ru{Реализуйте поиск.}
}

Finally, application \ru{И наконец, применение}. Observe that in the
substitution interpreter \ru{Обратите внимание что в интерпретатор с
подстановкой}, the only case that caused new substitutions to occur was
application \ru{единственным случаем вызывающим новые подстановки было
применение}. Therefore, this should be the case that constructs bindings
\ru{Таким образом, это должно быть случаем, когда создаются связки}. Let’s first
extract the function definition \ru{Давайте сначала выделим определение
функции}, just as before \ru{также как и раньше}:
\lsts{6/2/4.rkt}{rkt}
Previously, we substituted, then interpreted \ru{Раньше мы сначала подставляли,
а потом интерпретировали}. Because we have no substitution step \ru{Так как
теперь у нас нет шага подстановки}, we can proceed with interpretation \ru{мы
можем продолжать интерпретацию}, so long as we record the deferral of
substitution \ru{как только мы записали отсрочку подстановки}.
\lsts{6/2/5.rkt}{rkt}
That is \ru{То есть}, the set of function definitions remains unchanged
\ru{множество определений функций остается неизменным}; we’re interpreting the
body of the function, as before \ru{мы интерпретируем тело функции как и
раньше}; but we have to do it in an environment that binds the formal parameter
\ru{но мы должны делать это в среде, связывающей формальный параметр}. Let’s now
define that binding process \ru{Давайте определим процесс связывания}:
\lsts{6/2/6.rkt}{rkt}
the name being bound is the formal parameter \ru{имя было связано как формальный
параметр} (the same name that was substituted for, before \ru{это то же самое
имя которое подставлялось ранее}). It is bound to the result of interpreting the
argument \ru{Оно связано с результатом интерпретации аргумента} (because we’ve
decided on an eager application semantics \ru{потому что мы решили использовать
семантику жадного применения}). And finally, this extends the environment we
already have \ru{И наконец, оно расширяет среду которая у нас уже есть}.
Type-checking this helps to make sure \ru{Контроль типов помогает нам убедиться
что} we got all the little pieces right \ru{мы совместили все эти маленькие
кусочки правильно}.

Once we have a definition for lookup \ru{Как только мы получаем определение
для поиска}, we’d have a full interpreter \ru{мы получаем полный интерпретатор}.
So here’s one \ru{Вот он}:
\lsts{6/2/7.rkt}{rkt}

Observe that looking up a free identifier still produces an error \ru{Обратите
внимание что поиск свободного идентификатора все еще вызывает ошибку}, but it
has moved from the interpreter \ru{но она переместилась из интерпретатора}\ ---
which is by itself unable to determine whether or not an identifier is free
\ru{который сам по себе не способен определить свободен ли идентификатор}\ ---
to \ru{в} \verb|lookup|, which determines this based on the content of the
environment \ru{который определяет это на основе содержимого среды}.

Now we have a full interpreter \ru{Теперь у нас есть полный интерпретатор}. You
should of course test it make sure it works as you’d expect \ru{Естественно вы
должны протестировать его чтобы убедиться что он работает как вы ожидаете}.
For instance, these tests pass \ru{Например, с помощью этих тестов}:
\lsts{6/2/8_1.rkt}{rkt}
\lsts{6/2/8_2.rkt}{rkt}
\lsts{6/2/8_3.rkt}{rkt}
So we’re done, right \ru{Так что все сделано, правильно}\,?

\DoNow{
Spot the bug.\\\ru{Найдите ошибку.}
}
