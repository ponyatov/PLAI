\secrel{6 From Substitution to Environments 25}\secdown

Though we have a working definition of functions, you may feel a slight unease
about it. When the interpreter sees an identifier, you might have had a sense
that it needs to “look it up”. Not only did it not look up anything, we defined
its behavior to be an error! While absolutely correct, this is also a little
surprising. More importantly, we write interpreters to understand and explain
languages, and this implementation might strike you as not doing that, because
it doesn’t match our intuition.

There’s another difficulty with using substitution, which is the number of times
we traverse the source program. It would be nice to have to traverse only those
parts of the program that are actually evaluated, and then, only when necessary.
But substitution traverses everything—unvisited branches of conditionals, for
instance—and forces the program to be traversed once for substitution and once
again for interpretation.

\Exercise{
Does substitution have implications for the time complexity of evaluation?
}

There’s yet another problem with substitution, which is that it is defined in
terms of representations of the program source. Obviously, our interpreter has
and needs access to the source, to interpret it. However, other
implementations—such as compilers— have no need to store it for that purpose.
It would be nice to employ a mechanism that is more portable across
implementation strategies.
\note{Compilers might store versions of or information about the source for
other reasons, such as reporting runtime errors, and JITs may need it to
re-compile on demand.}

\secrel{Introducing the Environment\\\ru{Введение понятия ``Среда''}}

The intuition that addresses the first concern is to have \ru{Во-первых
интуитивно хочется иметь} the interpreter ``look up'' an identifier in some sort
of directory \ru{интерпретатор с процедурой ``поиска'' идентификатора в
некотором подобии каталога}. The intuition that addresses the second concern
\ru{Во-вторых интуиция подсказывает} is to \emph{defer} the substitution
\ru{\emph{избавиться} от подстановки}. Fortunately, these converge nicely in a
way that also addresses the third \ru{К счастью выполнение этих двух целей
сводится к третьей}. The directory records the \emph{intent to substitute}
\ru{Каталог хранит \emph{намерение замены}}, without actually rewriting the
program source \ru{без реального переписывания исходной программы}; by recording
the intent \ru{храня намерение замены}, rather than substituting immediately
\ru{без выполнения самой замены}, we can defer substitution \ru{мы можем
избавиться от замены}; and the resulting data structure \ru{и результирующая
структура}, which is called an \termdef{environment}{environment} \ru{которая
называется \termdef{среда}{среда}}, avoids the need for source-to-source
rewriting \ru{избавляет от необходимости source-to-source переписывания} and
maps nicely to low-level machine representations \ru{и хорошо отображается на
низкоуровневое машинное представление}. Each name association in the environment
\ru{Каждая ассоциация имени в среде} is called a \termdef{binding}{binding}
\ru{называется \termdef{связка}{связка}}.

Observe carefully \ru{Аккуратно рассмотрим} that what we are changing \ru{что мы
поменяли} is the \emph{implementation strategy} \ru{в \emph{стратегии
реализации}} for the programming language \ru{языка программирования}, \emph{not
the language itself} \ru{но \emph{не в самом языке}}. Therefore \ru{Таким
образом}, none of our datatypes for representing programs should change
\ru{никаких изменений не должно быть в наших типах данных представляющих
програму}, nor even should the answers \ru{также как и в результатах} that the
interpreter provides \ru{которые возвращает интерпретатор}. As a result \ru{В
результате}, we should think of the previous interpreter \ru{мы должны
рассматривать предыдущий интерпретатор} as a ''reference implementation''
\ru{как ``эталонную реализацию''} that the one we’re about to write should match
\ru{с поведением которой должно совпадать все что мы напишем}. Indeed \ru{В
самом деле}, we should create a generator \ru{нам следовало бы создать
генератор} that creates lots of tests \ru{который создает множество тестов},
runs them through both interpreters \ru{прогоняет их на обоих интерпретаторах},
and makes sure their answers are the same \ru{и подтверждает что их результаты
одинаковы}. Ideally \ru{В идеале}, we should \emph{prove} that the two
interpreters behave the same \ru{нам нужно \emph{доказать} что эти два
интерпретатора \term{эквивалентны}}, which is a good topic for advanced study
\ru{что является хорошей темой для отдельного исследования}.
\note{One subtlety is in defining precisely what “the same” means, especially
with regards to failure.}
\note{\ru{Одна из тонкостей\ --- что точно обозначает фраза ``то же
самое поведение'', особенно в случае ошибочных ситуаций.}}

Let’s first define our environment data structure \ru{Для начала давайте
определим структуру данных для нашей среды}.
An environment is a list of pairs of names associated with\ldots what\,?
\ru{Среда это список пар имен, ассоциированных с\ldots чем\,?}

\DoNow{
A natural question to ask here might be what the environment maps names to. But
a better, more fundamental, question is: How to determine the answer to the
“natural” question\,?\\
\ru{Здесь возникает естественный вопрос, на что именно среда отображает имена.
Но лучший, более фундаментальный вопрос: Как определить ответ на
``естественный'' вопрос\,?}
}

Remember that our environment was created to defer substitutions \ru{Вспомним,
что наша среда была создана чтобы избавиться от подстановки}. Therefore, the
answer lies in substitution \ru{Таким образом, ответ заключается в подстановке}.
We discussed earlier \ru{Ранее мы обсудили} \ref{waitmore} that we want
substitution to map names to answers \ru{что мы хотим чтобы подстановка
отображала имена на результаты вычислений}, corresponding to an eager function
application strategy \ru{в соответствие с жадной стратегией применения функций}.
Therefore, the environment should map names to answers \ru{Таким образом,
среда должна отображать имена на результаты вычислений}.
\lsts{6/1/1.rkt}{rkt}
\secrel{Interpreting with Environments\\
\ru{Интерпретация со средами}}

Now we can tackle the interpreter \ru{Теперь мы можем реализовать
интерпретатор}. One case is easy \ru{Одна ветка простая}, but we should revisit
all the others \ru{но другие нам нужно будет обсудить отдельно}:
\lsts{6/2/1.rkt}{rkt}
The arithmetic operations are easiest \ru{Арифметические опреации проще всего}.
Recall that before \ru{Как мы сказали ранее}, the interpreter recurred
\ru{интерпретатор вызывается рекурсивно} without performing any new
substitutions \ru{без выполнения новых подстановок}. As a result \ru{в
результате}, there are no new deferred substitutions to perform either \ru{нет
никаких отсроченных подстановок которые нужно выполнить}, which means the
environment does not change \ru{что значит что среда не меняется}:
\lsts{6/2/2.rkt}{rkt}

Now let’s handle identifiers \ru{Теперь обработаем идентификаторы}. Clearly
\ru{Ясно что}, encountering an identifier is no longer an error
\ru{неопределенный идентификатор больше не является ошибкой}:
this was the very motivation for this change \ru{это было одной из причин
изменения принципа интерпретации}. Instead, we must look up its value in the
directory \ru{Вместо этого мы должны выполнить поиск его значения в каталоге}:
\lsts{6/2/3.rkt}{rkt}

\DoNow{
Implement lookup.\\\ru{Реализуйте поиск.}
}

Finally, application \ru{И наконец, применение}. Observe that in the
substitution interpreter \ru{Обратите внимание что в интерпретатор с
подстановкой}, the only case that caused new substitutions to occur was
application \ru{единственным случаем вызывающим новые подстановки было
применение}. Therefore, this should be the case that constructs bindings
\ru{Таким образом, это должно быть случаем, когда создаются связки}. Let’s first
extract the function definition \ru{Давайте сначала выделим определение
функции}, just as before \ru{также как и раньше}:
\lsts{6/2/4.rkt}{rkt}
Previously, we substituted, then interpreted \ru{Раньше мы сначала подставляли,
а потом интерпретировали}. Because we have no substitution step \ru{Так как
теперь у нас нет шага подстановки}, we can proceed with interpretation \ru{мы
можем продолжать интерпретацию}, so long as we record the deferral of
substitution \ru{как только мы записали отсрочку подстановки}.
\lsts{6/2/5.rkt}{rkt}
That is \ru{То есть}, the set of function definitions remains unchanged
\ru{множество определений функций остается неизменным}; we’re interpreting the
body of the function, as before \ru{мы интерпретируем тело функции как и
раньше}; but we have to do it in an environment that binds the formal parameter
\ru{но мы должны делать это в среде, связывающей формальный параметр}. Let’s now
define that binding process \ru{Давайте определим процесс связывания}:
\lsts{6/2/6.rkt}{rkt}
the name being bound is the formal parameter \ru{имя было связано как формальный
параметр} (the same name that was substituted for, before \ru{это то же самое
имя которое подставлялось ранее}). It is bound to the result of interpreting the
argument \ru{Оно связано с результатом интерпретации аргумента} (because we’ve
decided on an eager application semantics \ru{потому что мы решили использовать
семантику жадного применения}). And finally, this extends the environment we
already have \ru{И наконец, оно расширяет среду которая у нас уже есть}.
Type-checking this helps to make sure \ru{Контроль типов помогает нам убедиться
что} we got all the little pieces right \ru{мы совместили все эти маленькие
кусочки правильно}.

Once we have a definition for lookup \ru{Как только мы получаем определение
для поиска}, we’d have a full interpreter \ru{мы получаем полный интерпретатор}.
So here’s one \ru{Вот он}:
\lsts{6/2/7.rkt}{rkt}

Observe that looking up a free identifier still produces an error \ru{Обратите
внимание что поиск свободного идентификатора все еще вызывает ошибку}, but it
has moved from the interpreter \ru{но она переместилась из интерпретатора}\ ---
which is by itself unable to determine whether or not an identifier is free
\ru{который сам по себе не способен определить свободен ли идентификатор}\ ---
to \ru{в} \verb|lookup|, which determines this based on the content of the
environment \ru{который определяет это на основе содержимого среды}.

Now we have a full interpreter \ru{Теперь у нас есть полный интерпретатор}. You
should of course test it make sure it works as you’d expect \ru{Естественно вы
должны протестировать его чтобы убедиться что он работает как вы ожидаете}.
For instance, these tests pass \ru{Например, с помощью этих тестов}:
\lsts{6/2/8_1.rkt}{rkt}
\lsts{6/2/8_2.rkt}{rkt}
\lsts{6/2/8_3.rkt}{rkt}
So we’re done, right \ru{Так что все сделано, правильно}\,?

\DoNow{
Spot the bug.\\\ru{Найдите ошибку.}
}

\secrel{Deferring Correctly\\
\ru{Правильная отсрочка}}

Here’s another test \ru{Вот другой тест}:
\lsts{6/3/1.rkt}{rkt}
In our interpreter \ru{В нашем интерпретаторе}, this evaluates to \ru{он
вычисляется в} 7. Should it \ru{Правильно ли это}\,?

Translated into \ru{При переводе на} \racket, this test corresponds to the
following two definitions and expression \ru{этот тест соответствует следующим
двум определениям и выражению}:
\lsts{6/3/2.rkt}{rkt}

What should this produce \ru{Что он должен вычислить}\,? \verb|(f1 3)|
substitutes \ru{подставляет} \verb|x| with \verb|3| in the body of
\ru{в теле} \verb|f1|, which then invokes \ru{что вызывает} \verb|(f2 4)|.
But notably, in \ru{Но заметим что} \verb|f2|, the identifier \ru{идентификатор}
\verb|x| is \emph{not bound} \ru{\emph{не связан}}\,! Sure enough, \racket\ will
produce an error \ru{Будьте уверены, \racket\ выдаст ошибку}.

In fact, so will our substitution-based interpreter \ru{Фактически то же
самое сделает наш интерпретатор с подстановкой}\,!

Why does the substitution process result in an error \ru{Почему процесс
подстановки приведет к ошибке}\,? It’s because \ru{Потому что}, when we replace
the representation of \ru{когда мы заменяем представление} \verb|x| with the
representation of \ru{на предславление} \verb|3| in the representation of \ru{в
представлении} \verb|f1|, we do so in \ru{мы это делаем только в} \verb|f1|
only.
\note{This “the representation of” is getting a little annoying, isn’t it\,?
Therefore, I’ll stop saying that, but do make sure you understand why I had to
say it. It’s an important bit of pedantry.}
\note{\ru{Это ``представление'' несколько надоедает, не так ли\,? Так что я
перестану говорить это, но убедитесь что понимаете почему я должен его говорить.
Это важный элемент педантизма.}}
(Obviously \ru{Очевидно}: \verb|x| is \ru{это параметр} \verb|f1|’s parameter;
even if another function had a parameter named \ru{даже если другая функция
имеет параметр} \verb|x|, that’s a \emph{different} \ru{это \emph{другой}}
\verb|x|.) Thus \ru{Так что}, when we get to evaluating the body of \ru{когда мы
вычисляем тело} \verb|f2|, its \ru{ее} \verb|x| hasn’t been substituted \ru{не
подставляется}, resulting in the error \ru{что приводит к ошибке}.

What went wrong when we switched to environments \ru{Что пошло не так, когда мы
переключились на среды}\,? Watch carefully \ru{Смотрите внимательнее}:
this is subtle \ru{это тонкая штука}. We can focus on applications \ru{Мы можем
сфокусироваться на применениях}, because only they affect the environment
\ru{потому что только они влияют на среду}. When we substituted the formal for
the value of the actual \ru{Когда мы заменили формальный параметр актуальным
значением}, we did so \ru{мы это сделали} by \emph{extending the current
environment} \ru{через \emph{расширение текущей среды}}. In terms of our example
\ru{В терминах нашего примера}, we asked the interpreter \ru{мы попросили
интерпретатор} to substitute not only \ru{заменить не только подстановку}
\verb|f2|’s substitution in \ru{в теле} \verb|f2|’s body, but also the current
ones \ru{но также и текущие вхождения} (those for the caller \ru{для
вызывающей}, \verb|f1|), and indeed all past ones as well \ru{а также вообще все
последующие}. That is, the environment only grows \ru{Так что среда только
растет}; it never shrinks \ru{она никогда не сокращается}.

Because we agreed that environments are only an alternate implementation
strategy for substitution \ru{Так как мы договорились что среды единственная
альтернативная стратегия реализации для подстановки}\ --- and in particular
\ru{в частности}, that the language’s meaning should not change \ru{семантика
языка не должна меняться}\ --- we have to alter the interpreter \ru{мы должны
поправить интерпретатор}. Concretely \ru{Конкретно}, we should not ask it to
carry around all past deferred substitution requests \ru{мы не должны просить
его хранить все отложенные запросы на подстановку}, but instead make it start
afresh for every new function \ru{вместо этого он должен начинать начисто для
каждой новой функции}, just as the substitution-based interpreter does \ru{также
как делает интерпретатор на подстановке}. This is an easy change \ru{Это простая
модификация}:
\lsts{6/3/3.rkt}{rkt}

Now we have truly reproduced the behavior of the substitution interpreter
\ru{Теперь мы на самом деле воспроизвели поведение подстановочного
интерпретатора}.
\note{In case you’re wondering how to write a test case that catches errors,
look up test/exn.}
\note{\ru{Если вам не понятно как написать тест который ловит ошибки,
посмотрите} test/exn}

\secrel{Scope\\\ru{Область видимости}}

The broken environment interpreter above \ru{Сломанный интерпретатор на средах
выше} implements what is known as \termdef{dynamic scope}{dynamic scope}
\ru{реализовывал то что называется \termdef{динамическая область
видимости}{динамическая область видимости}}. This means \ru{Это значит что} the
environment accumulates bindings as the program executes \ru{среда аккумулирует
связки в процессе выполнения программы}. As a result \ru{В результате}, whether
an identifier is even bound \ru{определение того является ли идентификатор
связанным} depends on the history of program execution \ru{зависит от истории
исполнения программы}. We should regard this unambiguously \ru{Мы должны
рассматривать это} as a flaw of programming language design \ru{как недостаток
дизайна языка программирования}. It adversely affects all tools \ru{Это
отрицательно влияет на все инструменты} that read and process programs
\ru{которые читают и обрабатывают программы}: compilers, IDEs, and humans
\ru{компиляторы, IDE и человеки}.

In contrast \ru{Наоборот}, substitution \ru{подстановка}\ --- and environments,
done correctly \ru{правильно реализованные среды}\ --- give us \termdef{lexical
scope}{lexical scope} or \termdef{static scope}{static scope} \ru{дают нам
\termdef{лексеческую}{лексическая область видимости} или
\termdef{статическую}{статическая область видимости} области видимости}.
``Lexical'' in this context means \ru{``Лексическая'' в этом контексте значит}
``as determined from the source program'' \ru{``определенная из исходного кода
программы''}, while ``static'' in computer science means \ru{в то время как
``статическая'' в информатике значит} ``without running the program'' \ru{``без
запуска программы''}, so these are appealing to the same intuition \ru{эти
названия следуют той же идее}. When we examine an identifier \ru{Когда мы
встречаем идентификатор}, we want to know two things \ru{мы хотим знать две
вещи}: (1) Is it bound \ru{Связан ли он}\,? (2) If so, where \ru{и если связан,
то где}\,? By “where” we mean \ru{``Где'' мы имеем в виду}: if there are
multiple bindings for the same name \ru{если существуют множественные связки для
одного имени}, which one governs this identifier \ru{какая из них контролирует
этот идентификатор}\,? Put differently \ru{Другими словами}, which one’s
substitution \ru{какая из подстановок} will give a value to this identifier
\ru{даст значение для этого идентификатора}\,? In general \ru{В общем}, these
questions cannot be answered statically \ru{на эти вопросы нельзя дать
статические ответы} in a dynamically-scoped language \ru{в языке с динамическими
областями видимости}: so your IDE \ru{так что ваша IDE}, for instance
\ru{например}, cannot overlay arrows to show you this information \ru{не может
расставить стрелки чтобы показать эту информацию} (as Dr\racket\ does \ru{как
это делает Dr\racket}).
\note{A different way to think about it is that in a dynamically-scoped
language, the answer to these questions is the same for all identifiers, and it
simply refers to the dynamic environment. In other words, it provides no useful
information.}
\note{\ru{Другой способ думать об этом для языка с динамической областью
видимости\ --- ответ на вопрос один для всех идентификаторов: так как указано
в динамической среде. То есть он не дает никакой полезной информации.}}
Thus \ru{Таким образом}, even though the rules of scope become more complex
\ru{по мере того как правила области видимости становятся сложнее} as the space
of names becomes richer \ru{и пространство типов имен становится богаче} (e.g.,
objects, threads, etc. \ru{например объекты, нити, и т.д.}), we should always
strive to preserve the spirit of static scoping \ru{мы всегда должны стремиться
сохранить дух статической области видимости}.

\secdown
\secrel{How Bad Is It\,?\\\ru{Насколько это плохо\,?}}

You might look at our running example \ru{Вы можете посмотреть на наш рабочий
пример} and wonder whether we’re creating a tempest in a teapot \ru{и удивиться
почему мы создаем бурю в стакане воды}. In return \ru{В свою очередь}, you
should consider two situations \ru{мы должны рассмотреть две ситуации}:
\begin{enumerate}

\item To understand the binding structure of your program \ru{Для понимания
структуры привязок в вашей программе}, you may need to look \emph{at the whole
program} \ru{вам потребуется рассматривать \emph{целиком всю программу}}. No
matter how much you’ve decomposed your program \ru{Не важно что вы разбили
программу} into small, understandable fragments \ru{на маленькие понятные
фрагменты}, it doesn’t matter if you have a free identifier anywhere \ru{не
важно если у вас есть где-то свободные идентификаторы}.

\item Understanding the binding structure \ru{Понимание структуры привязок} is
not only a function of the size of the program \ru{не только функция от размера
программы} but also of the complexity of its control flow \ru{но и от сложности
ее структур управления}. Imagine an interactive program with numerous callbacks
\ru{Представьте интерактивную программу со множеством обработчиков событий};
you’d have to track through every one of them \ru{вы должны отслеживать ее
поведение при срабатывании обработчиков в любом порядке}, too, to know which
binding governs an identifier \ru{чтобы понять как срабатывают привязки
идентификаторов}.

\end{enumerate}

Need a little more of a nudge \ru{Хотите еще пендаль}\,? Let’s replace the
expression of our example program with this one \ru{Давайте заменим выражение
в нашей программе-примере вот этим}:
\lsts{6/4/1.rkt}{rkt}
Suppose \ru{Предположим} \verb|moon-visible?| is a function \ru{это функция}
that presumably evaluates to \ru{которая предположительно вычисляется в}
\verb|false| on new-moon nights \ru{в ночи новолуния}, and \ru{и} \verb|true| at
other times \ru{в другое время}. Then \ru{Так что}, this program will evaluate
to an answer \ru{эта программа будет вычисляться до значения} except on new-moon
nights \ru{в зависимости от фазы луны}, when it will fail with an unbound
identifier error \ru{иногда она будет падать в ошибку несвязанного
идентификатора}.

\Exercise{
What happens on cloudy nights\,?\\\ru{Что случиться в облачную ночь}\,?
}

\secrel{6.4.2 The Top-Level Scope  . 31}

Matters become more complex when we contemplate top-level definitions in many
languages. For instance, some versions of Scheme (which is a paragon of lexical
scoping) allow you to write this:
\lsts{6/4/2.rkt}{rkt}
which seems to pretty clearly suggest where the y in the body of f will come
from, except:
\lsts{6/4/3.rkt}{rkt}
is legal and (f 10) produces 12. Wait, you might think, always take the last
one! But:
\lsts{6/4/4.rkt}{rkt}

Here, z is bound to the first value of y whereas the inner y is bound to the
second value. There is actually a valid explanation of this behavior in terms of
lexical scope, but it can become convoluted, and perhaps a more sensible option
is to prevent such redefinition. Racket does precisely this, thereby offering
the convenience of a top-level without its pain.
\note{Most “scripting” languages exhibit similar problems. As a result, on the
Web you will find enormous confusion about whether a certain language is
statically- or dynamically-scoped, when in fact readers are comparing behavior
inside functions (often static) against the top-level (usually dynamic).
Beware!}

\secup

\secrel{6.5 Exposing the Environment  . . 31}

If we were building the implementation for others to use, it would be wise and a
courtesy for the exported interpreter to take only an expression and list of
function definitions, and invoke our defined interp with the empty environment.
This both spares users an implementation detail, and avoids the use of an
interpreter with an incorrect environment. In some contexts, however, it can be
useful to expose the environment parameter. For instance, the environment can
represent a set of pre-defined bindings: e.g., if the language wishes to provide
pi automatically bound to 3.2 (in Indiana).

\secup

