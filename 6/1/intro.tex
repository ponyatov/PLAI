\secrel{Introducing the Environment\\\ru{Введение понятия ``Среда''}}

The intuition that addresses the first concern is to have \ru{Во-первых
интуитивно хочется иметь} the interpreter ``look up'' an identifier in some sort
of directory \ru{интерпретатор с процедурой ``поиска'' идентификатора в
некотором подобии каталога}. The intuition that addresses the second concern
\ru{Во-вторых интуиция подсказывает} is to \emph{defer} the substitution
\ru{\emph{избавиться} от подстановки}. Fortunately, these converge nicely in a
way that also addresses the third \ru{К счастью выполнение этих двух целей
сводится к третьей}. The directory records the \emph{intent to substitute}
\ru{Каталог хранит \emph{намерение замены}}, without actually rewriting the
program source \ru{без реального переписывания исходной программы}; by recording
the intent \ru{храня намерение замены}, rather than substituting immediately
\ru{без выполнения самой замены}, we can defer substitution \ru{мы можем
избавиться от замены}; and the resulting data structure \ru{и результирующая
структура}, which is called an \termdef{environment}{environment} \ru{которая
называется \termdef{среда}{среда}}, avoids the need for source-to-source
rewriting \ru{избавляет от необходимости source-to-source переписывания} and
maps nicely to low-level machine representations \ru{и хорошо отображается на
низкоуровневое машинное представление}. Each name association in the environment
\ru{Каждая ассоциация имени в среде} is called a \termdef{binding}{binding}
\ru{называется \termdef{связка}{связка}}.

Observe carefully \ru{Аккуратно рассмотрим} that what we are changing \ru{что мы
поменяли} is the \emph{implementation strategy} \ru{в \emph{стратегии
реализации}} for the programming language \ru{языка программирования}, \emph{not
the language itself} \ru{но \emph{не в самом языке}}. Therefore \ru{Таким
образом}, none of our datatypes for representing programs should change
\ru{никаких изменений не должно быть в наших типах данных представляющих
програму}, nor even should the answers \ru{также как и в результатах} that the
interpreter provides \ru{которые возвращает интерпретатор}. As a result \ru{В
результате}, we should think of the previous interpreter \ru{мы должны
рассматривать предыдущий интерпретатор} as a ''reference implementation''
\ru{как ``эталонную реализацию''} that the one we’re about to write should match
\ru{с поведением которой должно совпадать все что мы напишем}. Indeed \ru{В
самом деле}, we should create a generator \ru{нам следовало бы создать
генератор} that creates lots of tests \ru{который создает множество тестов},
runs them through both interpreters \ru{прогоняет их на обоих интерпретаторах},
and makes sure their answers are the same \ru{и подтверждает что их результаты
одинаковы}. Ideally \ru{В идеале}, we should \emph{prove} that the two
interpreters behave the same \ru{нам нужно \emph{доказать} что эти два
интерпретатора \term{эквивалентны}}, which is a good topic for advanced study
\ru{что является хорошей темой для отдельного исследования}.
\note{One subtlety is in defining precisely what “the same” means, especially
with regards to failure.}
\note{\ru{Одна из тонкостей\ --- что точно обозначает фраза ``то же
самое поведение'', особенно в случае ошибочных ситуаций.}}

Let’s first define our environment data structure \ru{Для начала давайте
определим структуру данных для нашей среды}.
An environment is a list of pairs of names associated with\ldots what\,?
\ru{Среда это список пар имен, ассоциированных с\ldots чем\,?}

\DoNow{
A natural question to ask here might be what the environment maps names to. But
a better, more fundamental, question is: How to determine the answer to the
“natural” question\,?\\
\ru{Здесь возникает естественный вопрос, на что именно среда отображает имена.
Но лучший, более фундаментальный вопрос: Как определить ответ на
``естественный'' вопрос\,?}
}

Remember that our environment was created to defer substitutions \ru{Вспомним,
что наша среда была создана чтобы избавиться от подстановки}. Therefore, the
answer lies in substitution \ru{Таким образом, ответ заключается в подстановке}.
We discussed earlier \ru{Ранее мы обсудили} \ref{waitmore} that we want
substitution to map names to answers \ru{что мы хотим чтобы подстановка
отображала имена на результаты вычислений}, corresponding to an eager function
application strategy \ru{в соответствие с жадной стратегией применения функций}.
Therefore, the environment should map names to answers \ru{Таким образом,
среда должна отображать имена на результаты вычислений}.
\lsts{6/1/1.rkt}{rkt}