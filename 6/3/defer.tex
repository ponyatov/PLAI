\secrel{Deferring Correctly\\
\ru{Правильная отсрочка}}

Here’s another test \ru{Вот другой тест}:
\lsts{6/3/1.rkt}{rkt}
In our interpreter \ru{В нашем интерпретаторе}, this evaluates to \ru{он
вычисляется в} 7. Should it \ru{Правильно ли это}\,?

Translated into \ru{При переводе на} \racket, this test corresponds to the
following two definitions and expression \ru{этот тест соответствует следующим
двум определениям и выражению}:
\lsts{6/3/2.rkt}{rkt}

What should this produce \ru{Что он должен вычислить}\,? \verb|(f1 3)|
substitutes \ru{подставляет} \verb|x| with \verb|3| in the body of
\ru{в теле} \verb|f1|, which then invokes \ru{что вызывает} \verb|(f2 4)|.
But notably, in \ru{Но заметим что} \verb|f2|, the identifier \ru{идентификатор}
\verb|x| is \emph{not bound} \ru{\emph{не связан}}\,! Sure enough, \racket\ will
produce an error \ru{Будьте уверены, \racket\ выдаст ошибку}.

In fact, so will our substitution-based interpreter \ru{Фактически то же
самое сделает наш интерпретатор с подстановкой}\,!

Why does the substitution process result in an error \ru{Почему процесс
подстановки приведет к ошибке}\,? It’s because \ru{Потому что}, when we replace
the representation of \ru{когда мы заменяем представление} \verb|x| with the
representation of \ru{на предславление} \verb|3| in the representation of \ru{в
представлении} \verb|f1|, we do so in \ru{мы это делаем только в} \verb|f1|
only.
\note{This “the representation of” is getting a little annoying, isn’t it\,?
Therefore, I’ll stop saying that, but do make sure you understand why I had to
say it. It’s an important bit of pedantry.}
\note{\ru{Это ``представление'' несколько надоедает, не так ли\,? Так что я
перестану говорить это, но убедитесь что понимаете почему я должен его говорить.
Это важный элемент педантизма.}}
(Obviously \ru{Очевидно}: \verb|x| is \ru{это параметр} \verb|f1|’s parameter;
even if another function had a parameter named \ru{даже если другая функция
имеет параметр} \verb|x|, that’s a \emph{different} \ru{это \emph{другой}}
\verb|x|.) Thus \ru{Так что}, when we get to evaluating the body of \ru{когда мы
вычисляем тело} \verb|f2|, its \ru{ее} \verb|x| hasn’t been substituted \ru{не
подставляется}, resulting in the error \ru{что приводит к ошибке}.

What went wrong when we switched to environments \ru{Что пошло не так, когда мы
переключились на среды}\,? Watch carefully \ru{Смотрите внимательнее}:
this is subtle \ru{это тонкая штука}. We can focus on applications \ru{Мы можем
сфокусироваться на применениях}, because only they affect the environment
\ru{потому что только они влияют на среду}. When we substituted the formal for
the value of the actual \ru{Когда мы заменили формальный параметр актуальным
значением}, we did so \ru{мы это сделали} by \emph{extending the current
environment} \ru{через \emph{расширение текущей среды}}. In terms of our example
\ru{В терминах нашего примера}, we asked the interpreter \ru{мы попросили
интерпретатор} to substitute not only \ru{заменить не только подстановку}
\verb|f2|’s substitution in \ru{в теле} \verb|f2|’s body, but also the current
ones \ru{но также и текущие вхождения} (those for the caller \ru{для
вызывающей}, \verb|f1|), and indeed all past ones as well \ru{а также вообще все
последующие}. That is, the environment only grows \ru{Так что среда только
растет}; it never shrinks \ru{она никогда не сокращается}.

Because we agreed that environments are only an alternate implementation
strategy for substitution \ru{Так как мы договорились что среды единственная
альтернативная стратегия реализации для подстановки}\ --- and in particular
\ru{в частности}, that the language’s meaning should not change \ru{семантика
языка не должна меняться}\ --- we have to alter the interpreter \ru{мы должны
поправить интерпретатор}. Concretely \ru{Конкретно}, we should not ask it to
carry around all past deferred substitution requests \ru{мы не должны просить
его хранить все отложенные запросы на подстановку}, but instead make it start
afresh for every new function \ru{вместо этого он должен начинать начисто для
каждой новой функции}, just as the substitution-based interpreter does \ru{также
как делает интерпретатор на подстановке}. This is an easy change \ru{Это простая
модификация}:
\lsts{6/3/3.rkt}{rkt}

Now we have truly reproduced the behavior of the substitution interpreter
\ru{Теперь мы на самом деле воспроизвели поведение подстановочного
интерпретатора}.
\note{In case you’re wondering how to write a test case that catches errors,
look up test/exn.}
\note{\ru{Если вам не понятно как написать тест который ловит ошибки,
посмотрите} test/exn}
