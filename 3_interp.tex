\secrel{A First Look at Interpretation \ru{Первый взгляд на
интерпретацию}}\secdown

Now that we have a representation of programs, there are many ways in which we
might want to manipulate them.
\ru{Теперь, когда мы имеем представление программ, существует множество
способов, которыми мы можем манипулировать ими.}
We might want to display a
program in an attractive way \ru{Мы можем захотеть выводить листинг программы в
красивом виде} (“pretty-print”), convert into code in some other format
\ru{преобразовать в код в какой-то другой формат} (“compilation”
\ru{``компиляция''/''трансляция''}), ask whether it obeys certain properties 
\ru{убедиться что она отвечает определенным требованиям}
(“verification” \ru{``верификация''}), and so on \ru{и так далее}.
For now, we’re going to focus on asking what value it corresponds to
(“\termdef{e\underline{val}uation}{evaluation}”\ --- the reduction of programs
to \emph{\underline{val}ues}).
\ru{Для начала, мы собираемся сфокусироваться на вопросе\ --- какому значению
соответствует программа (“\termdef{вычисление}{вычисление}"\ --- редукция
программы до \emph{значения})}

Let’s write an evaluator, in the form of an \termdef{interpreter}{interpreter},
for our arithmetic language.
\ru{Давайте напишем вычислитель, в форме
\termdef{интерпретатора}{интерпретатор}, для нашего арифметического языка.}
We choose arithmetic first for three reasons \ru{Мы выбрали арифметику прежде
всего по следующим трем причинам}:
\begin{itemize}[nosep]
  \item[(a)]
  you already know how it works, so we can focus on the mechanics of writing
evaluators;
\ru{вы уже знаете как работает арифметика, и мы можем сфокусироваться на
механике написания вычислителей;}
  \item[(b)]
  it’s contained in
every language we will encounter later, so we can build upwards and outwards from it;
\ru{она содержится в каждом языке, с которым мы столкнемся в дальнейшем, так
что мы будем расширять этот арифметический язык вверх и вширь;} and \ru{и}
  \item[(c)] 
it’s at once both small and big enough to illustrate many points
we’d like to get across.
\ru{этот язык минималистичен, но при этом достаточно большой, чтобы
проиллюстрировать многие моменты, которые мы хочем до вас донести.}
\end{itemize}

\secrel{Representing Arithmetic \ru{Представление арифметики}}\label{sec31}

Let’s first agree on how we will represent arithmetic expressions.
\ru{Давайте сначала договориться о том, как мы будем представлять арифметические
выражения.}
Let’s say we want to support only two operations\ --- addition and
multiplication\ --- in addition to primitive numbers.
\ru{Допустим, мы хотим поддерживать только две простые операции\ --- сложение и
умножение\ --- в дополнение к примитивным числам.}
We need to represent arithmetic \termdef{expressions}{expression}.
\ru{Нам необходимо представление для арифметических
\termdef{выражений}{выражение}}.
What are the rules that govern nesting of arithmetic expressions\,?
\ru{Какие правила регулируют вложенность арифметических выражений\,?} 
We’re actually free to nest any expression inside another.
\ru{На самом мы свободно можем владывать любое выражение внутрь любого другого.}

\DoNow{
Why did we not include division\,?
\ru{Почему мы не включили умножение\,?}
\\
What impact does it have on the remarks above\,?
\ru{Какое влияние это имеет на замечания выше\,?}
}

We’ve ignored division because it forces us into a discussion of what
expressions we might consider legal:
\ru{Мы игнорировали деление, потому что оно вовлекает нас в дикуссию о том,
какие выражения мы можем считать правильными:}
clearly the representation of $1/2$ ought to be legal;
\ru{ясно что представление $1/2$ должно быть легальным;} 
the representation of $1/0$ is much more debatable;
\ru{представление $1/0$ спорно;}
and that of \ru{и что-то типа} $1/(1-1)$ seems even more controversial.
\ru{кажется гораздо более спорным.}
 
We’d like to sidestep this controversy for now and return to it later
\ru{Мы хотели бы обойти сейчас это противоречие, в вернуться к нему позже}
\ref{}.

Thus, we want a representation for numbers and arbitrarily nestable
addition and multiplication.
\ru{Таким образом нам нужно представление для чисел и произвольно вложенных
сложений и умножений.} 
Here’s one we can use \ru{Вот то что мы можем использовать}
(used in \ru{использовано в} \ref{arithc} ):
\lstx{ArithC}{src/2/p12_1.rkt}{rkt}

\secrel{3.2 Writing an Interpreter . . . . . . . . . . . . . . . . . . . . . . . . . . 14}
\secrel{3.3 Did You Notice? . . . . . . . . . . . . . . . . . . . . . . . . . . . . 15}
\secrel{3.4 Growing the Language . . . . . . . . . . . . . . . . . . . . . . . . . 16}
\secup
