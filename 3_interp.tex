\secrel{A First Look at Interpretation \ru{Первый взгляд на
интерпретацию}}\secdown

Now that we have a representation of programs, there are many ways in which we
might want to manipulate them.
\ru{Теперь, когда мы имеем представление программ, существует множество
способов, которыми мы можем манипулировать ими.}
We might want to display a
program in an attractive way \ru{Мы можем захотеть выводить листинг программы в
красивом виде} (“pretty-print”), convert into code in some other format
\ru{преобразовать в код в какой-то другой формат} (“compilation”
\ru{``компиляция''/''трансляция''}), ask whether it obeys certain properties 
\ru{убедиться что она отвечает определенным требованиям}
(“verification” \ru{``верификация''}), and so on \ru{и так далее}.
For now, we’re going to focus on asking what value it corresponds to
(“\termdef{e\underline{val}uation}{evaluation}”\ --- the reduction of programs
to \emph{\underline{val}ues}).
\ru{Для начала, мы собираемся сфокусироваться на вопросе\ --- какому значению
соответствует программа (“\termdef{вычисление}{вычисление}"\ --- редукция
программы до \emph{значения})}

Let’s write an evaluator, in the form of an \termdef{interpreter}{interpreter},
for our arithmetic language.
\ru{Давайте напишем вычислитель, в форме
\termdef{интерпретатора}{интерпретатор}, для нашего арифметического языка.}
We choose arithmetic first for three reasons \ru{Мы выбрали арифметику прежде
всего по следующим трем причинам}:
\begin{itemize}[nosep]
  \item[(a)]
  you already know how it works, so we can focus on the mechanics of writing
evaluators;
\ru{вы уже знаете как работает арифметика, и мы можем сфокусироваться на
механике написания вычислителей;}
  \item[(b)]
  it’s contained in
every language we will encounter later, so we can build upwards and outwards from it;
\ru{она содержится в каждом языке, с которым мы столкнемся в дальнейшем, так
что мы будем расширять этот арифметический язык вверх и вширь;} and \ru{и}
  \item[(c)] 
it’s at once both small and big enough to illustrate many points
we’d like to get across.
\ru{этот язык минималистичен, но при этом достаточно большой, чтобы
проиллюстрировать многие моменты, которые мы хочем до вас донести.}
\end{itemize}

\secrel{Representing Arithmetic \ru{Представление арифметики}}\label{sec31}

Let’s first agree on how we will represent arithmetic expressions.
\ru{Давайте сначала договориться о том, как мы будем представлять арифметические
выражения.}
Let’s say we want to support only two operations\ --- addition and
multiplication\ --- in addition to primitive numbers.
\ru{Допустим, мы хотим поддерживать только две простые операции\ --- сложение и
умножение\ --- в дополнение к примитивным числам.}
We need to represent arithmetic \termdef{expressions}{expression}.
\ru{Нам необходимо представление для арифметических
\termdef{выражений}{выражение}}.
What are the rules that govern nesting of arithmetic expressions\,?
\ru{Какие правила регулируют вложенность арифметических выражений\,?} 
We’re actually free to nest any expression inside another.
\ru{На самом мы свободно можем владывать любое выражение внутрь любого другого.}

\DoNow{
Why did we not include division\,?
\ru{Почему мы не включили умножение\,?}
\\
What impact does it have on the remarks above\,?
\ru{Какое влияние это имеет на замечания выше\,?}
}

We’ve ignored division because it forces us into a discussion of what
expressions we might consider legal:
\ru{Мы игнорировали деление, потому что оно вовлекает нас в дикуссию о том,
какие выражения мы можем считать правильными:}
clearly the representation of $1/2$ ought to be legal;
\ru{ясно что представление $1/2$ должно быть легальным;} 
the representation of $1/0$ is much more debatable;
\ru{представление $1/0$ спорно;}
and that of \ru{и что-то типа} $1/(1-1)$ seems even more controversial.
\ru{кажется гораздо более спорным.}
 
We’d like to sidestep this controversy for now and return to it later
\ru{Мы хотели бы обойти сейчас это противоречие, в вернуться к нему позже}
\ref{}.

Thus, we want a representation for numbers and arbitrarily nestable
addition and multiplication.
\ru{Таким образом нам нужно представление для чисел и произвольно вложенных
сложений и умножений.} 
Here’s one we can use \ru{Вот то что мы можем использовать}
(used in \ru{использовано в} \ref{arithc} ):
\lstx{ArithC}{src/2/p12_1.rkt}{rkt}

\secrel{Writing an Interpreter \ru{Написание интерпретатора}}

Now let’s write an interpreter for this arithmetic language.
\ru{Теперь давайте напишем интерпретатор для этого арифметического языка.} 
First, we should think about what its type is.
\ru{Для начала нам надо подумать, какие типы он использует\,?} 
It clearly consumes a \verb|ArithC| value.
\ru{Совершенно ясно что на вход подается структура типа} \verb|ArithC|. 
What does it produce\,?
\ru{Что он возвращает\,?} 
Well, an interpreter evaluates
\ru{Ну, интерпретатор вычисляет} \ --- and what kind of value might arithmetic
expressions reduce to\,?
\ru{и к какому значению может редуцироваться арифметическое выражение\,?} 
Numbers, of course.
\ru{Конечно, числу.} 
So the interpreter is going to be a function from arithmetic expressions to
numbers.
\ru{Таким образом, интерпретатор должен быть функцией от арифметического
выражения, возвращающей число.}

\Exercise{
Write your test cases for the interpreter.
\ru{Напишите тесты для интерпретатора.}
}

Because we have a recursive datatype, it is natural to structure our interpreter
as a recursive function over it.
\ru{Так как мы имеем рекурсивный тип данных\note{допускаются произвольные
вложения того же типа}, нормально что структура нашего интерпретатора тоже
должна быть рекурсивной функций над выражением.}
Here’s a first template\note{Templates are
explained in great detail in \emph{How to Design Programs}.}
\ru{Вот первый шаблон\note{\ru{Шаблоны очень детально описаны в \emph{How to
Design Programs}}}} :
\lstx{ArithC.rkt}{src/3/p14_1.rkt}{rkt}
You’re probably tempted to jump straight to code, which you can:
\ru{Вероятно у вас есть соблазн сразу перейти к коду, который вы можете
написать:}
\lstx{ArithC.rkt}{src/3/p14_2.rkt}{rkt}

\DoNow{
Do you spot the errors\,?
\ru{Вы увидели ошибки\,?}
}

Instead, let’s expand the template out a step:
\ru{Вместо этого давайте расширим шаблон на один шаг:}
\lstx{ArithC.rkt}{src/3/p15_2.rkt}{rkt}
and now we can fill in the blanks:
\ru{и теперь мы можем заполнить пробелы:}
\lstx{ArithC.rkt}{src/3/p15_3.rkt}{rkt}

Later on \ru{Позже в} \ref{}, we’re going to wish we had returned a more complex
datatype than just numbers.
\ru{мы пожелаем возвращать более сложный тип данных, чем просто числа.}
But for now, this will do.
\ru{Но сейчас нам этого достаточно.}

Congratulations: you’ve written your first interpreter\,!
\ru{Поздравляем: вы только что написали свой первый интерпретатор\,!} 
I know, it’s very nearly an anticlimax\note{\ru{ситуация, когда проблема
казавшеяся очень сложной, решается с помощью чего-то тривиального //
Wikipedia}}\note{\ru{род морских улиток // там же}}.
\ru{Я знаю, это очень близко к разочарованию.}
But they’ll get harder\ --- much harder\ --- pretty soon, I promise.
\ru{Но все станет жестче\ --- намного жестче\ --- совсем скоро, я обещаю.}

\secrel{Did You Notice\,? \ru{Вы заметили\,?}}

I just slipped something by you:
\ru{Я только что утаил что-то от вас:}
\DoNow{
What is the ``meaning'' of addition and multiplication in this new language\,?
\ru{Каков ``смысл'' сложения и умножения в этом новом языке\,?}
}

That’s a pretty abstract question, isn’t it.
\ru{Это достаточно абстрактный вопрос, не так ли.} 
Let’s make it concrete.
\ru{Давайте его конкретизируем.} 
There are many kinds of addition in computer science:
\ru{В информатике существует множество видов сложения:}

\begin{itemize}
  \item 
First of all, there’s many different kinds of \termdef{numbers}{number}:
\ru{Прежде всего, существует множество видов \termdef{чисел}{число}:}
fixed-width \ru{фиксированной длины} (e.g., 32- bit \ru{например 32-битные})
integers \ru{целые}, signed fixed-width \ru{знаковые фиксированной длины} (e.g.,
31-bits plus a sign-bit \ru{например 31-битные плюс бит знака}) integers
\ru{целые}, arbitrary precision integers \ru{целые числа произвольной точности};
in some languages, rationals \ru{в некоторых языках\ --- натуральные дроби};
various formats of fixed- and floating-point numbers
\ru{различные форматы чисел с фиксированной и плавающей точкой}; in some
languages, complex numbers \ru{в некоторых языках комплектные числа}; and so on
\ru{и так далее}.
After the numbers have been chosen, addition may support only some combinations
of them.
\ru{После того как были выбраны определенные виды чисел, сложение может
поддерживать только некоторые их комбинации.}
  \item 
In addition, some languages permit the addition of datatypes such as matrices.
\ru{В дополнение, некоторые языки поддерживают сложение таких типов данных, как
матрицы.}
  \item 
Furthermore, many languages support ``addition'' of strings
\ru{Кроме того, многие языки поддерживают ``сложение'' строк} (
we use scare-quotes because we don’t really mean the mathematical concept of
addition, but rather the operation performed by an operator with the syntax +
\ru{Мы используем кавычки, так как предполагаем не математическую идею
сложения, а операцию, выполняемую оператором с синтаксисом +} ). 
In some languages this always means concatenation;
\ru{В некоторых языках это всегда значит конкатенацию строк;} 
in some others, it can result in numeric results (or numbers stored in strings).
\ru{в некоторых долбанутых языках иногда может получиться численный
результат (или числа хранимые в строках).}
\end{itemize}

These are all different meanings for addition. 
\ru{Все это является смыслом сложения.}
\termdef{Semantics}{semantics}
is the mapping of \emph{syntax} (e.g., +) to \emph{meaning} (e.g., some or all
of the above).
\ru{\termdef{Семантика}{семантика} это отображение \emph{синтаксиса} (например
+) на \emph{смысл} (например что-то или все из вышеперечисленного).}

This brings us to our first game of:
\ru{Это подводит нас к первоначальной игре:}
\begin{description}\item[\emph{Which of these is the same\,?} \ru{Что из
этого дает одинаковый результат\,?}]\
\\
\begin{itemize}[nosep]
  \item 1 + 2
  \item 1 + 2
  \item '1' + '2'
  \item '1' + '2'
\end{itemize}
\end{description}

Now return to the question above.
\ru{Теперь возвращаемся к предыдущему вопросу.} 
What semantics do we have\,?
\ru{Какую семантику мы имеем\,?}
We’ve adopted whatever semantics \racket\ provides, because we map + to
\racket’s +. 
\ru{Мы приняли ту же семантику, которую предоставляет \racket, потому что мы
отоборазили наш + на + в \racket е.}
In fact that’s not even quite true:
\ru{На самом деле это даже не совсем верно:} 
\racket\ may, for all we know, also enable +
to apply to strings, so we’ve chosen the restriction of \racket’s semantics to
numbers\note{though in fact \racket’s + doesn’t tolerate strings}.
\ru{\racket\ может, как все мы знаем, также использовать + к строкам, так что
мы выбрали ограничение семантики \racket а только для чисел\note{\ru{хотя на
самом деле \racket ский + нетолерантен к строкам}}.}

If we wanted a different semantics, we’d have to implement it explicitly.
\ru{Если мы хотим другую семантику, мы должны реализовать ее в явном виде.}

\Exercise{
What all would you have to change so that the number had signed- 32-bit
arithmetic\,?
\\
\ru{Что мы должны изменить, чтобы числа поддерживали знаковую 32-битную
арифметику\,?}
}

In general, we have to be careful about too readily borrowing from the host
language.
\ru{В общем, мы должны быть осторожными с заимствованиями из языка-носителя.}
We’ll return to this topic later \ru{Мы вернемся к этой теме позже} \ref{}.

\secrel{3.4 Growing the Language . . . . . . . . . . . . . . . . . . . . . . . . . 16}

\secup
