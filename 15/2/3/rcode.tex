\secrel{15.2.3 Recursion in Code  . . 139}

Now that we’ve obtained a basic programming language, let’s add recursion to it.
We saw earlier \ref{}\ that this could be done easily through desugaring. It’ll
prove to be a more complex story here.

\secdown
\secrel{A First Attempt at Typing Recursion}

Let’s now try to express a simple recursive function. The simplest is, of
course, one that loops forever. Can we write an infinite loop with just
functions? We already could simply with this program\ ---
\lsts{15/2/3/1.rkt}{rkt}
--- which we know we can represent in our language with functions as values.

\Exercise{
Why does this construct an infinite loop? What subtle dependency is it
making about the nature of function calls?
}

Now that we have a typed language, and one that forces us to annotate all
functions, let’s annotate it. For simplicity, from now on we’ll assume we’re
writing programs in a typed surface syntax, and that desugaring takes care of
constructing core language terms.

Observe, first, that we have two identical terms being applied to each other.
Historically, the overall term is called (capital omega in Greek) and each of
the identical sub-terms is called ! (lower-case omega in Greek). It is not a
given that identical terms must have precisely the same type, because it depends
on what invariants we want to assert of the context of use. In this case,
however, observe that x binds to !, so the second ! goes into both the first and
second positions. As a result, typing one effectively types both.

Therefore, let’s try to type !; let’s call this type 
. It’s clearly a function type, and
the function takes one argument, so it must be of the form. Now what is
% that argument? It’s ! itself. That is, the type of the value going into  is itself 
% . Thus, the
% type of ! is 
% , which is  ->  , which expands into ( ->  ) ->  , which further
% expands to (( ->  ) ->  ) ->  , and so on. In other words, this type cannot be
written as any finite string!

\DoNow{
Did you notice the subtle but important leap we just made?
}

\secrel{Program Termination}

We observed that the obvious typing of , which entails typing , seems to run
into serious problems. From that, however, we jumped to the conclusion that this
type cannot be written as any finite string, for which we’ve given only an
intuition, not a proof. In fact, something even stranger is true: in the type
system we’ve defined so far, we cannot type at all!
 
This is a strong statement, but we can actually say something much stronger. The
typed language we have so far has a property called strong normalization: every
expression that has a type will terminate computation after a finite number of
steps. In other words, this special (and peculiar) infinite loop program isn’t
the only one we can’t type; we can’t type any infinite loop (or even potential
infinite loop). A rough intuition that might help is that any type—which must be
a finite string—can have only a finite number of ->’s in it, and each
application discharges one, so we can perform only a finite number of
applications.

If our language permitted only straight-line programs, this would be
unsurprising. However, we have conditionals and even functions being passed
around as values, and with those we can encode any datatype we want. Yet, we
still get this guarantee! That makes this a somewhat astonishing result.

\Exercise{
Try to encode lists using functions in the untyped and then in the typed
language. What do you see? And what does that tell you about the impact of this
type system on the encoding?
}

This result also says something deeper. It shows that, contrary to what you may
believe—that a type system only prevents a few buggy programs from running—a
type system can change the semantics of a language. Whereas previously we could
write an infinite loop in just one to two lines, now we cannot write one at all.
It also shows that the type system can establish invariants not just about a
particular program, but about the language itself. If we want to absolutely
ensure that a program will terminate, we simply need to write it in this
language and pass the type checker, and the guarantee is ours!

What possible use is a language in which all programs terminate? For
generalpurpose programming, none, of course. But in many specialized domains,
it’s a tremendously useful guarantee to have. For instance, suppose you are
implementing a complex scheduling algorithm; you would like to know that your
scheduler is guaranteed to terminate so that the tasks being scheduled will
actually run. There are many other domains, too, where we would benefit from
such a guarantee: a packet-filter in a router; a real-time event processor; a
device initializer; a configuration file; the callbacks in single-threaded
JavaScript; and even a compiler or linker. In each case, we have an almost
unstated expectation that these programs will always terminate. And now we have
a language that can offer this guarantee—something it is impossible to test for,
no less!
\note{These are not hypothetical examples. In the Standard ML language, the
language for linking modules uses essentially this typed language for writing
module linking specifications. This means developers can write quite
sophisticated abstractions—they have functions-as-values, after all!—while still
being guaranteed that linking will always terminate, producing a program.}

\secrel{Typing Recursion}

What this says is, whereas before we were able to handle rec entirely through
desugaring, now we must make it an explicit part of the typed language. For
simplicity, we will consider a special case of rec—which nevertheless covers the
common uses— whereby the recursive identifier is bound to a function. Thus, in
the surface syntax, one might write
\lsts{15/2/3/2.rkt}{rkt}
for a summation function, where $\Sigma$ is the name of the function, n its
argument, and num the type consumed by and returned from the function. The expression
($\Sigma\ 10$) represents the use of this function to sum the number from 10
until 0.

How do we type such an expression? Clearly, we must have n bound in the body of
the function as we type it (but not of course, in the use of the function); this
much we know from typing functions. But what about $\Sigma$? Obviously it must
be bound in the type environment when checking the use (($\Sigma\ 10$)), and its
type must be num
% ->
num. But it must also be bound, to the same type, when checking the body of the
function. (Observe, too, that the type returned by the body must match its
declared return type.)

Now we can see how to break the shackles of the finiteness of the type. It is
certainly true that we can write only a finite number of ->’s in types in the
program source. However, this rule for typing recursion duplicates the -> in the
body that refers to itself, thereby ensuring that there is an inexhaustible
supply of applications. It’s our infinite quiver of arrows.

The code to implement this rule would be as follows. Assuming f is bound to the
function’s name, aT is the function’s argument type and rT is its return type, b
is the function’s body, and u is the function’s use:
\lsts{15/2/3/3.rkt}{rkt}

\secup
