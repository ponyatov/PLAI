\secrel{15.2.1 A Simple Type Checker  . . 134}

Before we can define a type checker, we have to fix two things: the syntax of
our typed core language and, hand-in-hand with that, the syntax of types
themselves.

To begin with, we’ll return to our language with functions-as-values \ref{}\ but
before we added mutation and other complications (some of which we’ll return to
later). To this language we have to add type annotations. Conventionally, we
don’t impose type annotations on constants or on primitive operations such as
addition; instead, we impose them on the boundaries of functions or methods.
Over the course of this study we will explore why this is a good locus for
annotations.

Given this decision, our typed core language becomes:
\lsts{15/2/1/1.rkt}{rkt}
That is, every procedure is annotated with the type of argument it expects and
type of argument it purports to produce.

Now we have to decide on a language of types. To do so, we follow the tradition
that the types abstract over the set of values. In our language, we have two
kinds of values:
\lsts{15/2/1/2.rkt}{rkt}
It follows that we should have two kinds of types: one for numbers and the other
for functions.

Even numeric types may not be straightforward: What information does a number
type need to record? In most languages, there are actually many numeric types,
and indeed there may not even be a single one that represents “numbers”.
However, we have ignored these gradations between numbers \ref{}, so it’s
sufficient for us to have just one. Having decided that, do we record additional
information about which number? We could in principle, but we would soon run
into decidability problems.

As for functions, we have more information: the type of expected argument, and
the type of claimed result. We might as well record this information we have
been given until and unless it has proven to not be useful. Combining these, we
obtain the following abstract language of types:
\lsts{15/2/1/3.rkt}{rkt}
Now that we’ve fixed both the term and type structure of the language, let’s
make sure we agree on what constitute type errors in our language (and, by fiat,
everything not a type error must pass the type checker). There are three obvious
forms of type errors:
\begin{itemize}[nosep]
  \item 
One or both arguments of + is not a number, i.e., is not a numT.
  \item 
One or both arguments of * is not a number.
  \item 
The expression in the function position of an application is not a function,
i.e., is not a funT.

\DoNow{
Any more?
}

We’re actually missing one:

  \item 
The expression in the function position of an application is a function but the
type of the actual argument does not match the type of the formal argument
expected by the function.
\end{itemize}
It seems clear all other programs in our language ought to type-check.

A natural starting signature for the type-checker would be that it is a
procedure that consumes an expression and returns a boolean value indicating
whether or not the expression type-checked. Because we know expressions contain
identifiers, it soon becomes clear that we will want a type environment, which
maps names to types, analogous to the value environment we have seen so far.

\Exercise{
Define the types and functions associated with type environments.
}

Thus, we might begin our program as follows:
\lsts{15/2/1/4.rkt}{rkt}

As the abbreviated set of cases above suggests, this approach will not work out.
We’ll soon see why.

Let’s begin with the easy case: numbers. Does a number type-check? Well, on its
own, of course it does; it may be that the surrounding context is not expecting
a number, but that error would be signaled elsewhere. Thus:
\lsts{15/2/1/5.rkt}{rkt}

Now let’s handle identifiers. Is an identifier well-typed? Again, on its own it
would appear to be, provided it is actually a bound identifier; it may not be
what the context desires, but hopefully that too would be handled elsewhere.
Thus we might write
\lsts{15/2/1/6.rkt}{rkt}

This should make you a little uncomfortable: lookup already signals an error if
an identifier isn’t bound, so there’s no need to repeat it (indeed, we will
never get to this error invocation). But let’s push on.

Now we tackle applications. We should type-check both the function part, to make
sure it’s a function, then ensure that the actual argument’s type is consistent
with what the function declares to be the type of its formal argument. For
instance, the function could be a number and the application could itself be a
function, or vice versa, and in either case we want to prevent such
mis-applications.

How does the code look?
\lsts{15/2/1/7.rkt}{rkt}

The recursive call to tc can only tell us whether the function expression
typechecks or not. If it does, how do we know what type it has? If we have an
immediate function, we could reach into its syntax and pull out the argument
(and return) types. But if we have a complex expression, we need some procedure
that will calculate the resulting type of that expression. Of course, such a
procedure could only provide a type if the expression is well-typed; otherwise
it would not be able to provide a coherent answer. In other words, our type
“calculator” has type “checking” as a special case! We should therefore
strengthen the inductive invariant on tc: that it not only tells us whether an
expression is typed but also what its type is. Indeed, by giving any type at all
it confirms that the expression types, and otherwise it signals an error.

Let’s now define this richer notion of a type “checker”.
\lsts{15/2/1/8.rkt}{rkt}

Now let’s fill in the pieces. Numbers are easy: they have the numeric type.
\lsts{15/2/1/9.rkt}{rkt}

Similarly, identifiers have whatever type the environment says they do (and if
they aren’t bound, this signals an error).
\lsts{15/2/1/10.rkt}{rkt}

Observe, so far, the similarity to and difference from interpreting: in the
identifier case we did essentially the same thing (except we returned a type
rather than an actual value), whereas in the numeric case we returned the
abstract “number” rather than indicate which specific number it was.

Let’s now examine addition. We must make sure both sub-expressions have numeric
type; only if they do will the overall expression evaluate to a number itself.
\lsts{15/2/1/11.rkt}{rkt}

We’ve usually glossed over multiplication after considering addition, but now it
will be instructive to handle it explicitly:
\lsts{15/2/1/12.rkt}{rkt}

\DoNow{
Did you see what’s different?
}

That’s right: practically nothing! (The multC instead of plusC in the type-case,
and the slightly different error message, notwithstanding.) That’s because, from
the perspective of type-checking (in this type language), there is no difference
between addition and multiplication, or indeed between any two functions that
consume two numbers and return one.

Observe another difference between interpreting and type-checking. Both care
that the arguments be numbers. The interpreter then returns a precise sum or
product, but the type-checker is indifferent to the differences between them:
therefore the expression that computes what it returns ((numT)) is a constant,
and the same constant in both cases.

Finally, the two hard cases: application and funcions. We’ve already discussed
what application must do: compute the value of the function and argument
expressions; ensure the function expression has function type; and check that
the argument expression is of compatible type. If all this holds up, then the
type of the overall application is whatever type the function body would return
(because the value that eventually returns at run-time is the result of
evaluating the function’s body).
\lsts{15/2/1/13.rkt}{rkt}

That leaves function definitions. The function has a formal parameter, which is
presumably used in the body; unless this is bound in the environment, the body
most probably will not type-check properly. Thus we have to extend the type
environment with the formal name bound to its type, and in that extended
environment type-check the body. Whatever value this computes must be the same
as the declared type of the body. If that is so, then the function itself has a
function type from the type of the argument to the type of the body.

\Exercise{
Why do I say “most probably” above?
\lsts{15/2/1/14.rkt}{rkt}
}

Observe another curious difference between the interpreter and type-checker. In
the interpreter, application was responsible for evaluating the argument
expression, extending the environment, and evaluating the body. Here, the
application case does check the argument expression, but leaves the environment
alone, and simply returns the type of the body without traversing it. Instead,
the body is actually traversed by the checker when checking a function
definition, so this is the point at which the environment actually extends.
