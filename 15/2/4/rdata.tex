\secrel{15.2.4 Recursion in Data   142}

We have seen how to type recursive programs, but this doesn’t yet enable us to
create recursive data. We already have one kind of recursive datum—the function
type—but this is built-in. We haven’t yet seen how developers can create their
own recursive datatypes.

\secdown

\secrel{Recursive Datatype Definitions}

When we speak of allowing programmers to create recursive data, we are actually
talking about three different facilities at once:
\begin{itemize}[nosep]
  \item 
Creating a new type.
  \item 
Letting instances of the new type have one or more fields.
  \item 
Letting some of these fields refer to instances of the same type.
\end{itemize}
In fact, once we allow the third, we must allow one more:
\begin{itemize}[nosep]
  \item Allowing non-recursive base-cases for the type.
\end{itemize}
This confluence of design criteria leads to what is commonly called an algebraic
datatype, such as the types supported by our typed language. For instance,
consider the following definition of a binary tree of numbers:
\note{Later \ref{}, we will discuss how types can be parameterized.}
\lsts{15/2/4/1.rkt}{rkt}
Observe that without a name for the new datatype, BTnum, we would not have been
able to refer back ot it in BTnd. Similarly, without the ability to have more
than one kind of BTnum, we would not have been able to define BTmt, and thus
wouldn’t have been able to terminate the recursion. Finally, of course, we need
multiple fields (as in BTnd) to construct useful and interesting data. In other
words, all three mechanisms are packaged together because they are most useful
in conjunction. (However, some langauges do permit the definition of stand-alone
structures. We will return to the impact of this design decision on the type
system later \ref{}.)

This concludes our initial presentation of recursive types, but it has a fatal
problem. We have not actually explained where this new type, BTnum, comes from.
That is because we have had to pretend it is baked into our type-checker.
However, it is simply impractical to keep changing our type-checker for each new
recursive type definition— it would be like modifying our interpreter each time
the program contains a recursive function! Instead, we need to find a way to
make such definitions intrinsic to the type language. We will return to this
problem later \ref{}.

This style of data definition is sometimes also known as a sum of products.
“Product” refers to the way fields combine in one variant: for instance, the
legal values of a BTnd are the cross-product of legal values in each of the
fields, supplied to the BTnd constructor. The “sum” is the aggregate of all
these variants: any given BTnum value is just one of these. (Think of “product”
as being “and”, and “sum” as being “or”.)

\secrel{Introduced Types}

Now, what impact does a datatype definition have? First, it introduces a new
type; then it uses this to define several constructors, predicates, and
selectors. For instance, in the above example, it first introduces BTnum, then
uses it to ascribe the following types:
\lsts{15/2/4/2.rkt}{rkt}

Observe a few salient facts:
\begin{itemize}
  \item 
Both the constructors create instances of BTnum, not something more refined.
We will discuss this design tradeoff later \ref{}.
  \item 
Both predicates consume values of type BTnum, not “any”. This is because the
type system can already tell us what type a value is. Thus, we only need to
distinguish between the variants of that one type.
  \item 
The selectors really only work on instances of the relevant variant—e.g., BTnd-
n can work only on instances of BTnd, not on instances of BTmt—but we don’t have
a way to express this in the static type system for lack of a suitable static
type. Thus, applying these can only result in a dynamic error, not a static one
caught by the type system.
\end{itemize}
There is more to say about recursive types, which we will return to shortly
\ref{}.

\secrel{Pattern-Matching and Desugaring}

Once we observe that these are the types, the only thing left is to provide an
account of pattern-matching. For instance, we can write the expression
\lsts{15/2/4/3.rkt}{rkt}
We have already seen \ref{}\ that this can be written in terms of the functions
defined above. We can simulate the binding done by this pattern-matcher using
let:
\lsts{15/2/4/4.rkt}{rkt}
In short, this can be done by a macro, so pattern-matching does not need to be
in the core language and can instead be delegated to desugaring. This, in turn,
means that one language can have many different pattern-matching mechanisms.

Except, that’s not quite true. Somehow, the macro that generates the code above
in terms of cond needs to know that the three positional selectors for a BTnd
are BTnd-n, BTnd-l, and BTnd-r, respectively. This information is explicit in
the type definition but only implicitly present in the use of the
pattern-matcher (that, indeed, being the point). Thus, somehow this information
must be communicated from definition to use. Thus the macro expander needs
something akin to the type environment to accomplish its task.

Observe, furthermore, that expressions such as e1 and e2 cannot be type-checked—
indeed, cannot even be reliable identified as expressions—until macro expansion
expands the use of type-case. Thus, expansion depends on the type environment,
while type-checking depends on the result of expansion. In other words, the two
are symbiotic and need to happen, not quite in “parallel”, but rather in
lock-step. Thus, building desugaring for a typed language, where the syntactic
sugar makes assumptions about types, is a little more intricate than doing so
for an untyped language.

\secup
