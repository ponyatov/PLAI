\secrel{15.2.6 Types and Mutation  . . 146}

We have now covered most of the basic features of our core language other than
mutation. In some ways, types have a simple interaction with mutation, and this
is because in a classical setting, they don’t interact at all. Consider, for
instance, the following untyped program:
\lsts{15/2/6/1.rkt}{rkt}
What is “the type” of x? It doesn’t really have one: for some time it’s a
number, and later (note the temporal word) it’s a string. We simply can’t give
it a type. In general, type checking is an atemporal activity: it is done once,
before the program runs, and must hence be independent of the specific order in
which programs execute. Keeping track of the precise values in the store is
hence beyond the reach of a type-checker.

The example above is, of course, easy to statically understand, but we should
never be mislead by simple examples. Suppose instead we had a program like
\lsts{15/2/6/2.rkt}{rkt}
Now it is literally impossible to reach any static conclusion about the type of
x after the conditional finishes, because only at run-time can we know what the
user might have entered.

To avoid this morass, traditional type checkers adopt a simple policy: types
must be invariant across mutation. That is, a mutation operation—whether
variable mutation or structure mutation—cannot change the type of the mutant.
Thus, the above examples would not type in our type language so far. How much
flexibility this gives the programmer is, however, a function of the type
language. For instance, if we were to admit a more flexible type that stands for
“number or string”, then the examples above would type, but x would always have
this, less precise, type, and all uses of x would have to contend with its
reduced specificity, an issue we will return to later \ref{}.

In short, mutation is easy to account for in a traditional type system because
its rule is simply that, while the value can change in ways below the level of
specificity of the type system, the type cannot change. In the case of an
operation like set! (or our core language’s setC), this means the type of the
assigned value must match that of the variable. In the case of structure
mutation, such as boxes, it means the assigned value must match that the box’s
contained type.
