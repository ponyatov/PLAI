\secrel{15.3.6 Recursive Types   172}

Now that we’ve seen union types, it pays to return to our original recursive
datatype formulation. If we accept the variants as type constructors, can we
write the recursive type as a union over these? For instance, returning to
BTnum, shouldn’t we be able to describe it as equivalent to
\lsts{15/3/6/1.rkt}{rkt}
thereby showing that BTmt is a zero-ary constructor, and BTnd takes three
parameters? Except, what are the types of those three parameters? In the type
we’ve written above, BTnum is either built into the type language (which is
unsatisfactory) or unbound. Perhaps we mean
\lsts{15/3/6/2.rkt}{rkt}
Except now we have an equation that has no obvious solution (remember !?).

This situation should be familiar from recursion in values \ref{}. Then, we
invented a recursive function constructor (and showed its implementation) to
circumvent this problem. We similarly need a recursive type constructor. This is
conventionally called $\mu$ (the Greek letter “mu”). With it, we would write the
above type as
\lsts{15/3/6/3.rkt}{rkt}
$\mu$ is a binding construct; it binds BTnum to the entire type written after
it, including the recursive binding of BTnum itself. In practice, of course,
this entire recursive type is the one we wish to call BTnum:
\lsts{15/3/6/4.rkt}{rkt}
Though this looks like a circular definition, notice that the name BTnum on the
right does not depend on the one to the left of the equation: i.e., we could
rewrite this as
\lsts{15/3/6/5.rkt}{rkt}
In other words, this definition of BTnum truly can be thought of as syntactic
sugar and replaced everywhere in the program without fear of infinite regress.

At a semantic level, there are usually two very different ways of thinking about
the meaning of types bound by $mu$: they can be interpreted as isorecursive or
equirecursive. The distinction between these is, however, subtle and beyond the
scope of this chapter. It suffices to note that a recursive type can be treated
as equivalent to its unfolding. For instance, if we define a numeric list type
as
\note{This material is covered especially well in Pierce’s book.}
\lsts{15/3/6/6.rkt}{rkt}
then
\lsts{15/3/6/7.rkt}{rkt}
and so on (iso- and equi-recursiveness differ in precisely what the notion of
equality is: definitional equality or isomorphism). At each step we simply
replace the T parameter with the entire type. As with value recursion, this
means we can “get another” ConsL constructor upon demand. Put differently, the
type of a list can be written as the union of zero or arbitrarily many elements;
this is the same as the type that consists of zero, one, or arbitrarily many
elements; and so on. Any lists of numbers fits all (and precisely) these types.

Observe that even with this informal understanding of $\mu$, we can now provide
a type to $\omega$, and hence to $\Omega$. 

\Exercise{
Ascribe types to $\omega$ and $\Omega$ 
}
