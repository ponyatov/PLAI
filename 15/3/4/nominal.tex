\secrel{15.3.4 Nominal Versus Structural Systems  . . 170}

In our initial type-checker, two types were considered equivalent if they had
the same structure. In fact, we offered no mechanism for naming types at all, so
it is not clear what alternative we had.

Now consider Typed Racket. A developer can write
\lsts{15/3/4/1.rkt}{rkt}
followed by
\lsts{15/3/4/2.rkt}{rkt}
Suppose the developer also defines the function
\lsts{15/3/4/3.rkt}{rkt}
and tries to apply f to v, i.e., (f v): should this application type or not?

There are two perfectly reasonable interpretations. One is to say that v was
declared to be of type NB1, which is a different name than NB2, and hence should
be considered a different type, so the above application should result in an
error. Such a system is called nominal, because the name of a type is paramount
for determining type equality.

In contrast, another interpretation is that because the structure of NB1 and NB2
are identical, there is no way for a developer to write a program that behaves
differently on values of these two types, so these two types should be
considered identical. Such a type system is called structural, and would
successfully type the above expression. (Typed Racket follows a structural
discipline, again to reduce the burden of importing existing untyped code,
which—in Racket—is usually written with a structural interpretation in mind. In
fact, Typed Racket not only types (f v), it prints the result as having type
NB1, despite the return type annotation on f!)
\note{If you want to get especially careful, you would note that there is a
difference between being considered the same and actually being the same. We
won’t go into this issue here, but consider the implication for a compiler
writer choosing representations of values, especially in a language that allows
run-time inspection of the static types of values.}

The difference between nominal and structural typing is most commonly
contentious in object-oriented languages, and we will return to this issue
briefly later \ref{}. However, the point of this section is to illustrate that
these questions are not intrinsically about “objects”. Any language that permits
types to be named—as all must, for programmer sanity—must contend with this
question: is naming merely a convenience, or are the choices of names intended
to be meaningful? Choosing the former answer leads to structural typing, while
choosing the latter leads down the nominal path.
