\secrel{15.3.5 Intersection Types  . . 171}

Since we’ve just explored union types, you must naturally wonder whether there
are also intersection types. Indeed there are.

If a union type means that a value (of that type) belongs to one of the types in
the union, an intersection type clearly means the value belongs to all the types
in the intersection: a conjunction, or “and”. This might seem strange: how can a
value belong to more than one type?

As a concrete answer, consider overloaded functions. For instance, in some
languages + operates on both numbers and strings; given two numbers it produces
a number, and given two strings it produces a string. In such a language, what
is the proper type for +? It is not (number number -> number) alone, because
that would reject its use on strings. By the same reasoning, it is not (string
string -> string) alone either. It is not even
\lsts{15/3/5/1.rkt}{rkt}
because + is not just one of these functions: it truly is both of them. We could
ascribe the type
\lsts{15/3/5/2.rkt}{rkt}
reflecting the fact that each argument, and the result, can be only one of these
types, not both. Doing so, however, leads to a loss of precision.

\DoNow{
In what way does this type lose precision?
}

Observe that with this type, the return type on all invocations is (number U
string). Thus, on every return we must distinguish beween numeric and string
returns, or else we will get a type error. Thus, even though we know that if
given two numeric arguments we will get a numeric result, this information is
lost to the type system.

More subtly, this type permits each argument’s type to be chosen independently
of the other. Thus, according to this type, the invocation (+ 3 "x") is
perfectly valid (and produces a value of type (number U string)). But of course
the addition operation we have specified is not defined for these inputs at all!

Thus the proper type to ascribe this form of addition is
\lsts{15/3/5/3.rkt}{rkt}
where $\Lambda$ should be reminiscent of the conjunction operator in logic. This
permits invocation with two numbers or two strings, but nothing else. An
invocation with two numbers has a numeric result type; one with two strings has
a string result type; and nothing else. This corresponds precisely to our
intended behavior for overloading (sometimes also called ad hoc polymorphism).
Observe that this only handles a finite number of overloaded cases.
