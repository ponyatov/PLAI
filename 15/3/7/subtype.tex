\secrel{15.3.7 Subtyping   . 173}

Imagine we have a typical binary tree definition; for simplicity, we’ll assume
that all the values are numbers. We will write this in Typed Racket to
illustrate a point:
\lsts{15/3/7/1.rkt}{rkt}
Now consider some concrete tree values:
\lst{15/3/7/2.rkt}
Observe that each structure constructor makes a value of its own type, not a
value of type BT. But consider the expression (nd 5 (mt) (mt)): the definition
of nd declares that the sub-trees must be of type BT, and yet we are able to
successfully give it values of type mt.

Obviously, it is not coincidental that we have defined BT in terms of mt and nd.
However, it does indicate that when type-checking, we cannot simply be checking
for function equality, at least not as we have so far. Instead, we must be
checking that one type “fits into” the other. This notion of fitting is called
subtyping (and the act of being fit, subsumption).

The essence of subtyping is to define a relation, usually denoted by <:, that
relates pairs of types. We say S <: T if a value of type S can be given where a
value of type T is expected: in other words, subtyping formalizes the notion of
substitutability (i.e., anywhere a value of type T was expected, it can be
replaced with—substituted by—a value of type S). When this holds, S is called
the subtype and T the supertype. It is useful (and usually accurate) to take a
subset interpretation: if the values of S are a subset of T, then an expression
expecting T values will not be unpleasantly surprised to receive only S values.

Subtyping has a pervasive effect on the type system. We have to reexamine every
kind of type and understand its interaction with subtyping. For base types, this
is usually quite obvious: disjoint types like number, string, etc., are all
unrelated to each other. (In languages where one base type is used to represent
another—for instance, in some scripting languages numbers are merely strings
written with a special syntax, and in other languages, booleans are merely
numbers—there might be subtyping relationships even between base types, but
these are not common.) However, we do have to consider how subtyping interacts
with every single compound type constructor

In fact, even our very diction about types has to change. Suppose we have an
expression of type T. Normally, we would say that it produces values of type T.
Now, we should be careful to say that it produces values of up to or at most T,
because it may only produce values of a subtype of T. Thus every reference to a
type should implicitly be cloaked in a reference to the potential for subtyping.
To avoid pestering you I will refrain from doing this, but be wary that it is
possible to make reasoning errors by not keeping this implicit interpretation in
mind.

\secdown

\secrel{Unions}

Let us see how unions interact with subtyping. Clearly, every sub-union is a
subtype of the entire union. In our running example, clearly every mt value is
also a BT; likewise for nd. Thus,
\lsts{15/3/7/3.rkt}{rkt}
As a result, (mt) also has type BT, thus enabling the expression (nd 5 (mt)
(mt)) to itself type, and to have the type nd—and hence, also the type BT. In
general,
\lsts{15/3/7/4.rkt}{rkt}
(we write what seems to be the same rule twice just to make clear it doesn’t
matter which “side” of the union the subtype is on). This says that a value of S
can be thought of as a value of S U T, because any expression of type S U T can
indeed contain a value of type S.

\secrel{Intersections}

While we’re at it, we should also briefly visit intersections. As you might
imagine, intersections behave dually:
\lsts{15/3/7/5.rkt}{rkt}
To convince yourself of this, take the subset interpretation: if a value is both
S and T, then clearly it is either one of them.

\DoNow{
Why are the following two not valid subsumptions?
\lsts{15/3/7/6.rkt}{rkt}
}

The first is not valid because a value of type T is a perfectly valid element of
type (S U T). For instance, a number is a member of type (string U number).
However, a number cannot be supplied where a value of type string is expected.

As for the second, in general, a value of type T is not also a value of type S.
% Any consumer of a ($S\ \Alpha\ T$) value is expecting to be able to treat it
% as both a T and a S, and the latter is not justified. For instance, given our
% overloaded + from before, if T is (number number -> number), then a function
% of this type
will not know how to operate on strings.

\secrel{Functions}

We have seen one more constructor: functions. We must therefore determine the
rules for subtyping when either type can be a function. Since we usually assume
functions are disjoint from all other types, we therefore only need to consider
when one function type is a subtype of another: i.e., when is
\note{We have also seen parametric datatypes. In this edition, exploring
subtyping for them is left as an exercise for the reader.}
\lsts{15/3/7/7.rkt}{rkt}
? For convenience, let us call the type (S1 -> T1) as f1, and (S2 -> T2) as f2.
The question then is, if an expression is expecting functions of the f2 type,
when can we safely give it functions with the f1 type? It is easiest to think
through this using the subset interpretation.

Consider a use of the f2 type. It returns values of type T2. Thus, the context
surrounding the function application is satisfied with values of type T2.
Clearly, if T1 is the same as T2, the use of f2 would continue to type;
similarly, if T1 consists of a subset of T2 values, it would still be fine. The
only problem is if T1 has more values than T2, because the context would then
encounter unexpected values that would result in undefined behavior. In other
words, we need that T1 <: T2. Observe that the “direction” of containment is the
same as that for the entire function type; this is called covariance (both vary
in the same direction). This is perhaps precisely what you expected.

By the same token, you might expect covariance in the argument position as well:
namely, that S1 <: S2. This would be predictable, and wrong. Let’s see why.

An application of a function with f2 type is providing parameter values of type
S2.
Suppose we instead substitute the function with one of type f1. If we had that
S1 <:
S2, that would mean that the new function accepts only values of typeS1—a
strictly smaller set. That means there may be some inputs—specifically those in
S2 that are not in S1—that the application is free to provide on which the
substituted function is not defined, again resulting in undefined behavior. To
avoid this, we have to make the subsumption go in the other direction: the
substituting function should accept at least as many inputs as the one it
replaces. Thus we need S2 <: S1, and say the function position is contravariant:
it goes against the direction of subtyping.

Putting together these two observations, we obtain a subtyping rule for
functions (and hence also methods):
\lsts{15/3/7/8.rkt}{rkt}

\secrel{Implementing Subtyping}

Of course, these rules assume that we have modified the type-checker to respect
subtyping. The essence of subtyping is a rule that says, if an expression e is
of type S, and S <: T, then e also has type T. While this sounds intuitive, it
is also immediately problematic for two reasons:
\begin{itemize}
  \item 
Until now all of our type rules have been syntax-driven, which is what enabled
us to write a recursive-descent type-checker. Now, however, we have a rule that
applies to all expressions, so we can no longer be sure when to apply it.
  \item 
There could be many levels of subtyping. As a result, it is no longer obvious
when to “stop” subtyping. In particular, whereas before type-checking was able
to calculate the type of an expression, now we have many possible types for each
expression; if we return the “wrong” one, we might get a type error (due to that
not being the type expected by the context) even though there exists some other
type that was the one expected by the context.
\end{itemize}
What these two issues point to is that the description of subtyping we are
giving here is fundamentally declarative: we are saying what must be true, but
not showing how to turn it into an algorithm. For each actual type language,
there is a less or more interesting problem in turning this into algorithmic
subtyping: an actual algorithm that realizes a type-checker (ideally one that
types exactly those programs that would have typed under the declarative regime,
i.e., one that is both sound and complete).

\secup