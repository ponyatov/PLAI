\secrel{15.3.8 Object Types   . . 176}

As we’ve mentioned earlier, types for objects are typically riven into two
camps: nominal and structural. Nominal types are familiar to most programmers
through Java, so I won’t say much about them here. Structural types for objects
dictate that an object’s type is itself a structured object, consisting of names
of fields and their types. For instance, an object with two methods, add1 and
sub1 \ref{}, would have the type
\lsts{15/3/8/1.rkt}{rkt}
(For future reference, let’s call this type addsub.) Type-checking would then
follow along predictable lines: for field access we would simply ensure the
field exists and would use its declared type for the dereference expression; for
method invocation we would have to ensure not only that the member exists but
that it has a function type. So far, so straightforward.

Object types become complicated for many reasons:
\note{Whole books are therefore devoted to this topic. Abadi and Carelli’s A
Theory of Objects is important but now somewhat dated. Bruce’s Foundations of
Object-Oriented Languages: Types and Semantics is more modern, and also offers
more gentle exposition. Pierce covers all the necessary theory beautifully.}
\begin{itemize}
  \item 
Self-reference. What is the type of self? It must the same type as the object
itself, since any operation that can be applied to the object from the “outside”
can also be applied to it from the “inside” using self. This means object types
are recursive types.
  \item 
Access controls: private, public, and other restrictions. These lead to a
distinction in the type of an object from “outside” and “inside”.
  \item 
Inheritance. Not only do we have to give a type to the parent object(s), what is
visible along inheritance paths may, again, differ from what is visible from the
“outside”.
  \item 
The interplay between multiple-inheritance and subtyping.
  \item 
The relationship between classes and interfaces in languages like Java, which
has a run-time cost.
  \item 
Mutation.
  \item 
Casts.
  \item 
Snakes on a plane.
\end{itemize}
and so on. Some of these problems simplify in the presence of nominal types,
because given a type’s name we can determine everything about its behavior (the
type declarations effectively become a dictionary through which the object’s
description can be looked up on demand), which is one argument in favor of
nominal typing.
\note{Note that Java’s approach is not the only way to build a nominal type
system. We have already argued that Java’s class system needlessly restricts the
expressive power of programmers \ref{}; in turn, Java’s nominal type system
needlessly conflates types (which are interface descriptions) and
implementations. It is, therefore, possible to have much better nominal type
systems than Java’s. Scala, for instance, takes significant steps in this
direction.}

A full exposition of these issues will take much more room than we have here.
For now, we will limit ourselves to one interesting question. Remember that we
said subtyping forces us to consider every type constructor? The structural
typing of objects introduces one more: the object type constructor. We therefore
have to understand its interaction with subtyping.

Before we do, let’s make sure we understand what an object type even means.
Consider the type addsub above, which lists two methods. What objects can be
given this type? Obviously, an object with just those two methods, with
precisely those two types, is eligible. Equally obviously, an object with only
one and not the other of those two methods, no matter what else it has, is not.
But the phrase “no matter what else it has” is meant to be leading. What if an
object represents an arithmetic package that also contains methods + and *, in
addition to the above two (all of the appropriate type)? In that case we
certainly have an object that can supply those two methods, so the arithmetic
package certainly has type addsub. Its other methods are simply inaccessible
using type addsub.

Let us write out the type of this package, in full, and call this type as+*:
\lsts{15/3/8/2.rkt}{rkt}
What we have just argued is that an object of type as+* should also be allowed
to claim the type addsub, which means it can be substituted in any context
expecting a value of type addsub. In other words, we have just said that we want
as+* <: addsub:
\lsts{15/3/8/3.rkt}{rkt}
This may momentarily look confusing: we’ve said that subtyping follows set
inclusion, so we would expect the smaller set on the left and the larger set on
the right. Yet, it looks like we have a “larger type” (certainly in terms of
character count) on the left and a “smaller type” on the right.

To understand why this is sound, it helps to develop the intuition that the
“larger” the type, the fewer values it can have. Every object that has the four
methods on the left clearly also has the two methods on the right. However,
there are many objects that have the two methods on the right that fail to have
all four on the left. If we think of a type as a constraint on acceptable value
shapes, the “bigger” type imposes more constraints and hence admits fewer
values. Thus, though the types may appear to be of the wrong sizes, everything
is well because the sets of values they subscribe are of the expected sizes.

More generally, this says that by dropping fields from an object’s type, we
obtain a supertype. This is called width subtyping, because the subtype is
“wider”, and we move up the subtyping hierarchy by adjusting the object’s
“width”. We see this even in the nominal world of Java: as we go up the
inheritance chain a class has fewer and fewer methods and fields, until we reach
Object, the supertype of all classes, which has the fewest. Thus for all class
types C in Java, C <: Object.
\note{Somewhat confusingly, the terms narrowing and widening are sometimes used,
but with what some might consider the opposite meaning.
To widen is to go from subtype to supertype, because it goes from a “narrower”
(smaller) to a “wider” (bigger) set. These terms evolved independently, but
unfortunately not consistently.}

As you might expect, there is another important form of subtyping, which is
within a given member. This simply says that any particular member can be
subsumed to a supertype in its corresponding position. For obvious reasons, this
form is called depth subtyping.

\Exercise{
Construct two examples of depth subtyping. In one, give the field itself an
object type, and use width subtyping to subtype that field. In the other, give
the field a function type.
}

Java has limited depth subtyping, preferring types to be invariant down the
object hierarchy because this is a safe option for conventional mutation.

The combination of width and depth subtyping cover the most interesting cases of
object subtyping. A type system that implemented only these two would, however,
needlessly annoy programmers. Other convenient (and mathematically necessary)
rules include the ability to permute names, reflexivity (every type is a subtype
of itself, because it is more convenient to interpret the subtype relationship
as $\subseteq$), and transitivity. Languages like Typed JavaScript employ all
these features to provide maximum flexibility to programmers.
