\secrel{15.3.3 Union Types   . . 164}

Suppose we want to construct a list of zoo animals, of which there are many
kinds: armadillos, boa constrictors, and so on. Currently, we are forced to
create a new datatype:
\note{“In Texas, there ain’t nothing in the middle of the road but a yellow line
and dead armadillos.”—Jim Hightower}
\lsts{15/3/3/1.rkt}{rkt}
and make a list of these: (listof Animal). The type Animal therefore represents
a “union” of armadillo and boa, except the only way to construct such unions is
to make a new type every time: if we want to represent the union of animals and
plants, we need
\lsts{15/3/3/2.rkt}{rkt}
so an actual animal is now one extra “level” deep. These datatypes are called
tagged unions or discriminated unions, because we must introduce explicit tags
(or discriminators), such as animal and plant, to tell them apart. In turn, a
structure can only reside inside a datatype declaration; we have had to create
datatypes with just one variant, such as
\lsts{15/3/3/3.rkt}{rkt}
to hold the datatype, and everywhere we’ve had to use the type Constraints
because eqCon is not itself a type, only a variant that can be distinguished at
run-time.

Either way, the point of a union type is to represent a disjunction, or “or”. A
value’s type is one of the types in the union. A value usually belongs to only
one of the types in the union, though this is a function of precisely how the
union types are defined, whether there are rules for normalizing them, and so
on.

\secdown

\secrel{Structures as Types}

A natural reaction to this might be, why not lift this restriction? Why not
allow each structure to exist on its own, and define a type to be a union of
some collection of structures? After all, in languages ranging from C to Racket,
programmers can define stand-alone structures without having to wrap them in
some other type with a tag constructor! For instance, in raw Racket, we can
write
\lsts{15/3/3/4.rkt}{rkt}
and a comment that says
\lsts{15/3/3/5.rkt}{rkt}
but without enforced static types, the comparison is messy. However, we can more
directly compare with Typed Racket, a typed form of Racket that is built into
DrRacket. Here is the same typed code:
\lsts{15/3/3/6.rkt}{rkt}
We can now define functions that consume values of type boa without any
reference to armadillos:
\lsts{15/3/3/7.rkt}{rkt}
In fact, if we apply this function to any other type, including an
armadillo—(big-one? (armadillo true))—we get a static error. This is because
armadillos are no more related to boas than numbers or strings are.

Of course, we can still define a union of these types:
\lsts{15/3/3/8.rkt}{rkt}
and functions over it:
\lsts{15/3/3/9.rkt}{rkt}
Whereas before we had one type with two variants, now we have three types. It
just so happens that two of the types form a union of convenience to define a
third.

\secrel{Untagged Unions}

It might appear that we still need to have discriminative tags, but we don’t. In
languages with union types, the effect of the optionof type constructor is often
obtained by combining the intended return type with a disjoint one representing
failure or noneness. For instance, here is the moral equivalent of (optionof
number):
\lsts{15/3/3/10.rkt}{rkt}
For that matter, Boolean may itself be a union of True and False, as it is in
Typed Racket, so a more accurate simulation of the option type would be:
\lsts{15/3/3/11.rkt}{rkt}

More generally, we could define
\lsts{15/3/3/12.rkt}{rkt}
which would work for all types, because none is a new, distinct type that cannot
be confused for any other. This gives us the same benefit as the optionof type,
except the value we want is not buried one level deep inside a some structure,
but is rather available immediately. For instance, consider member, which has
this Typed Racket type:
\lsts{15/3/3/13.rkt}{rkt}
If the element is not found, member returns false. Otherwise, it returns the
list starting from the element onward (i.e., the first element of the list will
be the desired element):
\lst{15/3/3/14.rkt}
To convert this to use Maybeof, we can write
\lsts{15/3/3/15.rkt}{rkt}
which, if the element is not found, returns the value (none), but if it is
found, still returns a list
\lsts{15/3/3/16.rkt}{rkt}
so that there is no need to remove the list from a some wrapper.

\secrel{Discriminating Untagged Unions}

It’s one thing to put values into unions; we have to also consider how to take
them out, in a well-typed manner. In our ML-like type system, we use a stylized
notation— type-case in our language, pattern-matching in ML—to identify and pull
apart the pieces. In particular, when we write
\lsts{15/3/3/17.rkt}{rkt}
the type of a remains the same in the entire expression. The identifiers a? and
l are bound to a boolean and numeric value, respectively, and big-one? must now
be written to consume those types, not armadillo and boa. Put in different
terms, we cannot have a function big-one? that consumes boas, because there is
no such type.

In contrast, with union types, we do have the boa type. Therefore, we follow the
principle that the act of asking predicates of a value narrows the type. For
instance, in the cond case
\lsts{15/3/3/18.rkt}{rkt}
though a begins as type Animal, after it passes the boa? test, the type checker
is expected to narrow its type to just the boa branch, so that the application
of big-one? is well-typed. In turn, in the rest of the conditional its type is
not boa—in this case, that leaves only one possibility, armadillo. This puts
greater pressure on the typechecker’s ability to test and recognize certain
patterns—known as if-splitting—without which it would be impossible to program
with union types; but it can always default to recognizing just those patterns
that the ML-like system would have recognized, such as pattern-matching or
type-case.

\secrel{Retrofitting Types}

It is unsurprising that Typed Racket uses union types. They are especially
useful when retrofitting types onto existing programming languages whose
programs were not defined with an ML-like type discipline in mind, such as in
scripting languages. A common principle of such retrofitted types is to
statically catch as many dynamic exceptions as possible. Of course, the checker
must ultimately reject some programs, and if it rejects too many programs that
would have run without an error, developers are unlikely to adopt it. Because
these programs were written without type-checking in mind, the type checker may
therefore need to go to heroic lengths to accept what are considered reasonable
idioms in the language.
\note{Unless it implements an interesting idea called soft typing, which rejects
no programs but provides information about points where the program would not
have been typeable.}
Consider the following JavaScript function:
\lsts{15/3/3/19.java}{java}
It consumes an array and two indices, and produces the sub-array between those
indices. For instance, slice([5, 7, 11, 13], 0, 2) produces [5, 7, 11].

In JavaScript, however, developers are free to leave out any or all trailing
arguments to a function. Every elided argument is given a special value,
undefined, and it is up to the function to cope with this. For instance, a
typical implementation of splice would let the user drop the third argument; the
following definition
\lsts{15/3/3/20.java}{java}
automatically returns the subarray until the end of the array: thus, slice([5,
7, 11, 13], 2) returns [11, 13].

In Typed JavaScript, a programmer can explicitly indicate a function’s
willingness to accept fewer arguments by giving a parameter the type U
Undefined, giving it the type
\note{Built at Brown by Arjun Guha and others. See
\href{http://www.jswebtools.org/}{our Web site}.}
\lsts{15/3/3/21.rkt}{rkt}
In principle, this means there is a potential type error at the expression stop
- start, because stop may not be a number. However, the assignment to stop sets
it to a numeric type precisely when it was elided by the user. In other words,
in all control paths, stop will eventually have a numeric type before the
subtraction occurs, so this function is well-typed. Of course, this requires the
type-checker to be able to reason about both control-flow (through the
conditional) and state (through the assignment) to ensure that this function is
well-typed; but Typed JavaScript can, and can thus bless functions such as this.

\secrel{Design Choices}

In languages with union types, it is common to have
\begin{itemize}[nosep]
  \item 
Stand-alone structure types (often represented using classes), rather than
datatypes with variants.
  \item 
Ad hoc collections of structures to represent particular types.
  \item 
The use of sentinel values to represent failure.
\end{itemize}
To convert programs written in this style to an ML-like type discipline would be
extremely onerous. Therefore, many retrofitted type systems adopt union types to
ease the process of typing.

Of the three properties above, the first seems morally neutral, but the other
two warrant more discussion. We will address them in reverse order.
\begin{itemize}
  \item 
Let’s tackle sentinels first. In many cases, sentinels ought to be replaced with
exceptions, but in many languages, exceptions can be very costly. Thus,
developers prefer to make a distinction between truly exceptional
situations—that ought not occur—and situations that are expected in the normal
course of operation. Checking whether an element is in a list and failing to
find it is clearly in the latter category (if we already knew the element was or
wasn’t present, there would be no need to run this predicate). In the latter
case, using sentinels is reasonable.

However, we must square this with the observation that failure to check for
exceptional sentinel values is a common source of error—and indeed, security
flaws—in C programs. This is easy to reconcile. In C, the sentinel is of the
same type (or at least, effectively the same type) as the regular return value,
and furthermore, there are no run-time checks. Therefore, the sentinel can be
used as a legitimate value without a type error. As a result, a sentinel of 0
can be treated as an address into which to allocate data, thus potentially
crashing the system. In contrast, our sentinel is of a truly new type that
cannot be used in any computation. We can easily reason about this by observing
that no existing functions in our language consume values of type none.

  \item 
Setting aside the use of “ad hoc”, which is pejorative, are different groupings
of a set of structures a good idea? In fact, such groupings occur even in
programs using an ML-like discipline, when programmers want to carve different
subuniverses of a larger one. For instance, ML programmers use a type like
\lsts{15/3/3/22.rkt}{rkt}
to represent s-expressions. If a function now operates on just some subset of
these terms—say just numbers and lists of numbers—they must create a fresh type,
and convert values between the two types even though their underlying
representations are essentially identical. As another example, consider the set
of CPS expressions. This is clearly a subset of all possible expressions, but if
we were to create a fresh datatype for it, we would not be able to use any
existing programs that process expressions—such as the interpreter.

\end{itemize}

In other words, union types appear to be a reasonable variation on the ML-style
type system we have seen earlier. However, even within union types there are
design variations, and these have consequences. For instance, can the type
system create new unions, or are user-defined (and named) unions permitted? That
is, can an expression like this
\lsts{15/3/3/23.rkt}{rkt}
be allowed to type (to (U Number Boolean)), or is it a type error to introduce
unions that have not previously been named and explicitly identified? Typed
Racket provides the former: it will construct truly ad hoc unions. This is
arguably better for importing existing code into a typed setting, because it is
more flexible. However, it is less clear whether this is a good design for
writing new code, because unions the programmer did not intend can occur and
there is no way to prevent them. This offers an unexplored corner in the design
space of programming languages.

\secup