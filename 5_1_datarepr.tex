\secrel{Defining Data Representations\\\ru{Представление определения}}

To keep things simple, let’s just consider functions of one argument.
\ru{Для упрощения будем рассматривать только функции от одного
аргумента.\note{\ru{подход имеет право на жизнь\ --- в языке Ы несколько
параметров передаются в контейнере-списке в сочетании с оператором @, не говоря
о \termdef{карринге}{карринг}, когда применение функции к аргументу \#1
возвращает анонимную лямбда-\emph{функцию}, применямую к аргументу
\#2 и т.д.: $f(x,y,z) \sim
(f(x)_{\lambda_x}(y))_{\lambda_{x,y}}(z)$}}} Here are some
\racket\ examples:
\ru{Вот несколько примеров на \racket:}
\lsts{src/5/p19_1.rkt}{rkt}

\Exercise{
When a function has multiple arguments, what simple but important criterion
governs the names of those arguments\,?\\
\ru{Если функция имеет несколько аргументов, какой простой но важный
критерий определяет имена этих аргументов\,?} }

What are the parts of a \termdef{function definition}{function definition}\,?
\ru{Из каких частей состоит \termdef{определение функции}{определение
функции}\,?}
It has a name \ru{Это имя}
(above, \verb|double|, \verb|quadruple|, and \verb|const5|),
which we’ll represent as a symbol
\ru{которые мы будем представлять символом}
('double, etc.);
its \termdef{formal parameter}{formal parameter}
\ru{ее \termdef{формальный параметр}{формальный параметр}}
or \termdef{argument}{argument}
\ru{или \termdef{аргумент}{аргумент}}
has a name \ru{имеющий имя} (e.g., x),
which too we can model as a symbol
\ru{который мы тоже смоделируем как символ} ('x);
and it has a \termdef{body}{function body}. \ru{и ее \termdef{тело}{тело
функции}.}
We’ll determine the body’s representation in stages,
\ru{Мы определим представление тела пошагово,}
but let’s start to lay out a datatype for function definitions:
\ru{но для начала выделим тип данных для всех определений функций:}
\lstx{FunDefC(ore)}{src/5/p20_2.rkt}{rkt}

What is the body\,? \ru{Что такое \term{тело функции}\,?}
Clearly, it has the form of an arithmetic expression,
\ru{Очевидно, оно имеет форму арифметичского выражения,}
and sometimes it can even be represented using the existing
\ru{и иногда оно даже может быть представлено используя существующий} 
\verb|ArithC(ore)| language \ru{язык}:
for instance, the body of \ru{например, тело} \verb|const5|
can be represented as \ru{может быть представлено как } \verb|(numC 5)|.
But representing the body of \ru{Но представление тела}
\verb|double| requires something more: \ru{требует кое-чего еще:}
not just addition (which we have), \ru{не только сложение (которое у нас есть),} 
but also \ru{но также и} ''x''.
You are probably used to calling this a \termdef{variable}{variable},
\ru{Вы возможно уже назвали это \termdef{переменной}{переменная}} 
but we will \emph{not use} that term for now.
\ru{но мы пока \emph{не будем использовать} этот термин.} 
Instead, we will call it an \termdef{identifier}{identifier}.\note{I promise
we’ll return to this issue of nomenclature later \ref{}.}
\ru{Вместо этого мы назовем его
\termdef{идентификатор}{идентификатор}.\note{\ru{Я вам обещаю, что мы
вернемся к этому вопросу номенклатуры позже \ref{}.}}}

\DoNow{Anything else\,?\\\ru{Что-то еще\,?}}

Finally, let’s look at the body of
\ru{Давайте посмотрим на тело} \verb|quadruple|.
It has yet another new construct:
\ru{Оно имеет другой новый конструкт:}
a function \termdef{application}{function application}.
\ru{\termdef{применение}{применение функции} функции.}
Be very careful to distinguish between a function \term{definition},
\ru{Будьте очень осторожны, и отличайте \term{определение} функции,} 
which describes what the function is,
\ru{которое описывает что есть некоторая функция,} 
and an \term{application}, which uses it.
\ru{от \term{применения}, т.е. ее использования.}
These are uses. \ru{Это виды использования.}
The \term{argument} (or \term{actual parameter})
\ru{\term{Аргумент} (или \term{фактический параметр})}
in the inner application of \ru{во внутреннем применении}
\verb|double| is \verb|x|;
the argument in the outer application is \ru{аргумент во внешнем применении}
\verb|(double x)|.
Thus, the argument can be any complex expression.
\ru{Таким образом, аргумент может быть любым сложным выражением.}

Let’s commit all this to a crisp datatype.
\ru{Давайте применим все это к нашему сверкающему типу данных.}
Clearly we’re extending what we had before 
\ru{Расширяем то что у нас уже было раньше}
(because we still want all of arithmetic
\ru{потому что нам все еще нужна вся арифметика}).
We’ll give a new name to our datatype
\ru{Мы дадим новое имя нашему типу данных}
to signify that it’s growing up:
\ru{чтобы отметить что он расширяется:}
\lstx{ExprC(ore)}{src/5/p20_3.rkt}{rkt}

Identifiers are closely related to formal parameters.
\ru{Идентификаторы близко связаны с формальными параметрами.}
When we apply a function by giving it a value for its parameter,
\ru{когда при применяем функцию придавая значение ее параметру,} 
we are in effect asking it
\ru{фактически мы просим функцию} 
to replace all instances of that formal parameter in the body
\ru{заменить все вхождения этого формального параметра в ее теле}\ ---
i.e., the identifiers with the same name as the formal parameter
\ru{т.е., идентификаторы с тем же именем что и формальный параметр}
--- with that value. \ru{этим значением.}
To simplify this process of search-and-replace,
\ru{Для упрощения этого процесса поиска/замены,}
we might as well use the same datatype to represent both.
\ru{мы могли бы так же использовать один и тот же тип данных для представления
обоих.}
We’ve already chosen symbols to represent formal parameters, so:
\ru{Мы уже выбрали символы для представления формальных параметров, так что:}
\note{Observe that we are being coy about a few issues: what kind of ''value''
\ref{} and when to replace \ref{}.}
\note{\ru{Заметим что были скромны в нескольких вопросах: какого рода
``значения'' \ref{} и когда их заменять \ref{}}}
\lsts{src/5/p21_1.rkt}{rkt}

Finally, applications.
\ru{В заключение, применения.}
They have two parts: \ru{Они имеют две части:} 
the function’s name, \ru{имя функции,}
and its argument. \ru{и ее аргумент.}
We’ve already agreed that the argument can be any full-fledged expression
\ru{мы уже убедились, что аргумент может быть любым полноценным выражением}
(including identifiers and other applications
\ru{включая идентификаторы и другие применения}).
As for the function name,
\ru{что касается имени функции,}
it again makes sense to use the same datatype
\ru{снова имеет смысл использовать тот же тип данных,}
as we did when giving the function its name in a function definition.
\ru{как мы сделали когда давали функции имя в ее определении.}
Thus: \ru{Так что:}
\lsts{src/5/p21_2.rkt}{rkt}
identifying which function to apply, and providing its argument.
\ru{определяет какая функция применяется, и предоставляет ее аргумент.}

Using these definitions,
\ru{используя эти определения,}
it’s instructive to write out the representations of the examples
we defined above:
\ru{полезно выписать представления примеров которые мы рассматривалии выше:}
\lsts{src/5/p21_3.rkt}{rkt}
We also need to choose a representation for a set of function definitions. 
\ru{Нам также необходимо выбрать представление для набора определений функций.}
It’s convenient to represent these by a list.
\ru{Удобно определить из через списки.}
\note{Look out\,! Did you notice that we spoke of a \term{set} of function
definitions, but chose a \emph{list} representation\,? That means we’re using an
\emph{ordered} collection of data to represent an \emph{unordered} entity. At
the very least, then, when testing, we should use any and all permutations of
definitions to ensure we haven’t subtly built in a dependence on the order.}
\note{\ru{Осторожнее\,! Вы заметили что мы говорим о \term{множестве}
определений функций, но используем представление \emph{списком}\,? Это значит,
что мы используем \emph{упорядоченный} набор данных для представления
\emph{неупорядоченной} сущности. По крайней мере при тестировании мы должны
использовать любые и все перестановки определений, чтобы убедиться что мы не
имеем неявной зависимости от порядка определений.}}
