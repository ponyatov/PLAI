\secrel{Extension: Binary Subtraction \ru{Расширение: бинарное вычитание}}

First, we’ll add subtraction.
\ru{Сначала мы добавим вычитание.}
Because our language already has numbers, addition, and multiplication, it’s
easy to define subtraction:
\ru{Так как наш язык уже поддерживает числа, сложение и умножение,
легко определить операцию вычитания:}
$a - b = a + -1 \times b$.

Okay, that was easy\,!
\ru{Ок, это было просто\,!}
But now we should turn this into concrete code.
\ru{Но теперь мы должны превратить это в рабочий код.}
To do so, we face a decision: where does this new subtraction operator reside\,?
\ru{Делая это, мы сталкиваемся с вопросом: где проживает этот новый оператор
вычитания\,?}
It is tempting, and perhaps seems natural, to just add one more rule to our
existing \class{ArithC}\ datatype.
\ru{Заманчиво, и кажется естественным, просто добавить еще одно правило к нашему
существующему класу \class{ArithC}.}

\DoNow{
What are the negative consequences of modifying \class{ArithC}\,?\\
\ru{Каковы негативные последствия модификации класса \class{ArithC}\,?}
}

This creates a few problems.
\ru{Это создает несколько проблем.}
The first, obvious, one is that we now have to modify all programs that process
\class{ArithC}.
\ru{Первой из них очевидно является то, что теперь мы должны скорректировать
все программы, обрабатывающие класс \class{ArithC}.}
So far that’s only our interpreter, which is pretty simple, but in a more
complex implementation, that could already be a concern.
\ru{До сих пор он используется только в нашем микроскопическом интерпретаторе,
который крайне прост, но в более сложной реализации языка это уже может быть
проблемой.}
Second, we were trying to add new constructs that we can define in terms of
existing ones;
\ru{Во вторых, мы пытаемся добавить новую языковую конструкцию, которую мы
можем оопределить в терминах уже существующих;}
it feels slightly self-defeating to do this in a way that isn’t modular.
\ru{кажется довольно ущербным делать это немодульно.}
Third, and most subtly, there’s something \emph{conceptually} wrong about
modifying \class{ArithC}.
\ru{В третьих, и это наиболее тонко, есть что-то \emph{концептуально}
неправильное в модификации класса \class{ArithC}.}
That’s because \class{ArithC} represents our \emph{core} language.
\ru{Это потому что \class{ArithC} представляет back-end \emph{ядро} нашего
языка.}
In contrast, subtraction and other additions represent our
user-facing, surface language.
\ru{В протиположность этому, вычитание и другие расширения представляют наш
пользовательский, front-end язык.}
It’s wise to record conceptually different ideas in distinct datatypes, rather
than shoehorn them into one.
\ru{Разумно инкапсулировать концептуально разные идеи в различные типы данных,
а не впихивать их в один.}
 
The separation can look a little unwieldy sometimes, but it makes the program
much easier for future developers to read and maintain.
\ru{Разделение иногда может выглядеть немного громоздко, но оно упрощает
дальнейшее развитие и поддержку программы.}
Besides, for different purposes you might want to layer on different extensions,
and separating the core from the surface enables that.
\ru{Кроме того, вы можете захотить разделить расширения для различных целей в
разные слои, и этому поможет выделение ядра языка.}

Therefore, we’ll define a new datatype to reflect our intended surface syntax
terms:
\ru{Таким образом, вы определяем новый тип данных для отображения термов нашего
front-end синтаксиса:}
\lstx{\class{ArithS}}{src/4/p17_1.rkt}{rkt}
This looks almost exactly like \class{ArithC}, other than the added case, which
follows the familiar recursive pattern.
\ru{Он выглядит почти так же, как и \class{ArithC}, кроме дополнительных
случаев, следуя знакомому рекурсивному шаблону.}

Given this datatype, we should do two things.
\ru{Для данного типа данных вы должны сделать две вещи.}
First, we should modify our parser to also parse\ --- expressions, and always
construct \class{ArithS} terms\note{rather than any \class{ArithC} ones}.
\ru{Прежде всего, мы должны модифицировать наш парсер чтобы он разбирал
выражения и создавал термы класса \class{ArithS}, а не ядерного \class{ArithC}.}
Second, we should implement a \termdef{desugar}{desugar} function that
translates \class{ArithS} values into \class{ArithC} ones.
\ru{Во вторых, мы должны реализовать функцию
\termdef{обессахаривания}{обессахаривание}, которая будет транслировать
структуры из \class{ArithS} значений в аналогичных структуры из экземпляров
\class{ArithC}.}

Let’s write the obvious part of desugar:
\ru{Напишем очевидную часть обессахароивания:}
\lstx{<desugar>::=}{src/4/p17_2.rkt}{rkt}
Now let’s convert the mathematical description of subtraction above into code:
\ru{Теперь давайте преобразуем математическое описание вычитания, приведенное
выше, в код:}
\lstx{<bminusS-case>::=}{src/4/p18_1.rkt}{rkt}

\DoNow{
It’s a common mistake to forget the recursive calls to desugar on $l$ and
$r$. What happens when you forget them\,? Try for yourself and see.\\
\ru{Распространенная ошибка\ --- забыть рекурсивные вызовы обессахаривания $l$ 
и $r$. Что произойдет если вы из забудете\,? Попробуйте сами, и посмотрите
результат.}
}