\secrel{Язык программирования используемый в книге}

Язык программирования используемый в книге\ --- 
\href{http://www.racket-lang.org/}{Racket}. Аналогично операционным системам,
Racket-система является исполняющей средой для целого ряда языков
программирования, так что \emph{вы должны указать Racketу на каком языке 
вы программируете}. Например, в Unix вы указываете в строку типа

\begin{verbatim}
#!/bin/sh
\end{verbatim}

в первой строке shell-скрипта; вы указываете веб-браузеру
тип документа, добавляя заголовок

\begin{verbatim}
<!DOCTYPE HTML PUBLIC "-//W3C//DTD HTML 4.01//EN" ...>
\end{verbatim}

Аналогично, Racket требует от вас указать какой язык вы будете использовать.
Диалекты языков Racket имеют тот же скобочный синтаксис, что и сам Racket,
но другую семантику; ту же семантику но другой синтаксис; или и то и то.
Так что каждая программа, которую может выполнять Racket-система, начинается со
строки \#lang за которой следует имя диалекта языка: по умолчанию, 
это оригинальный Racket (указыватся как \verb|racket|). В этой книге мы 
почти всегда будем использовать диалект\note{В DrRacket v.6.6,
выберите меню \menu{Язык > Выбрать язык\ldots > Start your program with \#lang
to specify the desired dialect}.}

\begin{verbatim}
#lang plai-typed
\end{verbatim}

Когда мы будем отклоняться от этого правила, это будет указано особо, так что
если не указано иное, добавляйте заголовок \verb|#lang plai-typed| в начало
каждого файла программы (предполагается что я тоже это сделал).


