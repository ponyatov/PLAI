\input{../../texheader/ebook}

\newcommand{\Exercise}[1]{
	\begin{description}
		\item{\textcolor{blue}{Упражнение}}\\#1
	\end{description}
}

\newcommand{\DoNow}[1]{
	\begin{description}
		\item{\textcolor{red}{Сделайте\,!}}\\#1
	\end{description}
}

\title{{\Huge{PLAI}}\\
Programming Languages: Application and Interpretation\\
{\small{second edition}}\\
{\Huge{языки программирования}}\\{\Huge{применение и реализация}}}

\author{\copyright\ Шрирам Кришнамурти\\
перевод Dmitry Ponyatov \email{dponyatov@gmail.com}}

\begin{document}
\maketitle
\tableofcontents
\secdown

\secrel{Introduction \ru{Введение}}\secdown
\secrel{Our Philosophy \ru{Наша философия}}

Please watch the video on \ru{Пожалуйста посмотрите это видео на}
\href{https://www.youtube.com/watch?v=3N__tvmZrzc}{YouTube}.

Someday there will be a textual description here instead.
\ru{Когда нибудь здесь будет полное текстовое описание того, о чем в нем
говориться.}

\secrel{The Structure of This Book \ru{Структура книги}}

В отличие от большинства учебников, эта книга не следует подходу "сверху вниз".
Скорее она имеет форму повествования с возвратами к предыдущим темам.
Мы часто будет строить программы инкрементно, так же как мы бы это делали в
парном программировании. Наш код будет включать ошибки, не потому что мы не
знаем правильный ответ, но потому что это лучший способ научить вас, 
углубляясь от поверхностного в детали. Включение намеренных ошибок делает
невозможным для вас читать материал пассивно: вы должны взаимодействовать с
ним, потому что вы никогда не можете быть уверены в правильности того что вы
читаете.

В конце концов вы всегда получите правильный ответ. Тем не меннее, это
нелинейное повествование немного раздражающе в краткосрочной перспективе (у вас
всегда будет соблазн сказать "Скажите же мне наконец ответ !"), и это также
делает эту книгу плохим справочником (вы не можете открыть произвольную
страницу и быть уверенным что на ней написана правда). Тем не менее, это чувство
разочарования\ --- ощущение обучения. Я не знаю другого способа.

\bigskip
В некоторых местах вы встретите следующие выделения:

\Exercise{Это упражнение. Попробуйте это сделать.}

Это традиционное для учебников упражнение.
Это то, что вам нужно сделать по своему усмотрению.
Если вы используете эту книгу как часть курса, это упражнение хорошо
задавать как домашнюю работу. В противоположность этому вы также можете найти
подобные вопросы, выделенные как 

\DoNow{Здесь предполагаются немедленные действия. Вы видите это ?}

Когда вы доберетесь до одного из этих блоков, остановитесь. Прочитайте,
подумайте, сформулируйте ответ перед тем как продолжить чтение. Вы должны
сделать это потому что это действительно упражнение, но ответ уже есть в книге,
чаще всего в тексте непосредственно после упражнения (т.е. в части которую вы
сейчас читаете) или это что-то, что вы можете получить самостоятельно,
запустив программу. Если вы просто продолжите читать, то вы увидете ответ без
его обдумывания (или не увидите его вообще, если это инструкции по запуску
программы), так что вы ни (а) проверите свои знания, ни (б) улучшите свое
понимание. Другими словами, это дополнительные, явные попытки стимулировать
ваше активное обучение. В конце концов, я могу только поощрять вас работать;
решение применять это или нет остается за вами.


\secrel{The Language of This Book \ru{Язык программирования используемый в
книге}}

The main programming language used in this book is
\ru{Язык программирования используемый в книге}\ --- 
\href{http://www.racket-lang.org/}{\racket}.
Like with all operating systems, however,
\ru{Аналогично операционным системам,}
\racket\ actually supports a host of programming languages,
\ru{\racket-система является исполняющей средой для целого ряда языков
программирования,}
so you must tell \racket\
\ru{так что вы должны указать \racket у}
\emph{which} language you’re programming in.
\ru{\emph{на каком} языке вы программируете.}
You inform the Unix shell by writing a line like
\ru{Например, в Unix вы пишете строку типа}

\begin{verbatim}
#!/bin/sh
\end{verbatim}
at the top of a script;
\ru{в первой строке shell-скрипта;}
you inform the browser by writing, say,
\ru{вы указываете веб-браузеру тип документа, добавляя заголовок}

\begin{verbatim}
<!DOCTYPE HTML PUBLIC "-//W3C//DTD HTML 4.01//EN" ...>
\end{verbatim}

Similarly, \racket\ asks that you declare which language you will be using.
\ru{Аналогично, \racket\ требует от вас указать какой язык вы будете
использовать.}
\racket\ languages can have the same parenthetical syntax as \racket\ but with a
different semantics;
\ru{Диалекты языков \racket\ имеют тот же скобочный синтаксис, что и сам
\racket, но другую семантику;}
the same semantics but a different syntax;
\ru{ту же семантику но другой синтаксис;}
or different syntax and semantics.
\ru{или различные синтаксис и семантику.}
Thus every \racket\ program
\ru{Так что каждая программа, которую может выполнять \racket-система,}
begins with \#lang followed by the name of some language:
\ru{начинается со строки \#lang за которой следует имя диалекта языка:}
by default, it’s \racket\ \ru{по умолчанию, это оригинальный \racket\ }
written as \ru{указыватся как} \verb|racket|).
In this book we’ll almost always use the language\note{In DrRacket v.5.3,
go to Language, then Choose Language, and select ``Use the language declared in
the source''.}
\ru{В этой книге мы почти всегда будем использовать диалект}\note{\ru{В DrRacket
v.6.6, выберите меню\\\menu{Язык > Выбрать язык\ldots > Start your program with
\#lang to specify the desired dialect}.}}
\begin{verbatim}
#lang plai-typed
\end{verbatim}
When we deviate we’ll say so explicitly,
\ru{Когда мы будем отклоняться от этого правила,}
so unless indicated otherwise, put
\ru{это будет указано особо, так что если не указано иное, добавляйте заголовок}
\verb|#lang plai-typed|
at the top of every file
\ru{в начало каждого файла программы}
(and assume I’ve done the same
\ru{предполагается что я тоже это
сделал})\note{В DrRacket v.6.6 требуется установить расширение
plai-typed:\\\menu{Файл>Install package\ldots>Package
Source:>\url{github://github.com/mflatt/plai-typed/master}>Install>\ldots>Закрыть}}.

The \termdef{Typed PLAI}{Typed PLAI}\ language differs from traditional \racket\
most importantly by being statically typed.
\ru{Язык \term{Typed PLAI}\ отличается от традиционного \racket\ в основном
\emph{статической типизацией}.}
It also gives you some useful new constructs:
\ru{Он также дает вам некоторые новые полезные конструкции:}
\verb|define-type| \ru{определение-типа}, \verb|type-case| \ru{выбор-по-типу},
and \verb|test|\note{There are additional commands for controlling the output
of testing, for instance. \ru{Также существуют дополнительные команды для
управления выводом тестов.} Be sure to read the documentation for the language.
\ru{Обязательно прочитайте документацию для языка.}
In DrRacket v.5.3, go to \menu{Help>Help Desk}, and in the Help Desk search bar,
type \menu{plai-typed}. \ru{В DrRacket v.6.6 идите в меню \menu{Help>Help
Desk}, и в поле поиска \menu{Help Desk} введите \menu{plai-typed}.}}
Here’s an example of each in use.
\ru{Вот примеры использования каждого из них.} 
We can introduce new datatypes
\ru{Мы можем создавать новые типы данных\note{запустить программу можно нажав
\keys{Ctrl+R}}}:
\lst{src/1/p8_1.rkt}
You can roughly think of this as analogous to the following in Java:
\ru{Вы можете примерно понять идею в терминах языка \java:}
an abstract class \term{абстрактный класс} \verb|MisspelledAnimal| and two
concrete sub-classes \ru{и два конкретизирующих подкласса}\ \verb|caml|
\ru{верблюд} and \verb|yacc| \ru{якк},
each of which has one numeric constructor argument named
\ru{каждый из которых имеет конструктор с числовым аргументом}
\verb|humps| \ru{горбы} and \verb|height| \ru{высота}, respectively
\ru{соответственно}.

In this language, we construct instances as follows:
\ru{На этом языке мы строим экземпляры классов следующим образом:}
\lst{src/1/p8_2.rkt}
As the name suggests \ru{Как следует из названия,}, \verb|define-type| creates a
type of the given name \ru{создает тип с заданным именем}.
We can use
this when, for instance, binding the above instances to names:
\ru{Мы можем это использовать например при связывании эксземпляров с именами:}
\lst{src/1/p8_3.rkt}
In fact you don’t need these particular type declarations, because \term{Typed
PLAI} will infer types for you here and in many other cases.
\ru{Фактически вам не нужны эти частные определения типов, так как \term{Typed
PLAI} в этом и других случаях будет сам делать для вас \term{вывод типов}.}
Thus you could just as well have written
\ru{Так что вы можете написать короче}

\lst{src/1/p8_4.rkt}

\noindent
but we prefer to write explicit type declarations as a matter of both discipline
and comprehensibility when we return to programs later.
\ru{но мы предпочтем писать полные объявления типов с точки зрения как
дисциплины, так и усвояемости, когда мы вернемся к программам позже.}

The type names can even be used recursively, as we will see repeatedly in this
book (for instance, section \ref{sec2_4}).
\ru{Имена типов даже могут быть использованы рекурсивно, как мы увидим
несколько позже в этой книге (например в разделе \ref{sec2_4}).}

The language provides a pattern-matcher for use when writing expressions, such
as a function’s body:
\ru{Язык предоставляет pattern-matcher для использования при написании
выражений, таких как тело функции:}

\lst{src/1/p9_1.rkt}

\noindent
In the expression \ru{Например в выражении} (>= humps 2), for instance,
\verb|humps| is the name given to whatever value was given as the argument to
the constructor \ru{имя humps соответствует любому значению, данному как
аргумент для конструктора} \verb|caml|.

Finally, you should write test cases, ideally before you’ve defined your
function, but also afterwards to protect against accidental changes:
\ru{И наконец, вы должны написать тесты, в идеале до того как вы ее реализовали,
или хотя бы после, чтобы защититься от внезапных несоответствий в ее поведении
при внесении изменений в код:}

\lst{src/1/p9_2.rkt}

\noindent
When you run the above program, the language will give you verbose output telling
you both tests passed.
\ru{При запуске тестов язык даст вам подробный отчет, что оба теста успешно
пройдены.}
Read the documentation to learn how to suppress most of these messages.
\ru{Прочитайте документацию, чтобы узнать как подавить вывод большей части
этих сообщений.}

Here’s something important that is obscured above.
\ru{Вот еще кое-что важное, что было неясно выше.}
We’ve used the same name,
\ru{Мы использовали одно и то же имя,}
humps (and height), in both the datatype definition and in the fields of the
patternmatch.
\ru{и в определении типа данных, и в полях объекта при проверке совпадения
шаблонов.}
This is absolutely unnecessary because the two are related by position, not
name.
\ru{Это совершенно необязательно, так как каждая пара связана по позиции, а не
по имени.}
Thus, we could have as well written the function as
\ru{Так что мы могли бы также написать эту функцию как}

\lst{src/1/p9_3.rkt}

\noindent
Because each h is only visible in the case branch in which it is introduced, the
two hs do not in fact clash.
\ru{Так как каждый h виден только в той case-секции, где он используется,
два h фактически не сталкиваются.}
You can therefore use convention and readability to
dictate your choices.
\ru{Таким образом вы можете использовать соглашения по оформлению кода для
улучшения читаемости, и диктовать свой выбор.}
In general, it makes sense to provide a long and
descriptive name when defining the datatype\note{because you probably won’t use
that name again}, but shorter names in the type-case because you’re
likely to use use those names one or more times.
\ru{В общем, имеет смысл использовать длинные описательные имена при определении
типа данных\note{\ru{потому что вы возможно больше никогда не будете
использовать это имя снова}}, и короткие имена в type-case, где они обычно
используются несколько раз.}

I did just say you’re unlikely to use the field descriptors introduced in the
datatype definition, but you can.
\ru{Также я хочу упомянуть декрипторы полей класса, которые вы возможно
захотите использовать.}
The language provides \term{selectors} to extract fields without the need for
pattern-matching: e.g., caml-humps.
\ru{Язык предоставляет \term{селекторы} для получения значений полей без
необходимости использовать pattern-matching, например caml-humps.}
Sometimes,
it’s much easier to use the selector directly rather than go through the
pattern-matcher.
\ru{Иногда намного проще использовать селектор, чем возиться с мэтчингом
шаблонов.}
It often isn’t, as when defining  above, but just to be
clear, let’s write it without pattern-matching:
\ru{Часто это не так, как в случае определения good?, но для ясности давайте
напишем без pattern-matching:}

\lst{src/1/p9_4.rkt}

\DoNow{
What happens if you mis-apply functions to the wrong kinds of values\,?\\
\ru{Что произойдет, если вы ошибочно примените функции к неправильным типам
значений\,?}

For instance, what if you give the caml constructor a string\,?\\
\ru{Например, что если вы дадите конструктору caml строковый аргумент\,?}

What if you send a number into each version of good? above\,?\\
\ru{Что произойдет если вы пошлете число к каждой версии good? описанных
выше\,?} }

\secup


\secrel{2 Everything (We Will Say) About Parsing 10}\secdown
\secrel{2.1 A Lightweight, Built-In First Half of a Parser . . . . . . . . . . . . . 10}
\secrel{2.2 A Convenient Shortcut . . . . . . . . . . . . . . . . . . . . . . . . . 10}
\secrel{2.3 Types for Parsing . . . . . . . . . . . . . . . . . . . . . . . . . . . . 11}
\secrel{2.4 Completing the Parser . . . . . . . . . . . . . . . . . . . . . .. . . 12}\label{sec2_4}
\secrel{2.5 Coda . . . . . . . . . . . . . . . . . . . . . . . . . . . . . . . . . . . 13}
\secup

\secrel{3 A First Look at Interpretation 13}\secdown
\secrel{3.1 Representing Arithmetic . . . . . . . . . . . . . . . . . . . . . . . . 14}
\secrel{3.2 Writing an Interpreter . . . . . . . . . . . . . . . . . . . . . . . . . . 14}
\secrel{3.3 Did You Notice? . . . . . . . . . . . . . . . . . . . . . . . . . . . . 15}
\secrel{3.4 Growing the Language . . . . . . . . . . . . . . . . . . . . . . . . . 16}
\secup

\secrel{4 A First Taste of Desugaring 16}\secdown
\secrel{4.1 Extension: Binary Subtraction . . . . . . . . . . . . . . . . . . . . . 17}
\secrel{4.2 Extension: Unary Negation . . . . . . . . . . . . . . . . . . . . . . . 18}
\secup

\secrel{5 Adding Functions to the Language 19}\secdown
\secrel{5.1 Defining Data Representations . . . . . . . . . . . . . . . . . . . . . 19}
\secrel{5.2 Growing the Interpreter . . . . . . . . . . . . . . . . . . . . . . . . . 21}
\secrel{5.3 Substitution . . . . . . . . . . . . . . . . . . . . . . . . . . . . . . . 22}
\secrel{5.4 The Interpreter, Resumed . . . . . . . . . . . . . . . . . . . . . . . . 23}
\secrel{5.5 Oh Wait, There’s More! . . . . . . . . . . . . . . . . . . . . . . . . . 25}
\secup

\secrel{6 From Substitution to Environments 25}\secdown
\secrel{6.1 Introducing the Environment . . . . . . . . . . . . . . . . . . . . . . 26}
\secrel{6.2 Interpreting with Environments . . . . . . . . . . . . . . . . . . . . . 27}
\secrel{6.3 Deferring Correctly . . . . . . . . . . . . . . . . . . . . . . . . . . . 29}
\secrel{6.4 Scope . . . . . . . . . . . . . . . . . . . . . . . . . . . . . . . . . . 30}
\secdown
\secrel{6.4.1 How Bad Is It? . . . . . . . . . . . . . . . . . . . . . . . . . 30}
\secrel{6.4.2 The Top-Level Scope . . . . . . . . . . . . . . . . . . . . . . 31}
\secup
\secrel{6.5 Exposing the Environment . . . . . . . . . . . . . . . . . . . . . . . 31}
\secup

\secrel{7 Functions Anywhere 31}\secdown
\secrel{7.1 Functions as Expressions and Values . . . . . . . . . . . . . . . . . . 32}
\secrel{7.2 Nested What? . . . . . . . . . . . . . . . . . . . . . . . . . . . . . . 35}
\secrel{7.3 Implementing Closures . . . . . . . . . . . . . . . . . . . . . . . . . 37}
\secrel{7.4 Substitution, Again . . . . . . . . . . . . . . . . . . . . . . . . . . . 38}
\secrel{7.5 Sugaring Over Anonymity . . . . . . . . . . . . . . . . . . . . . . . 39}
\secup

\secrel{8 Mutation: Structures and Variables 41}\secdown
\secrel{8.1 Mutable Structures . . . . . . . . . . . . . . . . . . . . . . . . . . . 41}
\secdown
\secrel{8.1.1 A Simple Model of Mutable Structures . . . . . . . . . . . . 41}
\secrel{8.1.2 Scaffolding . . . . . . . . . . . . . . . . . . . . . . . . . . . 42}
\secrel{8.1.3 Interaction with Closures . . . . . . . . . . . . . . . . . . . . 43}
\secrel{8.1.4 Understanding the Interpretation of Boxes . . . . . . . . . . . 44}
\secrel{8.1.5 Can the Environment Help? . . . . . . . . . . . . . . . . . . 46}
\secrel{8.1.6 Introducing the Store . . . . . . . . . . . . . . . . . . . . . . 48}
\secrel{8.1.7 Interpreting Boxes . . . . . . . . . . . . . . . . . . . . . . . 49}
\secrel{8.1.8 The Bigger Picture . . . . . . . . . . . . . . . . . . . . . . . 54}
\secup
\secrel{8.2 Variables . . . . . . . . . . . . . . . . . . . . . . . . . . . . . . . . 57}
\secdown
\secrel{8.2.1 Terminology . . . . . . . . . . . . . . . . . . . . . . . . . . 57}
\secrel{8.2.2 Syntax . . . . . . . . . . . . . . . . . . . . . . . . . . . . . 57}
\secrel{8.2.3 Interpreting Variables . . . . . . . . . . . . . . . . . . . . . . 58}
\secup
\secrel{8.3 The Design of Stateful Language Operations . . . . . . . . . . . . . . 59}
\secrel{8.4 Parameter Passing . . . . . . . . . . . . . . . . . . . . . . . . . . . . 60}
\secup

\secrel{9 Recursion and Cycles: Procedures and Data 62}\secdown
\secrel{9.1 Recursive and Cyclic Data . . . . . . . . . . . . . . . . . . . . . . . 62}
\secrel{9.2 Recursive Functions . . . . . . . . . . . . . . . . . . . . . . . . . . . 64}
\secrel{9.3 Premature Observation . . . . . . . . . . . . . . . . . . . . . . . . . 65}
\secrel{9.4 Without Explicit State . . . . . . . . . . . . . . . . . . . . . . . . . . 66}
\secup

\secrel{10 Objects 67}\secdown
\secrel{10.1 Objects Without Inheritance . . . . . . . . . . . . . . . . . . . . . . 67}
\secdown
\secrel{10.1.1 Objects in the Core . . . . . . . . . . . . . . . . . . . . . . . 68}
\secrel{10.1.2 Objects by Desugaring . . . . . . . . . . . . . . . . . . . . . 69}
\secrel{10.1.3 Objects as Named Collections . . . . . . . . . . . . . . . . . 69}
\secrel{10.1.4 Constructors . . . . . . . . . . . . . . . . . . . . . . . . . . 70}
\secrel{10.1.5 State . . . . . . . . . . . . . . . . . . . . . . . . . . . . . . 71}
\secrel{10.1.6 Private Members . . . . . . . . . . . . . . . . . . . . . . . . 71}
\secrel{10.1.7 Static Members . . . . . . . . . . . . . . . . . . . . . . . . . 72}
\secrel{10.1.8 Objects with Self-Reference . . . . . . . . . . . . . . . . . . 72}
\secrel{10.1.9 Dynamic Dispatch . . . . . . . . . . . . . . . . . . . . . . . 74}
\secup
\secrel{10.2 Member Access Design Space . . . . . . . . . . . . . . . . . . . . . 75}
\secrel{10.3 What (Goes In) Else? . . . . . . . . . . . . . . . . . . . . . . . . . . 75}
\secdown
\secrel{10.3.1 Classes . . . . . . . . . . . . . . . . . . . . . . . . . . . . . 76}
\secrel{10.3.2 Prototypes . . . . . . . . . . . . . . . . . . . . . . . . . . . 78}
\secrel{10.3.3 Multiple Inheritance . . . . . . . . . . . . . . . . . . . . . . 78}
\secrel{10.3.4 Super-Duper! . . . . . . . . . . . . . . . . . . . . . . . . . . 79}
\secrel{10.3.5 Mixins and Traits . . . . . . . . . . . . . . . . . . . . . . . . 79}
\secup
\secup

\secrel{11 Memory Management 81}\secdown
\secrel{11.1 Garbage . . . . . . . . . . . . . . . . . . . . . . . . . . . . . . . . . 81}
\secrel{11.2 What is “Correct” Garbage Recovery? . . . . . . . . . . . . . . . . . 81}
\secrel{11.3 Manual Reclamation . . . . . . . . . . . . . . . . . . . . . . . . . . 82}
\secdown
\secrel{11.3.1 The Cost of Fully-Manual Reclamation . . . . . . . . . . . . 82}
\secrel{11.3.2 Reference Counting . . . . . . . . . . . . . . . . . . . . . . 83}
\secup
\secrel{11.4 Automated Reclamation, or Garbage Collection . . . . . . . . . . . . 84}
\secdown
\secrel{11.4.1 Overview . . . . . . . . . . . . . . . . . . . . . . . . . . . . 84}
\secrel{11.4.2 Truth and Provability . . . . . . . . . . . . . . . . . . . . . . 85}
\secrel{11.4.3 Central Assumptions . . . . . . . . . . . . . . . . . . . . . . 85}
\secup
\secrel{11.5 Convervative Garbage Collection . . . . . . . . . . . . . . . . . . . . 86}
\secrel{11.6 Precise Garbage Collection . . . . . . . . . . . . . . . . . . . . . . . 87}
\secup

\secrel{12 Representation Decisions 87}\secdown
\secrel{12.1 Changing Representations . . . . . . . . . . . . . . . . . . . . . . . 87}
\secrel{12.2 Errors . . . . . . . . . . . . . . . . . . . . . . . . . . . . . . . . . . 89}
\secrel{12.3 Changing Meaning . . . . . . . . . . . . . . . . . . . . . . . . . . . 89}
\secrel{12.4 One More Example . . . . . . . . . . . . . . . . . . . . . . . . . . . 90}
\secup

\secrel{13 Desugaring as a Language Feature 91}\secdown
\secrel{13.1 A First Example . . . . . . . . . . . . . . . . . . . . . . . . . . . . . 91}
\secrel{13.2 Syntax Transformers as Functions . . . . . . . . . . . . . . . . . . . 93}
\secrel{13.3 Guards . . . . . . . . . . . . . . . . . . . . . . . . . . . . . . . . . . 95}
\secrel{13.4 Or: A Simple Macro with Many Features . . . . . . . . . . . . . . . 95}
\secdown
\secrel{13.4.1 A First Attempt . . . . . . . . . . . . . . . . . . . . . . . . . 95}
\secrel{13.4.2 Guarding Evaluation . . . . . . . . . . . . . . . . . . . . . . 97}
\secrel{13.4.3 Hygiene . . . . . . . . . . . . . . . . . . . . . . . . . . . . . 98}
\secup
\secrel{13.5 Identifier Capture . . . . . . . . . . . . . . . . . . . . . . . . . . . . 99}
\secrel{13.6 Influence on Compiler Design . . . . . . . . . . . . . . . . . . . . . 101}
\secrel{13.7 Desugaring in Other Languages . . . . . . . . . . . . . . . . . . . . 101}
\secup

\secrel{14 Control Operations 102}\secdown
\secrel{14.1 Control on the Web . . . . . . . . . . . . . . . . . . . . . . . . . . . 102}
\secdown
\secrel{14.1.1 Program Decomposition into Now and Later . . . . . . . . . 104}
\secrel{14.1.2 A Partial Solution . . . . . . . . . . . . . . . . . . . . . . . . 104}
\secrel{14.1.3 Achieving Statelessness . . . . . . . . . . . . . . . . . . . . 106}
\secrel{14.1.4 Interaction with State . . . . . . . . . . . . . . . . . . . . . . 107}
\secup
\secrel{14.2 Continuation-Passing Style . . . . . . . . . . . . . . . . . . . . . . . 109}
\secdown
\secrel{14.2.1 Implementation by Desugaring . . . . . . . . . . . . . . . . . 110}
\secrel{14.2.2 Converting the Example . . . . . . . . . . . . . . . . . . . . 114}
\secrel{14.2.3 Implementation in the Core . . . . . . . . . . . . . . . . . . 115}
\secup
\secrel{14.3 Generators . . . . . . . . . . . . . . . . . . . . . . . . . . . . . . . . 117}
\secdown
\secrel{14.3.1 Design Variations . . . . . . . . . . . . . . . . . . . . . . . . 117}
\secrel{14.3.2 Implementing Generators . . . . . . . . . . . . . . . . . . . . 119}
\secup
\secrel{14.4 Continuations and Stacks . . . . . . . . . . . . . . . . . . . . . . . . 121}
\secrel{14.5 Tail Calls . . . . . . . . . . . . . . . . . . . . . . . . . . . . . . . . 123}
\secrel{14.6 Continuations as a Language Feature . . . . . . . . . . . . . . . . . . 124}
\secdown
\secrel{14.6.1 Presentation in the Language . . . . . . . . . . . . . . . . . . 125}
\secrel{14.6.2 Defining Generators . . . . . . . . . . . . . . . . . . . . . . 126}
\secrel{14.6.3 Defining Threads . . . . . . . . . . . . . . . . . . . . . . . . 127}
\secrel{14.6.4 Better Primitives for Web Programming . . . . . . . . . . . . 131}
\secup
\secup

\secrel{15 Checking Program Invariants Statically: Types 131}\secdown
\secrel{15.1 Types as Static Disciplines . . . . . . . . . . . . . . . . . . . . . . . 133}
\secrel{15.2 A Classical View of Types . . . . . . . . . . . . . . . . . . . . . . . 134}
\secdown
\secrel{15.2.1 A Simple Type Checker . . . . . . . . . . . . . . . . . . . . 134}
\secrel{15.2.2 Type-Checking Conditionals . . . . . . . . . . . . . . . . . . 139}
\secrel{15.2.3 Recursion in Code . . . . . . . . . . . . . . . . . . . . . . . 139}
\secrel{15.2.4 Recursion in Data . . . . . . . . . . . . . . . . . . . . . . . . 142}
\secrel{15.2.5 Types, Time, and Space . . . . . . . . . . . . . . . . . . . . 144}
\secrel{15.2.6 Types and Mutation . . . . . . . . . . . . . . . . . . . . . . . 146}
\secrel{15.2.7 The Central Theorem: Type Soundness . . . . . . . . . . . . 147}
\secup
\secrel{15.3 Extensions to the Core . . . . . . . . . . . . . . . . . . . . . . . . . 148}
\secdown
\secrel{15.3.1 Explicit Parametric Polymorphism . . . . . . . . . . . . . . . 148}
\secrel{15.3.2 Type Inference . . . . . . . . . . . . . . . . . . . . . . . . . 155}
\secrel{15.3.3 Union Types . . . . . . . . . . . . . . . . . . . . . . . . . . 164}
\secrel{15.3.4 Nominal Versus Structural Systems . . . . . . . . . . . . . . 170}
\secrel{15.3.5 Intersection Types . . . . . . . . . . . . . . . . . . . . . . . 171}
\secrel{15.3.6 Recursive Types . . . . . . . . . . . . . . . . . . . . . . . . 172}
\secrel{15.3.7 Subtyping . . . . . . . . . . . . . . . . . . . . . . . . . . . . 173}
\secrel{15.3.8 Object Types . . . . . . . . . . . . . . . . . . . . . . . . . . 176}
\secup
\secup

\secrel{16 Checking Program Invariants Dynamically: Contracts 179}\secdown
\secrel{16.1 Contracts as Predicates . . . . . . . . . . . . . . . . . . . . . . . . . 181}
\secrel{16.2 Tags, Types, and Observations on Values . . . . . . . . . . . . . . . . 182}
\secrel{16.3 Higher-Order Contracts . . . . . . . . . . . . . . . . . . . . . . . . . 183}
\secrel{16.4 Syntactic Convenience . . . . . . . . . . . . . . . . . . . . . . . . . 187}
\secrel{16.5 Extending to Compound Data Structures . . . . . . . . . . . . . . . . 188}
\secrel{16.6 More on Contracts and Observations . . . . . . . . . . . . . . . . . . 189}
\secrel{16.7 Contracts and Mutation . . . . . . . . . . . . . . . . . . . . . . . . . 189}
\secrel{16.8 Combining Contracts . . . . . . . . . . . . . . . . . . . . . . . . . . 190}
\secrel{16.9 Blame . . . . . . . . . . . . . . . . . . . . . . . . . . . . . . . . . . 191}
\secup

\secrel{17 Alternate Application Semantics 195}\secdown
\secrel{17.1 Lazy Application . . . . . . . . . . . . . . . . . . . . . . . . . . . . 196}
\secdown
\secrel{17.1.1 A Lazy Application Example . . . . . . . . . . . . . . . . . . 196}
\secrel{17.1.2 What Are Values? . . . . . . . . . . . . . . . . . . . . . . . 197}
\secrel{17.1.3 What Causes Evaluation? . . . . . . . . . . . . . . . . . . . 198}
\secrel{17.1.4 An Interpreter . . . . . . . . . . . . . . . . . . . . . . . . . . 199}
\secrel{17.1.5 Laziness and Mutation . . . . . . . . . . . . . . . . . . . . . 201}
\secrel{17.1.6 Caching Computation . . . . . . . . . . . . . . . . . . . . . 201}
\secup
\secrel{17.2 Reactive Application . . . . . . . . . . . . . . . . . . . . . . . . . . 201}
\secdown
\secrel{17.2.1 Motivating Example: A Timer . . . . . . . . . . . . . . . . . 202}
\secrel{17.2.2 Callback Types are Four-Letter Words . . . . . . . . . . . . . 203}
\secrel{17.2.3 The Alternative: Reactive Languages . . . . . . . . . . . . . 204}
\secrel{17.2.4 Implementing Transparent Reactivity . . . . . . . . . . . . . 205}
\secup
\secup



\end{document}
