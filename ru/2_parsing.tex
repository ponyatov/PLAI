\secrel{2 Everything (We Will Say) About Parsing
\ru{Все (что мы будем говорить) о разборе}}\secdown

Parsing is the act of turning an input character stream into a more structured,
internal representation.
\ru{\termdef{Парсинг}{парсинг} или \termdef{разбор}{синтаксический разбор}\
--- процесс превращения входного потока одиночных символов в более
структурированное внутреннее представление\note{программы или данных, заданных в
текстовой синтаксической форме}.}
A common internal representation is as a tree, which programs can recursively
process.
\ru{Обычно используется внутреннее представление в виде дерева, которое может
быть обработано программой рекурсивно.}

For instance, given the stream
\ru{Например, для входного потока символов\note{включая пробелы, табуляции и
концы строк}}
\begin{verbatim}
23 + 5 - 6
\end{verbatim}
we might want a tree representing addition whose left (L) node represents the
number 23 and whose right (R) node represents subtraction of 6 from 5.
\ru{мы хотим получить деревянное представление сложения, в котором левая (L)
ветвь содержит число 23, а правая (R)\ --- вложенное представление вычитания 6
из 5.}
A parser is responsible for performing this transformation.
\ru{Парсер отвечает за выполнение такой транформации.}

\noindent
\begin{tabular}{c c}
\noindent\includegraphics[height=0.6\textheight]{tmp/2_p10_R.pdf}
&
\noindent\includegraphics[height=0.6\textheight]{tmp/2_p10_L.pdf}
\\
\emph{Право}ассоциативный разбор
&
(*) \emph{Лево}ассоциативный разбор
\\
\end{tabular}\bigskip

Parsing is a large, complex problem that is far from solved due to the
difficulties of ambiguity.
\ru{Парсинг\ --- большая проблема информатики, сложность которой
заключается в трудностях неоднозначности.}
For instance, an alternate parse tree (*) for the above input expression might
put subtraction at the top and addition below it.
\ru{Например, существует альтернативное (*) \termdef{дерево разбора}{дерево
разбора} для того же входного выражения, в котором мы можем поместить
вычитание на вершину дерева, а сложение будет вложенным поддеревом.}
We might also want to consider whether this addition operation is commutative
and hence whether the order of arguments can be switched.
\ru{Нас также может интересовать, является ли это сложение коммутативной
операцией, то есть можем ли мы изменить порядок аргументов\note{например для
оптимизации кода}.}
Everything only gets much, much worse when we get to full-fledged programming
languages (to say nothing of natural languages).
\ru{Все становится намного хуже по мере того, как мы пробираемся в сторону
полноценных языков программирования (даже не говоря о натуральных языках).}

\secrel{A Lightweight, Built-In First Half of a Parser
\ru{Легковесная встроенная часть парсера}}

These problems make parsing a worthy topic in its own right, and entire books,
tools, and courses are devoted to it.
\ru{Эти проблемы делают разбор достойной темой саму по себе, и ей посвящены
целые книги, утилиты и учебные курсы, например первая глава ``Книги Дракона''
\cite{dragon}.}
However, from our perspective parsing is mostly a distraction, because we want
to study the parts of programming languages that are not parsing.
\ru{Однако, с нашей точки зрения тема парсинга является сильным отвлечением,
так как есть другие более достойные темы, касающиеся реализации языков
программирования.}
We will therefore exploit a handy feature of Racket to manage the transformation
of input streams into trees: 
\ru{Поэтому мы сделаем финт ушами, и будем использовать встроенную фичу
\racket а для получения готовых деревьев разбора из входного потока: функцию}
\verb|read|.
\verb|read| is tied to the parenthetical form of the language, in that it parses
fully (and hence unambiguously) parenthesized terms into a built-in tree form.
\verb|read| \ru{привязана к скобочной форме языка, и полностью (и следовательно
однозначно) разбирает скобочные выражения во встроенное представление\ ---
дерево.}
For instance, running \ru{Например применение} \verb|(read)| on the
parenthesized form of the above input \ru{к следующему входному потоку символов
(включающему скобки)}\ ---
\begin{verbatim}
(+ 23 (- 5 6))
\end{verbatim}
--- will produce a list, whose first element is the symbol \ru{создаст список, в
котором первым элементом будет символ} \verb|'+|, second element is the number
\ru{вторым элементом число} 23, and third element is a list \ru{и третий
элемент список}: this list’s first element is the
symbol \ru{в котором первым элемент будет} \verb|'-|, second element is the
number \ru{второй элемент число} 5, and third element is the number \ru{и
третий элемент число} 6.

\fig{\ru{Дерево разбора для} (+ 23 (- 5 6))}{tmp/2_1.pdf}{height=0.7\textheight}

\lst{src/2/p10_1.rkt}

\secrel{2.2 A Convenient Shortcut . . . . . . . . . . . . . . . . . . . . . . . . . 10}
\secrel{2.3 Types for Parsing . . . . . . . . . . . . . . . . . . . . . . . . . . . . 11}
\secrel{2.4 Completing the Parser . . . . . . . . . . . . . . . . . . . . . .. . . 12}\label{sec2_4}
\secrel{2.5 Coda . . . . . . . . . . . . . . . . . . . . . . . . . . . . . . . . . . . 13}
\secup
