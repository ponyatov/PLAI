\secrel{Дерево Sym-классов}\secdown

\secrel{базовый класс \class{Sym}}

\lstx{базовый класс \class{Sym}}{core/py/sym01.py}{Python}
\begin{description}

\item[хранить значение]\ \\
\noindent\lstx{\class{val}}{core/py/sym02.py}{Python}

\item[идентифицировать свой тип]\ \\
идентификация типа для динамических языков возможна \emph{за счет использования
механизмов виртуальной машины языка реализации}, но для совместимости применим
общую схему тэгов:
\noindent\lstx{\class{tag}}{core/py/sym03.py}{Python}

\item[конструктор = лексемы]\ \\
\noindent\lstx{конструктор}{core/py/sym04.py}{Python}

\item[выводить себя в текстовом представлении]\ \\
\noindent\lstx{дамп}{core/py/sym05.py}{Python}

\item[хранить в себе вложенные элементы]\ \\
\noindent\lstx{вложенные элементы}{core/py/sym06.py}{Python}

\end{description}

\secup