\secrel{The Interpreter, Resumed\\\ru{Возвращаемся к
интерпретатору}}\label{interpresumed}

Phew! Now that we’ve completed the definition of substitution (or so we think),
let’s complete the interpreter. Substitution was a heavyweight step, but it also
does much of the work involved in applying a function. It is tempting to write
\lsts{src/5/5_4_1.rkt}{rkt}
Tempting, but wrong.

\DoNow{
Do you see why?
}

Reason from the types. What does the interpreter return? Numbers. What does
substitution return? Oh, that’s right, expressions! For instance, when we
substituted in the body of double, we got back the representation of (+ 5 5).
This is not a valid answer for the interpreter. Instead, it must be reduced to
an answer. That, of course, is precisely what the interpreter does:
\lsts{src/5/5_4_2.rkt}{rkt}
Okay, that leaves only one case: identifiers. What could possibly be complicated
about them? They should be just about as simple as numbers! And yet we’ve put
them off to the very end, suggesting something subtle or complex is afoot.

\DoNow{
Work through some examples to understand what the interpreter should do in the
identifier case.
}

Let’s suppose we had defined double as follows:
\lsts{src/5/5_4_3.rkt}{rkt}
When we substitute 5 for x, this produces the expression (+ 5 y). So far so
good, but what is left to substitute y? As a matter of fact, it should be clear
from the very outset that this definition of double is erroneous. The identifier
y is said to be free, an adjective that in this setting has negative
connotations.

In other words, the interpreter should never confront an identifier. All
identifiers ought to be parameters that have already been substituted (known as
bound identifiers—here, a positive connotation) before the interpreter ever sees
them. As a result, there is only one possible response given an identifier:
\lsts{src/5/5_4_4.rkt}{rkt}
And that’s it!

Finally, to complete our interpreter, we should define get-fundef:
\lsts{src/5/5_4_5.rkt}{rkt}
