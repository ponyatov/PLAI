\secrel{10.1.2 Objects by Desugaring  69}

While defining objects in the core language may be worthwhile, it’s an unwieldy
way to go about studying them. Instead, we’ll use Racket to represent objects,
sticking to the parts of the language we already know how to implement in our
interpreter. That is, we’ll assume that we are looking at the output of
desugaring. (For this reason, we’ll also stick to stylized code, potentially
writing unnecessary expressions on the grounds that this is what a simple
program generator would produce.)

\begin{framed}
Alert: All the code that follows will be in \verb|#lang plai|, not in the typed
language.
\end{framed}

\Exercise{
Why \#lang plai? What problems do you encounter when you try to type the
following code? Are some of them amenable to easy fixes, such as introducing a
new datatype and applying it consistently? How about if we make simplifications
for the purposes of modeling, such as assuming methods have only one argument?
Or are some of them less tractable?
}
