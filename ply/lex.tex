\secrel{Lex}\secdown

lex.py is used to tokenize an input string. For example, suppose you're writing
a programming language and a user supplied the following input string:
\begin{verbatim}
x = 3 + 42 * (s - t)
\end{verbatim}
A tokenizer splits the string into individual tokens
\begin{verbatim}
'x','=', '3', '+', '42', '*', '(', 's', '-', 't', ')'
\end{verbatim}
Tokens are usually given names to indicate what they are. For example:
\begin{verbatim}
'ID','EQUALS','NUMBER','PLUS','NUMBER','TIMES',
'LPAREN','ID','MINUS','ID','RPAREN'
\end{verbatim}
More specifically, the input is broken into pairs of token types and values. For
example:
\begin{verbatim}
('ID','x'), ('EQUALS','='), ('NUMBER','3'), 
('PLUS','+'), ('NUMBER','42), ('TIMES','*'),
('LPAREN','('), ('ID','s'), ('MINUS','-'),
('ID','t'), ('RPAREN',')'
\end{verbatim}
The identification of tokens is typically done by writing a series of regular
expression rules. The next section shows how this is done using lex.py.

\secrel{Lex Example}

The following example shows how lex.py is used to write a simple tokenizer.
\lstt{\file{calclex.py}}{ply/calclex.py}
To use the lexer, you first need to feed it some input text using its input()
method. After that, repeated calls to token() produce tokens. The following code
shows how this works:
\lstt{\file{calclex.py}}{ply/calclex_test.py}
When executed, the example will produce the following output:
\lstt{\file{calclex.py}}{ply/calclex.log}
Lexers also support the iteration protocol. So, you can write the above loop as
follows:
\lstx{iteration}{ply/iter.py}{Python}
The tokens returned by lexer.token() are instances of LexToken. This object has
attributes tok.type, tok.value, tok.lineno, and tok.lexpos. The following code
shows an example of accessing these attributes:
\lstx{tokenize}{ply/tokenize.py}{Python}
The tok.type and tok.value attributes contain the type and value of the token
itself. tok.line and tok.lexpos contain information about the location of the
token. tok.lexpos is the index of the token relative to the start of the input
text.

\secrel{The tokens list}

All lexers must provide a list tokens that defines all of the possible token
names that can be produced by the lexer. This list is always required and is
used to perform a variety of validation checks. The tokens list is also used by
the yacc.py module to identify terminals.

In the example, the following code specified the token names:
\lstx{\var{tokens}}{ply/tokens.py}{Python}

\secrel{Specification of tokens}

Each token is specified by writing a regular expression rule compatible with
Python's re module. Each of these rules are defined by making declarations with
a special prefix t\_ to indicate that it defines a token. For simple tokens, the
regular expression can be specified as strings such as this (note: Python raw
strings are used since they are the most convenient way to write regular
expression strings):
\lstx{\var{t\_PLUS}}{ply/plus.py}{Python}
In this case, the name following the t\_ must exactly match one of the names
supplied in tokens. If some kind of action needs to be performed, a token rule
can be specified as a function. For example, this rule matches numbers and
converts the string into a Python integer.
\lstx{\var{t\_NUMBER}}{ply/number.py}{Python}
When a function is used, the regular expression rule is specified in the
function documentation string. The function always takes a single argument which
is an instance of LexToken. This object has attributes of t.type which is the
token type (as a string), t.value which is the lexeme (the actual text matched),
t.lineno which is the current line number, and t.lexpos which is the position of
the token relative to the beginning of the input text. By default, t.type is set
to the name following the t\_ prefix. The action function can modify the
contents of the LexToken object as appropriate. However, when it is done, the
resulting token should be returned. If no value is returned by the action
function, the token is simply discarded and the next token read.

Internally, lex.py uses the re module to do its pattern matching. Patterns are
compiled using the re.VERBOSE flag which can be used to help readability.
However, be aware that unescaped whitespace is ignored and comments are allowed
in this mode. If your pattern involves whitespace, make sure you use \verb|\s|.
If you need to match the \verb|#| character, use \verb|[#]|.

When building the master regular expression, rules are added in the following
order:
\begin{enumerate}[nosep]
  \item 
All tokens defined by functions are added in the same order as they appear in
the lexer file.
  \item 
Tokens defined by strings are added next by sorting them in order of decreasing
regular expression length (longer expressions are added first).
\end{enumerate}

Without this ordering, it can be difficult to correctly match certain types of
tokens. For example, if you wanted to have separate tokens for "=" and "==", you
need to make sure that "==" is checked first. By sorting regular expressions in
order of decreasing length, this problem is solved for rules defined as strings.
For functions, the order can be explicitly controlled since rules appearing
first are checked first.

To handle reserved words, you should write a single rule to match an identifier
and do a special name lookup in a function like this:
\lstt{lookup}{ply/reserved.py}
This approach greatly reduces the number of regular expression rules and is
likely to make things a little faster.

\paragraph{Note}: You should avoid writing individual rules for reserved words.
For example, if you write rules like this,
\lstx{\ }{ply/for.py}{Python}
those rules will be triggered for identifiers that include those words as a
prefix such as "forget" or "printed". This is probably not what you want.

\secrel{Token values}

When tokens are returned by lex, they have a value that is stored in the value
attribute. Normally, the value is the text that was matched. However, the value
can be assigned to any Python object. For instance, when lexing identifiers, you
may want to return both the identifier name and information from some sort of
symbol table. To do this, you might write a rule like this:
\lstx{symbol table lookup}{ply/values.py}{Python}
It is important to note that storing data in other attribute names is not
recommended. The yacc.py module only exposes the contents of the value
attribute. Thus, accessing other attributes may be unnecessarily awkward. If you
need to store multiple values on a token, assign a tuple, dictionary, or
instance to value.

\secrel{Discarded tokens}

To discard a token, such as a comment, simply define a token rule that returns
no value. For example:
\lstx{comment token}{ply/comment.py}{Python}
Alternatively, you can include the prefix \verb|ignore_| in the token
declaration to force a token to be ignored. For example:
\lstx{ignore\_comment}{ply/commentign.py}{Python}
Be advised that if you are ignoring many different kinds of text, you may still
want to use functions since these provide more precise control over the order in
which regular expressions are matched (i.e., functions are matched in order of
specification whereas strings are sorted by regular expression length).

\secrel{Line numbers and positional information}

By default, lex.py knows nothing about line numbers. This is because lex.py
doesn't know anything about what constitutes a "line" of input (e.g., the
newline character or even if the input is textual data). To update this
information, you need to write a special rule. In the example, the
\verb|t_newline()| rule shows how to do this.
\lstx{track line numbers}{ply/newline.py}{Python}
Within the rule, the lineno attribute of the underlying lexer t.lexer is
updated. After the line number is updated, the token is simply discarded since
nothing is returned.

lex.py does not perform and kind of automatic column tracking. However, it does
record positional information related to each token in the lexpos attribute.
Using this, it is usually possible to compute column information as a separate
step. For instance, just count backwards until you reach a newline.
\lstx{compute column}{ply/column.py}{Python}
Since column information is often only useful in the context of error handling,
calculating the column position can be performed when needed as opposed to doing
it for each token.

\secrel{Ignored characters}

The special \verb|t_ignore| rule is reserved by lex.py for characters that
should be completely ignored in the input stream. Usually this is used to skip
over whitespace and other non-essential characters. Although it is possible to
define a regular expression rule for whitespace in a manner similar to
\verb|t_newline()|, the use of \verb|t_ignore| provides substantially better
lexing performance because it is handled as a special case and is checked in a
much more efficient manner than the normal regular expression rules.
\lstx{ignored characters}{ply/ignored.py}{Python}

The characters given in \verb|t_ignore| are not ignored when such characters are
part of other regular expression patterns. For example, if you had a rule to
capture quoted text, that pattern can include the ignored characters (which will
be captured in the normal way). The main purpose of \verb|t_ignore| is to ignore
whitespace and other padding between the tokens that you actually want to parse.

\secrel{Literal characters}

Literal characters can be specified by defining a variable literals in your
lexing module. For example:
\lstx{\var{literals}}{ply/litlist.py}{Python}
or alternatively
\lstx{\var{literals}}{ply/litstr.py}{Python}

A literal character is simply a single character that is returned "as is" when
encountered by the lexer. Literals are checked after all of the defined regular
expression rules. Thus, if a rule starts with one of the literal characters, it
will always take precedence.

When a literal token is returned, both its type and value attributes are set to
the character itself. For example, '+'.

It's possible to write token functions that perform additional actions when
literals are matched. However, you'll need to set the token type appropriately.
For example:
\lstx{literal actions}{ply/literals.py}{Python}

\secrel{Error handling}

The \verb|t_error()| function is used to handle lexing errors that occur when
illegal characters are detected. In this case, the \verb|t.value| attribute
contains the rest of the input string that has not been tokenized. In the
example, the error function was defined as follows:
\lstx{\var{t\_error}}{ply/error.py}{Python}
In this case, we simply print the offending character and skip ahead one
character by calling \verb|t.lexer.skip(1)|.

\secrel{EOF Handling}

The \verb|t_eof()| function is used to handle an end-of-file (EOF) condition in
the input. As input, it receives a token type \verb|'eof'| with the \var{lineno}
and \var{lexpos} attributes set appropriately. The main use of this function is
provide more input to the lexer so that it can continue to parse.
Here is an example of how this works:
\lstx{\var{t\_eof}}{ply/eof.py}{Python}
The EOF function should return the next available token (by calling
\verb|self.lexer.token()|) or \verb|None| to indicate no more data. Be aware
that setting more input with the \verb|self.lexer.input()| method does NOT reset
the lexer state or the lineno attribute used for position tracking. The lexpos
attribute is reset so be aware of that if you're using it in error reporting.

\secrel{Building and using the lexer}

To build the lexer, the function lex.lex() is used. For example:
\lstx{build lexer}{ply/lex.py}{Python}
This function uses Python reflection (or introspection) to read the regular
expression rules out of the calling context and build the lexer. Once the lexer
has been built, two methods can be used to control the lexer.
\begin{itemize}[nosep]
  \item 
\verb|lexer.input(data)|\\Reset the lexer and store a new input string.
  \item 
\verb|lexer.token()|\\Return the next token. Returns a special LexToken instance
on success or None if the end of the input text has been reached.
\end{itemize}

\secrel{The @TOKEN decorator}
\secrel{Optimized mode}
\secrel{Debugging}
\secrel{Alternative specification of lexers}
\secrel{Maintaining state}
\secrel{Lexer cloning}
\secrel{Internal lexer state}
\secrel{Conditional lexing and start conditions}
\secrel{Miscellaneous Issues}
\secup
