\secrel{PLY Overview}

PLY consists of two separate modules; lex.py and yacc.py, both of which are
found in a Python package called ply. The lex.py module is used to break input
text into a collection of tokens specified by a collection of regular expression
rules. yacc.py is used to recognize language syntax that has been specified in
the form of a context free grammar.

The two tools are meant to work together. Specifically, lex.py provides an
external interface in the form of a token() function that returns the next valid
token on the input stream. yacc.py calls this repeatedly to retrieve tokens and
invoke grammar rules. The output of yacc.py is often an Abstract Syntax Tree
(AST). However, this is entirely up to the user. If desired, yacc.py can also be
used to implement simple one-pass compilers.

Like its Unix counterpart, yacc.py provides most of the features you expect
including extensive error checking, grammar validation, support for empty
productions, error tokens, and ambiguity resolution via precedence rules. In
fact, almost everything that is possible in traditional yacc should be supported
in PLY.

The primary difference between yacc.py and Unix yacc is that yacc.py doesn't
involve a separate code-generation process. Instead, PLY relies on reflection
(introspection) to build its lexers and parsers. Unlike traditional lex/yacc
which require a special input file that is converted into a separate source
file, the specifications given to PLY are valid Python programs. This means that
there are no extra source files nor is there a special compiler construction
step (e.g., running yacc to generate Python code for the compiler). Since the
generation of the parsing tables is relatively expensive, PLY caches the results
and saves them to a file. If no changes are detected in the input source, the
tables are read from the cache. Otherwise, they are regenerated.

