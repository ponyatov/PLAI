\secrel{Introduction}

PLY is a pure-\py implementation of the popular compiler construction tools
\lex\ and \yacc. The main goal of PLY is to stay fairly faithful to the way in
which traditional \prog{lex}/\prog{yacc} tools work. This includes supporting
LALR(1) parsing as well as providing extensive input validation, error
reporting, and diagnostics. Thus, if you've used yacc in another programming
language, it should be relatively straightforward to use PLY.

Early versions of PLY were developed to support an Introduction to Compilers
Course I taught in 2001 at the University of Chicago. Since PLY was primarily
developed as an instructional tool, you will find it to be fairly picky about
token and grammar rule specification. In part, this added formality is meant to
catch common programming mistakes made by novice users. However, advanced users
will also find such features to be useful when building complicated grammars for
real programming languages. It should also be noted that PLY does not provide
much in the way of bells and whistles (e.g., automatic construction of abstract
syntax trees, tree traversal, etc.). Nor would I consider it to be a parsing
framework. Instead, you will find a bare-bones, yet fully capable lex/yacc
implementation written entirely in \py.

The rest of this document assumes that you are somewhat familiar with parsing
theory, syntax directed translation, and the use of compiler construction tools
such as lex and yacc in other programming languages. If you are unfamiliar with
these topics, you will probably want to consult an introductory text such as
"Compilers: Principles, Techniques, and Tools", by Aho, Sethi, and Ullman.
O'Reilly's "Lex and Yacc" by John Levine may also be handy. In fact, the
O'Reilly book can be used as a reference for PLY as the concepts are virtually
identical.
