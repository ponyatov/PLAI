\secrel{9.1 Recursive and Cyclic Data  . . 62}

Recursion in data can refer to one of two things. It can mean referring to
something of the same kind, or referring to the same thing itself.

Recursion of the same kind leads to what we traditionally call recursive data.
For instance, a tree is a recursive data structure: each vertex can have
multiple children, each of which is itself a tree. But if we write a procedure
to traverse the nodes of a tree, we expect it to terminate without having to
keep track of which nodes it has already visited. They are finite data
structures.

In contrast, a graph is often a cyclic datum: a node refers to another node,
which may refer back to the original one. (Or, for that matter, a node may refer
directly to itself.) When we traverse a graph, absent any explicit checks for
what we have already visited, we should expect a computation to diverge, i.e.,
not terminate. Instead, graph algorithms need a memory of what they have visited
to avoid repeating traversals.

Adding recursive data, such as lists and trees, to our language is quite
straightforward. We mainly require two things:
\begin{enumerate}[nosep]
  \item 
The ability to create compound structures (such as nodes that have references to
children).
  \item 
The ability to bottom-out the recursion (such as leaves).
\end{enumerate}

\Exercise{
Add lists and binary trees as built-in datatypes to the programming language.
}

Adding cyclic data is more subtle. Consider the simplest form of cyclic datum, a
cell referring back to itself:
\bigskip

Let’s try to define this in Racket. Here’s one attempt:
\lsts{src/9/1/1.rkt}{rkt}
But this doesn’t work: b on the right-hand side of the let isn’t bound. It’s
easy to see if we desugar it:
\lsts{src/9/1/2.rkt}{rkt}
and, for clarity, we can rename the b in the function:
\lsts{src/9/1/3.rkt}{rkt}
Now it’s patently clear that b is unbound.

Absent some magical Racket construct we haven’t yet seen, it becomes clear that
we can’t create a cyclic datum in one shot. Instead, we need to first create a
“place” for the datum, then refer to that place within itself. The use of
“then”—i.e., the introduction of time—should suggest a mutation operation.
Indeed, let’s try it with boxes.
\note{That construct would be shared, but virtually no other language has this
notational mechanism, so we won’t dwell on it here. In fact, what we are
studying is the main idea behind how shared actually works.}

Our plan is as follows. First, we want to create a box and bind it to some
identifier, say b. Now, we want to mutate the content of the box. What do we
want it to contain? A reference to itself. How does it obtain that reference? By
using the name, b, that is already bound to it. In this way, the mutation
creates the cycle:
\lsts{src/9/1/4.rkt}{rkt}
Note that this program will not run in Typed PLAI as written. We’ll return to
typing such programs later \ref{}. For now, run it in the untyped (\#lang plai)
language.

When the above program is Run, Racket prints this as: \verb|#0='#&#0#|. This
notation is in fact precisely what we want. Recall that \verb|#&| is how Racket
prints boxes. The \verb|#0=| (and similarly for other numbers) is how Racket
names pieces of cyclic data. Thus, Racket is saying, “\verb|#0| is bound to a
box whose content is \verb|#0#|, i.e., whatever is bound to \verb|#0|, i.e.,
itself”.

\Exercise{
Run the equivalent program through your interpreter for boxes and make sure it
produces a cyclic value. How do you check this?
}

The idea above generalizes to other datatypes. In this same way we can also
produce cyclic lists, graphs, and so on. The central idea is this two-step
process: first name an vacant placeholder; then mutate the placeholder so its
content is itself; to obtain “itself”, use the name previously bound. Of course,
we need not be limited to “self-cycles”: we can also have mutually-cyclic data
(where no one element is cyclic but their combination is).
