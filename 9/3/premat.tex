\secrel{9.3 Premature Observation   . 65}

Our preceding discussion of this pattern shows a clear temporal sequencing:
create, update, use. We can capture it in a desugaring rule. Suppose we add the
following new syntax:
\lsts{9/3/1.rkt}{rkt}
As an example of its use,
\lsts{9/3/2.rkt}{rkt}
would evaluate to the factorial of 10. This new syntax would desugar to:
\lsts{9/3/3.rkt}{rkt}
Where we assume that all references to name in value and body have been
rewritten to (unbox name), or alternatively that we instead use variables:
\lsts{9/3/4.rkt}{rkt}

This naturally inspires a question: what if we get these out of order? Most
interestingly, what if we try to use name before we’re done updating its true
value into place? Then we observe the state of the system right after creation,
i.e., we can see the placeholder in its raw form.

The simplest example that demonstrates this is as follows:
\lsts{9/3/5.rkt}{rkt}
or equivalently,
\lsts{9/3/6.rkt}{rkt}
In most Racket variants, this leaks the initial value given to the placeholder—a
value that was never meant for public consumption. This is troubling because it
is, after all, a legitimate value, which means it can probably be used in at
least some computations. If a developer accesses and uses it inadvertently,
however, they are effectively computing with nonsense.

There are generally three solutions to this problem:
\begin{enumerate}
  \item 
Make sure the value is sufficiently obscure so that it can never be used in a
meaningful context. This means values like 0 are especially bad, and indeed
most common datatypes should be shunned. Instead, the language might create
a new type of value just for use here. Passed to any other operation, this will
result in an error.
  \item 
Explicitly check every use of an identifier for belonging to this special “premature”
value. While this is technically feasible, it imposes an enormous performance
penalty on a program. Thus, it is usually only employed in teaching
languages.
  \item 
Allow the recursion constructor to be used only in the case of binding functions,
and then make sure that the right-hand side of the binding is syntactically a function.
Unfortunately, this solution can be a bit drastic because it precludes writing,
for instance, structures to create graphs.
\end{enumerate}
