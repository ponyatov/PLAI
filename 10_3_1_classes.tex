\secrel{10.3.1 Classes   . . 76}

Immediately we see a difficulty. Is this the constructor pattern?
\lsts{src/10/3/1/1.rkt}{rkt}
That suggests that the parent is at the “same level” as the object’s constructor
fields. That seems reasonable, in that once all these parameters are given, the
object is “fully defined”. However, we also still have
\lsts{src/10/3/1/2.rkt}{rkt}
Are we going to write all the parameters twice? (Whenever we write something
twice, we should worry that we may not do so consistently, thereby inducing
subtle errors.) Here’s an alternative: node/size can construct the instance of
node that is its parent. That is, node/size’s parent parameter is not the parent
object but rather the parent’s object maker.
\lsts{src/10/3/1/3.rkt}{rkt}
\lsts{src/10/3/1/4.rkt}{rkt}
Then the object constructor must remember to pass the parent-object maker on
every invocation:
\lsts{src/10/3/1/5.rkt}{rkt}
Obviously, this is something we might simplify with appropriate syntactic sugar.
We can confirm that both the old and new tests still work:
\lsts{src/10/3/1/6.rkt}{rkt}

\Exercise{
Rewrite this block of code using self-application instead of mutation.
}

What we have done is capture the essence of a class. Each function parameterized
over a parent is...well, it’s a bit tricky, really. Let’s call it a blob for
now. A blob corresponds to what a Java programmer defines when they write a
class:
\lsts{src/10/3/1/7.rkt}{rkt}

\DoNow{
So why are we going out of the way to not call it a “class”?
}

When a developer invokes a Java class’s constructor, it in effect constructs
objects all the way up the inheritance chain (in practice, a compiler might
optimize this to require only one constructor invocation and one object
allocation). These are private copies of the objects corresponding to the parent
classes (private, that is, up to the presence of static members). There is,
however, a question of how much of these objects is visible. Java chooses
that—unlike in our implementation above—only one method of a given name (and
signature) remains, no matter how many there might have been on the inheritance
chain, whereas every field remains in the result, and can be accessed by
casting. The latter makes some sense because each field presumably has
invariants governing it, so keeping them separate (and hence all present) is
wise. In contrast, it is easy to imagine an implementation that also makes all
the methods available, not only the ones lowest (i.e., most refined) in the
inheritance hierarchy. Many scripting languages take the latter approach.

\Exercise{
The code above is fundamentally broken. The self reference is to the same
syntactic object, whereas it needs to refer to the most-refined object:
this is known as open recursion. Modify the object representations so that self
always refers to the most refined version of the object. Hint: You will find the
self-application method (section 10.1.8.2) of recursion handy
\note{This demonstrates the other form of extensibility we get from traditional
objects: extensible recursion.}
}