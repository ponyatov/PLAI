\secrel{12.4 One More Example   90}

Let’s consider one more representation change. What is an environment?

An environment is a map from names to values (or locations, once we have
mutation). We’ve chosen to implement the mapping through a data structure,
but...do we have another way to represent maps? As functions, of course! An
environment, then, is a function that takes a name as an argument and return its
bound value (or an error):
\lsts{12/4/1.rkt}{rkt}
What is the empty environment? It’s the one that returns an error no matter what
name you try to look up:
\lsts{12/4/2.rkt}{rkt}
(In principle we should put a type annotation on the return, and it should be
Value, except of course thiis is vacuous.) Extending an environment with a
binding creates a function that takes a name and checks whether it is the name
just extended; if so it returns the corresponding value, otherwise it punts to
the environment being extended:
\lsts{12/4/3.rkt}{rkt}
Finally, how do we look up a name in an environment? We simply apply the
environment!
\lsts{12/4/4.rkt}{rkt}
And with that, we’re done!
