\secrel{12.1 Changing Representations  . . 87}

For a moment, let’s explore numbers. Racket’s numbers make a good target for
reuse because they are so powerful: we get arbitrary-sized integers (bignums),
rationals (which benefit from the bignum representation of integers), complex
numbers, and so on. Therefore, they can represent most ordinary programming
language number systems. However, that doesn’t mean they are what we want: they
could be too little or too much.
\begin{itemize}
  \item 
They are too much if what we want is a more restricted number system. For
instance, Java prescribes a very specific set of fixed-size representations
(e.g., int is specified to be 32-bit). Numbers that fall outside these sets
cannot be directly represented as numbers, and arithmetic must respect these
sets (e.g., overflowing so that adding 1 to 2147483647 does not produce
2147483648).
  \item 
They are too little if we want even richer numbers, whether quaternions or
numbers with associated probabilities.
\end{itemize}
Worse, we didn’t even stop and ask what we wanted, but blithely proceeded with
Racket numbers.

The reason we did so is because we weren’t really interested in the study of
numbers; rather, we were interested in programming language features such as
functions-asvalues. As language designers, however, you should be sure to ask
these hard questions up front.

Now let’s talk about our representation of closures. We could have instead
represented closures by exploiting Racket’s corresponding concept, and
correspondingly, function application with unvarnished Racket application.

\DoNow{
Replace the closure data structure with Racket functions representing
functions-as-values.
}

Here we go:
\lsts{src/12/1/1.rkt}{rkt}

\Exercise{
Observe a curious shift. In our previous implementation, the environment was
extended in the appC case. Here, it’s extended in the lamC case. Is one of these
incorrect? If not, why did this change occur?
}

This is certainly concise, but we’ve lost something very important:
understanding. Saying that a source language function corresponds to lambda
tells us virtually nothing: if we already knew precisely what lambda does we
might not be studying it, and if we didn’t, this mapping would teach us
absolutely nothing (and might, in fact, pile confusion on top of our ignorance).
For the same reason, we did not use Racket’s state to understand the varieties
of stateful operators \ref{}.

Once we’ve understood the feature, however, we should feel to use it as a
representation. Indeed, doing so might yield much more concise interpreters
because we aren’t doing everything manually. In fact, some later interpreters
\ref{}\ will become virtually unreadable if we did not exploit these richer
representations. Nevertheless, exploiting host language features has perils that
we should safeguard against.
\note{It’s a little like saying, “Now that we understand addition in terms of
increment-by-one, we can use addition to define multiplication: we don’t have to
use only increment-by-one to define it.”}
