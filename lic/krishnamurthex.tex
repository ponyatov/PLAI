\clearpage\label{krishna}
\begin{verbatim}
From: Shriram Krishnamurthi <sk@cs.brown.edu>
Date: Wed, 11 Jan 2017 08:11:22 -0500
To: Dmitry Ponyatov <dponyatov@gmail.com>
\end{verbatim}
% Subject: Re: What legal limitations does your PLAI book have on translation ?

\textcolor{BLUE}{
\noindent
I have some hesitancy on book extending. I'm going to add some extra sections
\begin{enumerate}[nosep]
  \item 
on using set of modern mainstream languages (\cpp, \java, \py, JavaScript)
with some lexer/parser tools for dynamic language core (DLR) realization,
  \item 
metaprogramming and transformational programming and
  \item 
knowledge engineering programming.
\end{enumerate} 
In current version I've tried to embed this extensions in original book, but it
gives lot of muddle, and later I will surely move all extra sections to end of
book, or move it to separate one or two books.
}

\bigskip
Thank you, but I ask that you do not add these extensions to the original book.

\bigskip
I'm authorizing you to \emph{translate} the book, not to rewrite it. At that
point you would be creating a different book with me as a co-author, and I do
not have the time (or language ability) to read and approve of the extensions.
Therefore, your idea of two separate books is ideal.

\bigskip
There is precedent: \'{E}ric Tanter did with his object-oriented extension of
PLAI: \url{https://users.dcc.uchile.cl/~etanter/ooplai/}. The same model seems
to make sense here. You would of course be the sole author of his other
document, and you are welcome to link to it from wherever you see fit in the
original.

\bigskip
The bottom line is, please keep these as two different documents, one written by
me and translated by you, and the other written entirely by you.

\bigskip
Note that the first edition of PLAI (\url{www.plai.org}) has an extensive set of
sections on metaprogramming and transformational programming. The problem is
that the programs in that section are based on untyped Racket rather than the
typed PLAI language. That's why they were not incorporated here.

However, if you don't mind switching languages for that portion, you can just
copy (translate) the sections out of the first edition, just making clear to
readers that you are now switching languages. When I teach in class, that is
actually what I do: just tell students we're switching languages and refer them
back to the chapters in the first edition.

\bigskip\copyright\ Shriram
