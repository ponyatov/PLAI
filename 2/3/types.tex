\secrel{Types for Parsing\\
\ru{Типы для разбора}}

Actually, I’ve lied a little.
\ru{На самом деле, я немного соврал.}
I said that \ru{я сказал что} \verb|(read)|\ --- or equivalently, using
quotation \ru{или использование квотирования, что эквивалентно}\ --- will
produce a \emph{list}, etc. \ru{создаст \emph{список}, блаблабла.}
That’s true in regular \racket, but in $Typed PLAI$
\ru{Это так для оригинального \racket, но в $Typed PLAI$} the type it
returns a distinct type called an \ru{возвращается специальный тип, который
называется} \termdef{s-expression}{s-expression}
\ru{s-выражение\index{s-выражение}}, written in $Typed PLAI$ as
\ru{который в $Typed PLAI$ записывается как}
\verb|s-expression|:
\begin{verbatim}
> (read)
- s-expression
[type in (+ 23 (- 5 6))]
'(+ 23 (- 5 6))
\end{verbatim}
\racket\ has a very rich language of s-expressions
\ru{имеет очень богатый язык на s-выражениях}
(it even has notation to represent cyclic structures
\ru{он даже имеет нотацию для представления циклических структур}), 
but we will use only the simple fragment of it.
\ru{но мы будем использовать только простейшую часть этого синтаксиса.}

In the typed language, an s-expression is treated distinctly from the other
types, such as numbers and lists.
\ru{В типизированном языке s-выражения обрабатыватся обособленно от других
типов, таких как числа и списки.}
\begin{framed}
Underneath, an s-expression is a large
recursive datatype that consists of all the base printable values—numbers,
strings, symbols, and so on—and printable collections (lists, vectors, etc.) of
s-expressions.
\ru{Далее s-выражение рассматривается как большой рекурсивный тип данных,
который содержит все базовые отображаемые (представимые в тексте) значения\
--- числа, строки, символы и т.д.\ --- и коллекции (списки, вектора и т.д.)
других s-выражений}.
\end{framed}
As a result, base types like numbers, symbols, and strings are
\emph{both} their own type and an instance of s-expression.
\ru{В результате такие базовые типы как числа, символы и строки, могут
\emph{одновременно} являться как собственным типом (число,..), так и
экземпляром s-выражениея}.
Typing such data can be fairly problematic, as we will discuss later
\ru{Типизация таких данных может быть очень проблематична, детальнее мы обсудим
это позже} \ref{}.

$Typed PLAI$ takes a simple approach.
\ru{$Typed PLAI$ применяет более простой подход.}
When written on their own, values like numbers are of those respective types.
\ru{Когда значания простых типов, типа чисел, написаны сами по себе, они
являются собственными типами (число).}
But when written inside a complex
s-expression—in particular, as created by read or quotation—they have type
s-expression.
\ru{Но когда они включены в состав сложного s-выражения\ --- в частности,
созданы через (read) или квотацию\ --- они имеют тип s-выражения.}
You have to then cast them to their native types.
\ru{Вы должны привести их к нативному типу.}
For instance \ru{Например}:
\lstl{2/p11_1.rkt}
This is similar to the casting that a Java programmer would have to insert.
\ru{Это похоже на явное приведение типов, которое должен вставить программист
на \java.}
We will study casting itself later \ru{Мы обсудим само приведение типов позже}
\ref{}.

Observe that the first element of the list is still not treated by the
type checker as a symbol:
\ru{Отметим, что первый элемент списка все еще не распознается контролером
типов как символ:}
a list-shaped s-expression is a list of s-expressions.
\ru{списко-образное s-выражение является списком s-выражений.}
Thus \ru{Таким образом},
\lst{2/p11_2.rkt}
whereas again, casting does the trick:
\ru{и снова приведение типов решает проблему :}
\lst{2/p11_3.rkt}
The need to cast s-expressions is a bit of a nuisance,
\ru{Необходимость приведения s-выражений немного геморна,}
but some complexity is unavoidable because of what we’re trying to accomplish:
\ru{но некоторая сложность неизбежна из-за того что мы пытаемся достичь:}
to convert an \emph{untyped input} stream into a \emph{typed output} stream
\ru{преобразование \emph{нетипизированного} входного потока в
\emph{типизированный} выходной поток}
through robustly typed means.
\ru{через средства робастной типизации.}
Somehow we have to make explicit our assumptions about that input stream.
\ru{Каким-то образом мы должны делать явные предположения об этом входном
потоке.}

Fortunately we will use s-expressions only in our parser, and our goal is to
\emph{get away from parsing as quickly as possible\,!}
\ru{К счастью, мы будем использовать s-выражения только в нашем парсере, и наша
цель состоит в том, чтобы \emph{уйти от разбора как можно быстрее\,!}}
Indeed, if anything this should be inducement to get away even quicker.
\ru{В самом деле, все эти заморочки являются побуждением сделать этот уход еще
быстрее.}
