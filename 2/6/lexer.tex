\secrel{\file{lpp.lpp} \ru{лексер}}

Для начала реализуем только лексер, который будет читать файл с простыми
выражениями, и выводить распознанные элементы:

\lstt{arith/src.src}{arith/src.src}

Для его компиляции и запуска используйте команды:

\begin{verbatim}
flex lpp.lpp
g++ -o ./exe.exe lex.yy.c
./exe.exe < src.src
\end{verbatim}

Под \linux\ их удобно записать в одну строку через \verb|&&|

\clearpage
При запуске на пустом файле \file{lpp.lpp} вы получите сообщение:
\begin{verbatim}
$> flex lpp.lpp && g++ -o ./exe.exe lex.yy.c && ./exe.exe < src.src 
lpp.lpp:1: premature EOF
\end{verbatim}

То есть для успешной трансляции у \term{файла лексера} \file{lpp.lpp} должна
быть определенная структура:

\verbatiminput{2/6/struc.lpp}

В простейшем случае мы можем попытаться скомпилировать самый элементарный
пустой лексер из одного обязального элемента: маркера начала блока правил
\lstt{arith/lpp.lp}{2/6/minimal0.lpp}

Команда \verb|flex lpp.lpp| выполнится без ошибок, и создаст 
\file{lex.yy.c}

Попытка его компиляции завершится ошибкой:
\begin{verbatim}
ponyatov@gac:~/PLAI/arith$ g++ -o ./exe.exe lex.yy.c 
/usr/lib/gcc/x86_64-linux-gnu/4.9/../../../x86_64-linux-gnu/crt1.o: In function `_start':
(.text+0x20): undefined reference to `main'
/tmp/cceTYbuX.o: In function `yylex()':
lex.yy.c:(.text+0x3d6): undefined reference to `yywrap'
collect2: error: ld returned 1 exit status
\end{verbatim}

Из отчета об ошибках вы видим проблемы со следующими функциями:

\begin{description}
\item[\fn{main()}]\ \\
отсутствует функция \fn{main()} которая должна вызывать лексер, и возможно
выполнить перед этим какую-то инициализацию, или завершающие операции. \flex\
предоставляет типовую реализацию \fn{main()} при использовании опции \var{main}.
\item[\fn{yywrap()}]\ \\
вызывается лексером при достижении конца файла; нам не нужно выполнять какие-то
особые действия, поэтому применим опцию \var{noyywrap}.
\item[\fn{yylex()}]\ \\
эта функция собственно и является лексером:
\begin{itemize}
\item 
каждый ее вызов выделяет из входного потока \emph{одну} \term{лексему},
\item
функция возвращает целое значение которое определяется командой \verb|return| в
каждом правиле,
\item
переменная \var{char* yytext} содержит ссылку на выделенную
строку\note{ASCIIZ-строка: массив 8-битные символов, оканчивающихся 0x00;
будьте осторожны с кириллицей, \flex\ не умеет из коробки работать с юникодом,
и поддерживает только младшую половину таблицы ASCII}
\end{itemize}
\end{description}