\secrel{$\oplus$ \class{Sym}: \ru{универсальный тип данных}}

Прежде чем мы продолжим с парсером, нам нужно уточнить, как мы
будем хранить результаты разбора. Выше в качестве хранилища \termdef{дерева
разбора}{дерево разбора} используются традиционное для \lisp-мира представление
в виде списков, где \emph{первым элементом идет вершина дерева}, а в следующих
элементах находятся поддеревья.

Но если мы внимательно рассмотрим синтаксис современных main\-stream языков
программирования\note{и текстовых форматов данных}, и учтем что нашей целью
является создание метаязыка для трансформации программ, более удобным и
правильным кажется использование не списков, а \termdef{атрибутных
деревьев}{атрибутное дерево}.

Вместо класса \class{ArithC} нам нужно дерево классов, наследованных от одного
\term{виртуального} базового класса \class{Sym}. Это требование вытекает из
жесткой типизации и отсутствии в \cpp\ поддержки \term{гетерогенных} структур
данных, способных хранить в себе \emph{разнотипные} элементы. Если мы
используем набор типов\note{которые будут хранить в себе элементы нашего дерева
разбора}, наследованных от одного виртуального класса, мы сможем оперировать ими
через указатели на базовый класс \class{Sym*}, и воспользоваться динамическими
хранилищами из библиотеки \cpp\ STL: \class{vector<Sym*>} и
\class{map<string,Sym*>}.

\lstxl{hpp.hpp}{2/6/sym/head.hpp}{C++}
\lstxl{cpp.cpp}{2/6/sym/head.cpp}{C++}

\clearpage
Любой элемент данных должен:
\begin{description}
\item[хранить свое значение и идентифицировать свой тип]\ \\\emph{универсальным
представлением любых данных является строка}:
\lstxl{hpp.hpp}{2/6/sym/tagval.hpp}{C++}

В \cpp\ есть средства идентификации класса по указателю на экземпляр (RTTI,
\fn{typeid()}), и по крайней мере тип тэга должен быть \class{static
Sym*}\note{указывать на экземпляр \class{Clazz::Sym *symbol, *number,
*string,..} т.е. на элемент данных типа ``класс'', существующий внутри нашей
DLR; собственно на текущий момент нам от него нужно только название класса} или
хотя бы \class{static string}, но для максимального \emph{упрощения кода} мы
используем просто строку.

\item[конструктор]\ \\создает элемент данных по паре тэг:значение <T:V>
\lstxl{hpp.hpp}{2/6/sym/constv.hpp}{C++}
\item[лексемы] (=\term{токены} =\term{терминалы})\ \\должны иметь конструктор из
строки, выделенной лексером
\lstxl{hpp.hpp}{2/6/sym/constoc.hpp}{C++}
\lstxl{cpp.cpp}{2/6/sym/constv.cpp}{C++}
\clearpage
\item[базовый класс]\ \\должен иметь \emph{по крайней мере} одну виртуальную
функцию
\item[выводить себя в текстовом представлении]\ \\для отладки или трансляции
\lstxl{hpp.hpp}{2/6/sym/dump.hpp}{C++}
\lstxl{cpp.cpp}{2/6/sym/dump.cpp}{C++}
\end{description}
