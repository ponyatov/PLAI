\secrel{A Convenient Shortcut \ru{Удобный трюк}}

As you know you need to test your programs extensively, which is hard to do when you
must manually type terms in over and over again.
\ru{Как вы знаете, нужно тщательно тестировать свои программы, что особенно
сложно, если вам нужно снова и снова вводить выражения вручную.}
Fortunately, as you might expect, the parenthetical syntax is integrated deeply
into \racket\ through the mechanism of quotation.
\ru{К счатью, как и следовало ожидать, скобочный синтаксис глубоко
интегрирован в \racket\ через механизм \termdef{квотирования}{квотирование}.}
That is, \ru{это то самое выражение} \verb|'<expr>|\ --—
which you saw a moment ago in the above example
\ru{которое вы видели только что при выполнении предыдущего примера}\ --- 
acts as if you had run \ru{действует так же, как если бы вы запустили}
\verb|(read)| and typed \ru{и ввели} <expr> at the prompt \ru{в текстовом поле
ввода} (and, of course, evaluates to the value the (read) would have
\ru{и конечно же вычисляется в то же значение, что дает} \verb|(read)|).
