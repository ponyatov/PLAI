\secrel{14.1.3 Achieving Statelessness  . . 106}

We haven’t actually achieved statelessness yet, because we have this large table
residing on the server, with no clear means to remove entries from it. It would
be better if we could avoid the server state entirely. This means we have to
move the relevant state to the client.

There are actually two ways in which the server holds state. One is that we have
reserved the right to create as many entries in the hash table as we wish,
rather than a constant number (i.e., linear in the size of the program itself).
The other is what we’re storing in the table: honest-to-goodness closures, which
might be holding on to an arbitrary amount of state. We’ll see this more clearly
soon.

Let’s start by eliminating the closure. Instead, let’s have each of the funtion
arguments to be named, top-level functions (which immediately forces us to have
only a fixed number of them, because the program’s size cannot be unbounded):
\lsts{src/14/1/3/1.rkt}{rkt}
Observe how each code block refers only to the name of the next, rather than to
a real closure. The value of the argument comes from the form. There’s just one
problem:
v1 in prog2 is a free identifier!

The way to fix this problem is, instead of creating a closure after one step, to
send v1 to the client to be stored there. Where do we store this? The browser
offers two mechanisms for doing this: cookies and hidden fields. Which one do we
use?
