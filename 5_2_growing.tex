\secrel{Growing the Interpreter\\\ru{Рост интерпретатора}}

Now we’re ready to tackle the interpreter proper. First, let’s remind ourselves
of what it needs to consume. Previously, it consumed only an expression to
evaluate. Now it also needs to take a list of function definitions:
\lsts{src/5/5_2_1.rkt}{rkt}
Let’s revisit our old interpreter \ref{firstinterp}. In the case of numbers,
clearly we still return the number as the answer. In the addition and
multiplication case, we still need to recur (because the sub-expressions might
be complex), but which set of function definitions do we use? Because the act of
evaluating an expression neither adds nor removes function \emph{definitions},
the set of definitions remains the same, and should just be passed along
unchanged in the recursive calls.
\lsts{src/5/5_2_2.rkt}{rkt}
Now let’s tackle application. First we have to look up the function definition,
for which we’ll assume we have a helper function of this type available:
\lsts{src/5/5_2_3.rkt}{rkt}
Assuming we find a function of the given name, we need to evaluate its body.
However, remember what we said about identifiers and parameters\,? We must
“search-and-replace”, a process you have seen before in school algebra called
\termdef{substitution}{substitution}. This is sufficiently important that we
should talk first about substitution before returning to the interpreter
\ref{interpresumed}.
