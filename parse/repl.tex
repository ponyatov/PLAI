\secrel{REPL: \ru{цикл} [R]ead/[E]val/[P]rint/[L]oop}

\begin{description}[nosep]
\item[R]ead \ru{\emph{читать} (и синтаксически разбирать) входной поток}
\item[E]val \ru{\emph{вычислять} полученное выражение}
\item[P]rint \ru{\emph{выводить} результат}
\item[L]oop \ru{\emph{повторить}}
\end{description}\bigskip

Обычно для интерактивных\note{и программ работающих в \term{пакетном режиме}: из
скриптов и ``батников''} лексических программ первым (стартовым\note{может
быть переназначено опцией \textbf{\%start}}) правилом идет реализация
\termdef{REPL}{REPL}-цикла в виде правила \emph{с левой рекурсией}:
\lstx{arith/ypp.ypp}{parse/repl0.ypp}{C++}

\begin{tabular}{l l}
\verb|REPL| & имя правила или \term{нетерминала} (*) \\
\verb|A   B| & B идет после A \\
\verb$A | B$ & A или B \\
\verb|SYM| & \term{терминал} который \emph{возвращает лексер} через вызов 
\fn{yylex()}\\
 \verb|{}| & \cpp\ код срабатывающий для части правила между
\verb$|$ или \verb|;|\\
\end{tabular}

\noindent(*)\note{правило, возвращающее какое-либо значение}

\lstt{arith/log.log}{parse/repl0.log}
