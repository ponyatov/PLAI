\secrel{\ru{Связка} \prog{flex}/\prog{bison}}

\term{Терминальные} элементы парсер получает вызовом лексера \fn{yylex()},
поэтому нам нужно реализовать \emph{связку синтаксического парсера и лексера}.

Прежде всего парсер должен уметь хранить распознанные грамматикой элементы, и
передавать их между правилами-\term{нетерминалами}. Для этого служит директива
\verb|%union|. \prog{bison} создает union-стурутуру \var{yylval}, которая
может хранить поля указанных типов.

Лексер должен иметь информацию о том, какие типы \term{терминалов} (токенов)
умеет обрабатывать парсер, и какие типы данных можно передать парсеру. Для
этого используется директива \verb|%defines|, которая заставляет \prog{bison}
создать файл \file{ypp.tab.hpp} с дефайнами для интерфейса
\prog{flex}/\prog{bison}.

\lstx{arith/ypp.ypp}{parse/union.ypp}{C++}

\prog{bison} создаст фай	л \file{ypp.tab.hpp}, который нужно подключить в
\file{hpp.hpp}:

\lstx{arith/ypp.ypp}{parse/lexyacc.ypp}{C++}
\lstx{arith/lpp.lpp}{parse/lexyacc.lpp}{C++}
\lstx{arith/hpp.hpp}{parse/parsinterface.hpp}{C++}
\lstx{arith/hpp.hpp}{parse/lexyacc.hpp}{C++}
\lstx{arith/hpp.cpp}{parse/lexyacc.cpp}{C++}
