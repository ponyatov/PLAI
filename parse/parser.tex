\secrel{$\oplus$ ypp.ypp: \ru{синтаксический парсер}}\secdown

Пока нам удалось выделить только плоский список базовых элементов. Для выделения
\emph{структуры} выражений регулярные выражения не подходят \cite{dragon}. Для
разбора арифместических выражений нам нужно добавить компонент, способный
распознавать \term{рекурсивные структурные шаблоны}: бинарные и префиксные
операторы, приоритет операций, допустимые сочетания элементов. Это можно сделать
с помощью генератора парсеров \prog{bison}.

Для добавления парсера нам нужно сделать широкий шаг: парсер завязывает между
собой лексер, грамматику языка перемешанную в правилах парсера с \cpp\ кодом,
конструкторы объектов вызываемых при генерации дерева разбора, вызов
интерпретатора и иногда различный вспомогательный код. Поэтому этот этап
потребует некоторых мыслительный усилий. Особенно проблемно самостоятельно
реализовать грамматику какого-нибудь языка\ --- требуется наработать навыки
написания правил для типовых синтаксических конструкций. В этом вам может помочь
GitHub: ищите проекты, использующие \prog{yacc}/\prog{bison}, и разбирайте код
синтаксических парсеров.

\clearpage
Формат файла парсера немного похож на файл лексера\note{уже из описания формата
видно, насколько непонятна может быть эта тема}:
\verbatiminput{parse/struc.ypp}

Попробуем пойти тем же пошаговым путем, как мы это делали с лексером. Создаем
пустой файл парсера, и получаем синтаксическую ошибку: \prog{bison} требует
чтобы в \file{ypp.ypp} было \emph{как минимум одно} \term{стартовое} правило.

\lstx{arith/ypp.ypp}{parse/0.ypp}{C++}
\lstt{arith/log.log}{parse/0y.log}

\secrel{REPL: \ru{цикл} [R]ead/[E]val/[P]rint/[L]oop}

\begin{description}[nosep]
\item[R]ead \ru{\emph{читать} (и синтаксически разбирать) входной поток}
\item[E]val \ru{\emph{вычислять} полученное выражение}
\item[P]rint \ru{\emph{выводить} результат}
\item[L]oop \ru{\emph{повторить}}
\end{description}\bigskip

Обычно для интерактивных\note{и программ работающих в \term{пакетном режиме}: из
скриптов и ``батников''} лексических программ первым (стартовым\note{может
быть переназначено опцией \textbf{\%start}}) правилом идет реализация
\termdef{REPL}{REPL}-цикла в виде правила \emph{с левой рекурсией}:
\lstx{arith/ypp.ypp}{parse/repl0.ypp}{C++}

\begin{tabular}{l l}
\verb|REPL| & имя правила или \term{нетерминала} (*) \\
\verb|A   B| & B идет после A \\
\verb$A | B$ & A или B \\
\verb|SYM| & \term{терминал} который \emph{возвращает лексер} через вызов 
\fn{yylex()}\\
 \verb|{}| & \cpp\ код срабатывающий для части правила между
\verb$|$ или \verb|;|\\
\end{tabular}

\noindent(*)\note{правило, возвращающее какое-либо значение}

\lstt{arith/log.log}{parse/repl0.log}

\secrel{\ru{Связка} \prog{flex}/\prog{bison}}


\secup