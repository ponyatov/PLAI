\secrel{$\oplus$ ypp.ypp: \ru{синтаксический парсер}}

Пока нам удалось выделить только плоский список базовых элементов. Для выделения
\emph{структуры} выражений регулярные выражения не подходят \cite{dragon}. Для
разбора арифместических выражений нам нужно добавить компонент, способный
распознавать \term{рекурсивные структурные шаблоны}: бинарные и префиксные
операторы, приоритет операций, допустимые сочетания элементов. Это можно сделать
с помощью генератора парсеров \prog{bison}.

Для добавления парсера нам нужно сделать широкий шаг: парсер завязывает между
собой лексер, грамматику языка перемешанную в правилах парсера с \cpp\ кодом,
конструкторы объектов вызываемых при генерации дерева разбора, вызов
интерпретатора и иногда различный вспомогательный код. Поэтому этот этап
потребует некоторых мыслительный усилий. Особенно проблемно самостоятельно
реализовать грамматику какого-нибудь языка\ --- требуется наработать навыки
написания правил для типовых синтаксических конструкций. В этом вам может помочь
GitHub: ищите проекты, использующие \prog{yacc}/\prog{bison}, и разбирайте код
синтаксических парсеров.

\clearpage
Формат файла парсера немного похож на файл лексера\note{уже из описания формата
видно, насколько непонятна может быть эта тема}:
\verbatiminput{parse/struc.ypp}

Попробуем пойти тем же пошаговым путем, как мы это делали с лексером. Создаем
пустой файл парсера, и получаем синтаксическую ошибку: \prog{bison} требует
чтобы в \file{ypp.ypp} было \emph{как минимум одно} \term{стартовое} правило.

\lstx{arith/ypp.ypp}{parse/0.ypp}{C++}
\lstt{arith/log.log}{parse/0y.log}