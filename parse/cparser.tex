\secrel{$\oplus$ \flex/\bison: classic approach\\
\ru{типовой рецепт реализации парсера}}\secdown

Если вас не устраивает использование тяжеловесной \racket-системы для реализации
своего языка, или по каким-то причином вы сильно ограничены в
ресурсах\note{например встраиваемая система или даже спец.железка на
микроконтроллере}, можно пойти более эффективным путем, и использовать комплект
из двух типовых утилит:
\begin{itemize}
  \item \flex: генератор \termdef{лексеров}{лексер}: программа создает
\cpp\ код в \file{lex.yy.c} на основе набора правил из \file{lpp.lpp}, состоящих
из двух частей, \emph{разделенных 1+ пробелом}:
\begin{enumerate}
  \item регулярное выражение (regexp)
  \item \cpp\ код выполняемый при обнаружении во \term{входном потоке} группы
  символов, удовлетворяющих этому regexpу.
\end{enumerate}

\item \bison: генератор синтаксических \termdef{парсеров}{парсер}: полученный
\file{ypp.tab.cpp} умеет распознавать вложенные синтаксичсекие структуры,
описанные в файле \file{ypp.ypp} как набор правил грамматики языка.
\end{itemize}

\secrel{$\oplus$ \file{arith/} \ru{парсер арифметических выражений}}

\begin{verbatim}
touch src.src log.log ypp.ypp lpp.lpp hpp.hpp cpp.cpp \
      Makefile .gitignore rc.rc
echo "#!/bin/sh" > rc.rc
echo gvim -p src.src log.log ... rc.rc >> rc.rc
chmod +x rc.rc
\end{verbatim}

Далее нам потребуется одновременно редактировать несколько файлов, поэтому в
качестве IDE воспользуемся легковесным редактором \gvim\note{сборка для \win\
доступна на \url{http://www.vim.org/download.php\#pc}}\ и будем запускать его из
скрипта \file{rc.rc} под \linux\ и \file{bat.bat} под \win.

\clearpage
\lstt{\file{arith/rc.rc}}{arith/rc.rc}
\lstt{\file{arith/bat.bat}}{arith/bat.bat}

Далее правки будут вноситься одновременно в несколько файлов, поэтому
внимательно смотрите на заголовки небольших секций кода далее.

\clearpage
\lstx{\file{arith/Makefile}}{tmp/mk.mk}{mk}
\clearpage

\secrel{\file{lpp.lpp} \ru{лексер}}

Для начала реализуем только лексер, который будет читать файл с простыми
выражениями, и выводить распознанные элементы:

\lstt{arith/src.src}{arith/src.src}

Для его компиляции и запуска используйте команды:

\begin{verbatim}
flex lpp.lpp
g++ -o ./exe.exe lex.yy.c
./exe.exe < src.src
\end{verbatim}

Под \linux\ их удобно записать в одну строку через \verb|&&|

\clearpage
При запуске на пустом файле \file{lpp.lpp} вы получите сообщение:
\begin{verbatim}
$> flex lpp.lpp && g++ -o ./exe.exe lex.yy.c && ./exe.exe < src.src 
lpp.lpp:1: premature EOF
\end{verbatim}

То есть для успешной трансляции у \term{файла лексера} \file{lpp.lpp} должна
быть определенная структура:

\verbatiminput{2/6/struc.lpp}

В простейшем случае мы можем попытаться скомпилировать самый элементарный
пустой лексер из одного обязального элемента: маркера начала блока правил
\lstt{arith/lpp.lp}{2/6/minimal0.lpp}

Команда \verb|flex lpp.lpp| выполнится без ошибок, и создаст 
\file{lex.yy.c}

Попытка его компиляции завершится ошибкой:
\begin{verbatim}
ponyatov@gac:~/PLAI/arith$ g++ -o ./exe.exe lex.yy.c 
/usr/lib/gcc/x86_64-linux-gnu/4.9/../../../x86_64-linux-gnu/crt1.o: In function `_start':
(.text+0x20): undefined reference to `main'
/tmp/cceTYbuX.o: In function `yylex()':
lex.yy.c:(.text+0x3d6): undefined reference to `yywrap'
collect2: error: ld returned 1 exit status
\end{verbatim}

Из отчета об ошибках вы видим проблемы со следующими функциями:

\begin{description}
\item[\fn{main()}]\ \\
отсутствует функция \fn{main()} которая должна вызывать лексер, и возможно
выполнить перед этим какую-то инициализацию, или завершающие операции. \flex\
предоставляет типовую реализацию \fn{main()} при использовании опции \var{main}.
\item[\fn{yywrap()}]\ \\
вызывается лексером при достижении конца файла; нам не нужно выполнять какие-то
особые действия, поэтому применим опцию \var{noyywrap}.
\item[\fn{yylex()}]\ \\
эта функция собственно и является лексером:
\begin{itemize}
\item 
каждый ее вызов выделяет из входного потока \emph{одну} \term{лексему},
\item
функция возвращает целое значение которое определяется командой \verb|return| в
каждом правиле,
\item
переменная \var{char* yytext} содержит ссылку на выделенную
строку\note{ASCIIZ-строка: массив 8-битные символов, оканчивающихся 0x00;
будьте осторожны с кириллицей, \flex\ не умеет из коробки работать с юникодом,
и поддерживает только младшую половину таблицы ASCII}
\end{itemize}
\end{description}
\secrel{\class{Sym}: universal [sym]bolic data type\\
\ru{универсальный [sym]вольный тип данных}}\secdown

Прежде чем мы продолжим с парсером, нам нужно уточнить, как мы будем хранить
результаты разбора. В PLAI в качестве хранилища \termdef{дерева разбора}{дерево
разбора} используется традиционное для \lisp-мира представление в виде списков,
где \emph{первым элементом идет вершина дерева}, а в следующих элементах
находятся поддеревья.

Но если мы внимательно рассмотрим синтаксис современных main\-stream языков
программирования\note{и текстовых форматов данных}, и учтем что нашей целью
является создание метаязыка для трансформации программ, более удобным и
правильным кажется использование не списков, а \termdef{атрибутных
деревьев}{атрибутное дерево}.

Вместо класса \class{ArithC} нам нужно дерево классов, наследованных от одного
\term{виртуального} базового класса \class{Sym}. Это требование вытекает из
жесткой типизации и отсутствия в \cpp\ поддержки \term{гетерогенных} структур
данных, способных хранить в себе \emph{разнотипные} элементы. Если мы используем
набор типов\note{которые будут хранить в себе элементы нашего дерева разбора},
наследованных от одного виртуального класса, мы сможем оперировать ими через
указатели на базовый класс \class{Sym*}, и воспользоваться динамическими
хранилищами из библиотеки \cpp\ STL: \class{deque<Sym*>} и
\class{map<string,Sym*>}.

\lstxl{hpp.hpp}{sym/head.hpp}{C++}
\lstxl{cpp.cpp}{sym/head.cpp}{C++}

\clearpage
Любой элемент данных должен:
\begin{description}
\item[хранить свое значение и идентифицировать свой тип]\ \\\emph{универсальным
представлением любых данных является строка}:
\lstxl{hpp.hpp}{sym/tagval.hpp}{C++}

В \cpp\ есть средства идентификации класса по указателю на экземпляр (RTTI,
\fn{typeid()}), и по крайней мере тип тэга должен быть \class{static
Sym*}\note{указывать на экземпляр \class{Clazz::Sym *symbol, *number,
*string,..} т.е. на элемент данных типа ``класс'', существующий внутри нашей
DLR; собственно на текущий момент нам от него нужно только название класса} или
хотя бы \class{static string}, но для максимального \emph{упрощения кода} мы
используем просто строку.

\item[конструктор]\ \\создает элемент данных по паре тэг:значение <T:V>
\lstxl{hpp.hpp}{sym/constv.hpp}{C++}
\item[лексемы] (=\term{токены} =\term{терминалы})\ \\должны иметь конструктор из
строки, выделенной лексером
\lstxl{hpp.hpp}{sym/constoc.hpp}{C++}
\lstxl{cpp.cpp}{sym/constv.cpp}{C++}
\clearpage
\item[базовый класс]\ \\должен иметь \emph{по крайней мере одну} виртуальную
функцию. Классически для этого рекомендуется использовать виртуальный
деструктор. Позже \ref{symgc} при реализации управления памятью мы так и
сделаем, а пока воспользуемся следущим требованием:
\item[выводить себя в текстовом представлении]\ \\для отладки или трансляции
\lstxl{hpp.hpp}{sym/dump.hpp}{C++}
\lstxl{cpp.cpp}{sym/dump.cpp}{C++}

\item[хранить в себе вложенные элементы]\ \\\emph{самое важное для наших целей
требование}. 

\item[1) с доступом по целочисленному индексу] (аналог массивов)\ \\
простейший случай, типичное решение для реализации деревьев с \emph{любым}
количеством ветвей, для реализации применим C++ шаблонный класс STL
\class{deque<Sym*>}. Каждый \class{Sym}-объект может выступать
в роли \term{массива с доступом на целому индексу}, \term{очереди} и
\term{стека}.

\lstxl{hpp.hpp}{sym/nest.hpp}{C++}
\lstxl{cpp.cpp}{sym/nest.cpp}{C++}

\item[2) с доступом по имени]\ \\необходимо для реализации таблиц
символов\note{подробно ср\'{е}ды и применение lookup таблиц рассмотрено в
\ref{env}}, адресации полей классов, хранения других атрибутов\note{например
поле \var{doc} для документирования аналогично \var{\_\_doc\_\_} в \py}, в общем
реализации \term{атрибутного} дерева.

\lstxl{hpp.hpp}{sym/lookup.hpp}{C++}

\item[выводить себя в текстовом виде в виде дерева]\ \\так как предполается что
мы работаем исключительно с атрибутными деревьями, необходимо наличие средства
просмотра не только содержимого, но и структуры любого элемента данных.

Мы должны иметь возможность визуально определять вложенность, а в идеале\ ---
иметь возможность копировать полученный \termdef{дамп}{дамп} в исходный код в
виде, который распознает парсер\note{сейчас это потребует значительного
усложнения синтаксиса и кода парсера, поэтому реализацию
\termdef{bootstrap}{bootstrap}-дампа пока отложим}.

\lstxl{hpp.hpp}{sym/dumptree.hpp}{C++}

\begin{tabular}{l l}
\fn{head()} & вывод заголовка \\
\fn{pad(int)} & отбивка табуляциями каждой строки дампа,\\
& определяется вложенностью \var{depth}\\
\fn{dump(depth++)} & \emph{рекурсивный} дамп дерева \\
\end{tabular}

\lstxl{cpp.cpp}{sym/dumptree.cpp}{C++}

Итак, у нас есть универсальный базовый тип для представления атрибутных
деревьев. Но для практического применения желательно добавить пару 
производных типов для представления чисел и операторов.
Через наследование \class{Sym} мы можем определять новые элементы данных,
отличающиеся особенным поведением\note{например \class{оператор}ы могут уметь
вычислять математические функции от своих вложенных аргументов}, или
\term{обертывать} классы\note{элементы GUI, интерфейсы СУБД, сетевые
соединения,\ldots} и произвольные \emph{библиотеки}, не связанные с ядром DLR.   

\end{description}

\secrel{\class{Num}: [num]bers \ru{числа}}

Базовый класс \class{Sym} хорошо подходит для хранения строк и идентификаторов,
числа также можно хранить в виде строк, но лучше определить отдельный
\term{класс-обертку} заменив тип \var{val} на \class{float}:

\lstxl{hpp.hpp}{sym/num.hpp}{C++}
\lstxl{cpp.cpp}{sym/num.cpp}{C++}
\begin{tabular}{l l}
\class{float} \var{val} & обернутый объект \\
\fn{Num(string)} & токен-конструктор (используется в лексере) \\
\fn{Num(float)} & конструктор обертки (будет использоваться при\\&упаковке
результатов вычислений в интерпретаторе \ref{cppinterp})\\
\fn{head()} & так как у нас изменился тип хранимого значения,\\&необходимо
переопределить функцию \var{val} $\rightarrow$ \class{string}\\ 
\end{tabular}

\lstxl{lpp.lpp}{parse/num.lpp}{C++}

\secrel{\ru{Классы-обертки}}

Можно обратить внимание, что определяя класс-обертку для каждого нового
типа\note{в нашем случае класс \class{Num} входит в ядро языка, но мы можем
рассмотреть на его примере расширение ядра новыми типами, например если вы
захотите добавить \term{комплексные числа} или битовые строки}, мы
переопределяем:
\begin{description}
\item[тип \var{val}]\ \\
в случае числа мы использовали примитивный тип \class{float}, но в общем случае
это может быть любой сложный объект: файл, соединение в БД, окно GUI,
спецификация формата файла или протокола,\ldots
\item[функцию \class{string} $\rightarrow$ \var{val}]
\item[обратную функцию \var{val}$\rightarrow$\class{string}]
\item[функцию создающую объект-обертку] из объекта обертываемого типа
\end{description}



\secup

\secup