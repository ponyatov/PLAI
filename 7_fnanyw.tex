\secrel{7 Functions Anywhere 31}\secdown

The introduction to the Scheme programming language definition establishes this
design principle:
\begin{framed}
Programming languages should be designed not by piling feature on top of
feature, but by removing the weaknesses and restrictions that make additional
features appear necessary. \ref{}
\end{framed}
As design principles go, this one is hard to argue with. (Some restrictions, of
course, have good reason to exist, but this principle forces us to argue for
them, not admit them by default.) Let’s now apply this to functions.

One of the things we stayed coy about when introducing functions (section 5) is
exactly where functions go. We may have suggested we’re following the model of
an idealized DrRacket, with definitions and their uses kept separate. But,
inspired by the Scheme design principle, let’s examine how necessary that is.

Why can’t functions definitions be expressions? In our current
arithmetic-centric language we face the uncomfortable question “What value does
a function definition represent?”, to which we don’t really have a good answer.
But a real programming language obviously computes more than numbers, so we no
longer need to confront the question in this form; indeed, the answer to the
above can just as well be, “A function value”. Let’s see how that might work
out.

What can we do with functions as values? Clearly, functions are a distinct kind
of value than a number, so we cannot, for instance, add them. But there is one
evident thing we can do: apply them to arguments! Thus, we can allow function
values to appear in the function position of an application. The behavior would,
naturally, be to apply the function. Thus, we’re proposing a language where the
following would be a valid program (where I’ve used brackets so we can easily
identify the function)
\lsts{src/7/7.rkt}{rkt}
and would evaluate to \verb|(+ 2 (* 4 3))|, or 14.\note{Did you see that I just
used substitution\,?}

\secrel{7.1 Functions as Expressions and Values  32}

Let’s first define the core language to include function definitions:
\lsts{src/7/7_1_1.rkt}{rkt}

For now, we’ll simply copy function definitions into the expression language.
We’re free to change this if necessary as we go along, but for now it at least
allows us to reuse our existing test cases.
\lsts{src/7/7_1_2.rkt}{rkt}

We also need to determine what an application looks like. What goes in the
function position of an application? We want to allow an entire function
definition, not just its name. Because we’ve lumped function definitions in with
all other expressions, let’s allow an arbitrary expression here, but with the
understanding that we want only function definition expressions:
\note{We might consider more refined datatypes that split function definitions
apart from other kinds of expressions. This amounts to trying to classify
different kinds of expressions, which we will return to when we study types.
\ref{}}
\lsts{src/7/7_1_3.rkt}{rkt}

With this definition of application, we no longer have to look up functions by
name, so the interpreter can get rid of the list of function definitions. If we
need it we can restore it later, but for now let’s just explore what happens
with function definitions are written at the point of application: so-called
immediate functions.

Now let’s tackle interp. We need to add a case to the interpreter for function
definitions, and this is a good candidate:
\lsts{src/7/7_1_4.rkt}{rkt}

\DoNow{
What happens when you add this?
}
Immediately, we see that we have a problem: the interpreter no longer always
returns numbers, so we have a type error.

We’ve alluded periodically to the answers computed by the interpreter, but never
bothered gracing these with their own type. It’s time to do so now.
\lsts{src/7/7_1_5.rkt}{rkt}

We’re using the suffix of V to stand for values, i.e., the result of evaluation.
The pieces of a funV will be precisely those of a fdC: the latter is input, the
former is output. By keeping them distinct we allow each one to evolve
independently as needed.

Now we must rewrite the interpreter. Let’s start with its type:
\lsts{src/7/7_1_6.rkt}{rkt}

This change naturally forces corresponding type changes to the Binding datatype
and to lookup.

\Exercise{
Modify Binding and lookup, appropriately.
}
\lsts{src/7/7_1_7.rkt}{rkt}

Clearly, numeric answers need to be wrapped in the appropriate numeric answer
constructor. Identifier lookup is unchanged. We have to slightly modify addition
and multiplication to deal with the fact that the interpreter returns Values,
not numbers:
\lsts{src/7/7_1_8.rkt}{rkt}

It’s worth examining the definition of one of these helper functions:
\lsts{src/7/7_1_9.rkt}{rkt}
Observe that it checks that both arguments are numbers before performing the
addition. This is an instance of a safe run-time system. We’ll discuss this
topic more when we get to types. \ref{}

There are two more cases to cover. One is function definitions. We’ve already
agreed these will be their own kind of value:
\lsts{src/7/7_1_10.rkt}{rkt}

That leaves one case, application. Though we no longer need to look up the
function definition, we’ll leave the code structured as similarly as possible:
\lsts{src/7/7_1_11.rkt}{rkt}

In place of the lookup, we reference f which is the function definition, sitting
right there. Note that, because any expression can be in the function definition
position, we really ought to harden the code to check that it is indeed a
function.

\DoNow{
What does is mean? That is, do we want to check that the function definition
position is syntactically a function definition (fdC), or only that it
evaluates to one (funV)? Is there a difference, i.e., can you write a program
that satisfies one condition but not the other?
}

We have two choices:
\begin{enumerate}[nosep]
  \item 
We can check that it syntactically is an fdC and, if it isn’t reject it as an
error.
  \item 
We can evaluate it, and check that the resulting value is a function (and signal
an error otherwise).
\end{enumerate}
We will take the latter approach, because this gives us a much more flexible
language. In particular, even if we can’t immediately imagine cases where we, as
humans, might need this, it might come in handy when a program needs to generate
code. And we’re writing precisely such a program, namely the desugarer! (See
section 7.5.) As a result, we’ll modify the application case to evaluate the
function position:
\lsts{src/7/7_1_12.rkt}{rkt}

\Exercise{
Modify the code to perform both versions of this check.
}

And with that, we’re done. We have a complete interpreter! Here, for instance,
are some of our old tests again:
\lsts{src/7/7_1_13.rkt}{rkt}

\secrel{7.2 Nested What?   35}

\secrel{7.3 Implementing Closures   . 37}

We need to change our representation of values to record closures rather than raw
function text:
\lsts{src/7/7_3_1.rkt}{rkt}
While we’re at it, we might as well alter our syntax for defining functions to drop
the useless name. This construct is historically called a lambda:
\lsts{src/7/7_3_2.rkt}{rkt}

When encountering a function definition, the interpreter must now remember to
save the substitutions that have been applied so far:
\note{“Save the environment! Create a closure today!”\ --- Cormac Flanagan}
\lsts{src/7/7_3_3.rkt}{rkt}

This saved set, not the empty environment, must be used when applying a
function:
\lsts{src/7/7_3_4.rkt}{rkt}

There’s actually another possibility: we could use the environment present at the
point of application:
\lsts{src/7/7_3_5.rkt}{rkt}

\Exercise{
What happens if we extend the dynamic environment instead?
}

In retrospect, it becomes even more clear why we interpreted the body of a
function in the empty environment. When a function is defined at the top-level,
it is not “closed over” any identifiers. Therefore, our previous function
applications have been special cases of this form of application.

\secrel{7.4 Substitution, Again   38}

\secrel{7.5 Sugaring Over Anonymity  . . 39}

Now let’s get back to the idea of naming functions, which has evident value for
program understanding. Observe that we do have a way of naming things: by
passing them to functions, where they acquire a local name (that of the formal
parameter). Anywhere within that function’s body, we can refer to that entity
using the formal parameter name.

Therefore, we can take a collection of function definitions and name them using
other...functions. For instance, the Racket code
\lsts{src/7/5/1.rkt}{rkt}
could first be rewritten as the equivalent
\lsts{src/7/5/2.rkt}{rkt}
We can of course just inline the definition of double, but to preserve the name,
we could write this as:
\lsts{src/7/5/3.rkt}{rkt}
Indeed, this pattern—which we will pronounce as “left-left-lambda”—is a local
naming mechanism. It is so useful that in Racket, it has its own special syntax:
\lsts{src/7/5/4.rkt}{rkt}
where let can be defined by desugaring as shown above.

Here’s a more complex example:
\lsts{src/7/5/5.rkt}{rkt}
This could be rewritten as
\lsts{src/7/5/6.rkt}{rkt}
which works just as we’d expect; but if we change the order, it no longer works—
\lsts{src/7/5/7.rkt}{rkt}
—because quadruple can’t “see” double. so we see that top-level binding is
different from local binding: essentially, the top-level has an “infinite
scope”. This is the source of both its power and problems.

There is another, subtler, problem: it has to do with recursion. Consider the
simplest infinite loop:
\lsts{src/7/5/8.rkt}{rkt}
Let’s convert it to let:
\lsts{src/7/5/9.rkt}{rkt}
Seems fine, right? Rewrite in terms of lambda:
\lsts{src/7/5/10.rkt}{rkt}
Clearly, the loop-forever on the last line isn’t bound!

This is another feature we get “for free” from the top-level. To eliminate this
magical force, we need to understand recursion explicitly, which we will do soon
\ref{}.
\secup
