\secrel{Functions as Expressions and Values\\
\ru{Функции как выражения и значения}}

Let’s first define the core language \ru{Для начала давайте определим ядреный
язык} to include function definitions \ru{включающий определения функций}:
\lsts{7/1/1.rkt}{rkt}

For now \ru{Теперь}, we’ll simply copy function definitions \ru{мы просто
скопируем определения функций} into the expression language \ru{в язык
выражений}. We’re free to change this \ru{Мы свободно можем изменить это} if
necessary \ru{при необходимости} as we go along \ru{далее}, but for now \ru{но
сейчас} it at least \ru{это как минимум} allows us \ru{позволяет нам} to reuse
\ru{повторно использовать} our existing test cases \ru{наши существующие наборы
тестов}.
\lsts{7/1/2.rkt}{rkt}

We also need to determine \ru{Нам также нужно определить} what an application
looks like \ru{как должно выглядить применение}. What goes in the function
position of an application \ru{Где находится позиция применяемой функции}\,? We
want to allow an entire function definition \ru{Мы хотим иметь возможность
использовать здесь определение функции}, not just its name \ru{а не только ее
имя}. Because we’ve lumped function definitions \ru{Так как мы скомпоновали
определения функций} in with all other expressions \ru{со всеми другими видами
выражений}, let’s allow an arbitrary expression here \ru{давайте позволим здесь
произвольное выражение}, but with the understanding \ru{но с пониманием} that we
want only function definition expressions \ru{что нас интересуют только
выражения определения функций}:
\note{We might consider more refined datatypes that split function definitions
apart from other kinds of expressions. This amounts to trying to classify
different kinds of expressions, which we will return to when we study types.
\ref{}}
\note{\ru{Мы могли бы рассмотреть более изощренные типы данных, которые
выделяют определения функций среди других видов выражений. Это равносильно
попытке классификации различных видов выражений, мы вернемся к этому когда будем
изучать типизацию \ref{}}}
\lsts{7/1/3.rkt}{rkt}

With this definition of application \ru{С таким определением применения}, we no
longer have to look up functions by name \ru{нам больше не нужно искать функции
по имени}, so the interpreter can get rid of the list of function definitions
\ru{поэтому интерпретатор может избавиться от списка определений функций}.
If we need it \ru{Если нам нужно} we can restore it later \ru{мы позже сможем
его восстановить}, but for now \ru{но пока} let’s just explore \ru{просто
давайте посмотрим} what happens \ru{что получиться} with function definitions
are written at the point of application \ru{когда определения функций написаны с
точки зрения применения}: so-called \ru{так называемые} \termdef{immediate
functions}{immediate function} \ru{\termdef{непосредственные
функции}{непосредственная функция}}.

Now let’s tackle \ru{Терерь давайте возьмемся за} \verb|interp|. We need to add
a case to the interpreter for function definitions \ru{Нам нужно добавить к
интерпретатору обработку случая определения функции}, and this is a good
candidate \ru{и вот это хороший кандидат}:
\lsts{7/1/4.rkt}{rkt}

\DoNow{
What happens when you add this?
}
Immediately, we see that we have a problem: the interpreter no longer always
returns numbers, so we have a type error.

We’ve alluded periodically to the answers computed by the interpreter, but never
bothered gracing these with their own type. It’s time to do so now.
\lsts{7/1/5.rkt}{rkt}

We’re using the suffix of V to stand for values, i.e., the result of evaluation.
The pieces of a funV will be precisely those of a fdC: the latter is input, the
former is output. By keeping them distinct we allow each one to evolve
independently as needed.

Now we must rewrite the interpreter. Let’s start with its type:
\lsts{7/1/6.rkt}{rkt}

This change naturally forces corresponding type changes to the Binding datatype
and to lookup.

\Exercise{
Modify Binding and lookup, appropriately.
}
\lsts{7/1/7.rkt}{rkt}

Clearly, numeric answers need to be wrapped in the appropriate numeric answer
constructor. Identifier lookup is unchanged. We have to slightly modify addition
and multiplication to deal with the fact that the interpreter returns Values,
not numbers:
\lsts{7/1/8.rkt}{rkt}

It’s worth examining the definition of one of these helper functions:
\lsts{7/1/9.rkt}{rkt}
Observe that it checks that both arguments are numbers before performing the
addition. This is an instance of a safe run-time system. We’ll discuss this
topic more when we get to types. \ref{}

There are two more cases to cover. One is function definitions. We’ve already
agreed these will be their own kind of value:
\lsts{7/1/10.rkt}{rkt}

That leaves one case, application. Though we no longer need to look up the
function definition, we’ll leave the code structured as similarly as possible:
\lsts{7/1/11.rkt}{rkt}

In place of the lookup, we reference f which is the function definition, sitting
right there. Note that, because any expression can be in the function definition
position, we really ought to harden the code to check that it is indeed a
function.

\DoNow{
What does is mean? That is, do we want to check that the function definition
position is syntactically a function definition (fdC), or only that it
evaluates to one (funV)? Is there a difference, i.e., can you write a program
that satisfies one condition but not the other?
}

We have two choices:
\begin{enumerate}[nosep]
  \item 
We can check that it syntactically is an fdC and, if it isn’t reject it as an
error.
  \item 
We can evaluate it, and check that the resulting value is a function (and signal
an error otherwise).
\end{enumerate}
We will take the latter approach, because this gives us a much more flexible
language. In particular, even if we can’t immediately imagine cases where we, as
humans, might need this, it might come in handy when a program needs to generate
code. And we’re writing precisely such a program, namely the desugarer! (See
section 7.5.) As a result, we’ll modify the application case to evaluate the
function position:
\lsts{7/1/12.rkt}{rkt}

\Exercise{
Modify the code to perform both versions of this check.
}

And with that, we’re done. We have a complete interpreter! Here, for instance,
are some of our old tests again:
\lsts{7/1/13.rkt}{rkt}
