\secrel{7.4 Substitution, Again   38}

We have seen that substitution is instructive in thinking through how to
implement lambda functions. However, we have to be careful with substitution
itself! Suppose we have the following expression (to give lambda functions their
proper Racket syntax):
\lsts{7/4/1.rkt}{rkt}
Now suppose we substitute for f the following expression: (lambda (y) (+ x y)).
Observe that it has a free identifier (x), so if it is ever evaluated, we would
expect to get an unbound identifier error. Substitution would appear to give:
\lsts{7/4/2.rkt}{rkt}
But observe that this latter program has no free identifiers!

That’s because we have too naive a version of substitution. To prevent
unexpected behavior like this (which is a form of dynamic binding), we need to
define capturefree substitution. It works roughly as follows: we first
consistently rename all bound identifiers to entirely previously unused (known
as fresh) names. Imagine that we give each identifier a numeric suffix to attain
freshness. Then the original expression becomes
\lsts{7/4/3.rkt}{rkt}
(Observe that we renamed f to f1 in both the binding and bound locations.) Now
let’s do the same with the expression we’re substituting:
\lsts{7/4/4.rkt}{rkt}
Now let’s substitute for f1:
\note{Why didn’t we rename x? Because x may be a reference to a top-level
binding, which should then also be renamed. This is simply another application
of the consistent renaming principle. In the current setting, the distinction is
irrelevant.}
\lsts{7/4/5.rkt}{rkt}
...and x is still free! This is a good form of substitution.

But one moment. What happens if we try the same example in our environmentbased
interpreter?
\DoNow{
Try it out.
}

Observe that it works correctly: it reports an unbound identifier error.
Environments automatically implement capture-free substitution!
\Exercise{
In what way does using an environment avoid the capture problem of substitution?
}
