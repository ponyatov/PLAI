\secrel{7.2 Nested What?   35}

The body of a function definition is an arbitrary expression. A function
definition is itself an expression. That means a function definition can contain
a\ldots function definition. For instance:
\lsts{7/2/1.rkt}{rkt}

Evaluating this isn’t very interesting:
\lsts{7/2/2.rkt}{rkt}
But suppose we apply the above function to something:
\lsts{7/2/3.rkt}{rkt}
Now the answer becomes more interesting:
\lsts{7/2/4.rkt}{rkt}
It’s almost as if applying the outer function had no impact on the inner
function at all. Well, why should it? The outer function introduces an
identifier which is promptly masked by the inner function introducing one of the
same name, thereby masking the outer definition if we obey static scope (as we
should!). But that suggests a different program:
\lsts{7/2/5.rkt}{rkt}
This evaluates to:
\lsts{7/2/6.rkt}{rkt}
Hmm, that’s interesting.
\DoNow{
What’s interesting?
}

To see what’s interesting, let’s apply this once more:
\lsts{7/2/7.rkt}{rkt}
This produces an error indicating that the identifier representing x isn’t
bound!

But it’s bound by the function named f1, isn’t it? For clarity, let’s switch to
representing it in our hypothetical Racket syntax:
\lsts{7/2/8.rkt}{rkt}
On applying the outer function, we would expect x to be substituted with 5,
resulting in
\lsts{7/2/9.rkt}{rkt}
which on further application and substitution yields (+ 5 4) or 9, not an error.

In other words, we’re again failing to faithfully capture what substitution
would have done. A function value needs to remember the substitutions that have
already been applied to it. Because we’re representing substitutions using an
environment, a function value therefore needs to be bundled with an environment.
This resulting data structure is called a closure.
\note{On the other hand, observe that with substitution, as we’ve defined it, we
would be replacing x with (numV 4), resulting in a function body of (plusC (numV
5) (idC 'y)), which does not type. That is, substitution is predicated on the
assumption that the type of answers is a form of syntax. It is actually possible
to carry through a study of even very advanced programming constructs under this
assumption, but we won’t take that path here.}

While we’re at it, observe that the appC case above uses funV-arg and funV-
body, but not funV-name. Come to think of it, why did a function need a name? so
that we could find it. But if we’re using the interpreter to find the function
for us, then there’s nothing to find and fetch. Thus the name is merely
descriptive, and might as well be a comment. In other words, a function no more
needs a name than any other immediate constant: we don’t name every use of 3,
for instance, so why should we name every use of a function? A function is
inherently anonymous, and we should separate its definition from its naming.

(But, you might say, this argument only makes sense if functions are always
written in-place. What if we want to put them somewhere else? Won’t they need
names then? They will, and we’ll return to this (section 7.5).)
