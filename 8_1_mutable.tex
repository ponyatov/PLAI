\secrel{8.1 Mutable Structures   41}
\secdown
\secrel{8.1.1 A Simple Model of Mutable Structures  41}

Objects are a generalization of structures, as we will soon see \ref{}.
Therefore, fields in objects are a generalization of fields in structures and to
understand mutation, it is mostly (but not entirely! \ref{}) sufficient to
understand mutable objects. To be even more reductionist, we don’t need a
structure to have many fields: a single one will suffice. We call this a box. In
Racket, boxes support just three operations:
\lsts{src/8/1/1.rkt}{rkt}
Thus, box takes a value and wraps it in a mutable container. unbox extracts the
current value inside the container. Finally, set-box! changes the value in the
container, and in a typed language, the new value is expected to be
type-consistent with what was there before. You can thus think of a box as
equivalent to a Java container class with parameterized type, which has a single
member field with a getter and setter: box is the constructor, unbox is the
getter, and set-box! is the setter. (Because there is only one field, its name
is irrelevant.)
\lsts{src/8/1/2.rkt}{rkt}

Because we must sometimes mutate in groups (e.g., removing money from one bank
account and depositing it in another), it is useful to be able to sequence a
group of mutable operations. In Racket, begin lets you write a sequence of
operations; it evaluates them in order and returns the value of the last one.

\Exercise{
Define begin by desugaring into let (and hence into lambda).
}

Even though it is possible to eliminate begin as syntactic sugar, it will prove
extremely useful for understanding how mutation works. Therefore, we will add a
simple, two-term version of sequencing to the core.

\secrel{8.1.2 Scaffolding   42}
\secrel{8.1.3 Interaction with Closures  . . 43}
\secrel{8.1.4 Understanding the Interpretation of Boxes . . 44}
\secrel{8.1.5 Can the Environment Help?  46}
\secrel{8.1.6 Introducing the Store  . 48}
\secrel{8.1.7 Interpreting Boxes  . . 49}
\secrel{8.1.8 The Bigger Picture  . . 54}
\secup
