\secrel{13 Desugaring as a Language Feature 91}\secdown

We have thus far extensively discussed and relied upon desugaring, but our
current desguaring mechanism have been weak. We have actually used desugaring in
two different ways. One, we have used it to shrink the language: to take a large
language and distill it down to its core \ref{}. But we have also used it to
grow the language: to take an existing language and add new features to it
\ref{}. This just shows that desugaring is a tremendously useful feature to
have. Indeed, it is so useful that we might ask two questions:
\begin{itemize}
  \item 
Because we create languages to simplify the creation of common tasks, what would
a language designed to support desugaring look like? Note that by “look” we
don’t mean only syntax but also its key behavioral properties.
  \item 
Given that general-purpose languages are often used as a target for desugaring,
why don’t they offer desugaring capabilities in the language itself ? For
instance, this might mean extending a base language with the additional language
that is the response to the previous question.
\end{itemize}
We are going to explore the answer to both questions simultaneously, by studying
the facilities provided for this by \racket.

\secrel{13.1 A First Example   . . 91}

\secrel{13.2 Syntax Transformers as Functions  . 93}

\secrel{13.3 Guards   . 95}

Now we can return to the problem that originally motivated the introduction of
syntax- case: ensuring that the binding position of a my-let-3 is syntactically
an identifier. For this, you need to know one new feature of syntax-case: each
rewriting rule can have two parts (as above), or three. If there are three
present, the middle one is treated as a guard: a predicate that must evaluate to
true for expansion to proceed rather than signal a syntax error. Especially
useful in this context is the predicate identifier?, which determines whether a
syntax object is syntactically an identifier (or variable).

\DoNow{
Write the guard and rewrite the rule to incorporate it.
}

Hopefully you stumbled on a subtlety: the argument to identifier? is of type
syntax. It needs to refer to the actual fragment of syntax bound to var. Recall
that var is bound in the syntax space, and \#' substitutes identifiers bound
there. Therefore, the correct way to write the guard is:
\lsts{src/13/3/1.rkt}{rkt}
With this information, we can now write the entire rule:
\lsts{src/13/3/2.rkt}{rkt}

\DoNow{
Now that you have a guarded rule definition, try to use the macro with a
non-identifier in the binding position and see what happens.
}

\secrel{13.4 Or: A Simple Macro with Many Features  95}\secdown

Consider or, which implements disjunction. It is natural, with prefix syntax, to
allow or to have an arbitrary number of sub-terms. We expand or into nested
conditionals that determine the truth of the expression.

\secrel{13.4.1 A First Attempt   . 95}

\secrel{13.4.2 Guarding Evaluation  . 97}

\secrel{13.4.3 Hygiene   . . 98}

\secup

\secrel{13.5 Identifier Capture   . 99}

Hygienic macros address a routine and important pain that creators of syntactic
sugar confront. On rare instances, however, a developer wants to intentionally
break hygiene. Returning to objects, consider this input program:
\lsts{src/13/5/1.rkt}{rkt}
What does the macro look like? Here’s an obvious candidate:
\lsts{src/13/5/2.rkt}{rkt}
Unfortunately, this macro produces the following error:
\lst{src/13/5/2.err}
which is referring to the self in the body of the method bound to first.

\Exercise{
Work through the hygienic expansion process to understand why error is the
expected outcome.
}

Before we jump to the richer macro, let’s consider a variant of the input term
that makes the binding explicit:
\lsts{src/13/5/3.rkt}{rkt}
The corresponding macro is a small variation on what we had before:
\lsts{src/13/5/4.rkt}{rkt}
This macro expands without difficulty.

\Exercise{
Work through the expansion of this version and see what’s different.
}

This offers a critical insight: had the identifier that goes in the binding
position been written by the macro user, there would have been no problem.
Therefore, we want to be able to pretend that the introduced identifier was
written by the user. The function datum->syntax converts the s-expression in its
second argument; its first argument is which syntax to pretend it was a part of
(in our case, the original macro use, which is bound to x). To introduce the
result into the environment used for expansion, we use with-syntax to bind it in
that environment:
\lsts{src/13/5/5.rkt}{rkt}

With this, we can go back to having self be implicit:
\lsts{src/13/5/6.rkt}{rkt}

\secrel{13.6 Influence on Compiler Design  101}

\secrel{13.7 Desugaring in Other Languages  . . 101}

\secup
