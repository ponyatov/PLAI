\secrel{14.2.1 Implementation by Desugaring  . . 110}

Because we already have good support for desguaring, let’s use to define the CPS
transform. Concretely, we’ll implement a CPS macro [REF]. To more cleanly
separate the source language from the target, we’ll use slightly different names
for most of the language constructs: a one-armed with and rec instead of let and
letrec; lam instead of lambda; cnd instead of if; seq for begin; and set for
set!. We’ll also give ourselves a sufficiently rich language to write some
interesting programs!
\note{The presentation that follows orders the cases of the macro from what I
believe are easiest to hardest. However, the code in the macro must avoid
non-overlapping patterns, and hence follows a diffent order.}
\lsts{src/14/2/1/1.rkt}{rkt}

Our representation in CPS will be to turn every expression into a procedure of
one argument, the continuation. The converted expression will eventually either
supply a value to the continuation or will pass the continuation on to some
other expression that will—by preserving this invariant inductively—supply it
with a value. Thus, all output from CPS will look like (lambda (k) ...) (and we
will rely on hygiene [REF] to keep all these introduced k’s from clashing with
one another).

First let’s dispatch with the easy case, which is atomic values. Though
conceptually easiest, we have written this last because otherwise this pattern
would shadow all the other cases. (Ideally, we should have written it first and
provided a guard expression that precisely defines the syntactic cases we want
to treat as atomic. We’re playing loose here becuase our focus is on more
interesting cases.) In the atomic case, we already have a value, so we simply
need to supply it to the continuation:
\lsts{src/14/2/1/2.rkt}{rkt}
Similarly for quoted constants:
\lsts{src/14/2/1/3.rkt}{rkt}

Also, we already know, from [REF] and [REF], that we can treat with and rec as
macros, respectively:
\lsts{src/14/2/1/4.rkt}{rkt}
\lsts{src/14/2/1/5.rkt}{rkt}

Mutation is easy: we have to evaluate the new value, and then perform the actual
update:
\lsts{src/14/2/1/6.rkt}{rkt}

Sequencing is also straightforward: we perform each operation in turn. Observe
how this preserves the semantics of sequencing: not only does it obey the order
of operations, the value of the first sub-term (e1) is not mentioned anywhere in
the body of the second (e2), so the name given to the identifier holding its
value is irrelevant.
\lsts{src/14/2/1/7.rkt}{rkt}

When handling conditionals, we need to create a new continuation to remember
that we are waiting for the test expression to evaluate. Once we have its value,
however, we can dispatch on the result and return to the existing continuations:
\lsts{src/14/2/1/8.rkt}{rkt}

When we get to applications, we have two cases to consider. We absolutely need
to handle the treatment of procedures created in the language: those with one
argument. For the purposes of writing example programs, however, it is useful to
be able to employ primitives such as + and *. Thus, we will assume for
simplicity that oneargument procedures are written by the user, and hence need
conversion to CPS, while two-argument ones are primitives that will not perform
any Web or other control operations and hence can be invoked directly; we will
also assume that the primitive will be written in-line (i.e., the application
position will not be a complex expression that can itself, say, perform a Web
interaction).

For an application we have to evaluate both the function and argument
expressions. Once we’ve obtained these, we are ready to apply the function.
Therefore, it is tempting to write
\lsts{src/14/2/1/9.rkt}{rkt}

\DoNow{
Do you see why this is wrong?
}

The problem is that, though the function is now a value, that value is a closure
with a potentially complicated body: evaluating the body can, for example,
result in further Web interactions, at which point the rest of the function’s
body, as well as the pending (k ...) (i.e., the rest of the program), will all
be lost. To avoid this, we have to supply k to the function’s value, and let the
inductive invariant ensure that k will eventually be invoked with the value of
applying fv to av:
\lsts{src/14/2/1/10.rkt}{rkt}

Treating the special case of built-in binary operations is easier:
\lsts{src/14/2/1/11.rkt}{rkt}

The very pattern we could not use for user-defined procedures we employ here,
because we assume that the application of f will always return without any
unusual transfers of control.

A function is itself a value, so it should be returned to the pending
computation. The application case above, however, shows that we have to
transform functions to take an extra argument, namely the continuation at the
point of invocation. This leaves us with a quandary: which continuation do we
supply to the body?
\lsts{src/14/2/1/12.rkt}{rkt}

That is, in place of ..., which continuation do we supply: k or dyn-k?

\DoNow{
Which continuation should we supply?
}

The former is the continuation at the point of closure creation. The latter is
the continuation at the point of closure invocation. In other words, the former
is “static” and the latter is “dynamic”. In this case, we need to use the
dynamic continuation, otherwise something very strange would happen: the program
would return to the point where the closure was created, rather than where it is
being used! This would result in seemingly very strange program behavior, so we
wish to avoid it. Observe that we are consciously choosing the dynamic
continuation just as, where scope was concerned, we chose the static
environment.
\lsts{src/14/2/1/13.rkt}{rkt}

Finally, for the purpose of modeling Web programming, we can add our input and
output procedures. Output follows the application pattern we’ve already seen:
\lsts{src/14/2/1/14.rkt}{rkt}

Finally, for input, we can use the pre-existing read-number/suspend, but this
time generate its uses rather than force the programmer to construct them:
\lsts{src/14/2/1/15.rkt}{rkt}

Notice that the continuation bound to k is precisely the continuation that we
need to stash at the point of a Web interaction.

Testing any code converted to CPS is slightly annoying because all CPS terms
expect a continuation. The initial continuation is one that simply either (a)
consumes a value and returns it, or (b) consumes a value and prints it, or (c)
consumes a value, prints it, and gets ready for another computation (as the
prompt in the DrRacket Interactions window does). All three of these are
effectively just the identity function in various guises. Thus, the following
definition is helpful for testing:
\lsts{src/14/2/1/16.rkt}{rkt}
For instance,
\lsts{src/14/2/1/17.rkt}{rkt}
We can also test our old Web program:
\lsts{src/14/2/1/18.rkt}{rkt}

Lest you get lost in the myriad of code, let me highlight the important lesson
here: We’ve recovered our code structure. That is, we can write the program in
direct style, with properly nested expressions, and a compiler—in this case, the
CPS converter— takes care of making it work with a suitable underlying API. This
is what good programming languages ought to do!
