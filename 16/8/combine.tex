\secrel{16.8 Combining Contracts   . . 190}

Now that we’ve discussed combinators for all the basic datatypes, it’s natural
to discuss combining contracts. Just as we saw unions \ref{}\ and intersections
\ref{}\ for types, we should be considering unions and intersections
(respectively, “or”s and “and”s), ; for that matter, we might also consider
negation. However, contracts are only superficially like types, so we have to
consider these questions in their own light for contracts rather than try to map
the meanings we have learned from types to the sphere of contracts.

As always, the immediate case is straightforward. Union contracts combine with
disjunction—indeed, being predicates, their results can literally be combined
with or— and intersection contracts with conjunction. We apply the predicates in
turn, with shortcircuiting, and either generate an error or return the
contracted value. Intersection contracts combine with conjunction (and). And
negation contracts are simply the original immediate contract applied and the
decision negated (with not).

Contract combination is much harder in the deferred, higher-order case. For
instance, consider the negation of a function contract from numbers to numbers.
What exactly does it mean to negate it? Does it mean the function should not
accept numbers? Or that if it does, it should not produce them? Or both? And in
particular, how do we enforce such a contract? How, for instance, do we check
that a function does not accept numbers—are we expecting that when given a
number, it produces an error? But now consider the identity function wrapped
with such a contract; since it clearly does not result in an error when given a
number (or indeed any other value), does that mean we should wait until it
produces a value, and if it does produce a number, reject it? But worst of all,
note that this means we will be running functions on domains on which they are
not defined: a sure recipe for destroying program invariants, polluting the
heap, or crashing the program.

Intersection contracts require values to pass all the sub-contracts. This means
rewrapping the higher-order value in something that checks all the domain
sub-contracts as well as all the range sub-contracts. Failing to meet even one
sub-contract means the value has failed the entire intersction.

Union contracts are more subtle, because failing to meet any one sub-contract is
not grounds for rejection. Rather, it simply means that that one sub-contract is
no longer a candidate contract representing the wrapped value; the other
sub-contracts might still be candidates, and only when no others are left must
be reject the value. This means the implementation of union contracts must
maintain memory of which sub-contracts have and have not yet passed—memory, in
this case, being a sophisticated term for the use of mutation. As each
sub-contract fails, it is removed from the list of candidates, while all the
remaining ones continue to applied. When no candidates remain, the contract
system must report a violation. The error report would presumably provide the
actual values that eliminated each part of each sub-contract (keeping in mind
that these may be nested multiple functions deep).
\note{In a multi-threaded language like Racket, this also requires locks to
avoid race conditions.}

The implemented versions of contract constructors and combinators in Racket
place restrictions on the acceptable forms of sub-contracts. These enable
implementations that are both efficient and yield useful error messages.
Furthermore, the more extreme situations discussed above rarely occur in
practice—though now you know how to implement them if you need to.
