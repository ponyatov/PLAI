\secrel{16.5 Extending to Compound Data Structures  . 188}

As we have already discussed, it appears easy to extend contracts to structured
datatypes such as lists, vectors, and user-defined recursive datatypes. This
only requires that the appropriate set of run-time observations be available.
This will usually be the case, up to the resolution of types in the language.
For instance, as we have discussed \ref{}, a language with datatypes does not
require type predicates but will still offer predicates to distinguish the
variants; this is case where type-level “contract” checking is best (and perhaps
must) be left to the static type system, while the contacts assert more refined
structural properties.

However, this strategy can run into significant performance problems. For
instance, suppose we built a balanced binary search-tree to perform asymptotic
logarithmic time (in the size of the tree) insertions and lookups. Now say we
have wrapped this tree in a suitable contract. Sadly, the mere act of checking
the contract visits the entire tree, thereby taking linear time! Ideally,
therefore, we would prefer a strategy whereby the contract was already
checked—incrementally—at the time of construction, and does not need to be
checked again at the time of lookup.

Worse, both balancing and search-tree ordering are recursive properties. In
principle, therefore, they attach to every sub-tree, and so should be applied on
every recursive call. During insertion, which is a recursive procedure, the
contract would be checked on every visited sub-tree. In a tree of size t, the
contract predicate applies to a sub-tree of t 2 elements, then to a sub-sub-tree
of t 4 elements, and so on, resulting—in the worst case—in visiting a total of t
%  2 + t 4 +    + t t 
elements...making our intended logarithmictime insertion process take linear
time.

In both cases, there is ready mitigation available in many cases. Each value
needs to be associated (either intrinsically, or by storage in a hash table)
with the set of contracts it has already passed. Then, when a contract is ready
to apply, it first checks whether the value has already been checked and, if it
has, does not check again. This is essentially a form of memoization of contract
checking and can thus reduce the algorithmic complexity of checking. Again, like
memoization, this works best when the values are immutable. If the values can
mutate and the contracts perform arbitrary computations, it may not be sound to
perform this optimization.

There is a subtler way in which we might examine the issue of data structures.
As an example, consider the contract we wrote earlier to check that all values
in a numeric list are even. Suppose we have wrapped a list in this contract, but
are interested only in the first element of the list. Naturally, we are paying
the cost of checking all the values in the list, which may take a very long
time. More importantly, however, a user might argue that reporting a violation
about the second element of the list is itself a violation of our expectation
about contract-checking, since we did not actually use that element.

This suggests deferring checking even for some values that could be checked
immediately. For instance, the entire list could be turned into a wrapped value
containing a deferred check, and each value is checked only when it is visited.
This strategy might be attractive, but it is not trivial to code, and especially
runs into problems in the presence of aliasing: if two different identifiers are
referring to the same list, one with a contract guard and the other without, we
have to ensure both of them function as expected (which usually means we cannot
store any mutable state in the list itself).
