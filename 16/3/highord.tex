\secrel{16.3 Higher-Order Contracts   . 183}

To determine what to do, it helps to recall what sort of guarantee contracts
provide in the first place. In real-sqrt-1 above, we demanded that the argument
be nonnegative. However, this is only checked if—and when—real-sqrt-1 is
actually used, and then only on the actual values that are passed to it. For
instance, if the program contains this fragment
\lsts{16/3/1.rkt}{rkt}
but this thunk is never invoked, the programmer would never see this contract
violation. In fact, it may be that the thunk is not invoked on this run of the
program, but in a later run it will be; thus, the program has a lurking contract
error. For this reason, it is usually preferable to express invariants through
static types; but where we do use contracts, we understand that it is with the
caveat that we will only be notified of errors when the program is suitably
exercised.

This is a useful insight, because it offers a solution to our problem with
functions. We check, immediately, that the purported function value truly is a
function. However, instead of ignoring the domain and range contracts, we defer
them. We check the domain contract when (and each time) the function is actually
applied to a value, and we check the range contract when the function actually
returns a value.

This is clearly a different pattern than make-contract followed. Thus, we should
give make-contract a more descriptive name: it checks immediate contracts (i.e.,
those that can be checked in their entirety now).
\note{In the Racket contract system, immediate contracts are called flat. This
term is slightly misleading, since they can also protect data structures.}
\lsts{16/3/2.rkt}{rkt}

In contrast, a function contract takes two contracts as arguments—representing
checks to be made on the domain and range—and returns a predicate. This is the
predicate to apply on values purporting to satisfy that contract. First, this
checks that the given value actually is a function: this part is still
immediate. Then, we create a surrogate procedure that applies the “residual”
contracts—to check the domain and range—but otherwise behaves the same as the
original function.

This creation of a surrogate represents a departure from the traditional
assertion mechanism, which simply checks values and then leaves them alone.
Instead, for functions we must use the created surrogate if we want contract
checking. In general, therefore, it is useful to have a wrapper that consumes a
contract and value, and creates a guarded version of that value:
\lsts{16/3/3.rkt}{rkt}

As a very simple example, let us suppose we want to wrap the add1 function in
numeric contracts (with function, the constructor of function contracts, to be
defined momentarily):
\lsts{16/3/4.rkt}{rkt}
We want a1 to be bound to essentially the following code:
\lsts{16/3/5.rkt}{rkt}
Here, the (lambda (x) ...) is the surrogate; it applies two numeric contracts
around the invocation of add1. Recall that contracts must behave like the
identity function in the absence of violations, so this procedure has precisely
the same behavior as add1 on non-violating uses.

To achieve this, we use the following definition of function. Remember that we
have to also ensure that the given value is truly a function (as add1 above
indeed is, and can be checked immediately, which is why the check has
disappeared by the time we bind the surrogate to a1):
\note{For simplicity we assume single-argument functions here, but the extension
to multiple arity is straightforward. Indeed, more complex contracts can even
check for relationships between the arguments.}
\lsts{16/3/6.rkt}{rkt}

To understand how this works, let us substitute arguments. To keep the resulting
code readable, we will first construct the number? contract checker and give it
a name:
\lsts{16/3/7.rkt}{rkt}
Now let’s return to the definition of a1. First we apply guard:
\lsts{16/3/8.rkt}{rkt}
Now we apply the function contract constructor:
\lsts{16/3/9.rkt}{rkt}
Applying the left-left-lambda gives:
\lsts{16/3/10.rkt}{rkt}
Notice that this immediately checks that the guarded value is indeed a function.
Thus we get
\lsts{16/3/11.rkt}{rkt}
which is precisely the surrogate we desired, with the behavior of add1 on
non-violating executions.

\DoNow{
How many ways are there to violate the above contract for add1?
}

There are three ways, corresponding to the three contract constructors:
\begin{enumerate}[nosep]
  \item 
the value wrapped might not be a function;
  \item 
the wrapped value might be a function that is applied to a non-numeric value;
or,
  \item 
the wrapped value might be a function that consumes numbers but produces values
of non-numeric type.
\end{enumerate}

\Exercise{
Write examples that perform each of these three violations, and observe the
behavior of the contract system. Can you improve the error messages to better
distinguish these cases?
}

The same wrapping technique works for d/dx as well:

\lsts{16/3/12.rkt}{rkt}

\Exercise{
There are seven ways to violate this contract, corresponding to each of the
seven contract constructors. Violate each of them by passing arguments or
modifying code, as needed. Can you improve error reporting to correctly identify
each kind of violation?
}

Notice that the nested function contract defers the checking of the immediate
contracts for two applications, rather than one. This is what we should expect,
because immediate contracts only report problems with actual values, so they
cannot report anything until applied to actual values. However, this does mean
that the notion of “violation”’ is subtle: the function value passed to d/dx may
in fact truly be in violation of the contract, but this violation will not be
observed until numeric values are passed or returned.
