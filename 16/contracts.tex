\secrel{16 Checking Program Invariants Dynamically: Contracts 179}\secdown

Type systems offer rich and valuable ways to represent program invariants.
However, they also represent an important trade-off, because not all non-trivial
properties of programs can be verified statically. Furthermore, even if we can
devise a method to settle a certain property statically, the burdens of
annotation and computational complexity may be too great. Thus, it is inevitable
that some of the properties we care about must either be ignored or settled only
at run-time. Here, we will discuss run-time enforcement.
\note{This is a formal property, known as
\href{https://en.wikipedia.org/wiki/Rice's_theorem}{Rice’s Theorem}.}

Virtually every programming language has some form of assertion mechanism that
enables programmers to write properties that are richer than the language’s
static type system permits. In languages without static types, these properties
might start with simple type-like assertions: whether a parameter is numeric,
for instance. However, the language of assertions is often the entire
programming language, so any predicate can be used as an assertion: for
instance, an implementation of a cryptography package might want to ensure
certain parameters pass a primality test, or a balanced binary search-tree might
want to ensure that its subtrees are indeed balanced and preserve the
search-tree ordering.

\secrel{16.1 Contracts as Predicates   . 181}

\secrel{16.2 Tags, Types, and Observations on Values  . 182}

At this point we’ve reproduced the essence of assertion systems in most
languages. What else is there to say? Let’s suppose for a moment that our
language is not statically typed. Then we will want to write assertions that
reproduce at least traditional typelike invariants, if not more. make-contract
above can capture all standard type-like properties such as checking for
numbers, strings, and so on, assuming the appropriate predicates are either
provided by the language or can be fashioned from the ones given.
Or can it?

Recall that even our simplest type language had not just base types, like
numbers, but also constructed types. While some of these, like lists and
vectors, appear to not be very challenging, they are once we care about
mutation, performance, and blame, which we discuss below. However, functions are
immediately problematic.

As a working example, we will take the following function:
\lsts{16/2/1.rkt}{rkt}
Statically, we would give this the type
\lsts{16/2/2.rkt}{rkt}
(it consumes a function, and produces its derivative—another function). Let us
suppose we want to guard this with contracts.

The fundamental problem is that in most languages, we cannot directly express
this as a predicate. Most language run-time systems store very limited
information about the types of values—so limited that, relative to the types we
have seen so far, we should use a different name to describe this information;
traditionally they are called tags. Sometimes tags coincide with what we might
regard as types: for instance, a number will have a tag identifying it as a
number (perhaps even a specific kind of number), a string will have a tag
identifying it as a string, and so forth. Thus we can write predicates based on
the values of these tags.
\note{There have been a few efforts to preserve rich type information from the
source program through lower levels of abstraction all the way down to assembly
language, but these are research efforts.}

When we get to structured values, however, the situation is more complex. A
vector would have a tag declaring it to be a vector, but not dictating what
kinds of values its elements are (and they may not even all be of the same
kind); however, a program can usually also obtain its size, and thus traverse
it, to gather this information. (There is, however, more to be said about
structured values below \ref{}.)

\DoNow{
Write a contract that checks that a list consists solely of even numbers.
}

Here it is:
\lsts{16/2/3.rkt}{rkt}
(Again, note that the first two questions need not be asked if we know,
statically, that we have a list of numbers.) Similarly, an object might simply
identify itself as an object, not providing additional information. But in
languages that permit reflection on the object’s structure, a contract can still
gather the information it needs.

In every language, however, this becomes problematic when we encounter
functions. We might think of a function as having a type for its domain and
range, but to a run-time system, a function is just an opaque object with a
function tag, and perhaps some very limited metadata (such as the function’s
arity). The run-time system can hardly even tell whether the function consumes
and produces functions—as opposed to other kinds of values—much less whether
they it consumes and produces ones of (number -> number) type.

This problem is nicely embodied in the (misnamed) typeof operator in JavaScript.
Given values of base types like numbers and strings, typeof returns a string to
that effect (e.g., "number"). For objects, it returns "object". Most
importantly, for functions it returns "function", with no additional
information.
\note{For this reason, perhaps typeof is a bad name for this operator. It should
have been called tagof instead, leaving open the possibility that future static
type systems for JavaScript could provide a true typeof.}

To summarize, this means that at the point of being confronted with a function,
a function contract can only check that it is, indeed, a function (and if it is
not, that is clearly an error). It cannot check anything about the domain and
range of the function.
Should we give up?

\secrel{16.3 Higher-Order Contracts   . 183}

To determine what to do, it helps to recall what sort of guarantee contracts
provide in the first place. In real-sqrt-1 above, we demanded that the argument
be nonnegative. However, this is only checked if—and when—real-sqrt-1 is
actually used, and then only on the actual values that are passed to it. For
instance, if the program contains this fragment
\lsts{16/3/1.rkt}{rkt}
but this thunk is never invoked, the programmer would never see this contract
violation. In fact, it may be that the thunk is not invoked on this run of the
program, but in a later run it will be; thus, the program has a lurking contract
error. For this reason, it is usually preferable to express invariants through
static types; but where we do use contracts, we understand that it is with the
caveat that we will only be notified of errors when the program is suitably
exercised.

This is a useful insight, because it offers a solution to our problem with
functions. We check, immediately, that the purported function value truly is a
function. However, instead of ignoring the domain and range contracts, we defer
them. We check the domain contract when (and each time) the function is actually
applied to a value, and we check the range contract when the function actually
returns a value.

This is clearly a different pattern than make-contract followed. Thus, we should
give make-contract a more descriptive name: it checks immediate contracts (i.e.,
those that can be checked in their entirety now).
\note{In the Racket contract system, immediate contracts are called flat. This
term is slightly misleading, since they can also protect data structures.}
\lsts{16/3/2.rkt}{rkt}

In contrast, a function contract takes two contracts as arguments—representing
checks to be made on the domain and range—and returns a predicate. This is the
predicate to apply on values purporting to satisfy that contract. First, this
checks that the given value actually is a function: this part is still
immediate. Then, we create a surrogate procedure that applies the “residual”
contracts—to check the domain and range—but otherwise behaves the same as the
original function.

This creation of a surrogate represents a departure from the traditional
assertion mechanism, which simply checks values and then leaves them alone.
Instead, for functions we must use the created surrogate if we want contract
checking. In general, therefore, it is useful to have a wrapper that consumes a
contract and value, and creates a guarded version of that value:
\lsts{16/3/3.rkt}{rkt}

As a very simple example, let us suppose we want to wrap the add1 function in
numeric contracts (with function, the constructor of function contracts, to be
defined momentarily):
\lsts{16/3/4.rkt}{rkt}
We want a1 to be bound to essentially the following code:
\lsts{16/3/5.rkt}{rkt}
Here, the (lambda (x) ...) is the surrogate; it applies two numeric contracts
around the invocation of add1. Recall that contracts must behave like the
identity function in the absence of violations, so this procedure has precisely
the same behavior as add1 on non-violating uses.

To achieve this, we use the following definition of function. Remember that we
have to also ensure that the given value is truly a function (as add1 above
indeed is, and can be checked immediately, which is why the check has
disappeared by the time we bind the surrogate to a1):
\note{For simplicity we assume single-argument functions here, but the extension
to multiple arity is straightforward. Indeed, more complex contracts can even
check for relationships between the arguments.}
\lsts{16/3/6.rkt}{rkt}

To understand how this works, let us substitute arguments. To keep the resulting
code readable, we will first construct the number? contract checker and give it
a name:
\lsts{16/3/7.rkt}{rkt}
Now let’s return to the definition of a1. First we apply guard:
\lsts{16/3/8.rkt}{rkt}
Now we apply the function contract constructor:
\lsts{16/3/9.rkt}{rkt}
Applying the left-left-lambda gives:
\lsts{16/3/10.rkt}{rkt}
Notice that this immediately checks that the guarded value is indeed a function.
Thus we get
\lsts{16/3/11.rkt}{rkt}
which is precisely the surrogate we desired, with the behavior of add1 on
non-violating executions.

\DoNow{
How many ways are there to violate the above contract for add1?
}

There are three ways, corresponding to the three contract constructors:
\begin{enumerate}[nosep]
  \item 
the value wrapped might not be a function;
  \item 
the wrapped value might be a function that is applied to a non-numeric value;
or,
  \item 
the wrapped value might be a function that consumes numbers but produces values
of non-numeric type.
\end{enumerate}

\Exercise{
Write examples that perform each of these three violations, and observe the
behavior of the contract system. Can you improve the error messages to better
distinguish these cases?
}

The same wrapping technique works for d/dx as well:

\lsts{16/3/12.rkt}{rkt}

\Exercise{
There are seven ways to violate this contract, corresponding to each of the
seven contract constructors. Violate each of them by passing arguments or
modifying code, as needed. Can you improve error reporting to correctly identify
each kind of violation?
}

Notice that the nested function contract defers the checking of the immediate
contracts for two applications, rather than one. This is what we should expect,
because immediate contracts only report problems with actual values, so they
cannot report anything until applied to actual values. However, this does mean
that the notion of “violation”’ is subtle: the function value passed to d/dx may
in fact truly be in violation of the contract, but this violation will not be
observed until numeric values are passed or returned.

\secrel{16.4 Syntactic Convenience   . 187}

\secrel{16.5 Extending to Compound Data Structures  . 188}

As we have already discussed, it appears easy to extend contracts to structured
datatypes such as lists, vectors, and user-defined recursive datatypes. This
only requires that the appropriate set of run-time observations be available.
This will usually be the case, up to the resolution of types in the language.
For instance, as we have discussed \ref{}, a language with datatypes does not
require type predicates but will still offer predicates to distinguish the
variants; this is case where type-level “contract” checking is best (and perhaps
must) be left to the static type system, while the contacts assert more refined
structural properties.

However, this strategy can run into significant performance problems. For
instance, suppose we built a balanced binary search-tree to perform asymptotic
logarithmic time (in the size of the tree) insertions and lookups. Now say we
have wrapped this tree in a suitable contract. Sadly, the mere act of checking
the contract visits the entire tree, thereby taking linear time! Ideally,
therefore, we would prefer a strategy whereby the contract was already
checked—incrementally—at the time of construction, and does not need to be
checked again at the time of lookup.

Worse, both balancing and search-tree ordering are recursive properties. In
principle, therefore, they attach to every sub-tree, and so should be applied on
every recursive call. During insertion, which is a recursive procedure, the
contract would be checked on every visited sub-tree. In a tree of size t, the
contract predicate applies to a sub-tree of t 2 elements, then to a sub-sub-tree
of t 4 elements, and so on, resulting—in the worst case—in visiting a total of t
%  2 + t 4 +    + t t 
elements...making our intended logarithmictime insertion process take linear
time.

In both cases, there is ready mitigation available in many cases. Each value
needs to be associated (either intrinsically, or by storage in a hash table)
with the set of contracts it has already passed. Then, when a contract is ready
to apply, it first checks whether the value has already been checked and, if it
has, does not check again. This is essentially a form of memoization of contract
checking and can thus reduce the algorithmic complexity of checking. Again, like
memoization, this works best when the values are immutable. If the values can
mutate and the contracts perform arbitrary computations, it may not be sound to
perform this optimization.

There is a subtler way in which we might examine the issue of data structures.
As an example, consider the contract we wrote earlier to check that all values
in a numeric list are even. Suppose we have wrapped a list in this contract, but
are interested only in the first element of the list. Naturally, we are paying
the cost of checking all the values in the list, which may take a very long
time. More importantly, however, a user might argue that reporting a violation
about the second element of the list is itself a violation of our expectation
about contract-checking, since we did not actually use that element.

This suggests deferring checking even for some values that could be checked
immediately. For instance, the entire list could be turned into a wrapped value
containing a deferred check, and each value is checked only when it is visited.
This strategy might be attractive, but it is not trivial to code, and especially
runs into problems in the presence of aliasing: if two different identifiers are
referring to the same list, one with a contract guard and the other without, we
have to ensure both of them function as expected (which usually means we cannot
store any mutable state in the list itself).

\secrel{16.6 More on Contracts and Observations  189}

\secrel{16.7 Contracts and Mutation   . 189}

\secrel{16.8 Combining Contracts   . . 190}

Now that we’ve discussed combinators for all the basic datatypes, it’s natural
to discuss combining contracts. Just as we saw unions \ref{}\ and intersections
\ref{}\ for types, we should be considering unions and intersections
(respectively, “or”s and “and”s), ; for that matter, we might also consider
negation. However, contracts are only superficially like types, so we have to
consider these questions in their own light for contracts rather than try to map
the meanings we have learned from types to the sphere of contracts.

As always, the immediate case is straightforward. Union contracts combine with
disjunction—indeed, being predicates, their results can literally be combined
with or— and intersection contracts with conjunction. We apply the predicates in
turn, with shortcircuiting, and either generate an error or return the
contracted value. Intersection contracts combine with conjunction (and). And
negation contracts are simply the original immediate contract applied and the
decision negated (with not).

Contract combination is much harder in the deferred, higher-order case. For
instance, consider the negation of a function contract from numbers to numbers.
What exactly does it mean to negate it? Does it mean the function should not
accept numbers? Or that if it does, it should not produce them? Or both? And in
particular, how do we enforce such a contract? How, for instance, do we check
that a function does not accept numbers—are we expecting that when given a
number, it produces an error? But now consider the identity function wrapped
with such a contract; since it clearly does not result in an error when given a
number (or indeed any other value), does that mean we should wait until it
produces a value, and if it does produce a number, reject it? But worst of all,
note that this means we will be running functions on domains on which they are
not defined: a sure recipe for destroying program invariants, polluting the
heap, or crashing the program.

Intersection contracts require values to pass all the sub-contracts. This means
rewrapping the higher-order value in something that checks all the domain
sub-contracts as well as all the range sub-contracts. Failing to meet even one
sub-contract means the value has failed the entire intersction.

Union contracts are more subtle, because failing to meet any one sub-contract is
not grounds for rejection. Rather, it simply means that that one sub-contract is
no longer a candidate contract representing the wrapped value; the other
sub-contracts might still be candidates, and only when no others are left must
be reject the value. This means the implementation of union contracts must
maintain memory of which sub-contracts have and have not yet passed—memory, in
this case, being a sophisticated term for the use of mutation. As each
sub-contract fails, it is removed from the list of candidates, while all the
remaining ones continue to applied. When no candidates remain, the contract
system must report a violation. The error report would presumably provide the
actual values that eliminated each part of each sub-contract (keeping in mind
that these may be nested multiple functions deep).
\note{In a multi-threaded language like Racket, this also requires locks to
avoid race conditions.}

The implemented versions of contract constructors and combinators in Racket
place restrictions on the acceptable forms of sub-contracts. These enable
implementations that are both efficient and yield useful error messages.
Furthermore, the more extreme situations discussed above rarely occur in
practice—though now you know how to implement them if you need to.

\secrel{16.9 Blame   . 191}

\secup

