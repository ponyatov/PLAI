\secrel{16 Checking Program Invariants Dynamically: Contracts 179}\secdown

Type systems offer rich and valuable ways to represent program invariants.
However, they also represent an important trade-off, because not all non-trivial
properties of programs can be verified statically. Furthermore, even if we can
devise a method to settle a certain property statically, the burdens of
annotation and computational complexity may be too great. Thus, it is inevitable
that some of the properties we care about must either be ignored or settled only
at run-time. Here, we will discuss run-time enforcement.
\note{This is a formal property, known as
\href{https://en.wikipedia.org/wiki/Rice's_theorem}{Rice’s Theorem}.}

Virtually every programming language has some form of assertion mechanism that
enables programmers to write properties that are richer than the language’s
static type system permits. In languages without static types, these properties
might start with simple type-like assertions: whether a parameter is numeric,
for instance. However, the language of assertions is often the entire
programming language, so any predicate can be used as an assertion: for
instance, an implementation of a cryptography package might want to ensure
certain parameters pass a primality test, or a balanced binary search-tree might
want to ensure that its subtrees are indeed balanced and preserve the
search-tree ordering.

\secrel{16.1 Contracts as Predicates   . 181}

\secrel{16.2 Tags, Types, and Observations on Values  . 182}

At this point we’ve reproduced the essence of assertion systems in most
languages. What else is there to say? Let’s suppose for a moment that our
language is not statically typed. Then we will want to write assertions that
reproduce at least traditional typelike invariants, if not more. make-contract
above can capture all standard type-like properties such as checking for
numbers, strings, and so on, assuming the appropriate predicates are either
provided by the language or can be fashioned from the ones given.
Or can it?

Recall that even our simplest type language had not just base types, like
numbers, but also constructed types. While some of these, like lists and
vectors, appear to not be very challenging, they are once we care about
mutation, performance, and blame, which we discuss below. However, functions are
immediately problematic.

As a working example, we will take the following function:
\lsts{16/2/1.rkt}{rkt}
Statically, we would give this the type
\lsts{16/2/2.rkt}{rkt}
(it consumes a function, and produces its derivative—another function). Let us
suppose we want to guard this with contracts.

The fundamental problem is that in most languages, we cannot directly express
this as a predicate. Most language run-time systems store very limited
information about the types of values—so limited that, relative to the types we
have seen so far, we should use a different name to describe this information;
traditionally they are called tags. Sometimes tags coincide with what we might
regard as types: for instance, a number will have a tag identifying it as a
number (perhaps even a specific kind of number), a string will have a tag
identifying it as a string, and so forth. Thus we can write predicates based on
the values of these tags.
\note{There have been a few efforts to preserve rich type information from the
source program through lower levels of abstraction all the way down to assembly
language, but these are research efforts.}

When we get to structured values, however, the situation is more complex. A
vector would have a tag declaring it to be a vector, but not dictating what
kinds of values its elements are (and they may not even all be of the same
kind); however, a program can usually also obtain its size, and thus traverse
it, to gather this information. (There is, however, more to be said about
structured values below \ref{}.)

\DoNow{
Write a contract that checks that a list consists solely of even numbers.
}

Here it is:
\lsts{16/2/3.rkt}{rkt}
(Again, note that the first two questions need not be asked if we know,
statically, that we have a list of numbers.) Similarly, an object might simply
identify itself as an object, not providing additional information. But in
languages that permit reflection on the object’s structure, a contract can still
gather the information it needs.

In every language, however, this becomes problematic when we encounter
functions. We might think of a function as having a type for its domain and
range, but to a run-time system, a function is just an opaque object with a
function tag, and perhaps some very limited metadata (such as the function’s
arity). The run-time system can hardly even tell whether the function consumes
and produces functions—as opposed to other kinds of values—much less whether
they it consumes and produces ones of (number -> number) type.

This problem is nicely embodied in the (misnamed) typeof operator in JavaScript.
Given values of base types like numbers and strings, typeof returns a string to
that effect (e.g., "number"). For objects, it returns "object". Most
importantly, for functions it returns "function", with no additional
information.
\note{For this reason, perhaps typeof is a bad name for this operator. It should
have been called tagof instead, leaving open the possibility that future static
type systems for JavaScript could provide a true typeof.}

To summarize, this means that at the point of being confronted with a function,
a function contract can only check that it is, indeed, a function (and if it is
not, that is clearly an error). It cannot check anything about the domain and
range of the function.
Should we give up?

\secrel{16.3 Higher-Order Contracts   . 183}

\secrel{16.4 Syntactic Convenience   . 187}

\secrel{16.5 Extending to Compound Data Structures  . 188}

\secrel{16.6 More on Contracts and Observations  189}

A general problem for any contract implementation—which is exacerbated by
complex data—is a curious one. Earlier, we complained that it was difficult to
check function contracts because we have insufficient power to observe: all we
can check is that a value is a function, and no more. In real languages, the
problem for data structures is actually the opposite: we have too much ability
to observe. For instance, if we implement a strategy of deferring checking of a
list, we quite possibly need to use a structure to hold the actual list, and
modify first and rest to get their values through this structure (after checking
contracts). However, a procedure like list? might now return false rather than
true because structures are not lists; therefore, list? needs to be re-bound to
a procedure that also returns true on structures that represent these special
deferredcontract lists. But the contract system author needs to also remember to
tackle cons?, pair?, and goodness knows how many other procedures that all
perform observations.

In general, one observation is essentially impossible to “fix”: eq?. Normally,
we have the property that every value is eq? to itself, even for functions.
However, the wrapped value of a function is a new procedure that not only isn’t
eq? to itself but probably shouldn’t be, because its behavior truly is different
(though only on contract violations, and only after enough values have been
supplied to observe the violation). However, this means that a program cannot
surreptitiously guard itself, because the act of guarding can be observed. As a
result, a malicious module can sometimes detect whether it is being passed
guarded values, behaving normally when it is and abnormally only when it is not!

\secrel{16.7 Contracts and Mutation   . 189}

We should rightly be concerned about the interaction between contracts and
mutation, and even more so when we have contracts that are either inherently
deferred or have been implemented in a deferred fashion. There are two things to
be concerned about. One is storing a contracted value in mutable state. The
other is writing a contract for mutable state.

When we store a contracted value, the strategy of wrapping ensures that contract
checking works gracefully. At each stage, a contract checks as much as it can
with the value at hand, and creates a wrapped value embodying the residual
check. Thus, even if this wrapped value is stored in mutable state and retrieved
for use later, it still contains these checks, and they will be performed when
the value is eventually used.

The other issue is writing contracts for mutable data, such as boxes and
vectors. In this case we probably have to create a wrapper for the entire
datatype that records the intended contract. Then, when a value inside the datatype is
replaced with a new one, the operation that performs the update—such as
set-box!—needs to retrieve the intended contract from the wrapper, apply it to
the value, and store the wrapped value. Therefore, this requires changing the
behavior of the data structure mutation operators to be sensitive to contracted
values. However, mutation does not change the point at which violations are
caught: right away for immediate contracts, upon (in)appropriate use for
deferred ones.

\secrel{16.8 Combining Contracts   . . 190}

\secrel{16.9 Blame   . 191}

\secup

