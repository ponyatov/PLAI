\secrel{16.6 More on Contracts and Observations  189}

A general problem for any contract implementation—which is exacerbated by
complex data—is a curious one. Earlier, we complained that it was difficult to
check function contracts because we have insufficient power to observe: all we
can check is that a value is a function, and no more. In real languages, the
problem for data structures is actually the opposite: we have too much ability
to observe. For instance, if we implement a strategy of deferring checking of a
list, we quite possibly need to use a structure to hold the actual list, and
modify first and rest to get their values through this structure (after checking
contracts). However, a procedure like list? might now return false rather than
true because structures are not lists; therefore, list? needs to be re-bound to
a procedure that also returns true on structures that represent these special
deferredcontract lists. But the contract system author needs to also remember to
tackle cons?, pair?, and goodness knows how many other procedures that all
perform observations.

In general, one observation is essentially impossible to “fix”: eq?. Normally,
we have the property that every value is eq? to itself, even for functions.
However, the wrapped value of a function is a new procedure that not only isn’t
eq? to itself but probably shouldn’t be, because its behavior truly is different
(though only on contract violations, and only after enough values have been
supplied to observe the violation). However, this means that a program cannot
surreptitiously guard itself, because the act of guarding can be observed. As a
result, a malicious module can sometimes detect whether it is being passed
guarded values, behaving normally when it is and abnormally only when it is not!
