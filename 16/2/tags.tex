\secrel{16.2 Tags, Types, and Observations on Values  . 182}

At this point we’ve reproduced the essence of assertion systems in most
languages. What else is there to say? Let’s suppose for a moment that our
language is not statically typed. Then we will want to write assertions that
reproduce at least traditional typelike invariants, if not more. make-contract
above can capture all standard type-like properties such as checking for
numbers, strings, and so on, assuming the appropriate predicates are either
provided by the language or can be fashioned from the ones given.
Or can it?

Recall that even our simplest type language had not just base types, like
numbers, but also constructed types. While some of these, like lists and
vectors, appear to not be very challenging, they are once we care about
mutation, performance, and blame, which we discuss below. However, functions are
immediately problematic.

As a working example, we will take the following function:
\lsts{16/2/1.rkt}{rkt}
Statically, we would give this the type
\lsts{16/2/2.rkt}{rkt}
(it consumes a function, and produces its derivative—another function). Let us
suppose we want to guard this with contracts.

The fundamental problem is that in most languages, we cannot directly express
this as a predicate. Most language run-time systems store very limited
information about the types of values—so limited that, relative to the types we
have seen so far, we should use a different name to describe this information;
traditionally they are called tags. Sometimes tags coincide with what we might
regard as types: for instance, a number will have a tag identifying it as a
number (perhaps even a specific kind of number), a string will have a tag
identifying it as a string, and so forth. Thus we can write predicates based on
the values of these tags.
\note{There have been a few efforts to preserve rich type information from the
source program through lower levels of abstraction all the way down to assembly
language, but these are research efforts.}

When we get to structured values, however, the situation is more complex. A
vector would have a tag declaring it to be a vector, but not dictating what
kinds of values its elements are (and they may not even all be of the same
kind); however, a program can usually also obtain its size, and thus traverse
it, to gather this information. (There is, however, more to be said about
structured values below \ref{}.)

\DoNow{
Write a contract that checks that a list consists solely of even numbers.
}

Here it is:
\lsts{16/2/3.rkt}{rkt}
(Again, note that the first two questions need not be asked if we know,
statically, that we have a list of numbers.) Similarly, an object might simply
identify itself as an object, not providing additional information. But in
languages that permit reflection on the object’s structure, a contract can still
gather the information it needs.

In every language, however, this becomes problematic when we encounter
functions. We might think of a function as having a type for its domain and
range, but to a run-time system, a function is just an opaque object with a
function tag, and perhaps some very limited metadata (such as the function’s
arity). The run-time system can hardly even tell whether the function consumes
and produces functions—as opposed to other kinds of values—much less whether
they it consumes and produces ones of (number -> number) type.

This problem is nicely embodied in the (misnamed) typeof operator in JavaScript.
Given values of base types like numbers and strings, typeof returns a string to
that effect (e.g., "number"). For objects, it returns "object". Most
importantly, for functions it returns "function", with no additional
information.
\note{For this reason, perhaps typeof is a bad name for this operator. It should
have been called tagof instead, leaving open the possibility that future static
type systems for JavaScript could provide a true typeof.}

To summarize, this means that at the point of being confronted with a function,
a function contract can only check that it is, indeed, a function (and if it is
not, that is clearly an error). It cannot check anything about the domain and
range of the function.
Should we give up?
