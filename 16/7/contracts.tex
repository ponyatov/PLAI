\secrel{16.7 Contracts and Mutation   . 189}

We should rightly be concerned about the interaction between contracts and
mutation, and even more so when we have contracts that are either inherently
deferred or have been implemented in a deferred fashion. There are two things to
be concerned about. One is storing a contracted value in mutable state. The
other is writing a contract for mutable state.

When we store a contracted value, the strategy of wrapping ensures that contract
checking works gracefully. At each stage, a contract checks as much as it can
with the value at hand, and creates a wrapped value embodying the residual
check. Thus, even if this wrapped value is stored in mutable state and retrieved
for use later, it still contains these checks, and they will be performed when
the value is eventually used.

The other issue is writing contracts for mutable data, such as boxes and
vectors. In this case we probably have to create a wrapper for the entire
datatype that records the intended contract. Then, when a value inside the datatype is
replaced with a new one, the operation that performs the update—such as
set-box!—needs to retrieve the intended contract from the wrapper, apply it to
the value, and store the wrapped value. Therefore, this requires changing the
behavior of the data structure mutation operators to be sensitive to contracted
values. However, mutation does not change the point at which violations are
caught: right away for immediate contracts, upon (in)appropriate use for
deferred ones.
