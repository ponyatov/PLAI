\secrel{16.4 Syntactic Convenience   . 187}

Earlier we saw two styles of using flat contracts, as embodied in real-sqrt-1
and real-sqrt-2. Both styles have disadvantages. The latter, which is
reminiscent of traditional assertion systems, simply does not work for
higher-order values, because it is the wrapped value that must be used in the
computation. (Not surprisingly, traditional assertion systems only handle
immediate contracts, so they fail to notice this subtlety.) The style in the
former, where we wrap each use with a contract, works in theory but suffers from
three downsides:
\begin{enumerate}[nosep]
  \item 
The developer may forget to wrap some uses.
  \item 
The contract is checked once per use, which is wasteful when there is more than
one use.
  \item 
The program comingles contract checking with its functional behavior, reducing
readability.
\end{enumerate}
Fortunately, a judicious use of syntactic sugar can solve this problem in common
cases. For instance, suppose we want to make it easy to attach contracts to
function parameters, so a developer could write
\lsts{16/4/1.rkt}{rkt}
with the intent of guarding x with positive?, but performing the check only
once, on function invocation. This should translate to, say,
\lsts{16/4/2.rkt}{rkt}
That is, the macro generates a fresh name for each identifier, then associates
the name given by the user to the wrapped version of the value supplied to that
fresh name. The following macro implements exactly this:
\lsts{16/4/3.rkt}{rkt}
With conveniences like this, designers of contract languages can improve the
readability, efficiency, and robustness of contract use.
