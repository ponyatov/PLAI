\secrel{Writing an Interpreter \ru{Написание интерпретатора}}

Now let’s write an interpreter for this arithmetic language.
\ru{Теперь давайте напишем интерпретатор для этого арифметического языка.} 
First, we should think about what its type is.
\ru{Для начала нам надо подумать, какие типы он использует\,?} 
It clearly consumes a \verb|ArithC| value.
\ru{Совершенно ясно что на вход подается структура типа} \verb|ArithC|. 
What does it produce\,?
\ru{Что он возвращает\,?} 
Well, an interpreter evaluates
\ru{Ну, интерпретатор вычисляет} \ --- and what kind of value might arithmetic
expressions reduce to\,?
\ru{и к какому значению может редуцироваться арифметическое выражение\,?} 
Numbers, of course.
\ru{Конечно, числу.} 
So the interpreter is going to be a function from arithmetic expressions to
numbers.
\ru{Таким образом, интерпретатор должен быть функцией от арифметического
выражения, возвращающей число.}

\Exercise{
Write your test cases for the interpreter.
\ru{Напишите тесты для интерпретатора.}
}

Because we have a recursive datatype, it is natural to structure our interpreter
as a recursive function over it.
\ru{Так как мы имеем рекурсивный тип данных\note{допускаются произвольные
вложения того же типа}, нормально что структура нашего интерпретатора тоже
должна быть рекурсивной функций над выражением.}
Here’s a first template\note{Templates are
explained in great detail in \emph{How to Design Programs}.}
\ru{Вот первый шаблон\note{\ru{Шаблоны очень детально описаны в \emph{How to
Design Programs}}}} :
\lstx{ArithC.rkt}{src/3/p14_1.rkt}{rkt}
You’re probably tempted to jump straight to code, which you can:
\ru{Вероятно у вас есть соблазн сразу перейти к коду, который вы можете
написать:}
\lstx{ArithC.rkt}{src/3/p14_2.rkt}{rkt}

\DoNow{
Do you spot the errors\,?
\ru{Вы увидели ошибки\,?}
}

Instead, let’s expand the template out a step:
\ru{Вместо этого давайте расширим шаблон на один шаг:}
\lstx{ArithC.rkt}{src/3/p15_2.rkt}{rkt}
and now we can fill in the blanks:
\ru{и теперь мы можем заполнить пробелы:}
\lstx{ArithC.rkt}{src/3/p15_3.rkt}{rkt}

Later on \ru{Позже в} \ref{}, we’re going to wish we had returned a more complex
datatype than just numbers.
\ru{мы пожелаем возвращать более сложный тип данных, чем просто числа.}
But for now, this will do.
\ru{Но сейчас нам этого достаточно.}

Congratulations: you’ve written your first interpreter\,!
\ru{Поздравляем: вы только что написали свой первый интерпретатор\,!} 
I know, it’s very nearly an anticlimax\note{\ru{ситуация, когда проблема
казавшеяся очень сложной, решается с помощью чего-то тривиального //
Wikipedia}}\note{\ru{род морских улиток // там же}}.
\ru{Я знаю, это очень близко к разочарованию.}
But they’ll get harder\ --- much harder\ --- pretty soon, I promise.
\ru{Но все станет жестче\ --- намного жестче\ --- совсем скоро, я обещаю.}
