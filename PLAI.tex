\input{../texheader/ebook}

\usepackage{../texheader/lstrkt}\lstdefinestyle{rkt}{language=rkt}

\newcommand{\Exercise}[1]{
	\begin{description}
		\item{\textcolor{red}{Упражнение}}\\#1
	\end{description}
}

\newcommand{\DoNow}[1]{
	\begin{description}
		\item{\textcolor{red}{Сделайте\,!}}\\#1
	\end{description}
}

\title{{\Large{PLAI}}\\
{\Large{Programming Languages: Application and Interpretation}}\\
{\small{second edition}}\\
{\Large{\ru{языки программирования:\\применение и интерпретация}}}}

\author{\copyright\ Shriram Krishnamurthy\\
\ru{перевод Dmitry Ponyatov \email{dponyatov@gmail.com}}}

\begin{document}
\maketitle
\tableofcontents
\secdown

% \secrel{Введение}\secdown
\secrel{Наша философия}

Пожалуйста посмотрите это
\href{https://www.youtube.com/watch?v=3N__tvmZrzc}{видео на YouTube}.

Когда нибудь здесь будет полное текстовое описание того, о чем в нем говориться.


\secrel{Структура книги}

В отличие от большинства учебников, эта книга не следует подходу "сверху вниз".
Скорее она имеет форму повествования с возвратами к предыдущим темам.
Мы часто будет строить программы инкрементно, так же как мы бы это делали в
парном программировании. Наш код будет включать ошибки, не потому что мы не
знаем правильный ответ, но потому что это лучший способ научить вас, 
углубляясь от поверхностного в детали. Включение намеренных ошибок делает
невозможным для вас читать материал пассивно: вы должны взаимодействовать с
ним, потому что вы никогда не можете быть уверены в правильности того что вы
читаете.

В конце концов вы всегда получите правильный ответ. Тем не меннее, это
нелинейное повествование немного раздражающе в краткосрочной перспективе (у вас
всегда будет соблазн сказать "Скажите же мне наконец ответ !"), и это также
делает эту книгу плохим справочником (вы не можете открыть произвольную
страницу и быть уверенным что на ней написана правда). Тем не менее, это чувство
разочарования\ --- ощущение обучения. Я не знаю другого способа.

\bigskip
В некоторых местах вы встретите следующие выделения:

\Exercise{Это упражнение. Попробуйте это сделать.}

Это традиционное для учебников упражнение.
Это то, что вам нужно сделать по своему усмотрению.
Если вы используете эту книгу как часть курса, это упражнение хорошо
задавать как домашнюю работу. В противоположность этому вы также можете найти
подобные вопросы, выделенные как 

\DoNow{Здесь предполагаются немедленные действия. Вы видите это ?}

Когда вы доберетесь до одного из этих блоков, остановитесь. Прочитайте,
подумайте, сформулируйте ответ перед тем как продолжить чтение. Вы должны
сделать это потому что это действительно упражнение, но ответ уже есть в книге,
чаще всего в тексте непосредственно после упражнения (т.е. в части которую вы
сейчас читаете) или это что-то, что вы можете получить самостоятельно,
запустив программу. Если вы просто продолжите читать, то вы увидете ответ без
его обдумывания (или не увидите его вообще, если это инструкции по запуску
программы), так что вы ни (а) проверите свои знания, ни (б) улучшите свое
понимание. Другими словами, это дополнительные, явные попытки стимулировать
ваше активное обучение. В конце концов, я могу только поощрять вас работать;
решение применять это или нет остается за вами.


\secrel{Язык программирования используемый в книге}

Язык программирования используемый в книге\ --- 
\href{http://www.racket-lang.org/}{Racket}. Аналогично операционным системам,
Racket-система является исполняющей средой для целого ряда языков
программирования, так что \emph{вы должны указать Racketу на каком языке 
вы программируете}. Например, в Unix вы указываете в строку типа

\begin{verbatim}
#!/bin/sh
\end{verbatim}

в первой строке shell-скрипта; вы указываете веб-браузеру
тип документа, добавляя заголовок

\begin{verbatim}
<!DOCTYPE HTML PUBLIC "-//W3C//DTD HTML 4.01//EN" ...>
\end{verbatim}

Аналогично, Racket требует от вас указать какой язык вы будете использовать.
Диалекты языков Racket имеют тот же скобочный синтаксис, что и сам Racket,
но другую семантику; ту же семантику но другой синтаксис; или и то и то.
Так что каждая программа, которую может выполнять Racket-система, начинается со
строки \#lang за которой следует имя диалекта языка: по умолчанию, 
это оригинальный Racket (указыватся как \verb|racket|). В этой книге мы 
почти всегда будем использовать диалект\note{В DrRacket v.6.6,
выберите меню \menu{Язык > Выбрать язык\ldots > Start your program with \#lang
to specify the desired dialect}.}

\begin{verbatim}
#lang plai-typed
\end{verbatim}

Когда мы будем отклоняться от этого правила, это будет указано особо, так что
если не указано иное, добавляйте заголовок \verb|#lang plai-typed| в начало
каждого файла программы (предполагается что я тоже это сделал).



\secup

% \secrel{Everything (We Will Say) About Parsing
\ru{Все (что мы будем говорить) о разборе}}\secdown
\clearpage

Parsing is the act of turning an input character stream into a more structured,
internal representation.
\ru{\termdef{Парсинг}{парсинг} или \termdef{разбор}{синтаксический разбор}\
--- процесс превращения входного потока одиночных символов в более
структурированное внутреннее представление\note{программы или данных, заданных в
текстовой синтаксической форме}.}
A common internal representation is as a tree, which programs can recursively
process.
\ru{Обычно используется внутреннее представление в виде дерева, которое может
быть обработано программой рекурсивно.}

For instance, given the stream
\ru{Например, для входного потока символов\note{включая пробелы, табуляции и
концы строк}}
\begin{verbatim}
23 + 5 - 6
\end{verbatim}
we might want a tree representing addition whose left (L) node represents the
number 23 and whose right (R) node represents subtraction of 6 from 5.
\ru{мы хотим получить деревянное представление сложения, в котором левая (L)
ветвь содержит число 23, а правая (R)\ --- вложенное представление вычитания 6
из 5.}
A parser is responsible for performing this transformation.
\ru{Парсер отвечает за выполнение такой транформации.}

\noindent
\begin{tabular}{c c}
\noindent\includegraphics[height=0.8\textheight]{tmp/2_p10_R.pdf}
&
\noindent\includegraphics[height=0.8\textheight]{tmp/2_p10_L.pdf}
\\
\emph{Право}ассоциативный разбор
&
(*) \emph{Лево}ассоциативный разбор
\\
\end{tabular}\bigskip

Parsing is a large, complex problem that is far from solved due to the
difficulties of ambiguity.
\ru{Парсинг\ --- большая проблема информатики, сложность которой
заключается в трудностях неоднозначности.}
For instance, an alternate parse tree (*) for the above input expression might
put subtraction at the top and addition below it.
\ru{Например, существует альтернативное (*) \termdef{дерево разбора}{дерево
разбора} для того же входного выражения, в котором мы можем поместить
вычитание на вершину дерева, а сложение будет вложенным поддеревом.}
We might also want to consider whether this addition operation is commutative
and hence whether the order of arguments can be switched.
\ru{Нас также может интересовать, является ли это сложение коммутативной
операцией, то есть можем ли мы изменить порядок аргументов\note{например для
оптимизации кода}.}
Everything only gets much, much worse when we get to full-fledged programming
languages (to say nothing of natural languages).
\ru{Все становится намного хуже по мере того, как мы пробираемся в сторону
полноценных языков программирования (даже не говоря о натуральных языках).}

\secrel{2.1 A Lightweight, Built-In First Half of a Parser . . . . . . . . . . . . . 10}

\input{2_2_shortcut}
\input{2_3_types}
\secrel{Completing the Parser \ru{Заканчиваем с парсером}}\label{sec2_4}

In principle, we can think of \ru{В принципе, мы можем думать о} \verb|read|\
as a complete parser \ru{как о законченном парсере}.
However, its output is generic \ru{Тем не менее, его вывод все еще сырой}:
it represents the token structure without offering any comment on its intent.
\ru{он содержит структуру токенов не предлагая каких-либо комментариев об их
назначении.}
We would instead prefer to have a representation that tells us something about
the \emph{intended meaning} of the terms in our language, just as we wrote at
the very beginning: “representing addition”, “represents a number”, and so on.
\ru{Вместо этого мы предпочли бы иметь представление, которое говорит нам
что-то о \emph{предполагаемом значении} термов нашего языка, так же как мы
писали в самом начале: ``представление сложения'', ``представление числа'' и
так далее.}

To do this, we must first introduce a datatype that captures this
representation. \ru{Чтобы сделать это, мы сначала введем тип данных, который
зафиксирует это представление.} We will separately discuss \ru{Мы отдельно
рассмотрим} (section \ru{в разделе} \ref{sec31}) how and why we obtained this
datatype \ru{как и зачем мы применяем этот тип}, but for now let’s say it’s
given to us \ru{но сейчас пока будем считать, что он нам задан}:
\lstx{ArithC.rkt}{src/2/p12_1.rkt}{rkt}
We now need a function that will convert s-expressions into instances of this
datatype.
\ru{Теперь нам нужна функция, которая преобразует s-выражение в структуру из
экземпляров этого типа.}
This is the other half of our parser \ru{Это вторая половина нашего парсера}:
\lstx{ArithC.rkt}{src/2/p12_2.rkt}{rkt}

Thus\note{typing in \racket\ console \emph{after program run}} \ru{Таким
образом\note{\ru{введя выражение в \racket-консоли \emph{после выполнения
программы}}}}
\begin{verbatim}
> (parse '(+ (* 1 2) (+ 2 3)))
- ArithC
(plusC
    (multC (numC 1) (numC 2))
    (plusC (numC 2) (numC 3)))
\end{verbatim}
\lstx{ArithC.rkt}{src/2/p12_3.rkt}{rkt}
\begin{verbatim}
(good
  (parse '(+ (* 1 2) (+ 2 3)))
  (plusC (multC (numC 1) (numC 2)) (plusC (numC 2) (numC 3)))
  (plusC (multC (numC 1) (numC 2)) (plusC (numC 2) (numC 3)))
  "at line 26")
\end{verbatim}

Congratulations\,! \ru{Мои поздравления\,!}
You have just completed your first representation of a program.
\ru{Вы только что завершили ваше первое представление программы.}
From now on we can focus entirely on programs represented as recursive trees,
ignoring the vagaries of surface syntax and how to get them into the tree form.
\ru{С этого момента мы можем полностью сосредоточиться на программах,
представленных в виде рекурсивных деревьев, не обращая внимания на капризы
наносного синтаксиса и процесс получения из него дерева разбора.}
We’re finally ready to start studying programming languages\,!
\ru{Мы, наконец, готовы приступить к изучению языков программирования\,!}

\Exercise{
What happens if you forget to quote the argument to the 
\ru{Что случиться, если вы забудете заквотить аргумент вызова}
parser\,?
Why\,? \ru{Почему\,?}
}
\input{2_5_coda}
\secup

% \secrel{A First Look at Interpretation \ru{Первый взгляд на
интерпретацию}}\secdown

Now that we have a representation of programs, there are many ways in which we
might want to manipulate them.
\ru{Теперь, когда мы имеем представление программ, существует множество
способов, которыми мы можем манипулировать ими.}
We might want to display a
program in an attractive way \ru{Мы можем захотеть выводить листинг программы в
красивом виде} (“pretty-print”), convert into code in some other format
\ru{преобразовать в код в какой-то другой формат} (“compilation”
\ru{``компиляция''/''трансляция''}), ask whether it obeys certain properties 
\ru{убедиться что она отвечает определенным требованиям}
(“verification” \ru{``верификация''}), and so on \ru{и так далее}.
For now, we’re going to focus on asking what value it corresponds to
(“\termdef{e\underline{val}uation}{evaluation}”\ --- the reduction of programs
to \emph{\underline{val}ues}).
\ru{Для начала, мы собираемся сфокусироваться на вопросе\ --- какому значению
соответствует программа (“\termdef{вычисление}{вычисление}"\ --- редукция
программы до \emph{значения})}

Let’s write an evaluator, in the form of an \termdef{interpreter}{interpreter},
for our arithmetic language.
\ru{Давайте напишем вычислитель, в форме
\termdef{интерпретатора}{интерпретатор}, для нашего арифметического языка.}
We choose arithmetic first for three reasons \ru{Мы выбрали арифметику прежде
всего по следующим трем причинам}:
\begin{itemize}[nosep]
  \item[(a)]
  you already know how it works, so we can focus on the mechanics of writing
evaluators;
\ru{вы уже знаете как работает арифметика, и мы можем сфокусироваться на
механике написания вычислителей;}
  \item[(b)]
  it’s contained in
every language we will encounter later, so we can build upwards and outwards from it;
\ru{она содержится в каждом языке, с которым мы столкнемся в дальнейшем, так
что мы будем расширять этот арифметический язык вверх и вширь;} and \ru{и}
  \item[(c)] 
it’s at once both small and big enough to illustrate many points
we’d like to get across.
\ru{этот язык минималистичен, но при этом достаточно большой, чтобы
проиллюстрировать многие моменты, которые мы хочем до вас донести.}
\end{itemize}

\secrel{Representing Arithmetic \ru{Представление арифметики}}\label{sec31}

\input{3_2_interp}
\input{3_3_notice}
\secrel{3.4 Growing the Language . . . . . . . . . . . . . . . . . . . . . . . . . 16}

\secup

% \secrel{4 A First Taste of Desugaring 16}\label{sec4}\secdown

We’ve begun with a very spartan arithmetic language.
\ru{Мы начали с очень спартанского арифметического языка.}
Let’s look at how we might extend it with more arithmetic operations that can
nevertheless be expressed in terms of existing ones.
\ru{Давайте посмотрим, как мы могли бы расширить его б\'ольшим количеством
арифметических операций, которые тем не менее могут быть выражены в терминах
существующих операторов.}
We’ll add just two, because these will suffice to illustrate the point.
\ru{Мы добавим только два, потому что этого будет достаточно для иллюстрации
этого метода.}

\secrel{4.1 Extension: Binary Subtraction . . . . . . . . . . . . . . . . . . . . . 17}
\secrel{4.2 Extension: Unary Negation . . . . . . . . . . . . . . . . . . . . . . . 18}
\secup

% 
% \secrel{Adding Functions to the Language\\
\ru{Добавление функций в язык}}\secdown

Let’s start turning this into a real programming language.
\ru{Теперь давайте превратим нашу поделку в настоящий язык программирования.}
We could add intermediate features such as conditionals,
\ru{Мы можем добавить вспомогательные фичи типа условных конструкций,}
but to do almost anything interesting
\ru{но чтобы сделать что-то интересное}
we’re going to need functions or their moral equivalent, 
\ru{нам нужны \termdef{функции}{функция} или их эквивалент,}
so let’s get to it.
\ru{так что начнем.}

\Exercise{
Add conditionals to your language.
\ru{Добавьте условные конструкции в ваш язык.}
You can either add boolean datatypes
\ru{Вы можете добавить булевый тип данных}
or, if you want to do something quicker,
\ru{или, если вы хотите сделать это побыстрее,}
add a conditional that treats \emph{0} as \term{false}
\ru{добавьте условие что \emph{0} считается \term{ложью},}
and \emph{everything else} as \term{true}.
\ru{ и не-ноль\ --- \term{истиной}.}

What are the important test cases you should write\,?\\
\ru{Какие варианты тестов вы должны написать\,?}
}

Imagine, therefore, that we’re modeling a system like Dr\racket.
\ru{Представьте что мы моделируюем систему типа Dr\racket.}
The developer defines functions in the definitions window, 
\ru{Разработчик определяет функции в окне определений,}
and uses them in the interactions window.
\ru{и использует их к интерактивном окне.}
For now, let’s assume all definitions go in the definitions window only
\ru{Предполагается что все определения вводятя \emph{только} в окне определений}
(we’ll relax this soon \ru{позже мы ослабим это требование} \ref{}),
and all expressions in the interactions window only.
\ru{а все выражения только в интерактивном окне.}
Thus, running a program simply loads definitions.
\ru{Таким образом, запуск программы просто загружает определения.}
Because our interpreter corresponds to the interactions window prompt,
\ru{Так как наш интерпретатор соответствует строке ввода интерактивного окна,}
we’ll therefore assume it is supplied with a set of definitions.
\ru{мы также предполагаем что интерпретатор укомплектован набором этих
определений.}
\note{A \emph{set} of definitions suggests \emph{no ordering}, which means,
presumably, any definition can refer to any other. That’s what I intend here,
but when you are designing your own language, be sure to think about this.}
\note{\ru{\emph{Набор} определений предполагает \emph{отсутствие
упорядочивания}, то есть любое определение может ссылаться на любое другое. Вот
что я здесь предполагаю сделать, но когда вы разрабатываете ваш собственный
язык, подумайте об этом.}}
\clearpage

\secrel{5.1 Defining Data Representations  19}

\secrel{Growing the Interpreter\\\ru{Рост интерпретатора}}

Now we’re ready to tackle the interpreter proper. First, let’s remind ourselves
of what it needs to consume. Previously, it consumed only an expression to
evaluate. Now it also needs to take a list of function definitions:
\lsts{src/5/5_2_1.rkt}{rkt}
Let’s revisit our old interpreter \ref{firstinterp}. In the case of numbers,
clearly we still return the number as the answer. In the addition and
multiplication case, we still need to recur (because the sub-expressions might
be complex), but which set of function definitions do we use? Because the act of
evaluating an expression neither adds nor removes function \emph{definitions},
the set of definitions remains the same, and should just be passed along
unchanged in the recursive calls.
\lsts{src/5/5_2_2.rkt}{rkt}
Now let’s tackle application. First we have to look up the function definition,
for which we’ll assume we have a helper function of this type available:
\lsts{src/5/5_2_3.rkt}{rkt}
Assuming we find a function of the given name, we need to evaluate its body.
However, remember what we said about identifiers and parameters\,? We must
“search-and-replace”, a process you have seen before in school algebra called
\termdef{substitution}{substitution}. This is sufficiently important that we
should talk first about substitution before returning to the interpreter
\ref{interpresumed}.

\input{5_3_subst}
\input{5_4_resumed}
\input{5_5_more}
\secup

% \input{6_env}
% \secrel{7 Functions Anywhere 31}\secdown

The introduction to the Scheme programming language definition establishes this
design principle:
\begin{framed}
Programming languages should be designed not by piling feature on top of
feature, but by removing the weaknesses and restrictions that make additional
features appear necessary. \ref{}
\end{framed}
As design principles go, this one is hard to argue with. (Some restrictions, of
course, have good reason to exist, but this principle forces us to argue for
them, not admit them by default.) Let’s now apply this to functions.

One of the things we stayed coy about when introducing functions (section 5) is
exactly where functions go. We may have suggested we’re following the model of
an idealized DrRacket, with definitions and their uses kept separate. But,
inspired by the Scheme design principle, let’s examine how necessary that is.

Why can’t functions definitions be expressions? In our current
arithmetic-centric language we face the uncomfortable question “What value does
a function definition represent?”, to which we don’t really have a good answer.
But a real programming language obviously computes more than numbers, so we no
longer need to confront the question in this form; indeed, the answer to the
above can just as well be, “A function value”. Let’s see how that might work
out.

What can we do with functions as values? Clearly, functions are a distinct kind
of value than a number, so we cannot, for instance, add them. But there is one
evident thing we can do: apply them to arguments! Thus, we can allow function
values to appear in the function position of an application. The behavior would,
naturally, be to apply the function. Thus, we’re proposing a language where the
following would be a valid program (where I’ve used brackets so we can easily
identify the function)
\lsts{src/7/7.rkt}{rkt}
and would evaluate to \verb|(+ 2 (* 4 3))|, or 14.\note{Did you see that I just
used substitution\,?}

\secrel{7.1 Functions as Expressions and Values  32}

Let’s first define the core language to include function definitions:
\lsts{src/7/7_1_1.rkt}{rkt}

For now, we’ll simply copy function definitions into the expression language.
We’re free to change this if necessary as we go along, but for now it at least
allows us to reuse our existing test cases.
\lsts{src/7/7_1_2.rkt}{rkt}

We also need to determine what an application looks like. What goes in the
function position of an application? We want to allow an entire function
definition, not just its name. Because we’ve lumped function definitions in with
all other expressions, let’s allow an arbitrary expression here, but with the
understanding that we want only function definition expressions:
\note{We might consider more refined datatypes that split function definitions
apart from other kinds of expressions. This amounts to trying to classify
different kinds of expressions, which we will return to when we study types.
\ref{}}
\lsts{src/7/7_1_3.rkt}{rkt}

With this definition of application, we no longer have to look up functions by name,
so the interpreter can get rid of the list of function definitions. If we need it we can
restore it later, but for now let’s just explore what happens with function definitions are
written at the point of application: so-called immediate functions.

Now let’s tackle interp. We need to add a case to the interpreter for function
definitions, and this is a good candidate:
\lsts{src/7/7_1_4.rkt}{rkt}

\DoNow{
What happens when you add this?
}
Immediately, we see that we have a problem: the interpreter no longer always
returns numbers, so we have a type error.

We’ve alluded periodically to the answers computed by the interpreter, but never
bothered gracing these with their own type. It’s time to do so now.
\lsts{src/7/7_1_5.rkt}{rkt}

We’re using the suffix of V to stand for values, i.e., the result of evaluation.
The pieces of a funV will be precisely those of a fdC: the latter is input, the
former is output. By keeping them distinct we allow each one to evolve
independently as needed.

Now we must rewrite the interpreter. Let’s start with its type:
\lsts{src/7/7_1_6.rkt}{rkt}

This change naturally forces corresponding type changes to the Binding datatype
and to lookup.

\Exercise{
Modify Binding and lookup, appropriately.
}
\lsts{src/7/7_1_7.rkt}{rkt}

Clearly, numeric answers need to be wrapped in the appropriate numeric answer
constructor. Identifier lookup is unchanged. We have to slightly modify addition
and multiplication to deal with the fact that the interpreter returns Values,
not numbers:
\lsts{src/7/7_1_8.rkt}{rkt}

It’s worth examining the definition of one of these helper functions:
\lsts{src/7/7_1_9.rkt}{rkt}
\secrel{7.2 Nested What?   35}

The body of a function definition is an arbitrary expression. A function
definition is itself an expression. That means a function definition can contain
a\ldots function definition. For instance:
\lsts{src/7/7_2_1.rkt}{rkt}

Evaluating this isn’t very interesting:
\lsts{src/7/7_2_2.rkt}{rkt}
But suppose we apply the above function to something:
\lsts{src/7/7_2_3.rkt}{rkt}
Now the answer becomes more interesting:
\lsts{src/7/7_2_4.rkt}{rkt}
It’s almost as if applying the outer function had no impact on the inner
function at all. Well, why should it? The outer function introduces an
identifier which is promptly masked by the inner function introducing one of the
same name, thereby masking the outer definition if we obey static scope (as we
should!). But that suggests a different program:
\lsts{src/7/7_2_5.rkt}{rkt}
This evaluates to:
\lsts{src/7/7_2_6.rkt}{rkt}
Hmm, that’s interesting.
\DoNow{
What’s interesting?
}

To see what’s interesting, let’s apply this once more:
\lsts{src/7/7_2_7.rkt}{rkt}
This produces an error indicating that the identifier representing x isn’t
bound!

But it’s bound by the function named f1, isn’t it? For clarity, let’s switch to
representing it in our hypothetical Racket syntax:
\lsts{src/7/7_2_8.rkt}{rkt}
On applying the outer function, we would expect x to be substituted with 5, resulting
in
\lsts{src/7/7_2_9.rkt}{rkt}
which on further application and substitution yields (+ 5 4) or 9, not an error.

In other words, we’re again failing to faithfully capture what substitution
would have done. A function value needs to remember the substitutions that have
already been applied to it. Because we’re representing substitutions using an
environment, a function value therefore needs to be bundled with an environment.
This resulting data structure is called a closure.
\note{On the other hand, observe that with substitution, as we’ve defined it, we
would be replacing x with (numV 4), resulting in a function body of (plusC (numV
5) (idC 'y)), which does not type. That is, substitution is predicated on the
assumption that the type of answers is a form of syntax. It is actually possible
to carry through a study of even very advanced programming constructs under this
assumption, but we won’t take that path here.}

While we’re at it, observe that the appC case above uses funV-arg and funV-
body, but not funV-name. Come to think of it, why did a function need a name? so
that we could find it. But if we’re using the interpreter to find the function for us, then
there’s nothing to find and fetch. Thus the name is merely descriptive, and might as
well be a comment. In other words, a function no more needs a name than any other
immediate constant: we don’t name every use of 3, for instance, so why should we
name every use of a function? A function is inherently anonymous, and we should
separate its definition from its naming.

(But, you might say, this argument only makes sense if functions are always written
in-place. What if we want to put them somewhere else? Won’t they need names then?
They will, and we’ll return to this (section 7.5).)


\secrel{7.3 Implementing Closures   . 37}

\secrel{7.4 Substitution, Again   38}

We have seen that substitution is instructive in thinking through how to
implement lambda functions. However, we have to be careful with substitution
itself! Suppose we have the following expression (to give lambda functions their
proper Racket syntax):
\lsts{src/7/7_4_1.rkt}{rkt}
Now suppose we substitute for f the following expression: (lambda (y) (+ x y)).
Observe that it has a free identifier (x), so if it is ever evaluated, we would
expect to get an unbound identifier error. Substitution would appear to give:
\lsts{src/7/7_4_2.rkt}{rkt}
But observe that this latter program has no free identifiers!

That’s because we have too naive a version of substitution. To prevent
unexpected behavior like this (which is a form of dynamic binding), we need to
define capturefree substitution. It works roughly as follows: we first
consistently rename all bound identifiers to entirely previously unused (known
as fresh) names. Imagine that we give each identifier a numeric suffix to attain
freshness. Then the original expression becomes
\lsts{src/7/7_4_3.rkt}{rkt}
(Observe that we renamed f to f1 in both the binding and bound locations.) Now let’s
do the same with the expression we’re substituting:
\lsts{src/7/7_4_4.rkt}{rkt}
Now let’s substitute for f1:
\note{Why didn’t we rename x? Because x may be a reference to a top-level
binding, which should then also be renamed. This is simply another application
of the consistent renaming principle. In the current setting, the distinction is
irrelevant.}
\lsts{src/7/7_4_5.rkt}{rkt}
...and x is still free! This is a good form of substitution.

But one moment. What happens if we try the same example in our environmentbased
interpreter?
\DoNow{
Try it out.
}

Observe that it works correctly: it reports an unbound identifier error.
Environments automatically implement capture-free substitution!
\Exercise{
In what way does using an environment avoid the capture problem of substitution?
}

\secrel{7.5 Sugaring Over Anonymity  . . 39}

\secup

% \secrel{8 Mutation: Structures and Variables 41}\secdown

It’s time for another

\bigskip
\textbf{Which of these is the same?}
\begin{itemize}
  \item \verb|f = 3|
  \item \verb|o.f = 3|
  \item \verb|f = 3|
\end{itemize}

Assuming all three are in Java, the first and third could behave exactly like
each other or exactly like the second: it all depends on whether f is a local
identifier (such as a parameter) or a field of the object (i.e., the code is
really this.f = 3).

In either case, we are asking the evaluator to permanently change the value
bound to f. This has important implications for other observers. Until now, for
a given set of inputs, a computation always returned the same value. Now, the
answer depends on when it was invoked: above, it depends on whether it was
invoked before or after the value of f was changed. The introduction of time has
profound effects on reasoning about programs.

However, there are really two quite different notions of change buried in the
uniform syntax above. Changing the value of a field (o.f = 3 or this.f = 3) is
extremely different from changing that of an identifier (f = 3 where f is bound
inside the method, not by the object). We will explore these in turn. We’ll
tackle fields below, and return to identifiers in section \ref{8_2_vars}.

\secrel{8.1 Mutable Structures   41}
\secdown
\input{8_1_1_model}
\input{8_1_2_scaffold}
\secrel{8.1.3 Interaction with Closures  . . 43}

Consider a simple counter:
\lsts{src/8/1/3_1.rkt}{rkt}
Every time it is invoked, it produces the next integer:
\lst{src/8/1/3_1.log}
Why does this work? It’s because the box is created only once, and bound to n,
and then closed over. All subsequent mutations affect the same box. In contrast,
swapping two lines makes a big difference:
\lsts{src/8/1/3_2.rkt}{rkt}
Observe:
\lst{src/8/1/3_2.log}
In this case, a new box is allocated on every invocation of the function, so the
answer each time is the same (despite the mutation inside the procedure). Our
implementation of boxes should be certain to preserve this distinction.

The examples above hint at an implementation necessity. Clearly, whatever the
environment closes over in new-loc must refer to the same box each time. Yet
something also needs to make sure that the value in that box is different each
time! Look at it more carefully: it must be lexically the same, but dynamically
different. This distinction will be at the heart of our implementation.

\secrel{8.1.4 Understanding the Interpretation of Boxes . . 44}

Let’s begin by reproducing our current interpreter:
\lst{src/8/1/4_1.rkt}
Because we’ve introduced a new kind of value, the box, we have to update the set
of values:
\lst{src/8/1/4_2.rkt}
Two of these cases should be easy. When we’re given a box expression, we simply
evaluate it and return it wrapped in a boxV:
\lst{src/8/1/4_3.rkt}
Similarly, extracting a value from a box is easy:
\lst{src/8/1/4_4.rkt}

By now, you should be constructing a healthy set of test cases to make sure
these behave as you’d expect.

Of course, we haven’t done any hard work yet. All the interesting behavior is,
presumably, hidden in the treatment of setboxC. It may therefore surprise you
that we’re going to look at seqC first instead (and you’ll see why we included
it in the core).

Let’s take the most natural implementation of a sequence of two instructions:
\lst{src/8/1/4_5.rkt}

That is, we evaluate the first term, then the second, and return the result of
the second.

You should immediately spot something troubling. We bound the result of
evaluating the first term, but didn’t subsequently do anything with it. That’s
okay: presumably the first term contained a mutation expression of some sort,
and its value is uninteresting (indeed, note that set-box! returns a void
value). Thus, another implementation might be this:
\lst{src/8/1/4_6.rkt}

Not only is this slightly dissatisfying in that it just uses the analogous
Racket sequencing construct, it still can’t possibly be right! This can only
work only if the result of the mutation is being stored somewhere. But because
our interpreter only computes values, and does not perform any mutation itself,
any mutations in (interp b1 env) are completely lost. This is obviously not what
we want.

\input{8_1_5_envhelp}
\secrel{8.1.6 Introducing the Store  . 48}
\secrel{8.1.7 Interpreting Boxes  . . 49}
\secrel{8.1.8 The Bigger Picture  . . 54}
\secup

\secrel{8.2 Variables   . . 57}\label{8_2_vars}

Now that we’ve got structure mutation worked out, let’s consider the other case:
variable mutation.

\secdown
\secrel{8.2.1 Terminology   . . 57}

First, our choice of terms. We’ve insisted on using the word “identifier” before because
we wanted to reserve “variable” for what we’re about to study. In Java, when we say
(assuming x is locally bound, e.g., as a method parameter)
\lsts{src/8/2/1.rkt}{java}
we’re asking to change the value of x. After the first assignment, the value of
x is 1; after the second one, it’s 3. Thus, the value of x varies over the
course of the execution of the method.

Now, we also use the term “variable” in mathematics to refer to function
parameters. For instance, in f(y) = y + 3 we say that y is a “variable”. That is
called a variable because it varies across invocations; however, within each
invocation, it has the same value in its scope. Our identifiers until now have
corresponded to this notion of a variable. In contrast, programming variables
can vary even within each invocation, like the Java x above.
\note{If the identifier was bound to a box, then it remained bound to the same
box value. It’s the content of the box that changed, not which box the
identifier was bound to.}

Henceforth, we will use variable when we mean an identifier whose value can
change within its scope, and identifier when this cannot happen. If in doubt, we
might play it safe and use “variable”; if the difference doesn’t really matter,
we might use either one. It is less important to get caught up in these specific
terms than to understand that they represent a distinction that matters \ref{}.

\secrel{8.2.2 Syntax   . . 57}

\secrel{8.2.3 Interpreting Variables  . 58}

\secup

\secrel{8.3 The Design of Stateful Language Operations  . . 59}

Though most programming languages include one or both kinds of state we have
studied, their admission should not be regarded as a trivial or foregone matter.
On the one hand, state brings some vital benefits:

\begin{itemize}
  \item 
State provides a form of modularity. As our very interpreter demonstrates,
without explicit stateful operations, to achieve the same effect:
\begin{itemize}
  \item 
We would need to add explicit parameters and return values that pass the
equivalent of the store around.
  \item
These changes would have to be made to all procedures that may be involved in a
communication path between producers and consumers of state.
\end{itemize}

Thus, a different way to think of state in a programming language is that it is
an implicit parameter already passed to and returned from all procedures,
without imposing that burden on the programmer. This enables procedures to
communicate “at a distance” without all the intermediaries having to be aware of
the communication.

  \item
State makes it possible to construct dynamic, cyclic data structures, or at
least to do so in a relatively straightforward manner \ref{sec9}.

  \item
State gives procedures memory, such as new-loc above. If a procedure could not
remember things for itself, the callers would need to perform the remembering on
its behalf, employing the moral equivalent of store-passing. This is not only
unwieldy, it creates the potential for a caller to interfere with the memory for
its own nefarious purposes (e.g., a caller might purposely send back an old
store, thereby obtaining a reference already granted to some other party,
through which it might launch a correctness or security attack).

\end{itemize}

On the other hand, state imposes real costs on programmers as well as on
programs that process programs (such as compilers). One is “aliasing”, which we
discuss later \ref{}. Another is “referential transparency”, which too I
hope to return to \ref{}. Finally, we have described above how state provides a
form of modularity. However, this same description could be viewed as that of a
back-channel of communication that the intermediaries did not know and could not
monitor. In some (especially security and distributed system) settings, such
back-channels can lead to collusion, and can hence be extremely dangerous and
undesirable.

Because there is no optimal answer, it is probably wise to include mutation
operators but to carefully delinate them. In Standard ML, for instance, there is
no variable mutation, because it is considered unnecessary. Instead, the
language has the equivalent of boxes (called refs). One can easily simulate
variables using boxes (e.g., see new-loc and consider how it would be written
with variables instead), so no expressive power is lost, though it does create
more potential for aliasing than variables alone would have (aliasing
\ref{aliasing}) if the boxes are not used carefully.

In return, however, developers obtain expressive types: every data structure is
considered immutable unless it contains a ref, and the presence of a ref is a
warning to both developers and programs (such as compilers) that the underlying
value may keep changing. Thus, for instance, if b is a box, a developer should
be aware that replacing all instances of (unbox b) with v, where v is bound to
(unbox b), is unwise: the former always fetches the current value in the box,
while the latter may be referring to an older content. (Conversely, if the
developer wants the value at a certain point in time, oblivious to future
mutations to the box, they should be sure to retrieve and bind it rather than
always use unbox.)

\secrel{8.4 Parameter Passing   . 60}

\secup

% \secrel{9 Recursion and Cycles:\\Procedures and Data 62}\label{sec9}\secdown

Recursion is the act of self-reference. When we speak of recursion in
programming languages, we may have one of (at least) two meanings in mind:
recursion in data, and recursion in control (i.e., of program behavior—that is
to say, of functions).

\secrel{9.1 Recursive and Cyclic Data  . . 62}

\secrel{9.2 Recursive Functions   64}

\secrel{9.3 Premature Observation   . 65}

\secrel{9.4 Without Explicit State   . . 66}

\secup

% 
% \input{10_objects}
% \secrel{11 Memory Management 81}\secdown
\secrel{11.1 Garbage   81}

We use the term garbage to refer to allocated memory that is no longer
necessary. There are two distinct kinds of allocations that a typical
programming language runtime system performs. One kind is for the environment;
this follows a push-and-pop discipline consistent with the nature of static
scope. Returning from a procedure returns that procedure’s allocated environment
space for subsequent use, seemingly free of cost. In contrast, allocation on the
store has to follow an value’s lifetime, which could outlive that of the scope
in which it was created—indeed, it may live forever. Therefore, we need a
different strategy for recovering space consumed by store-allocated garbage.
\note{It’s not free! The machine has to execute an explicit “pop” instruction to
recover that space. As a result, it is not necessarily cheaper than other memory
management strategies.}

There are many methods for recovering this space, but they largely fall into two
camps: manual and automatic. Manual collection depends on the developer being
able to know and correctly discard unwated memory. Traditionally, humans have
not proven especially good at this (though in some cases they have knowledge a
machine might not \ref{}). Over several decades, therefore, automated methods
have become nearly ubiquitous.

\secrel{11.2 What is “Correct” Garbage Recovery?  . . 81}

\secrel{11.3 Manual Reclamation   . . 82}\secdown

The most manual approach would be to entrust all de-allocation to the human. In
C, for instance, there are two basic primitives: malloc for allocation and free
to reclaim. malloc consumes a size and returns a reference to a store-allocated
value; free consumes the references and reclaims its assocated memory.
\note{“Moloch has been used figuratively in English literature from John
Milton’s Paradise Lost (1667) to Allen Ginsberg’s ‘Howl’ (1955), to refer to a
person or thing demanding or requiring a very costly sacrifice.”\
--— \href{https://en.wikipedia.org/wiki/Moloch}{Wikipedia on Moloch}}
\note{“I do not consider it coincidental that this name sounds like malloc.”
—Ian Barland}

\secrel{11.3.1 The Cost of Fully-Manual Reclamation  82}

\secrel{11.3.2 Reference Counting  . 83}

\secup

\secrel{11.4 Automated Reclamation, or Garbage Collection  84}
\secdown
\secrel{11.4.1 Overview   . 84}

\secrel{11.4.2 Truth and Provability  . 85}

\secrel{11.4.3 Central Assumptions  . 85}

\secup

\secrel{11.5 Convervative Garbage Collection  . . 86}

\secrel{11.6 Precise Garbage Collection  . . 87}

\secup

% \secrel{12 Representation Decisions 87}\secdown

Go back and look again at our interpreter for function as values \ref{}. Do you
see something curiously non-uniform about it?

\DoNow{
No, really, do. Do you?
}

Consider how we chose to represent our two different kinds of values: numbers
and functions. Ignoring the superficial numV and closV wrappers, focus on the
underlying data representations. We represented the interpreted language’s
numbers as Racket numbers, but we did not represent the interpreted language’s
functions (closures) as Racket functions (closures).

That’s our non-uniformity. It would have been more uniform to use Racket’s
representations for both, or also to not use Racket’s representation for either.
So why did we make this particular choice?

We were trying to illustrate and point, and that point is what we will explore
right now.

\secrel{12.1 Changing Representations  . . 87}

For a moment, let’s explore numbers. Racket’s numbers make a good target for
reuse because they are so powerful: we get arbitrary-sized integers (bignums),
rationals (which benefit from the bignum representation of integers), complex
numbers, and so on. Therefore, they can represent most ordinary programming
language number systems. However, that doesn’t mean they are what we want: they
could be too little or too much.
\begin{itemize}
  \item 
They are too much if what we want is a more restricted number system. For
instance, Java prescribes a very specific set of fixed-size representations
(e.g., int is specified to be 32-bit). Numbers that fall outside these sets
cannot be directly represented as numbers, and arithmetic must respect these
sets (e.g., overflowing so that adding 1 to 2147483647 does not produce
2147483648).
  \item 
They are too little if we want even richer numbers, whether quaternions or
numbers with associated probabilities.
\end{itemize}
Worse, we didn’t even stop and ask what we wanted, but blithely proceeded with
Racket numbers.

The reason we did so is because we weren’t really interested in the study of
numbers; rather, we were interested in programming language features such as
functions-asvalues. As language designers, however, you should be sure to ask
these hard questions up front.

Now let’s talk about our representation of closures. We could have instead
represented closures by exploiting Racket’s corresponding concept, and
correspondingly, function application with unvarnished Racket application.

\DoNow{
Replace the closure data structure with Racket functions representing
functions-as-values.
}

Here we go:
\lsts{src/12/1/1.rkt}{rkt}

\Exercise{
Observe a curious shift. In our previous implementation, the environment was
extended in the appC case. Here, it’s extended in the lamC case. Is one of these
incorrect? If not, why did this change occur?
}

This is certainly concise, but we’ve lost something very important:
understanding. Saying that a source language function corresponds to lambda
tells us virtually nothing: if we already knew precisely what lambda does we
might not be studying it, and if we didn’t, this mapping would teach us
absolutely nothing (and might, in fact, pile confusion on top of our ignorance).
For the same reason, we did not use Racket’s state to understand the varieties
of stateful operators \ref{}.

Once we’ve understood the feature, however, we should feel to use it as a
representation. Indeed, doing so might yield much more concise interpreters
because we aren’t doing everything manually. In fact, some later interpreters
\ref{}\ will become virtually unreadable if we did not exploit these richer
representations. Nevertheless, exploiting host language features has perils that
we should safeguard against.
\note{It’s a little like saying, “Now that we understand addition in terms of
increment-by-one, we can use addition to define multiplication: we don’t have to
use only increment-by-one to define it.”}

\secrel{12.2 Errors   . 89}

When programs go wrong, programmers need a careful presentation of errors. Using
host language features runs the risk that users will see host language errors,
which they will not understand. Therefore, we have to carefully translate error
conditions into terms that the user of our language will understand, without
letting the host language “leak through”.

Worse, programs that should error might not! For instance, suppose we decide
that functions should only appear in top-level positions. If we fail to
expressly check for this, desugaring into the more permissive lambda may result
in an interpreter that produces answers where it should have halted with an
error. Therefore, we have to take great care to permit only the intended surface
language to be mapped to the host language.

As another example, consider the different mutation operations. In our language,
attempting to mutate an unbound variable produces an error. In some languages,
doing so results in the variable being defined. Failing to pin down our intended
semantics is a common language designer error, saying instead, “It is whatever
the implementation does”. This attitude (a) is lazy and sloppy, (b) may yield
unexpected and negative consequences, and (c) makes it hard for you to move your
language from one implementation platform to another. Don’t ever make this
mistake!

\secrel{12.3 Changing Meaning   89}

Mapping functions-as-values to lambda works especially because we intend for the
two to have the same meaning. However, this makes it difficult to change the
meaning of what a function does. Lemme give ya’ a hypothetic: suppose we wanted
our language to implement dynamic scope. In our original interpreter, this was
easy (almost too easy, as history shows). But try to make the interpreter that
uses lambda implement dynamic scope. It can similarly be difficult or at least
subtle to map eager evaluation onto a language with lazy application \ref{}.
\note{Don’t let this go past the hypothetical stage, please.}

\Exercise{
Convert the above interpreter to use dynamic scope.
}

The point is that the raw data structure representation does not make anything
especially easy, but it usually doesn’t get in the way, either. In contrast,
mapping to host language features can make some intents—mainly, those match what
the host language already does!—especially easy, and others subtle or difficult.
There is the added danger that we may not be certain of what the host language’s
feature does (e.g., does its “lambda” actually implement static scope?).

The moral is that this is a good property to exploit only we want to “pass
through” the base language’s meaning—-and then it is especially wise because it
ensures that we don’t accidentally change its meaning. If, however, we want to
exploit a significant part of the base language and only augment its meaning,
perhaps other implementation strategies \ref{}\ will work just as well instead
of writing an interpreter.

\secrel{12.4 One More Example   90}

\secup

% \secrel{13 Desugaring as a Language Feature 91}\secdown

We have thus far extensively discussed and relied upon desugaring, but our
current desguaring mechanism have been weak. We have actually used desugaring in
two different ways. One, we have used it to shrink the language: to take a large
language and distill it down to its core \ref{}. But we have also used it to
grow the language: to take an existing language and add new features to it
\ref{}. This just shows that desugaring is a tremendously useful feature to
have. Indeed, it is so useful that we might ask two questions:
\begin{itemize}
  \item 
Because we create languages to simplify the creation of common tasks, what would
a language designed to support desugaring look like? Note that by “look” we
don’t mean only syntax but also its key behavioral properties.
  \item 
Given that general-purpose languages are often used as a target for desugaring,
why don’t they offer desugaring capabilities in the language itself ? For
instance, this might mean extending a base language with the additional language
that is the response to the previous question.
\end{itemize}
We are going to explore the answer to both questions simultaneously, by studying
the facilities provided for this by \racket.

\secrel{13.1 A First Example   . . 91}

\secrel{13.2 Syntax Transformers as Functions  . 93}

\secrel{13.3 Guards   . 95}

Now we can return to the problem that originally motivated the introduction of
syntax- case: ensuring that the binding position of a my-let-3 is syntactically
an identifier. For this, you need to know one new feature of syntax-case: each
rewriting rule can have two parts (as above), or three. If there are three
present, the middle one is treated as a guard: a predicate that must evaluate to
true for expansion to proceed rather than signal a syntax error. Especially
useful in this context is the predicate identifier?, which determines whether a
syntax object is syntactically an identifier (or variable).

\DoNow{
Write the guard and rewrite the rule to incorporate it.
}

Hopefully you stumbled on a subtlety: the argument to identifier? is of type
syntax. It needs to refer to the actual fragment of syntax bound to var. Recall
that var is bound in the syntax space, and \#' substitutes identifiers bound
there. Therefore, the correct way to write the guard is:
\lsts{src/13/3/1.rkt}{rkt}
With this information, we can now write the entire rule:
\lsts{src/13/3/2.rkt}{rkt}

\DoNow{
Now that you have a guarded rule definition, try to use the macro with a
non-identifier in the binding position and see what happens.
}

\secrel{13.4 Or: A Simple Macro with Many Features  95}\secdown

Consider or, which implements disjunction. It is natural, with prefix syntax, to
allow or to have an arbitrary number of sub-terms. We expand or into nested
conditionals that determine the truth of the expression.

\secrel{13.4.1 A First Attempt   . 95}

\secrel{13.4.2 Guarding Evaluation  . 97}

\secrel{13.4.3 Hygiene   . . 98}

\secup

\secrel{13.5 Identifier Capture   . 99}

Hygienic macros address a routine and important pain that creators of syntactic
sugar confront. On rare instances, however, a developer wants to intentionally
break hygiene. Returning to objects, consider this input program:
\lsts{src/13/5/1.rkt}{rkt}
What does the macro look like? Here’s an obvious candidate:
\lsts{src/13/5/2.rkt}{rkt}
Unfortunately, this macro produces the following error:
\lst{src/13/5/2.err}
which is referring to the self in the body of the method bound to first.

\Exercise{
Work through the hygienic expansion process to understand why error is the
expected outcome.
}

Before we jump to the richer macro, let’s consider a variant of the input term
that makes the binding explicit:
\lsts{src/13/5/3.rkt}{rkt}
The corresponding macro is a small variation on what we had before:
\lsts{src/13/5/4.rkt}{rkt}
This macro expands without difficulty.

\Exercise{
Work through the expansion of this version and see what’s different.
}

This offers a critical insight: had the identifier that goes in the binding
position been written by the macro user, there would have been no problem.
Therefore, we want to be able to pretend that the introduced identifier was
written by the user. The function datum->syntax converts the s-expression in its
second argument; its first argument is which syntax to pretend it was a part of
(in our case, the original macro use, which is bound to x). To introduce the
result into the environment used for expansion, we use with-syntax to bind it in
that environment:
\lsts{src/13/5/5.rkt}{rkt}

With this, we can go back to having self be implicit:
\lsts{src/13/5/6.rkt}{rkt}

\secrel{13.6 Influence on Compiler Design  101}

\secrel{13.7 Desugaring in Other Languages  . . 101}

\secup

\secrel{14 Control Operations 102}\secdown

The term control refers to any programming language instruction that causes
evaluation to proceed, because it “controls” the program counter of the machine.
In that sense, even a simple arithmetic expression should qualify as “control”,
and operations such as sequential program execution, or function calls and
returns, most certainly do. However, in practice we use the term to refer
primarily to those operations that cause non-local transfer of control,
especially beyond that of mere functions and procedures, and the next step up,
namely exceptions. We will study such operations in this chapter.

As we study the following control operators, it’s worth remembering that even
without them, we still have languages that are Turing-complete, and therefore
have no more “power”. Therefore, what control operators do is change and
potentially improve the way we express our intent, and therefore enhance the
structure of programs. Thus, it pays to being our study by focusing on program
structure.

\secrel{14.1 Control on the Web   102}

Let us begin our study by examining the structure of Web programs. Consider the
following program:
\note{ Henceforth, we’ll call this our “addition server”.
You should, of course, understand this as a stand-in for more sophisticated
applications. For instance, the two prompts might ask for starting and ending
points for a trip, and in place of addition we might compute a route or compute
airfares. There might even be computation between the two steps: e.g., after
entering the first city, the airline might prompt us with choices of where it
flies from there.}
\lsts{src/14/1/1.rkt}{rkt}
To test these ideas, here’s an implementation of read-number:
\lsts{src/14/1/2.rkt}{rkt}
When run at the console or in DrRacket, this program prompts us for one number,
then another, and then displays their sum.

Now suppose we want to run this on a Web server. We immediately encounter a
difficulty: the structure of server-side Web programs is such that they generate
a single Web page—such as the one asking for the first number—and then halt. As
a result, the rest of the program—which in this case prompts for the second
number, then adds them, and then prints that result, is lost.

\DoNow{
Why do Web servers behave in such a strange way?
}

There are at least two reasons for this behavior: one perhaps historical, and
the other technical. The historical reason is that Web servers were initially
designed to serve pages, i.e., static content. Any program that ran had to
generate its output to a file, from which a server could offer it. Naturally,
developers wondered why that same program couldn’t run on demand. This made Web
content dynamic. Terminating the program after generating a single piece of
output was the simplest incremental step towards programs, not pages, on the
Web.

The more important reason—and the one that has stayed with us—is technical.
Imagine our addition server has generated its first prompt. Recall that there is
considerable pending computation: the second prompt, the addition, and the
display of the result. This computation must suspend waiting for the user’s
input. If there are millions of users, then millions of computations must be
suspended, creating an enormous performance problem. Furthermore, suppose a user
does not actually complete the computation—analogous to searching at an on-line
bookstore or airline site, but not completing the purchase. How does the server
know when or even whether to terminate the computation? And until it does, the
resources associated with that computation remain in use.

Conceptually, therefore, the Web protocol was designed to be stateless: it would
not store state on the server associated with intermediate computations.
Instead, Web program developers would be forced to maintain all necessary state
elsewhere, and each request would need to be able to resume the computation in
full. In practice, the Web has not proven to be stateless at all, but it still
hews largely in this direction, and studying the structure of such programs is
very instructive.

Now consider client-side Web programs: those that run inside the browser,
written in or compiled to JavaScript. Suppose such a computation needs to
communicate with a server. The primitive for this is called XMLHttpRequest. The
user makes an instance of this primitive and invokes its send method to send a
message to the server. Communicating with a server is not, however,
instantaneous (and indeed may never complete, depending on the state of the
network). This leaves the sending process suspended.

The designers of JavaScript decided to make the language single-threaded: i.e.,
there would be only one thread of execution at a time. This avoids the various
perils that arise from combining mutation with threads. As a result, however,
the JavaScript process locks up awaiting the response, and nothing else can
happen: e.g., other handlers on the page no longer respond.
\note{Due to the structuring problems this causes, there are now various
proposals to, in effect, add “safe” threads to JavaScript. The ideas described
in this chapter can be viewed as an alternative that offer similar structuring
benefits.}

To avoid this problem, the design of XMLHttpRequest demands that the developer
provide a procedure that responds to the request if and when it arrives. This
callback procedure is registered with the system. It needs to embody the rest of
the processing of that request. Thus, for entirely different reasons—not
performance, but avoiding the problems of synchronization, non-atomicity, and
deadlocks—the client-side Web has evolved to demand the same pattern of
developers. Let us now better understand that pattern.

\secdown
\secrel{14.1.1 Program Decomposition into Now and Later 104}

Let us consider what it takes to make our above program work in a stateless
setting, such as on a Web server. First we have to determine the first
interaction. This is the prompt for the first number, because Racket evaluates
arguments from left to right. It is instructive to divide the program into two
parts: what happens to generate the first interaction (which can all run now),
and what needs to happen after it (which must be “remembered” somehow). The
former is easy:
\lsts{src/14/1/1/1.rkt}{rkt}
We’ve already explained in prose what’s left, but now it’s time to write it as a
program. It seems to be something like
\note{We’re intentionally
ignoring
read-number for
now, but we’ll
return to it. For
now, let’s pretend
it’s built-in.}
\lsts{src/14/1/1/2.rkt}{rkt}
A Web server can’t execute the above, however, because it evidently isn’t a
program. We instead need some way of writing this as one.

Let’s observe a few characteristics of this computation:
\begin{itemize}[nosep]
  \item 
It needs to be a legitimate program.
  \item 
It needs to stay suspended until the request comes in.
  \item 
It needs a way—such as a parameter—to refer to the value from the first
interaction.
\end{itemize}

Put together these characteristics and we have a clear representation—a
function:
\lsts{src/14/1/1/3.rkt}{rkt}

\secrel{14.1.2 A Partial Solution   104}

\secrel{14.1.3 Achieving Statelessness  . . 106}

\secrel{14.1.4 Interaction with State  . 107}

\secup

\secrel{14.2 Continuation-Passing Style  . . 109}

The functions we’ve been writing have a name. Though we’ve presented ideas in
terms of theWeb, we’re relying on a much older idea: the functions are called
continuations, and this style of programs is called continuation-passing style
(CPS). This is worth studying in its own right, because it is the basis for
studying a variety of other nontrivial control operations—such as generators.
\note{We will take the liberty of using CPS as both a noun and verb: a
particular structure of code and the process that converts code into it.}

Earlier, we converted programs so that no Web input operation was nested inside
another. The motivation was simple: when the program terminates, all nested
computations are lost. A similar argument applies, in a more local sense, in the
case of XMLHttpRequest: any computation depending on the result of a response
from aWeb server needs to reside in the callback associated with the request to
the server.

In fact, we don’t need to transform every expression. We only care about
expressions that involve actual Web interaction. For example, if we computed a
more complex mathematical expression than just addition, we wouldn’t need to
transform it. If, however, we had a function call, we’d either have to be
absolutely certain the function didn’t have any Web invocations either inside
it, or in the functions in invokes, or the ones they invoke...or else, to be
defensive, we should transform them all. Therefore, we have to transform every
expression that we can’t be sure performs noWeb interactions.

The heart of our transformation is therefore to turn every one-argument
function, f, into one with an extra argument. This extra argument is the
continuation, which represents the rest of the computation. The continuation is
itself a function of one argument. This argument takes the value that would have
been returned by f and passes it to the rest of the computation. f, instead of
returning a value, instead passes the value it would have returned to its
continuation.

CPS is a general transformation, which we can apply to any program. Because it’s
a program transformation, we can think of it as a special kind of desugaring: in
particular, instead of transforming programs from a larger language to a smaller
one (as macros do), or from one language to entirely another (as compilers do),
it transforms programs within the same language: from the full language to a
more restricted version that obeys the pattern we’ve been discussing. As a
result, we can reuse an evaluator for the full language to also evaluate
programs in the CPS subset.

\secdown
\secrel{14.2.1 Implementation by Desugaring  . . 110}

\secrel{14.2.2 Converting the Example  . . 114}

\secrel{14.2.3 Implementation in the Core  115}

\secup

\input{14_3_gens}
\secrel{14.4 Continuations and Stacks   121}

\secrel{14.5 Tail Calls   . . 123}

\secrel{14.6 Continuations as a Language Feature  124}
\secdown
\secrel{14.6.1 Presentation in the Language  125}
\secrel{14.6.2 Defining Generators  . 126}
\secrel{14.6.3 Defining Threads   127}
\secrel{14.6.4 Better Primitives for Web Programming  131}
\secup

\secup


\secrel{15 Checking Program Invariants Statically: Types 131}\secdown
\secrel{15.1 Types as Static Disciplines  . . 133}
\secrel{15.2 A Classical View of Types  . . 134}
\secdown
\secrel{15.2.1 A Simple Type Checker  . . 134}
\secrel{15.2.2 Type-Checking Conditionals  139}
\secrel{15.2.3 Recursion in Code  . . 139}
\secrel{15.2.4 Recursion in Data   142}
\secrel{15.2.5 Types, Time, and Space  . . 144}
\secrel{15.2.6 Types and Mutation  . . 146}
\secrel{15.2.7 The Central Theorem: Type Soundness  147}
\secup
\secrel{15.3 Extensions to the Core   . 148}
\secdown
\secrel{15.3.1 Explicit Parametric Polymorphism  148}
\secrel{15.3.2 Type Inference   . 155}
\secrel{15.3.3 Union Types   . . 164}
\secrel{15.3.4 Nominal Versus Structural Systems  . . 170}
\secrel{15.3.5 Intersection Types  . . 171}
\secrel{15.3.6 Recursive Types   172}
\secrel{15.3.7 Subtyping   . 173}
\secrel{15.3.8 Object Types   . . 176}
\secup
\secup

\secrel{16 Checking Program Invariants Dynamically: Contracts 179}\secdown
\secrel{16.1 Contracts as Predicates   . 181}
\secrel{16.2 Tags, Types, and Observations on Values  . 182}
\secrel{16.3 Higher-Order Contracts   . 183}
\secrel{16.4 Syntactic Convenience   . 187}
\secrel{16.5 Extending to Compound Data Structures  . 188}
\secrel{16.6 More on Contracts and Observations  189}
\secrel{16.7 Contracts and Mutation   . 189}
\secrel{16.8 Combining Contracts   . . 190}
\secrel{16.9 Blame   . 191}
\secup

\secrel{17 Alternate Application Semantics 195}\secdown
\secrel{17.1 Lazy Application . . . . . . . . . . . . . . . . . . . . . . . . . . . . 196}
\secdown
\secrel{17.1.1 A Lazy Application Example . . . . . . . . . . . . . . . . . . 196}
\secrel{17.1.2 What Are Values? . . . . . . . . . . . . . . . . . . . . . . . 197}
\secrel{17.1.3 What Causes Evaluation? . . . . . . . . . . . . . . . . . . . 198}
\secrel{17.1.4 An Interpreter . . . . . . . . . . . . . . . . . . . . . . . . . . 199}
\secrel{17.1.5 Laziness and Mutation . . . . . . . . . . . . . . . . . . . . . 201}
\secrel{17.1.6 Caching Computation . . . . . . . . . . . . . . . . . . . . . 201}
\secup
\secrel{17.2 Reactive Application . . . . . . . . . . . . . . . . . . . . . . . . . . 201}
\secdown
\secrel{17.2.1 Motivating Example: A Timer . . . . . . . . . . . . . . . . . 202}
\secrel{17.2.2 Callback Types are Four-Letter Words . . . . . . . . . . . . . 203}
\secrel{17.2.3 The Alternative: Reactive Languages . . . . . . . . . . . . . 204}
\secrel{17.2.4 Implementing Transparent Reactivity . . . . . . . . . . . . . 205}
\secup
\secup



\clearpage
\addcontentsline{toc}{chapter}{Index \ru{Предметный указатель}}\printindex

\end{document}
