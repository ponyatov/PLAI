\input{../texheader/ebook}

\usepackage{../texheader/lstrkt}\lstdefinestyle{rkt}{language=rkt}

\newcommand{\Exercise}[1]{
	\begin{description}
		\item{\textcolor{red}{Упражнение}}\\#1
	\end{description}
}

\newcommand{\DoNow}[1]{
	\begin{description}
		\item{\textcolor{red}{Сделайте\,!}}\\#1
	\end{description}
}

\title{{\Huge{PLAI}}\\
Programming Languages: Application and Interpretation\\
{\small{second edition}}\\
{\Huge{языки программирования}}\\{\Huge{применение и реализация}}}

\author{\copyright\ Шрирам Кришнамурти\\
перевод Dmitry Ponyatov \email{dponyatov@gmail.com}}

\begin{document}
\maketitle
\tableofcontents
\secdown

\secrel{Introduction \ru{Введение}}\secdown
\secrel{Our Philosophy \ru{Наша философия}}

Please watch the video on \ru{Пожалуйста посмотрите это видео на}
\href{https://www.youtube.com/watch?v=3N__tvmZrzc}{YouTube}.

Someday there will be a textual description here instead.
\ru{Когда нибудь здесь будет полное текстовое описание того, о чем в нем
говориться.}

\secrel{The Structure of This Book \ru{Структура книги}}

В отличие от большинства учебников, эта книга не следует подходу "сверху вниз".
Скорее она имеет форму повествования с возвратами к предыдущим темам.
Мы часто будет строить программы инкрементно, так же как мы бы это делали в
парном программировании. Наш код будет включать ошибки, не потому что мы не
знаем правильный ответ, но потому что это лучший способ научить вас, 
углубляясь от поверхностного в детали. Включение намеренных ошибок делает
невозможным для вас читать материал пассивно: вы должны взаимодействовать с
ним, потому что вы никогда не можете быть уверены в правильности того что вы
читаете.

В конце концов вы всегда получите правильный ответ. Тем не меннее, это
нелинейное повествование немного раздражающе в краткосрочной перспективе (у вас
всегда будет соблазн сказать "Скажите же мне наконец ответ !"), и это также
делает эту книгу плохим справочником (вы не можете открыть произвольную
страницу и быть уверенным что на ней написана правда). Тем не менее, это чувство
разочарования\ --- ощущение обучения. Я не знаю другого способа.

\bigskip
В некоторых местах вы встретите следующие выделения:

\Exercise{Это упражнение. Попробуйте это сделать.}

Это традиционное для учебников упражнение.
Это то, что вам нужно сделать по своему усмотрению.
Если вы используете эту книгу как часть курса, это упражнение хорошо
задавать как домашнюю работу. В противоположность этому вы также можете найти
подобные вопросы, выделенные как 

\DoNow{Здесь предполагаются немедленные действия. Вы видите это ?}

Когда вы доберетесь до одного из этих блоков, остановитесь. Прочитайте,
подумайте, сформулируйте ответ перед тем как продолжить чтение. Вы должны
сделать это потому что это действительно упражнение, но ответ уже есть в книге,
чаще всего в тексте непосредственно после упражнения (т.е. в части которую вы
сейчас читаете) или это что-то, что вы можете получить самостоятельно,
запустив программу. Если вы просто продолжите читать, то вы увидете ответ без
его обдумывания (или не увидите его вообще, если это инструкции по запуску
программы), так что вы ни (а) проверите свои знания, ни (б) улучшите свое
понимание. Другими словами, это дополнительные, явные попытки стимулировать
ваше активное обучение. В конце концов, я могу только поощрять вас работать;
решение применять это или нет остается за вами.


\secrel{The Language of This Book \ru{Язык программирования используемый в
книге}}

The main programming language used in this book is
\ru{Язык программирования используемый в книге}\ --- 
\href{http://www.racket-lang.org/}{\racket}.
Like with all operating systems, however,
\ru{Аналогично операционным системам,}
\racket\ actually supports a host of programming languages,
\ru{\racket-система является исполняющей средой для целого ряда языков
программирования,}
so you must tell \racket\
\ru{так что вы должны указать \racket у}
\emph{which} language you’re programming in.
\ru{\emph{на каком} языке вы программируете.}
You inform the Unix shell by writing a line like
\ru{Например, в Unix вы пишете строку типа}

\begin{verbatim}
#!/bin/sh
\end{verbatim}
at the top of a script;
\ru{в первой строке shell-скрипта;}
you inform the browser by writing, say,
\ru{вы указываете веб-браузеру тип документа, добавляя заголовок}

\begin{verbatim}
<!DOCTYPE HTML PUBLIC "-//W3C//DTD HTML 4.01//EN" ...>
\end{verbatim}

Similarly, \racket\ asks that you declare which language you will be using.
\ru{Аналогично, \racket\ требует от вас указать какой язык вы будете
использовать.}
\racket\ languages can have the same parenthetical syntax as \racket\ but with a
different semantics;
\ru{Диалекты языков \racket\ имеют тот же скобочный синтаксис, что и сам
\racket, но другую семантику;}
the same semantics but a different syntax;
\ru{ту же семантику но другой синтаксис;}
or different syntax and semantics.
\ru{или различные синтаксис и семантику.}
Thus every \racket\ program
\ru{Так что каждая программа, которую может выполнять \racket-система,}
begins with \#lang followed by the name of some language:
\ru{начинается со строки \#lang за которой следует имя диалекта языка:}
by default, it’s \racket\ \ru{по умолчанию, это оригинальный \racket\ }
written as \ru{указыватся как} \verb|racket|).
In this book we’ll almost always use the language\note{In DrRacket v.5.3,
go to Language, then Choose Language, and select ``Use the language declared in
the source''.}
\ru{В этой книге мы почти всегда будем использовать диалект}\note{\ru{В DrRacket
v.6.6, выберите меню\\\menu{Язык > Выбрать язык\ldots > Start your program with
\#lang to specify the desired dialect}.}}
\begin{verbatim}
#lang plai-typed
\end{verbatim}
When we deviate we’ll say so explicitly,
\ru{Когда мы будем отклоняться от этого правила,}
so unless indicated otherwise, put
\ru{это будет указано особо, так что если не указано иное, добавляйте заголовок}
\verb|#lang plai-typed|
at the top of every file
\ru{в начало каждого файла программы}
(and assume I’ve done the same
\ru{предполагается что я тоже это
сделал})\note{В DrRacket v.6.6 требуется установить расширение
plai-typed:\\\menu{Файл>Install package\ldots>Package
Source:>\url{github://github.com/mflatt/plai-typed/master}>Install>\ldots>Закрыть}}.

The \termdef{Typed PLAI}{Typed PLAI}\ language differs from traditional \racket\
most importantly by being statically typed.
\ru{Язык \term{Typed PLAI}\ отличается от традиционного \racket\ в основном
\emph{статической типизацией}.}
It also gives you some useful new constructs:
\ru{Он также дает вам некоторые новые полезные конструкции:}
\verb|define-type| \ru{определение-типа}, \verb|type-case| \ru{выбор-по-типу},
and \verb|test|\note{There are additional commands for controlling the output
of testing, for instance. \ru{Также существуют дополнительные команды для
управления выводом тестов.} Be sure to read the documentation for the language.
\ru{Обязательно прочитайте документацию для языка.}
In DrRacket v.5.3, go to \menu{Help>Help Desk}, and in the Help Desk search bar,
type \menu{plai-typed}. \ru{В DrRacket v.6.6 идите в меню \menu{Help>Help
Desk}, и в поле поиска \menu{Help Desk} введите \menu{plai-typed}.}}
Here’s an example of each in use.
\ru{Вот примеры использования каждого из них.} 
We can introduce new datatypes
\ru{Мы можем создавать новые типы данных\note{запустить программу можно нажав
\keys{Ctrl+R}}}:
\lst{src/1/p8_1.rkt}
You can roughly think of this as analogous to the following in Java:
\ru{Вы можете примерно понять идею в терминах языка \java:}
an abstract class \term{абстрактный класс} \verb|MisspelledAnimal| and two
concrete sub-classes \ru{и два конкретизирующих подкласса}\ \verb|caml|
\ru{верблюд} and \verb|yacc| \ru{якк},
each of which has one numeric constructor argument named
\ru{каждый из которых имеет конструктор с числовым аргументом}
\verb|humps| \ru{горбы} and \verb|height| \ru{высота}, respectively
\ru{соответственно}.

In this language, we construct instances as follows:
\ru{На этом языке мы строим экземпляры классов следующим образом:}
\lst{src/1/p8_2.rkt}
As the name suggests \ru{Как следует из названия,}, \verb|define-type| creates a
type of the given name \ru{создает тип с заданным именем}.
We can use
this when, for instance, binding the above instances to names:
\ru{Мы можем это использовать например при связывании эксземпляров с именами:}
\lst{src/1/p8_3.rkt}
In fact you don’t need these particular type declarations, because \term{Typed
PLAI} will infer types for you here and in many other cases.
\ru{Фактически вам не нужны эти частные определения типов, так как \term{Typed
PLAI} в этом и других случаях будет сам делать для вас \term{вывод типов}.}
Thus you could just as well have written
\ru{Так что вы можете написать короче}

\lst{src/1/p8_4.rkt}

\noindent
but we prefer to write explicit type declarations as a matter of both discipline
and comprehensibility when we return to programs later.
\ru{но мы предпочтем писать полные объявления типов с точки зрения как
дисциплины, так и усвояемости, когда мы вернемся к программам позже.}

The type names can even be used recursively, as we will see repeatedly in this
book (for instance, section \ref{sec2_4}).
\ru{Имена типов даже могут быть использованы рекурсивно, как мы увидим
несколько позже в этой книге (например в разделе \ref{sec2_4}).}

The language provides a pattern-matcher for use when writing expressions, such
as a function’s body:
\ru{Язык предоставляет pattern-matcher для использования при написании
выражений, таких как тело функции:}

\lst{src/1/p9_1.rkt}

\noindent
In the expression \ru{Например в выражении} (>= humps 2), for instance,
\verb|humps| is the name given to whatever value was given as the argument to
the constructor \ru{имя humps соответствует любому значению, данному как
аргумент для конструктора} \verb|caml|.

Finally, you should write test cases, ideally before you’ve defined your
function, but also afterwards to protect against accidental changes:
\ru{И наконец, вы должны написать тесты, в идеале до того как вы ее реализовали,
или хотя бы после, чтобы защититься от внезапных несоответствий в ее поведении
при внесении изменений в код:}

\lst{src/1/p9_2.rkt}

\noindent
When you run the above program, the language will give you verbose output telling
you both tests passed.
\ru{При запуске тестов язык даст вам подробный отчет, что оба теста успешно
пройдены.}
Read the documentation to learn how to suppress most of these messages.
\ru{Прочитайте документацию, чтобы узнать как подавить вывод большей части
этих сообщений.}

Here’s something important that is obscured above.
\ru{Вот еще кое-что важное, что было неясно выше.}
We’ve used the same name,
\ru{Мы использовали одно и то же имя,}
humps (and height), in both the datatype definition and in the fields of the
patternmatch.
\ru{и в определении типа данных, и в полях объекта при проверке совпадения
шаблонов.}
This is absolutely unnecessary because the two are related by position, not
name.
\ru{Это совершенно необязательно, так как каждая пара связана по позиции, а не
по имени.}
Thus, we could have as well written the function as
\ru{Так что мы могли бы также написать эту функцию как}

\lst{src/1/p9_3.rkt}

\noindent
Because each h is only visible in the case branch in which it is introduced, the
two hs do not in fact clash.
\ru{Так как каждый h виден только в той case-секции, где он используется,
два h фактически не сталкиваются.}
You can therefore use convention and readability to
dictate your choices.
\ru{Таким образом вы можете использовать соглашения по оформлению кода для
улучшения читаемости, и диктовать свой выбор.}
In general, it makes sense to provide a long and
descriptive name when defining the datatype\note{because you probably won’t use
that name again}, but shorter names in the type-case because you’re
likely to use use those names one or more times.
\ru{В общем, имеет смысл использовать длинные описательные имена при определении
типа данных\note{\ru{потому что вы возможно больше никогда не будете
использовать это имя снова}}, и короткие имена в type-case, где они обычно
используются несколько раз.}

I did just say you’re unlikely to use the field descriptors introduced in the
datatype definition, but you can.
\ru{Также я хочу упомянуть декрипторы полей класса, которые вы возможно
захотите использовать.}
The language provides \term{selectors} to extract fields without the need for
pattern-matching: e.g., caml-humps.
\ru{Язык предоставляет \term{селекторы} для получения значений полей без
необходимости использовать pattern-matching, например caml-humps.}
Sometimes,
it’s much easier to use the selector directly rather than go through the
pattern-matcher.
\ru{Иногда намного проще использовать селектор, чем возиться с мэтчингом
шаблонов.}
It often isn’t, as when defining  above, but just to be
clear, let’s write it without pattern-matching:
\ru{Часто это не так, как в случае определения good?, но для ясности давайте
напишем без pattern-matching:}

\lst{src/1/p9_4.rkt}

\DoNow{
What happens if you mis-apply functions to the wrong kinds of values\,?\\
\ru{Что произойдет, если вы ошибочно примените функции к неправильным типам
значений\,?}

For instance, what if you give the caml constructor a string\,?\\
\ru{Например, что если вы дадите конструктору caml строковый аргумент\,?}

What if you send a number into each version of good? above\,?\\
\ru{Что произойдет если вы пошлете число к каждой версии good? описанных
выше\,?} }

\secup

\secrel{Everything (We Will Say) About Parsing
\ru{Все (что мы будем говорить) о разборе}}\secdown
\clearpage

Parsing is the act of turning an input character stream into a more structured,
internal representation.
\ru{\termdef{Парсинг}{парсинг} или \termdef{разбор}{синтаксический разбор}\
--- процесс превращения входного потока одиночных символов в более
структурированное внутреннее представление\note{программы или данных, заданных в
текстовой синтаксической форме}.}
A common internal representation is as a tree, which programs can recursively
process.
\ru{Обычно используется внутреннее представление в виде дерева, которое может
быть обработано программой рекурсивно.}

For instance, given the stream
\ru{Например, для входного потока символов\note{включая пробелы, табуляции и
концы строк}}
\begin{verbatim}
23 + 5 - 6
\end{verbatim}
we might want a tree representing addition whose left (L) node represents the
number 23 and whose right (R) node represents subtraction of 6 from 5.
\ru{мы хотим получить деревянное представление сложения, в котором левая (L)
ветвь содержит число 23, а правая (R)\ --- вложенное представление вычитания 6
из 5.}
A parser is responsible for performing this transformation.
\ru{Парсер отвечает за выполнение такой транформации.}

\noindent
\begin{tabular}{c c}
\noindent\includegraphics[height=0.8\textheight]{tmp/2_p10_R.pdf}
&
\noindent\includegraphics[height=0.8\textheight]{tmp/2_p10_L.pdf}
\\
\emph{Право}ассоциативный разбор
&
(*) \emph{Лево}ассоциативный разбор
\\
\end{tabular}\bigskip

Parsing is a large, complex problem that is far from solved due to the
difficulties of ambiguity.
\ru{Парсинг\ --- большая проблема информатики, сложность которой
заключается в трудностях неоднозначности.}
For instance, an alternate parse tree (*) for the above input expression might
put subtraction at the top and addition below it.
\ru{Например, существует альтернативное (*) \termdef{дерево разбора}{дерево
разбора} для того же входного выражения, в котором мы можем поместить
вычитание на вершину дерева, а сложение будет вложенным поддеревом.}
We might also want to consider
\ru{Нас также может интересовать,}
whether this addition operation is commutative
\ru{является ли это сложение коммутативной операцией,}
and hence whether the order of arguments can be switched.
\ru{то есть можем ли мы изменить порядок аргументов\note{например для
оптимизации кода}.}
Everything only gets much, much worse when we get to full-fledged programming
languages (to say nothing of natural languages).
\ru{Все становится намного хуже для полноценных языков программирования
(не говоря о натуральных языках).}

\secrel{A Lightweight, Built-In First Half of a Parser
\ru{Легковесная встроенная часть парсера}}

These problems make parsing a worthy topic in its own right, and entire books,
tools, and courses are devoted to it.
\ru{Эти проблемы делают разбор достойной темой саму по себе, и ей посвящены
целые книги, утилиты и учебные курсы, например первая глава ``Книги Дракона''
\cite{dragon}.}
However, from our perspective parsing is mostly a distraction, because we want
to study the parts of programming languages that are not parsing.
\ru{Однако, с нашей точки зрения тема парсинга является сильным отвлечением,
так как есть другие более достойные темы, касающиеся реализации языков
программирования.}
We will therefore exploit a handy feature of Racket to manage the transformation
of input streams into trees: 
\ru{Поэтому мы сделаем финт ушами, и будем использовать встроенную фичу
\racket а для получения готовых деревьев разбора из входного потока: функцию}
\verb|read|.
\verb|read| is tied to the parenthetical form of the language, in that it parses
fully (and hence unambiguously) parenthesized terms into a built-in tree form.
\verb|read| \ru{привязана к скобочной форме языка, и полностью (и следовательно
однозначно) разбирает скобочные выражения во встроенное представление\ ---
дерево.}
For instance, running \ru{Например применение} \verb|(read)| on the
parenthesized form of the above input \ru{к следующему входному потоку символов
(включающему скобки)}\ ---
\begin{verbatim}
(+ 23 (- 5 6))
\end{verbatim}
--- will produce a list, whose first element is the symbol \ru{создаст список, в
котором первым элементом будет символ} \verb|'+|, second element is the number
\ru{вторым элементом число} 23, and third element is a list \ru{и третий
элемент список}: this list’s first element is the
symbol \ru{в котором первым элемент будет} \verb|'-|, second element is the
number \ru{второй элемент число} 5, and third element is the number \ru{и
третий элемент число} 6.

\fig{\ru{Дерево разбора для} (+ 23 (- 5 6))}{tmp/2_1.pdf}{height=0.7\textheight}

\lst{src/2/p10_1.rkt}

\secrel{A Convenient Shortcut \ru{Удобный трюк}}

As you know you need to test your programs extensively, which is hard to do when you
must manually type terms in over and over again.
\ru{Как вы знаете, нужно тщательно тестировать свои программы, что особенно
сложно, если вам нужно снова и снова вводить выражения вручную.}
Fortunately, as you might expect, the parenthetical syntax is integrated deeply
into \racket\ through the mechanism of quotation.
\ru{К счатью, как и следовало ожидать, скобочный синтаксис глубоко
интегрирован в \racket\ через механизм \termdef{квотирования}{квотирование}.}
That is, \ru{это то самое выражение} \verb|'<expr>|\ --—
which you saw a moment ago in the above example
\ru{которое вы видели только что при выполнении предыдущего примера}\ --- 
acts as if you had run \ru{действует так же, как если бы вы запустили}
\verb|(read)| and typed \ru{и ввели} <expr> at the prompt \ru{в текстовом поле
ввода} (and, of course, evaluates to the value the (read) would have
\ru{и конечно же вычисляется в то же значение, что дает} \verb|(read)|).

\secrel{Types for Parsing \ru{Типы для разбора}}

Actually, I’ve lied a little.
\ru{На самом деле, я немного соврал.}
I said that \ru{я сказал что} \verb|(read)|\ --- or equivalently, using
quotation \ru{или использование квотирования, что эквивалентно}\ --- will
produce a \emph{list}, etc. \ru{создаст \emph{список}, блаблабла.}
That’s true in regular \racket, but in $Typed PLAI$
\ru{Это так для оригинального \racket, но в $Typed PLAI$} the type it
returns a distinct type called an \ru{возвращается специальный тип, который
называется} \termdef{s-expression}{s-expression}
\ru{s-выражение\index{s-выражение}}, written in $Typed PLAI$ as
\ru{который в $Typed PLAI$ записывается как}
\verb|s-expression|:
\begin{verbatim}
> (read)
- s-expression
[type in (+ 23 (- 5 6))]
'(+ 23 (- 5 6))
\end{verbatim}
\racket\ has a very rich language of s-expressions
\ru{имеет очень богатый язык на s-выражениях}
(it even has notation to represent cyclic structures
\ru{он даже имеет нотацию для представления циклических структур}), 
but we will use only the simple fragment of it.
\ru{но мы будем использовать только простейшую часть этого синтаксиса.}

In the typed language, an s-expression is treated distinctly from the other
types, such as numbers and lists.
\ru{В типизированном языке s-выражения обрабатыватся обособленно от других
типов, таких как числа и списки.}
\begin{framed}
Underneath, an s-expression is a large
recursive datatype that consists of all the base printable values—numbers,
strings, symbols, and so on—and printable collections (lists, vectors, etc.) of
s-expressions.
\ru{Далее s-выражение рассматривается как большой рекурсивный тип данных,
который содержит все базовые отображаемые (представимые в тексте) значения\
--- числа, строки, символы и т.д.\ --- и коллекции (списки, вектора и т.д.)
других s-выражений}.
\end{framed}
As a result, base types like numbers, symbols, and strings are
\emph{both} their own type and an instance of s-expression.
\ru{В результате такие базовые типы как числа, символы и строки, могут
\emph{одновременно} являться как собственным типом (число,..), так и
экземпляром s-выражениея}.
Typing such data can be fairly problematic, as we will discuss later
\ru{Типизация таких данных может быть очень проблематична, детальнее мы обсудим
это позже} \ref{}.

$Typed PLAI$ takes a simple approach.
\ru{$Typed PLAI$ применяет более простой подход.}
When written on their own, values like numbers are of those respective types.
\ru{Когда значания простых типов, типа чисел, написаны сами по себе, они
являются собственными типами (число).}
But when written inside a complex
s-expression—in particular, as created by read or quotation—they have type
s-expression.
\ru{Но когда они включены в состав сложного s-выражения\ --- в частности,
созданы через (read) или квотацию\ --- они имеют тип s-выражения.}
You have to then cast them to their native types.
\ru{Вы должны привести их к нативному типу.}
For instance \ru{Например}:
\lstl{src/2/p11_1.rkt}
This is similar to the casting that a Java programmer would have to insert.
\ru{Это похоже на явное приведение типов, которое должен вставить программист
на \java.}
We will study casting itself later \ru{Мы обсудим само приведение типов позже}
\ref{}.

Observe that the first element of the list is still not treated by the
type checker as a symbol:
\ru{Отметим, что первый элемент списка все еще не распознается контролером
типов как символ:}
a list-shaped s-expression is a list of s-expressions.
\ru{списко-образное s-выражение является списком s-выражений.}
Thus \ru{Таким образом},
\lst{src/2/p11_2.rkt}
whereas again, casting does the trick:
\ru{и снова приведение типов решает проблему :}
\lst{src/2/p11_3.rkt}
The need to cast s-expressions is a bit of a nuisance,
\ru{Необходимость приведения s-выражений немного геморна,}
but some complexity is unavoidable because of what we’re trying to accomplish:
\ru{но некоторая сложность неизбежна из-за того что мы пытаемся достичь:}
to convert an \emph{untyped input} stream into a \emph{typed output} stream
\ru{преобразование \emph{нетипизированного} входного потока в
\emph{типизированный} выходной поток}
through robustly typed means.
\ru{через средства робастной типизации.}
Somehow we have to make explicit our assumptions about that input stream.
\ru{Каким-то образом мы должны делать явные предположения об этом входном
потоке.}

Fortunately we will use s-expressions only in our parser, and our goal is to
\emph{get away from parsing as quickly as possible\,!}
\ru{К счастью, мы будем использовать s-выражения только в нашем парсере, и наша
цель состоит в том, чтобы \emph{уйти от разбора как можно быстрее\,!}}
Indeed, if anything this should be inducement to get away even quicker.
\ru{В самом деле, все эти заморочки являются побуждением сделать этот уход еще
быстрее.}

\secrel{Completing the Parser \ru{Заканчиваем с парсером}}\label{sec2_4}

In principle, we can think of \ru{В принципе, мы можем думать о} \verb|read|\
as a complete parser \ru{как о законченном парсере}.
However, its output is generic \ru{Тем не менее, его вывод все еще сырой}:
it represents the token structure without offering any comment on its intent.
\ru{он содержит структуру токенов не предлагая каких-либо комментариев об их
назначении.}
We would instead prefer to have a representation
\ru{Вместо этого мы предпочли бы иметь представление,}
that tells us something about the \emph{intended meaning} of the terms in our
language,
\ru{которое говорит нам что-то о \emph{предполагаемом значении} термов нашего
языка,}
just as we wrote at the very beginning: “representing addition”, “represents a
number”, and so on.
\ru{так же как мы писали в самом начале: ``представление сложения'',
``представление числа'' и так далее.}

To do this, we must first introduce a datatype
\ru{Чтобы сделать это, мы сначала введем тип данных,}
that captures this representation.
\ru{который зафиксирует это представление.}
We will separately discuss \ru{Мы
отдельно рассмотрим} (section \ru{в разделе} \ref{sec31}) how and why we obtained this
datatype \ru{как и зачем мы применяем этот тип}, but for now let’s say it’s
given to us \ru{но сейчас пока будем считать, что он нам задан}:
\lstx{ArithC.rkt}{src/2/p12_1.rkt}{rkt}\label{arithc}
We now need a function that will convert s-expressions into instances of this
datatype.
\ru{Теперь нам нужна функция, которая преобразует s-выражение в структуру из
экземпляров этого типа.}
This is the other half of our parser \ru{Это вторая половина нашего парсера}:
\lstxl{ArithC.rkt}{src/2/p12_2.rkt}{rkt}

Thus\note{typing in \racket\ console \emph{after program run}} \ru{Таким
образом\note{\ru{введя выражение в \racket-консоли \emph{после выполнения
программы}}}}
\lst{src/2/v1.rkt}
\lstx{ArithC.rkt}{src/2/p12_3.rkt}{rkt}
\lst{src/2/v2.rkt}

Congratulations\,! \ru{Мои поздравления\,!}
You have just completed your first representation of a program.
\ru{Вы только что завершили ваше первое представление программы.}
From now on we can focus entirely on programs
\ru{С этого момента мы можем полностью сосредоточиться на программах,}
represented as recursive trees,
\ru{представленных в виде рекурсивных деревьев,}
ignoring the vagaries of surface syntax
\ru{не обращая внимания на капризы наносного синтаксиса}
and how to get them into the tree form.
\ru{и процесс получения из него дерева разбора.}
We’re finally ready to start studying programming languages\,!
\ru{Мы, наконец, готовы приступить к изучению языков программирования\,!}

\Exercise{
What happens if you forget to quote the argument to the 
\ru{Что случиться, если вы забудете заквотить аргумент вызова}
parser\,?
Why\,? \ru{Почему\,?}
}
\secrel{Coda \ru{Кода}}

\racket’s syntax, which it inherits from Scheme and \lisp, is controversial.
\ru{Синтаксис \racket а, который он наследует от Scheme и Lisp, спорен.}
Observe, however, something deeply valuable that we get from it.
\ru{Заметим, однако, что мы получаем от него нечто глубоко ценное.} 
While parsing traditional languages can be very complex,
\ru{В то время как парсинг традиционных языков может быть очень сложным,}
parsing this syntax is virtually trivial.
\ru{разбор этого синтаксиса практически тривиален.}
Given a sequence of tokens corresponding to the input,
\ru{Для заданной последовательности лексем, соответствующих входному потоку,}
it is absolutely straightforward to turn paren\-the\-sized sequences into
s-expressions;
\ru{абсолютно тривиально превратить скобочные последовательности в s-выражения;}
it is equally straightforward (as we see above) to turn sexpressions into proper
syntax trees.
\ru{столь же просто (как мы видим выше) преобразовать s-выражения в правильные
синтаксические деревья.}
I like to call such two-level languages \term{bicameral}, in loose analogy to
government legislative houses:
\ru{Мне нравится называть такие двухуровневые языки \term{двухпалатными}, в
свободной аналогии к государственным законодательным учреждениям:}
the lower-level does rudimentary well-formedness checking, while the upper-level
does deeper validity checking.
\ru{нижний уровень делает рудиментарную проверку правильности оформления, в то
время как верхний уровень выполняет глубокую проверку валидности.}
(We haven’t done any of the latter yet, but we will
\ru{Мы еще не делали последнего, но мы будем}
\ref{}.)

The virtues of this syntax are thus manifold.
\ru{Достоинства этого синтаксиса, таким образом, многообразны.}
The amount of code it requires is small, and can easily be embedded in many
contexts.
\ru{Объем кода, который он требует, очень мал, и может быть встроен во многих
контекстах.}
By integrating the syntax into the language, it becomes easy for programs to
manipulate representations of programs (as we will see more of in \ref{}).
\ru{Интеграция синтаксиса в язык делает простой программную манипуляцию
представлением программ (как мы увидим в \ref{}).}
It’s therefore no surprise that even though many Lisp-based syntaxes have had
wildly different semantics, they all share this syntactic legacy.
\ru{Поэтому неудивительно, что множество основанных на \lisp е синтаксисов,
имеющих дико разную семантику, все равно разделяют это общее синтаксическое
наследие.}

Of course, we could just use XML instead.
\ru{Конечно, мы могли бы использовать XML.}
That would be much better.
\ru{Это было бы намного лучше.}
Or JSON.
\ru{Или JSON.}
Because that wouldn’t be anything like an s-expression at all.
\ru{Потому что все равно это в итоге было бы тем же s-выражением.}

\secup

\secrel{A First Look at Interpretation \ru{Первый взгляд на
интерпретацию}}\label{firstinterp}\secdown

Now that we have a representation of programs, there are many ways in which we
might want to manipulate them.
\ru{Теперь, когда мы имеем представление программ, существует множество
способов, которыми мы можем манипулировать ими.}
We might want to display a
program in an attractive way \ru{Мы можем захотеть выводить листинг программы в
красивом виде} (“pretty-print”), convert into code in some other format
\ru{преобразовать в код в какой-то другой формат} (“compilation”
\ru{``компиляция''/''трансляция''}), ask whether it obeys certain properties 
\ru{убедиться что она отвечает определенным требованиям}
(“verification” \ru{``верификация''}), and so on \ru{и так далее}.
For now, we’re going to focus on asking what value it corresponds to
(“\termdef{e\underline{val}uation}{evaluation}”\ --- the reduction of programs
to \emph{\underline{val}ues}).
\ru{Для начала, мы собираемся сфокусироваться на вопросе\ --- какому значению
соответствует программа (“\termdef{вычисление}{вычисление}"\ --- редукция
программы до \emph{значения})}

Let’s write an evaluator, in the form of an \termdef{interpreter}{interpreter},
for our arithmetic language.
\ru{Давайте напишем вычислитель, в форме
\termdef{интерпретатора}{интерпретатор}, для нашего арифметического языка.}
We choose arithmetic first for three reasons \ru{Мы выбрали арифметику прежде
всего по следующим трем причинам}:
\begin{itemize}[nosep]
  \item[(a)]
  you already know how it works, so we can focus on the mechanics of writing
evaluators;
\ru{вы уже знаете как работает арифметика, и мы можем сфокусироваться на
механике написания вычислителей;}
  \item[(b)]
  it’s contained in
every language we will encounter later, so we can build upwards and outwards from it;
\ru{она содержится в каждом языке, с которым мы столкнемся в дальнейшем, так
что мы будем расширять этот арифметический язык вверх и вширь;} and \ru{и}
  \item[(c)] 
it’s at once both small and big enough to illustrate many points
we’d like to get across.
\ru{этот язык минималистичен, но при этом достаточно большой, чтобы
проиллюстрировать многие моменты, которые мы хочем до вас донести.}
\end{itemize}

\secrel{Representing Arithmetic \ru{Представление арифметики}}\label{sec31}

Let’s first agree on how we will represent arithmetic expressions.
\ru{Давайте сначала договориться о том, как мы будем представлять арифметические
выражения.}
Let’s say we want to support only two operations\ --- addition and
multiplication\ --- in addition to primitive numbers.
\ru{Допустим, мы хотим поддерживать только две простые операции\ --- сложение и
умножение\ --- в дополнение к примитивным числам.}
We need to represent arithmetic \termdef{expressions}{expression}.
\ru{Нам необходимо представление для арифметических
\termdef{выражений}{выражение}}.
What are the rules that govern nesting of arithmetic expressions\,?
\ru{Какие правила регулируют вложенность арифметических выражений\,?} 
We’re actually free to nest any expression inside another.
\ru{На самом мы свободно можем владывать любое выражение внутрь любого другого.}

\DoNow{
Why did we not include division\,?
\ru{Почему мы не включили умножение\,?}
\\
What impact does it have on the remarks above\,?
\ru{Какое влияние это имеет на замечания выше\,?}
}

We’ve ignored division because it forces us into a discussion of what
expressions we might consider legal:
\ru{Мы игнорировали деление, потому что оно вовлекает нас в дикуссию о том,
какие выражения мы можем считать правильными:}
clearly the representation of $1/2$ ought to be legal;
\ru{ясно что представление $1/2$ должно быть легальным;} 
the representation of $1/0$ is much more debatable;
\ru{представление $1/0$ спорно;}
and that of \ru{и что-то типа} $1/(1-1)$ seems even more controversial.
\ru{кажется гораздо более спорным.}
 
We’d like to sidestep this controversy for now and return to it later
\ru{Мы хотели бы обойти сейчас это противоречие, в вернуться к нему позже}
\ref{}.

Thus, we want a representation for numbers and arbitrarily nestable
addition and multiplication.
\ru{Таким образом нам нужно представление для чисел и произвольно вложенных
сложений и умножений.} 
Here’s one we can use \ru{Вот то что мы можем использовать}
(used in \ru{использовано в} \ref{arithc} ):
\lstx{ArithC}{src/2/p12_1.rkt}{rkt}

\secrel{Writing an Interpreter \ru{Написание интерпретатора}}

Now let’s write an interpreter for this arithmetic language.
\ru{Теперь давайте напишем интерпретатор для этого арифметического языка.} 
First, we should think about what its type is.
\ru{Для начала нам надо подумать, какие типы он использует\,?} 
It clearly consumes a \verb|ArithC| value.
\ru{Совершенно ясно что на вход подается структура типа} \verb|ArithC|. 
What does it produce\,?
\ru{Что он возвращает\,?} 
Well, an interpreter evaluates
\ru{Ну, интерпретатор вычисляет} \ --- and what kind of value might arithmetic
expressions reduce to\,?
\ru{и к какому значению может редуцироваться арифметическое выражение\,?} 
Numbers, of course.
\ru{Конечно, числу.} 
So the interpreter is going to be a function from arithmetic expressions to
numbers.
\ru{Таким образом, интерпретатор должен быть функцией от арифметического
выражения, возвращающей число.}

\Exercise{
Write your test cases for the interpreter.
\ru{Напишите тесты для интерпретатора.}
}

Because we have a recursive datatype, it is natural to structure our interpreter
as a recursive function over it.
\ru{Так как мы имеем рекурсивный тип данных\note{допускаются произвольные
вложения того же типа}, нормально что структура нашего интерпретатора тоже
должна быть рекурсивной функций над выражением.}
Here’s a first template\note{Templates are
explained in great detail in \emph{How to Design Programs}.}
\ru{Вот первый шаблон\note{\ru{Шаблоны очень детально описаны в \emph{How to
Design Programs}}}} :
\lstx{ArithC.rkt}{src/3/p14_1.rkt}{rkt}
You’re probably tempted to jump straight to code, which you can:
\ru{Вероятно у вас есть соблазн сразу перейти к коду, который вы можете
написать:}
\lstx{ArithC.rkt}{src/3/p14_2.rkt}{rkt}

\DoNow{
Do you spot the errors\,?
\ru{Вы увидели ошибки\,?}
}

Instead, let’s expand the template out a step:
\ru{Вместо этого давайте расширим шаблон на один шаг:}
\lstx{ArithC.rkt}{src/3/p15_2.rkt}{rkt}
and now we can fill in the blanks:
\ru{и теперь мы можем заполнить пробелы:}
\lstx{ArithC.rkt}{src/3/p15_3.rkt}{rkt}

Later on \ru{Позже в} \ref{}, we’re going to wish we had returned a more complex
datatype than just numbers.
\ru{мы пожелаем возвращать более сложный тип данных, чем просто числа.}
But for now, this will do.
\ru{Но сейчас нам этого достаточно.}

Congratulations: you’ve written your first interpreter\,!
\ru{Поздравляем: вы только что написали свой первый интерпретатор\,!} 
I know, it’s very nearly an anticlimax\note{\ru{ситуация, когда проблема
казавшеяся очень сложной, решается с помощью чего-то тривиального //
Wikipedia}}\note{\ru{род морских улиток // там же}}.
\ru{Я знаю, это очень близко к разочарованию.}
But they’ll get harder\ --- much harder\ --- pretty soon, I promise.
\ru{Но все станет жестче\ --- намного жестче\ --- совсем скоро, я обещаю.}

\secrel{Did You Notice\,? \ru{Вы заметили\,?}}

I just slipped something by you:
\ru{Я только что утаил что-то от вас:}
\DoNow{
What is the ``meaning'' of addition and multiplication in this new language\,?
\ru{Каков ``смысл'' сложения и умножения в этом новом языке\,?}
}

That’s a pretty abstract question, isn’t it.
\ru{Это достаточно абстрактный вопрос, не так ли.} 
Let’s make it concrete.
\ru{Давайте его конкретизируем.} 
There are many kinds of addition in computer science:
\ru{В информатике существует множество видов сложения:}

\begin{itemize}
  \item 
First of all, there’s many different kinds of \termdef{numbers}{number}:
\ru{Прежде всего, существует множество видов \termdef{чисел}{число}:}
fixed-width \ru{фиксированной длины} (e.g., 32- bit \ru{например 32-битные})
integers \ru{целые}, signed fixed-width \ru{знаковые фиксированной длины} (e.g.,
31-bits plus a sign-bit \ru{например 31-битные плюс бит знака}) integers
\ru{целые}, arbitrary precision integers \ru{целые числа произвольной точности};
in some languages, rationals \ru{в некоторых языках\ --- натуральные дроби};
various formats of fixed- and floating-point numbers
\ru{различные форматы чисел с фиксированной и плавающей точкой}; in some
languages, complex numbers \ru{в некоторых языках комплектные числа}; and so on
\ru{и так далее}.
After the numbers have been chosen, addition may support only some combinations
of them.
\ru{После того как были выбраны определенные виды чисел, сложение может
поддерживать только некоторые их комбинации.}
  \item 
In addition, some languages permit the addition of datatypes such as matrices.
\ru{В дополнение, некоторые языки поддерживают сложение таких типов данных, как
матрицы.}
  \item 
Furthermore, many languages support ``addition'' of strings
\ru{Кроме того, многие языки поддерживают ``сложение'' строк} (
we use scare-quotes because we don’t really mean the mathematical concept of
addition, but rather the operation performed by an operator with the syntax +
\ru{Мы используем кавычки, так как предполагаем не математическую идею
сложения, а операцию, выполняемую оператором с синтаксисом +} ). 
In some languages this always means concatenation;
\ru{В некоторых языках это всегда значит конкатенацию строк;} 
in some others, it can result in numeric results (or numbers stored in strings).
\ru{в некоторых долбанутых языках иногда может получиться численный
результат (или числа хранимые в строках).}
\end{itemize}

These are all different meanings for addition. 
\ru{Все это является смыслом сложения.}
\termdef{Semantics}{semantics}
is the mapping of \emph{syntax} (e.g., +) to \emph{meaning} (e.g., some or all
of the above).
\ru{\termdef{Семантика}{семантика} это отображение \emph{синтаксиса} (например
+) на \emph{смысл} (например что-то или все из вышеперечисленного).}

This brings us to our first game of:
\ru{Это подводит нас к первоначальной игре:}
\begin{description}\item[\emph{Which of these is the same\,?} \ru{Что из
этого дает одинаковый результат\,?}]\
\\
\begin{itemize}[nosep]
  \item 1 + 2
  \item 1 + 2
  \item '1' + '2'
  \item '1' + '2'
\end{itemize}
\end{description}

Now return to the question above.
\ru{Теперь возвращаемся к предыдущему вопросу.} 
What semantics do we have\,?
\ru{Какую семантику мы имеем\,?}
We’ve adopted whatever semantics \racket\ provides, because we map + to
\racket’s +. 
\ru{Мы приняли ту же семантику, которую предоставляет \racket, потому что мы
отоборазили наш + на + в \racket е.}
In fact that’s not even quite true:
\ru{На самом деле это даже не совсем верно:} 
\racket\ may, for all we know, also enable +
to apply to strings, so we’ve chosen the restriction of \racket’s semantics to
numbers\note{though in fact \racket’s + doesn’t tolerate strings}.
\ru{\racket\ может, как все мы знаем, также использовать + к строкам, так что
мы выбрали ограничение семантики \racket а только для чисел\note{\ru{хотя на
самом деле \racket ский + нетолерантен к строкам}}.}

If we wanted a different semantics, we’d have to implement it explicitly.
\ru{Если мы хотим другую семантику, мы должны реализовать ее в явном виде.}

\Exercise{
What all would you have to change so that the number had signed- 32-bit
arithmetic\,?
\\
\ru{Что мы должны изменить, чтобы числа поддерживали знаковую 32-битную
арифметику\,?}
}

In general, we have to be careful about too readily borrowing from the host
language.
\ru{В общем, мы должны быть осторожными с заимствованиями из языка-носителя.}
We’ll return to this topic later \ru{Мы вернемся к этой теме позже} \ref{}.

\secrel{3.4 Growing the Language . . . . . . . . . . . . . . . . . . . . . . . . . 16}

\secup

\secrel{4 A First Taste of Desugaring 16}\label{sec4}\secdown

We’ve begun with a very spartan arithmetic language.
\ru{Мы начали с очень спартанского арифметического языка.}
Let’s look at how we might extend it with more arithmetic operations that can
nevertheless be expressed in terms of existing ones.
\ru{Давайте посмотрим, как мы могли бы расширить его б\'ольшим количеством
арифметических операций, которые тем не менее могут быть выражены в терминах
существующих операторов.}
We’ll add just two, because these will suffice to illustrate the point.
\ru{Мы добавим только два, потому что этого будет достаточно для иллюстрации
этого метода.}

\secrel{4.1 Extension: Binary Subtraction . . . . . . . . . . . . . . . . . . . . . 17}
\secrel{4.2 Extension: Unary Negation . . . . . . . . . . . . . . . . . . . . . . . 18}
\secup


\secrel{5 Adding Functions to the Language 19}\secdown
\secrel{5.1 Defining Data Representations . . . . . . . . . . . . . . . . . . . . . 19}
\secrel{5.2 Growing the Interpreter . . . . . . . . . . . . . . . . . . . . . . . . . 21}
\secrel{5.3 Substitution . . . . . . . . . . . . . . . . . . . . . . . . . . . . . . . 22}
\secrel{5.4 The Interpreter, Resumed . . . . . . . . . . . . . . . . . . . . . . . . 23}
\secrel{5.5 Oh Wait, There’s More! . . . . . . . . . . . . . . . . . . . . . . . . . 25}
\secup

\secrel{6 From Substitution to Environments 25}\secdown
\secrel{6.1 Introducing the Environment . . . . . . . . . . . . . . . . . . . . . . 26}
\secrel{6.2 Interpreting with Environments . . . . . . . . . . . . . . . . . . . . . 27}
\secrel{6.3 Deferring Correctly . . . . . . . . . . . . . . . . . . . . . . . . . . . 29}
\secrel{6.4 Scope . . . . . . . . . . . . . . . . . . . . . . . . . . . . . . . . . . 30}
\secdown
\secrel{6.4.1 How Bad Is It? . . . . . . . . . . . . . . . . . . . . . . . . . 30}
\secrel{6.4.2 The Top-Level Scope . . . . . . . . . . . . . . . . . . . . . . 31}
\secup
\secrel{6.5 Exposing the Environment . . . . . . . . . . . . . . . . . . . . . . . 31}
\secup

\secrel{7 Functions Anywhere 31}\secdown
\secrel{7.1 Functions as Expressions and Values . . . . . . . . . . . . . . . . . . 32}
\secrel{7.2 Nested What? . . . . . . . . . . . . . . . . . . . . . . . . . . . . . . 35}
\secrel{7.3 Implementing Closures . . . . . . . . . . . . . . . . . . . . . . . . . 37}
\secrel{7.4 Substitution, Again . . . . . . . . . . . . . . . . . . . . . . . . . . . 38}
\secrel{7.5 Sugaring Over Anonymity . . . . . . . . . . . . . . . . . . . . . . . 39}
\secup

\secrel{8 Mutation: Structures and Variables 41}\secdown
\secrel{8.1 Mutable Structures . . . . . . . . . . . . . . . . . . . . . . . . . . . 41}
\secdown
\secrel{8.1.1 A Simple Model of Mutable Structures . . . . . . . . . . . . 41}
\secrel{8.1.2 Scaffolding . . . . . . . . . . . . . . . . . . . . . . . . . . . 42}
\secrel{8.1.3 Interaction with Closures . . . . . . . . . . . . . . . . . . . . 43}
\secrel{8.1.4 Understanding the Interpretation of Boxes . . . . . . . . . . . 44}
\secrel{8.1.5 Can the Environment Help? . . . . . . . . . . . . . . . . . . 46}
\secrel{8.1.6 Introducing the Store . . . . . . . . . . . . . . . . . . . . . . 48}
\secrel{8.1.7 Interpreting Boxes . . . . . . . . . . . . . . . . . . . . . . . 49}
\secrel{8.1.8 The Bigger Picture . . . . . . . . . . . . . . . . . . . . . . . 54}
\secup
\secrel{8.2 Variables . . . . . . . . . . . . . . . . . . . . . . . . . . . . . . . . 57}
\secdown
\secrel{8.2.1 Terminology . . . . . . . . . . . . . . . . . . . . . . . . . . 57}
\secrel{8.2.2 Syntax . . . . . . . . . . . . . . . . . . . . . . . . . . . . . 57}
\secrel{8.2.3 Interpreting Variables . . . . . . . . . . . . . . . . . . . . . . 58}
\secup
\secrel{8.3 The Design of Stateful Language Operations . . . . . . . . . . . . . . 59}
\secrel{8.4 Parameter Passing . . . . . . . . . . . . . . . . . . . . . . . . . . . . 60}
\secup

\secrel{9 Recursion and Cycles: Procedures and Data 62}\secdown
\secrel{9.1 Recursive and Cyclic Data . . . . . . . . . . . . . . . . . . . . . . . 62}
\secrel{9.2 Recursive Functions . . . . . . . . . . . . . . . . . . . . . . . . . . . 64}
\secrel{9.3 Premature Observation . . . . . . . . . . . . . . . . . . . . . . . . . 65}
\secrel{9.4 Without Explicit State . . . . . . . . . . . . . . . . . . . . . . . . . . 66}
\secup

\secrel{10 Objects 67}\secdown
\secrel{10.1 Objects Without Inheritance . . . . . . . . . . . . . . . . . . . . . . 67}
\secdown
\secrel{10.1.1 Objects in the Core . . . . . . . . . . . . . . . . . . . . . . . 68}
\secrel{10.1.2 Objects by Desugaring . . . . . . . . . . . . . . . . . . . . . 69}
\secrel{10.1.3 Objects as Named Collections . . . . . . . . . . . . . . . . . 69}
\secrel{10.1.4 Constructors . . . . . . . . . . . . . . . . . . . . . . . . . . 70}
\secrel{10.1.5 State . . . . . . . . . . . . . . . . . . . . . . . . . . . . . . 71}
\secrel{10.1.6 Private Members . . . . . . . . . . . . . . . . . . . . . . . . 71}
\secrel{10.1.7 Static Members . . . . . . . . . . . . . . . . . . . . . . . . . 72}
\secrel{10.1.8 Objects with Self-Reference . . . . . . . . . . . . . . . . . . 72}
\secrel{10.1.9 Dynamic Dispatch . . . . . . . . . . . . . . . . . . . . . . . 74}
\secup
\secrel{10.2 Member Access Design Space . . . . . . . . . . . . . . . . . . . . . 75}
\secrel{10.3 What (Goes In) Else? . . . . . . . . . . . . . . . . . . . . . . . . . . 75}
\secdown
\secrel{10.3.1 Classes . . . . . . . . . . . . . . . . . . . . . . . . . . . . . 76}
\secrel{10.3.2 Prototypes . . . . . . . . . . . . . . . . . . . . . . . . . . . 78}
\secrel{10.3.3 Multiple Inheritance . . . . . . . . . . . . . . . . . . . . . . 78}
\secrel{10.3.4 Super-Duper! . . . . . . . . . . . . . . . . . . . . . . . . . . 79}
\secrel{10.3.5 Mixins and Traits . . . . . . . . . . . . . . . . . . . . . . . . 79}
\secup
\secup

\secrel{11 Memory Management 81}\secdown
\secrel{11.1 Garbage . . . . . . . . . . . . . . . . . . . . . . . . . . . . . . . . . 81}
\secrel{11.2 What is “Correct” Garbage Recovery? . . . . . . . . . . . . . . . . . 81}
\secrel{11.3 Manual Reclamation . . . . . . . . . . . . . . . . . . . . . . . . . . 82}
\secdown
\secrel{11.3.1 The Cost of Fully-Manual Reclamation . . . . . . . . . . . . 82}
\secrel{11.3.2 Reference Counting . . . . . . . . . . . . . . . . . . . . . . 83}
\secup
\secrel{11.4 Automated Reclamation, or Garbage Collection . . . . . . . . . . . . 84}
\secdown
\secrel{11.4.1 Overview . . . . . . . . . . . . . . . . . . . . . . . . . . . . 84}
\secrel{11.4.2 Truth and Provability . . . . . . . . . . . . . . . . . . . . . . 85}
\secrel{11.4.3 Central Assumptions . . . . . . . . . . . . . . . . . . . . . . 85}
\secup
\secrel{11.5 Convervative Garbage Collection . . . . . . . . . . . . . . . . . . . . 86}
\secrel{11.6 Precise Garbage Collection . . . . . . . . . . . . . . . . . . . . . . . 87}
\secup

\secrel{12 Representation Decisions 87}\secdown
\secrel{12.1 Changing Representations . . . . . . . . . . . . . . . . . . . . . . . 87}
\secrel{12.2 Errors . . . . . . . . . . . . . . . . . . . . . . . . . . . . . . . . . . 89}
\secrel{12.3 Changing Meaning . . . . . . . . . . . . . . . . . . . . . . . . . . . 89}
\secrel{12.4 One More Example . . . . . . . . . . . . . . . . . . . . . . . . . . . 90}
\secup

\secrel{13 Desugaring as a Language Feature 91}\secdown
\secrel{13.1 A First Example . . . . . . . . . . . . . . . . . . . . . . . . . . . . . 91}
\secrel{13.2 Syntax Transformers as Functions . . . . . . . . . . . . . . . . . . . 93}
\secrel{13.3 Guards . . . . . . . . . . . . . . . . . . . . . . . . . . . . . . . . . . 95}
\secrel{13.4 Or: A Simple Macro with Many Features . . . . . . . . . . . . . . . 95}
\secdown
\secrel{13.4.1 A First Attempt . . . . . . . . . . . . . . . . . . . . . . . . . 95}
\secrel{13.4.2 Guarding Evaluation . . . . . . . . . . . . . . . . . . . . . . 97}
\secrel{13.4.3 Hygiene . . . . . . . . . . . . . . . . . . . . . . . . . . . . . 98}
\secup
\secrel{13.5 Identifier Capture . . . . . . . . . . . . . . . . . . . . . . . . . . . . 99}
\secrel{13.6 Influence on Compiler Design . . . . . . . . . . . . . . . . . . . . . 101}
\secrel{13.7 Desugaring in Other Languages . . . . . . . . . . . . . . . . . . . . 101}
\secup

\secrel{14 Control Operations 102}\secdown
\secrel{14.1 Control on the Web . . . . . . . . . . . . . . . . . . . . . . . . . . . 102}
\secdown
\secrel{14.1.1 Program Decomposition into Now and Later . . . . . . . . . 104}
\secrel{14.1.2 A Partial Solution . . . . . . . . . . . . . . . . . . . . . . . . 104}
\secrel{14.1.3 Achieving Statelessness . . . . . . . . . . . . . . . . . . . . 106}
\secrel{14.1.4 Interaction with State . . . . . . . . . . . . . . . . . . . . . . 107}
\secup
\secrel{14.2 Continuation-Passing Style . . . . . . . . . . . . . . . . . . . . . . . 109}
\secdown
\secrel{14.2.1 Implementation by Desugaring . . . . . . . . . . . . . . . . . 110}
\secrel{14.2.2 Converting the Example . . . . . . . . . . . . . . . . . . . . 114}
\secrel{14.2.3 Implementation in the Core . . . . . . . . . . . . . . . . . . 115}
\secup
\secrel{14.3 Generators . . . . . . . . . . . . . . . . . . . . . . . . . . . . . . . . 117}
\secdown
\secrel{14.3.1 Design Variations . . . . . . . . . . . . . . . . . . . . . . . . 117}
\secrel{14.3.2 Implementing Generators . . . . . . . . . . . . . . . . . . . . 119}
\secup
\secrel{14.4 Continuations and Stacks . . . . . . . . . . . . . . . . . . . . . . . . 121}
\secrel{14.5 Tail Calls . . . . . . . . . . . . . . . . . . . . . . . . . . . . . . . . 123}
\secrel{14.6 Continuations as a Language Feature . . . . . . . . . . . . . . . . . . 124}
\secdown
\secrel{14.6.1 Presentation in the Language . . . . . . . . . . . . . . . . . . 125}
\secrel{14.6.2 Defining Generators . . . . . . . . . . . . . . . . . . . . . . 126}
\secrel{14.6.3 Defining Threads . . . . . . . . . . . . . . . . . . . . . . . . 127}
\secrel{14.6.4 Better Primitives for Web Programming . . . . . . . . . . . . 131}
\secup
\secup

\secrel{15 Checking Program Invariants Statically: Types 131}\secdown
\secrel{15.1 Types as Static Disciplines . . . . . . . . . . . . . . . . . . . . . . . 133}
\secrel{15.2 A Classical View of Types . . . . . . . . . . . . . . . . . . . . . . . 134}
\secdown
\secrel{15.2.1 A Simple Type Checker . . . . . . . . . . . . . . . . . . . . 134}
\secrel{15.2.2 Type-Checking Conditionals . . . . . . . . . . . . . . . . . . 139}
\secrel{15.2.3 Recursion in Code . . . . . . . . . . . . . . . . . . . . . . . 139}
\secrel{15.2.4 Recursion in Data . . . . . . . . . . . . . . . . . . . . . . . . 142}
\secrel{15.2.5 Types, Time, and Space . . . . . . . . . . . . . . . . . . . . 144}
\secrel{15.2.6 Types and Mutation . . . . . . . . . . . . . . . . . . . . . . . 146}
\secrel{15.2.7 The Central Theorem: Type Soundness . . . . . . . . . . . . 147}
\secup
\secrel{15.3 Extensions to the Core . . . . . . . . . . . . . . . . . . . . . . . . . 148}
\secdown
\secrel{15.3.1 Explicit Parametric Polymorphism . . . . . . . . . . . . . . . 148}
\secrel{15.3.2 Type Inference . . . . . . . . . . . . . . . . . . . . . . . . . 155}
\secrel{15.3.3 Union Types . . . . . . . . . . . . . . . . . . . . . . . . . . 164}
\secrel{15.3.4 Nominal Versus Structural Systems . . . . . . . . . . . . . . 170}
\secrel{15.3.5 Intersection Types . . . . . . . . . . . . . . . . . . . . . . . 171}
\secrel{15.3.6 Recursive Types . . . . . . . . . . . . . . . . . . . . . . . . 172}
\secrel{15.3.7 Subtyping . . . . . . . . . . . . . . . . . . . . . . . . . . . . 173}
\secrel{15.3.8 Object Types . . . . . . . . . . . . . . . . . . . . . . . . . . 176}
\secup
\secup

\secrel{16 Checking Program Invariants Dynamically: Contracts 179}\secdown
\secrel{16.1 Contracts as Predicates . . . . . . . . . . . . . . . . . . . . . . . . . 181}
\secrel{16.2 Tags, Types, and Observations on Values . . . . . . . . . . . . . . . . 182}
\secrel{16.3 Higher-Order Contracts . . . . . . . . . . . . . . . . . . . . . . . . . 183}
\secrel{16.4 Syntactic Convenience . . . . . . . . . . . . . . . . . . . . . . . . . 187}
\secrel{16.5 Extending to Compound Data Structures . . . . . . . . . . . . . . . . 188}
\secrel{16.6 More on Contracts and Observations . . . . . . . . . . . . . . . . . . 189}
\secrel{16.7 Contracts and Mutation . . . . . . . . . . . . . . . . . . . . . . . . . 189}
\secrel{16.8 Combining Contracts . . . . . . . . . . . . . . . . . . . . . . . . . . 190}
\secrel{16.9 Blame . . . . . . . . . . . . . . . . . . . . . . . . . . . . . . . . . . 191}
\secup

\secrel{17 Alternate Application Semantics 195}\secdown
\secrel{17.1 Lazy Application . . . . . . . . . . . . . . . . . . . . . . . . . . . . 196}
\secdown
\secrel{17.1.1 A Lazy Application Example . . . . . . . . . . . . . . . . . . 196}
\secrel{17.1.2 What Are Values? . . . . . . . . . . . . . . . . . . . . . . . 197}
\secrel{17.1.3 What Causes Evaluation? . . . . . . . . . . . . . . . . . . . 198}
\secrel{17.1.4 An Interpreter . . . . . . . . . . . . . . . . . . . . . . . . . . 199}
\secrel{17.1.5 Laziness and Mutation . . . . . . . . . . . . . . . . . . . . . 201}
\secrel{17.1.6 Caching Computation . . . . . . . . . . . . . . . . . . . . . 201}
\secup
\secrel{17.2 Reactive Application . . . . . . . . . . . . . . . . . . . . . . . . . . 201}
\secdown
\secrel{17.2.1 Motivating Example: A Timer . . . . . . . . . . . . . . . . . 202}
\secrel{17.2.2 Callback Types are Four-Letter Words . . . . . . . . . . . . . 203}
\secrel{17.2.3 The Alternative: Reactive Languages . . . . . . . . . . . . . 204}
\secrel{17.2.4 Implementing Transparent Reactivity . . . . . . . . . . . . . 205}
\secup
\secup



\clearpage
\addcontentsline{toc}{chapter}{Index \ru{Предметный указатель}}\printindex

\end{document}
