\input{../texheader/ebook}

\usepackage{../texheader/lstrkt}\lstdefinestyle{rkt}{language=rkt}

\newcommand{\Exercise}[1]{
	\begin{description}
		\item{\textcolor{red}{Упражнение}}\\#1
	\end{description}
}

\newcommand{\DoNow}[1]{
	\begin{description}
		\item{\textcolor{red}{Сделайте\,!}}\\#1
	\end{description}
}

\title{{\Large{PLAI}}\\
{\Large{Programming Languages: Application and Interpretation}}\\
{\small{second edition}}\\
{\Large{\ru{языки программирования:\\применение и интерпретация}}}}

\author{\copyright\ Shriram Krishnamurthy\\
\ru{перевод Dmitry Ponyatov \email{dponyatov@gmail.com}}}

\begin{document}
\maketitle
\tableofcontents
\secdown

\secrel{Introduction\\
\ru{Введение}}\secdown
\clearpage
\secdown\secrel{\ru{$\oplus$ От переводчика}}

Я наткнулся на эту книгу в поисках информации по реализации динамических языков.
Первоначально целью был только (подстрочный) перевод, чтобы разобраться в теме
самому, и заодно заполнить дыру в отечественной учебной литературе по реализации
языков программирования.
% Кто не согласен с дырой\ --- назовите хотя бы пару mainstream \ru{ходовых}
% языков программирования, созданных в России\note{Рефал и 1С.Васик}.

Потом мне захотелось добавить кое-что из своих находок, а поскольку лицензия
оригинальной книги\note{Creative Commons
\href{https://creativecommons.org/licenses/by-nc-sa/3.0/us/}{BY-NC-SA} 3.0 US}
позволяет ShareAlike \ru{модификации}, я добавил свою реализацию
DLR\note{[D]ynamic [L]anguage [R]untime} и интерпретатора\note{\ru{мне наиболее
интересно направление мета-/трансформаторного программирования, и обработка
текстовых форматов данных}}\ на flex/bison/\cpp. Сразу предупрежу что
правильность моего кода была принесена в жертву его простоте, поэтому не
удивляйтесь диким расходам и утечкам памяти, и топорному \cpp\ кун-фу.

\secup
\clearpage
\secrel{Our Philosophy\\\ru{Наша философия}}

Please watch the video on \ru{Пожалуйста посмотрите это видео на}
\href{https://www.youtube.com/watch?v=3N__tvmZrzc}{YouTube}.

Someday there will be a textual description here instead.
\ru{Когда нибудь здесь будет полное текстовое описание того, о чем в нем
говориться.}

\secrel{\ru{Метапрограммирование}}\secdown

Метапрограммирование\ --- вид программирования, связанный с созданием программ,
которые порождают другие программы как результат своей работы (в частности, на
стадии компиляции их исходного кода), либо программы, которые меняют себя во
время выполнения (самомодифицирующийся код).

\secrel{\ru{Генерация кода}}

При этом подходе код программы не пишется вручную, а создаётся автоматически
программой-генератором на основе другой программы\note{часто на каком-нибудь
специализированном сверхвысокоуровневом DSL-языке} или спецификации требуемого
приложения. Такой подход приобретает смысл, когда у вас вырабываются различные
шаблоны кода, часто достаточно сложные (высокоуровневые парадигмы, выполнение
требований внешних библиотек, стереотипные методы реализации приложений и т.п.).
При этом б\'{о}льшая часть кода теряет содержательный смысл, и становится лишь
механическим выполнением правил реализации/кодирования. В большинстве
практических случаев, при написании типовых приложений, особенно если вы
работаете в одной-двух прикладных областях\note{ERP/бухучет, расчетные
программы, АРМ, САПР, веб-приложения,\ldots}, эта часть становится настолько
значительной, что возникает мысль задавать вручную лишь спецификаторную часть
бизнес-логики, а остальной типовой код добавлять автоматически. Это и
проделывает метапрограмма-генератор.

\secrel{\ru{Мультиплатформенные приложения}}

Проблема поддержки нескольких платформ выполнения для приложений стала особенно
актуальной в последние 5-10 лет с связи с заметным увеличением количества
активно используемых операционных систем, взрывным ростом рынка мобильных
устройств, и развитием облачных технологий. Применение метапрограммирования
позволяет упростить реализацию серверов облачных сервисов и программ-клиентов
для десятков вариантов аппаратных $\times$ программных платформ.

\secrel{Transformational programming (TP)\\
\ru{Трансформаторное программирование}}

A program transformation is any operation that takes a computer program and
generates another program. In many cases the transformed program is required to
be semantically equivalent to the original, relative to a particular formal
semantics and in fewer cases the transformations result in programs that
semantically differ from the original in predictable ways.[1]

While the transformations can be performed manually, it is often more practical
to use a program transformation system that applies specifications of the
required transformations. Program transformations may be specified as automated
procedures that modify compiler data structures (e.g. abstract syntax trees)
representing the program text, or may be specified more conveniently using
patterns representing parameterized source code text fragments.

A practical requirement for source code transformation systems is that they be
able to effectively process programs written in a programming language. This
usually requires integration of a full front-end for the programming language of
interest, including source code parsing, building internal program
representations of code structures, the meaning of program symbols, useful
static analyses, and regeneration of valid source code from transformed program
representations. The problem of building and integrating adequate front ends for
conventional languages (Java, C++, PHP, ...) may be of equal difficulty as
building the program transformation system itself because of the complexity of
such languages. To be widely useful, a transformation system must be able to
handle many target programming languages, and must provide some means of
specifying such front ends.

A generalisation of semantic equivalence is the notion of program refinement:
one program is a refinement of another if it terminates on all the initial
states for which the original program terminates, and for each such state it is
guaranteed to terminate in a possible final state for the original program. In
other words, a refinement of a program is more defined and more deterministic
than the original program. If two programs are refinements of each other, then
the programs are equivalent.

A transformation language is a computer language designed to transform some
input text in a certain formal language into a modified output text that meets
some specific goal[clarification needed].

Program transformation systems such as Stratego/XT, TXL, Tom, DMS, and ASF+SDF
all have transformation languages as a major component. The transformation
languages for these systems are driven by declarative descriptions of the
structure of the input text (typically a grammar), allowing them to be applied
to wide variety of formal languages and documents.

Macro languages are a kind of transformation languages to transform a meta
language into specific higher programming language like Java, C++, Fortran or
into lower-level Assembly language.

In the model-driven engineering technical space, there are model transformation
languages (MTLs), that take as input models conforming to a given metamodel and
produce as output models conforming to a different metamodel. An example of such
a language is the QVT OMG standard.

There are also low-level languages such as the Lx family[1] implemented by the
bootstrapping method. The L0 language may be considered as assembler for
transformation languages. There is also a high-level graphical language built on
upon Lx called MOLA.[2]

There are a number of XML transformation languages. These include Tritium, XSLT,
XQuery, STX, FXT, XDuce, CDuce, HaXml, XMLambda, and FleXML.

\secrel{\ru{Основные методы реализации}}\secdown

\secrel{Homoiconicity \ru{Гомоиконичность}}

In computer programming \ru{В программировании},
\termdef{homoiconicity}{homoiconicity}
\ru{\termdef{гомоиконичность}{гомоиконичность}}\note{from the Greek words \ru{от
греческих слов} \emph{homo} meaning the same \ru{обозначает \emph{само-}} and
\ru{и} \emph{icon} meaning representation \ru{обозначает \emph{представление}}}\
is a property of some programming languages \ru{это свойство некоторых языков
программирования} in which the program structure is similar to its syntax
\ru{когда структура программы аналогична ее синтаксису}, and therefore \ru{и
таким образом} the program's internal representation \ru{внутреннее представление
программы} can be inferred by reading the text's layout \ru{может быть выведено
через чтение текстового описания}.\ref{}
If a language is homoiconic \ru{Если язык гомоиконичен}, it means that the
language text \ru{это значит что текст программы на этом языке} has the same
structure \ru{имеет ту же структуру} as its abstract syntax tree \ru{что и ее
синтаксическое дерево} (i.e. the AST and the syntax are \term{isomorphic} \ru{то
есть AST и синтаксис \term{изоморфны}}).
This allows all code in the language to be accessed \ru{Это позволяет иметь
доступ ко всему коду на таком языке} and transformed as data \ru{и
трансформировать его как данные}, using the same representation \ru{используя
одно представление}.

In a homoiconic language \ru{В гомоиконичном языке} the primary representation
of programs \ru{первичное представлние программы} is also a \emph{data
structure} \ru{также само является \emph{структурой данных}} in a primitive type
of the language itself \ru{примитивного для этого языка типа}.
This makes metaprogramming easier \ru{Это свойство языка делает
метапрограммирование проще} than in a language without this property \ru{чем в
языке без него}, since code can be treated as data \ru{так как код
рассматривается как данные}:
reflection in the language \ru{интроспекция/рефлексия в языке} (examining the
program's entities at runtime \ru{обработка элементов программы в рантайме})
depends on a single, homogeneous structure \ru{выполняется на простой гомогенной
структуре данных}, and it does not have to handle several different structures
\ru{и не требует поддержки нескольких разнотипных элементов данных} that would
appear in a complex syntax \ru{необходимой при сложном синтаксисе}.
To put that another way \ru{Другими словами}, homoiconicity is where a program's
source code \ru{гомоиконичность это когда исходный код программы} is written as
a basic data structure \ru{является базовой структурой данных} that the
programming language knows how to access \ru{с которой язык умеет работать
непосредственно}.

A typical, commonly cited example \ru{Типичный, широко упоминаемый пример} is
the programming language \ru{это язык программирования} \lisp, which was created
to be easy \ru{который был создан специально} for lists manipulation \ru{для
манипуляции списками} and where the structure is given by \term{S-expressions}
\ru{в котором структура заданная \term{S-выражениями}} that take the form of
nested lists \ru{принимает форму вложенных списков}. \lisp\ programs are written
in the form of lists \ru{Программы на \lisp\ записивыаются в форме списков}; the
result is that the program can access its own functions and procedures while
running \ru{в результате программа может оперировать собственными функциями и
процедурами в процессе выполнения}, and programmatically reprogram itself on the
fly \ru{и репрограммировать себя на лету}. Homoiconic languages typically
include full support of syntactic macros \ru{Гомоиконичные языки обычно включают
полную поддержку синтаксических макросов} allowing the programmer to express
program transformations \ru{позволяющих программисту выражать трансформации
программ} in a concise way \ru{в сжатой форме}. Examples are the programming
languages \ru{Другие примеры таких языков программирования} $Clojure$ (a
contemporary dialect of \ru{современный диалект} \lisp), $Rebol$ and Refal \ru{и
Рефал}.

\secrel{Интроспекция}

\termdef{Интроспекция}{интроспекция}\ --- представление внутренних структур
языка в виде переменных встроенных типов с возможностью доступа к ним из
программы.

Позволяет во время выполнения просматривать, создавать и изменять определения
типов, стек вызовов, обращаться к переменной по имени, получаемому динамически и
пр.

\begin{itemize}[nosep]
  \item 
Пространство имён \verb|System.Reflection| и тип \verb|System.Type| в .NET;
  \item 
классы \verb|Class|, \verb|Method|, \verb|Field| в \java;
  \item 
представление пространств имен и определений типов через встроенные типы данных
в \py;
  \item 
стандартные встроенные возможности в $Forth$ по доступу к ресурсам виртуальной
машины;
  \item 
получение значения и изменение свойств почти любого из объектов в ECMAScript (с
оговорками).
\end{itemize}

\secrel{Интерпретация произвольного кода}

Интерпретация произвольного кода, представленного в виде строки.

Существует естественным образом во множестве интерпретируемых языков (впервые
функция \verb|eval| была реализована в \lisp, а точнее, непосредственно перед
ним, ставшим её первым реализованным интерпретатором).

Компилятор \href{https://ru.wikipedia.org/wiki/Tiny_C_Compiler}{TinyC} позволяет
``на лету'' компилировать и исполнять код на языке \ci, представленный в виде
строки символов. В
\href{https://ru.wikipedia.org/wiki/%D0%A4%D0%BE%D1%80%D1%82_(%D1%8F%D0%B7%D1%8B%D0%BA_%D0%BF%D1%80%D0%BE%D0%B3%D1%80%D0%B0%D0%BC%D0%BC%D0%B8%D1%80%D0%BE%D0%B2%D0%B0%D0%BD%D0%B8%D1%8F)}{$Forth$}
предусмотрена интерпретация из строки словом \verb|EVALUATE|.

\secup

\secup

\secrel{The Structure of This Book \ru{Структура книги}}

Unlike some other textbooks, this one does not follow a top-down narrative.
\ru{В отличие от большинства учебников, эта книга не следует подходу ``сверху
вниз''.}
Rather it has the flow of a conversation, with backtracking.
\ru{Скорее она имеет форму повествования с возвратами к предыдущим темам.}
We will often build up programs incrementally, just as a pair of programmers
would.
\ru{Мы часто будет строить программы инкрементно, так же как мы бы это делали в
парном программировании.}
We will include mistakes,
\ru{Наш код будет включать ошибки,}
not because I don’t know the answer,
\ru{не потому что мы не знаем правильный ответ,}
but because \emph{this is the best way for you to learn.}
\ru{но потому что \emph{это лучший способ научить вас.}}
Including mistakes makes it impossible for you to read passively:
\ru{Включение намеренных ошибок делает невозможным для вас читать материал
пассивно:}
you must instead engage with the material,
\ru{вы должны взаимодействовать с ним,}
because you can never be sure of the veracity of what you’re reading.
\ru{потому что вы никогда не можете быть уверены в правильности того что вы
читаете.}

At the end, you’ll always get to the right answer.
\ru{В конце концов вы всегда получите правильный ответ.}
However, this non-linear path is more frustrating in the short term
\ru{Тем не меннее, это нелинейное повествование немного раздражающе в
краткосрочной перспективе}
(you will often be tempted to say,
\ru{у вас всегда будет соблазн сказать}
"Just tell me the answer, already\,!
\ru{Скажите же мне наконец ответ\,!}"),
and it makes the book a poor reference guide
\ru{и это также делает эту книгу плохим справочником}
(you can’t open up to a random page and be sure what it says is correct
\ru{вы не можете открыть произвольную страницу и быть уверенным что на ней
написана правда}).
However, that feeling of frustration is the sensation of learning.
\ru{Тем не менее, это чувство разочарования\ --- ощущение обучения.}
I don’t know of a way around it.
\ru{Я не знаю другого способа.}

\bigskip
At various points you will encounter this:
\ru{В некоторых местах вы встретите следующие выделения:}

\Exercise{
This is an exercise. Do try it.\\
\ru{Это упражнение. Попробуйте это сделать.}
}

This is a traditional textbook exercise.
\ru{Это традиционное для учебников упражнение.}
It’s something you need to do on your own.
\ru{Это то, что вам нужно сделать по своему усмотрению.}
If you’re using this book as part of a course,
\ru{Если вы используете эту книгу как часть курса,}
this may very well have been assigned as homework.
\ru{это упражнение хорошо задавать как домашнюю работу.}
In contrast, you will also find exercise-like questions that look like this:
\ru{В противоположность этому вы также можете найти подобные вопросы, выделенные
как}

\DoNow{
There’s an activity here\,! Do you see it\,?\\
\ru{Здесь предполагаются немедленные действия\,! Вы видите это\,?}
}

When you get to one of these, \emph{stop}.
\ru{Когда вы доберетесь до одного из этих блоков, \emph{остановитесь}.}
Read, think, and formulate an answer before you proceed.
\ru{Прочитайте, подумайте, сформулируйте ответ перед тем как продолжить чтение.}
You must do this because this is actually an \emph{exercise},
\ru{Вы должны сделать это потому что это действительно \emph{упражнение},}
but the answer is already in the book
\ru{но ответ уже есть в книге}\ ---
most often in the text immediately following
\ru{чаще всего в тексте непосредственно после упражнения}
(i.e., in the part you’re reading right now
\ru{т.е. в части которую вы сейчас читаете})\ ---
or is something you can determine for yourself by running a program.
\ru{или это что-то, что вы можете получить самостоятельно, запустив программу.}
If you just read on,
\ru{Если вы просто продолжите читать,}
you’ll see the answer without having thought about it
\ru{то вы увидете ответ без его обдумывания}
(or not see it at all, if the instructions are to run a program
\ru{или не увидите его вообще, если это инструкции по запуску программы}),
so you will get to neither
\ru{так что вы ни}
(a) test your knowledge, nor \ru{проверите свои знания, ни}
(b) improve your intuitions. \ru{улучшите свое понимание.}
In other words, these are additional, explicit attempts to encourage active
learning.
\ru{Другими словами, это дополнительные, явные попытки стимулировать ваше
активное обучение.}
Ultimately, however, I can only encourage it;
\ru{В конце концов, я могу только поощрять вас работать;}
it’s up to you to practice it.
\ru{решение применять это или нет остается за вами.}

\secrel{The Language of This Book\\\ru{Язык программирования используемый в
книге}}

The main programming language used in this book is
\ru{Язык программирования используемый в книге}\ --- 
\href{http://www.racket-lang.org/}{\racket}.
Like with all operating systems, however,
\ru{Аналогично операционным системам,}
\racket\ actually supports a host of programming languages,
\ru{\racket-система является исполняющей средой для целого ряда языков
программирования,}
so you must tell \racket\
\ru{так что вы должны указать \racket у}
\emph{which} language you’re programming in.
\ru{\emph{на каком} языке вы программируете.}
You inform the Unix shell by writing a line like
\ru{Например, в Unix вы пишете строку типа}

\begin{verbatim}
#!/bin/sh
\end{verbatim}
at the top of a script;
\ru{в первой строке shell-скрипта;}
you inform the browser by writing, say,
\ru{вы указываете веб-браузеру тип документа, добавляя заголовок}

\begin{verbatim}
<!DOCTYPE HTML PUBLIC "-//W3C//DTD HTML 4.01//EN" ...>
\end{verbatim}

Similarly, \racket\ asks that you declare which language you will be using.
\ru{Аналогично, \racket\ требует от вас указать какой язык вы будете
использовать.}
\racket\ languages can have the same parenthetical syntax as \racket\ but with a
different semantics;
\ru{Диалекты языков \racket\ имеют тот же скобочный синтаксис, что и сам
\racket, но другую семантику;}
the same semantics but a different syntax;
\ru{ту же семантику но другой синтаксис;}
or different syntax and semantics.
\ru{или различные синтаксис и семантику.}
Thus every \racket\ program
\ru{Так что каждая программа, которую может выполнять \racket-система,}
begins with \#lang followed by the name of some language:
\ru{начинается со строки \#lang за которой следует имя диалекта языка:}
by default, it’s \racket\ \ru{по умолчанию, это оригинальный \racket\ }
written as \ru{указыватся как} \verb|racket|).
In this book we’ll almost always use the language\note{In DrRacket v.5.3,
go to\\
Language, then Choose Language, and select ``Use the language declared in
the source''.}
\ru{В этой книге мы почти всегда будем использовать диалект}\note{\ru{В DrRacket
v.6.6, выберите меню\\\menu{Язык > Выбрать язык\ldots > Start your program with
\#lang to specify the desired dialect}.}}
\begin{verbatim}
#lang plai-typed
\end{verbatim}
When we deviate we’ll say so explicitly,
\ru{Когда мы будем отклоняться от этого правила, это будет указано особо,}
so unless indicated otherwise, put
\ru{так что если не указано иное, добавляйте заголовок}
\verb|#lang plai-typed|
at the top of every file
\ru{в начало каждого файла программы}
(and assume I’ve done the same
\ru{предполагается что я тоже это
сделал})\note{В DrRacket v.6.6 требуется установить расширение
plai-typed:\\\menu{Файл>Install package\ldots>Package
Source:>\url{github://github.com/mflatt/plai-typed/master}>Install>\ldots>Закрыть}}.

The \termdef{Typed PLAI}{Typed PLAI}\ language differs from traditional \racket\
most importantly by being statically typed.
\ru{Язык \term{Typed PLAI}\ отличается от традиционного \racket\ в основном
\emph{статической типизацией}.}
It also gives you some useful new constructs:
\ru{Он также дает вам некоторые новые полезные конструкции:}
\verb|define-type| \ru{определение-типа}, \verb|type-case| \ru{выбор-по-типу},
and \verb|test|\note{There are additional commands for controlling the output
of testing, for instance. \ru{Также существуют дополнительные команды для
управления выводом тестов.} Be sure to read the documentation for the language.
\ru{Обязательно прочитайте документацию для языка.}
In DrRacket v.5.3, go to \menu{Help>Help Desk}, and in the Help Desk search bar,
type \menu{plai-typed}. \ru{В DrRacket v.6.6 идите в меню \menu{Help>Help
Desk>plai-typed}.}}
Here’s an example of each in use.
\ru{Вот примеры использования каждого из них.} 
We can introduce new datatypes
\ru{Мы можем создавать новые типы данных\note{запустить программу можно нажав
\keys{Ctrl+R}}}:
\lst{1/3/p8_1.rkt}
You can roughly think of this as analogous to the following in Java:
\ru{Вы можете примерно понять идею в терминах языка \java:}
an abstract class \term{абстрактный класс} \verb|MisspelledAnimal| and two
concrete sub-classes \ru{и два конкретизирующих подкласса}\ \verb|caml|
\ru{верблд} and \verb|yacc| \ru{якк},
each of which has one numeric constructor argument named
\ru{каждый из которых имеет конструктор с числовым аргументом}
\verb|humps| \ru{горбы} and \verb|height| \ru{высота}, respectively
\ru{соответственно}.

In this language, we construct instances as follows:
\ru{На этом языке мы строим экземпляры классов следующим
образом:\note{выполните это \emph{в командной строке \racket}, и посмотрите на
класс результата}}
\lst{1/3/p8_2.rkt}
As the name suggests, \ru{Как следует из названия,} \verb|define-type| creates
a type of the given name \ru{создает тип с заданным именем}.
We can use
this when, for instance, binding the above instances to names:
\ru{Мы можем это использовать например при связывании экземпляров с именами:}
\lst{1/3/p8_3.rkt}
In fact you don’t need these particular type declarations,
\ru{Фактически вам не нужны эти частные определения типов,} 
because \term{Typed PLAI} will infer types for you here and in many other cases.
\ru{так как \term{Typed PLAI} в этом и других случаях будет сам делать для вас
\term{вывод типов}.}
Thus you could just as well have written
\ru{Так что вы можете написать короче}
\lst{1/3/p8_4.rkt}
but we prefer to write explicit type declarations
\ru{но мы предпочтем писать полные объявления типов}
as a matter of both discipline and comprehensibility when we return to programs
later.
\ru{с точки зрения как
дисциплины, так и усвояемости, когда мы вернемся к программам позже.}

The type names can even be used recursively, as we will see repeatedly in this
book (for instance, section \ref{sec2_4}).
\ru{Имена типов даже могут быть использованы рекурсивно, как мы увидим
несколько позже в этой книге (например в разделе \ref{sec2_4}).}

The language provides a pattern-matcher for use when writing expressions, such
as a function’s body:
\ru{Язык предоставляет pattern-matcher для использования при написании
выражений, таких как тело функции:}
\lst{1/3/p9_1.rkt}
In the expression \ru{Например в выражении} (>= humps 2), for instance,
\verb|humps| is the name given to whatever value was given as the argument to
the constructor \ru{имя humps соответствует любому значению, данному как
аргумент для конструктора} \verb|caml|.

Finally, you should write test cases, ideally before you’ve defined your
function, but also afterwards to protect against accidental changes:
\ru{И наконец, вы должны написать тесты, в идеале до того как вы ее реализовали,
или хотя бы после, чтобы защититься от внезапных несоответствий в ее поведении
при внесении изменений в код:}
\lst{1/3/p9_2.rkt}
When you run the above program, the language will give you verbose output telling
you both tests passed.
\ru{При запуске тестов язык даст вам подробный отчет, что оба теста успешно
пройдены.}
Read the documentation to learn how to suppress most of these messages.
\ru{Прочитайте документацию, чтобы узнать как подавить вывод большей части
этих сообщений.}

Here’s something important that is obscured above.
\ru{Вот еще кое-что важное, что было неясно выше.}
We’ve used the same name,
\ru{Мы использовали одно и то же имя,}
humps (and height), in \emph{both} the datatype definition and in the fields of
the patternmatch.
\ru{и в определении типа данных, и в полях объекта при проверке совпадения
шаблонов.}
This is absolutely unnecessary because the two are related by \emph{position},
not name.
\ru{Это совершенно необязательно, так как каждая пара связана по \emph{позиции},
а не по имени.}
Thus, we could have as well written the function as
\ru{Так что мы могли бы также написать эту функцию как}
\lst{1/3/p9_3.rkt}
Because each $h$ is only visible in the case branch in which it is introduced,
the two $h$s do not in fact clash.
\ru{Так как каждый $h$ виден только в той case-секции, где он используется,
два $h$ фактически не сталкиваются.}
You can therefore use convention and readability to
dictate your choices.
\ru{Таким образом вы можете использовать соглашения по оформлению кода для
улучшения читаемости, и диктовать свой выбор.}
In general, it makes sense to provide a long and descriptive name
\ru{В общем, имеет смысл использовать длинные описательные имена}
when defining the datatype\note{because you probably won’t use
that name again},
\ru{при определении типа данных\note{\ru{потому что вы возможно больше никогда
не будете использовать это имя снова}},}
but shorter names in the \class{type-case} because you’re likely to use use
those names one or more times.
\ru{и короткие имена в \class{type-case}, где они обычно используются несколько
раз.}

I did just say you’re unlikely to use the field descriptors introduced in the
datatype definition, but you can.
\ru{Также я хочу упомянуть декрипторы полей класса, которые вы возможно
захотите использовать.}
The language provides \term{selectors} to extract fields without the need for
pattern-matching: e.g., \class{caml-humps}.
\ru{Язык предоставляет \term{селекторы} для получения значений полей без
необходимости использовать pattern-matching, например \class{caml-humps}.}
Sometimes,
it’s much easier to use the selector directly rather than go through the
pattern-matcher.
\ru{Иногда намного проще использовать селектор, чем возиться с мэтчингом
шаблонов.}
It often isn’t, as when defining \class{good?}\ above,
\ru{Часто это не так, как в случае определения \class{good?},}
but just to be clear, let’s write it without pattern-matching:
\ru{но для ясности давайте напишем без pattern-matching:}
\lst{1/3/p9_4.rkt}

\DoNow{
What happens if you mis-apply functions to the wrong kinds of values\,?\\
\ru{Что произойдет, если вы ошибочно примените функции к неправильным типам
значений\,?}

For instance, what if you give the \class{caml}\ constructor a string\,?\\
\ru{Например, что если вы дадите конструктору \class{caml}\ строковый
аргумент\,?}

What if you send a number into each version of \class{good?}\ above\,?\\
\ru{Что произойдет если вы пошлете число к каждой версии \class{good?}\
описанных выше\,?} }

\secup

\input{2/parsing}
\secrel{Interpreting with Environments\\
\ru{Интерпретация со средами}}

Now we can tackle the interpreter \ru{Теперь мы можем реализовать
интерпретатор}. One case is easy \ru{Одна ветка простая}, but we should revisit
all the others \ru{но другие нам нужно будет обсудить отдельно}:
\lsts{6/2/1.rkt}{rkt}
The arithmetic operations are easiest \ru{Арифметические опреации проще всего}.
Recall that before \ru{Как мы сказали ранее}, the interpreter recurred
\ru{интерпретатор вызывается рекурсивно} without performing any new
substitutions \ru{без выполнения новых подстановок}. As a result \ru{в
результате}, there are no new deferred substitutions to perform either \ru{нет
никаких отсроченных подстановок которые нужно выполнить}, which means the
environment does not change \ru{что значит что среда не меняется}:
\lsts{6/2/2.rkt}{rkt}

Now let’s handle identifiers \ru{Теперь обработаем идентификаторы}. Clearly
\ru{Ясно что}, encountering an identifier is no longer an error
\ru{неопределенный идентификатор больше не является ошибкой}:
this was the very motivation for this change \ru{это было одной из причин
изменения принципа интерпретации}. Instead, we must look up its value in the
directory \ru{Вместо этого мы должны выполнить поиск его значения в каталоге}:
\lsts{6/2/3.rkt}{rkt}

\DoNow{
Implement lookup.\\\ru{Реализуйте поиск.}
}

Finally, application \ru{И наконец, применение}. Observe that in the
substitution interpreter \ru{Обратите внимание что в интерпретатор с
подстановкой}, the only case that caused new substitutions to occur was
application \ru{единственным случаем вызывающим новые подстановки было
применение}. Therefore, this should be the case that constructs bindings
\ru{Таким образом, это должно быть случаем, когда создаются связки}. Let’s first
extract the function definition \ru{Давайте сначала выделим определение
функции}, just as before \ru{также как и раньше}:
\lsts{6/2/4.rkt}{rkt}
Previously, we substituted, then interpreted \ru{Раньше мы сначала подставляли,
а потом интерпретировали}. Because we have no substitution step \ru{Так как
теперь у нас нет шага подстановки}, we can proceed with interpretation \ru{мы
можем продолжать интерпретацию}, so long as we record the deferral of
substitution \ru{как только мы записали отсрочку подстановки}.
\lsts{6/2/5.rkt}{rkt}
That is \ru{То есть}, the set of function definitions remains unchanged
\ru{множество определений функций остается неизменным}; we’re interpreting the
body of the function, as before \ru{мы интерпретируем тело функции как и
раньше}; but we have to do it in an environment that binds the formal parameter
\ru{но мы должны делать это в среде, связывающей формальный параметр}. Let’s now
define that binding process \ru{Давайте определим процесс связывания}:
\lsts{6/2/6.rkt}{rkt}
the name being bound is the formal parameter \ru{имя было связано как формальный
параметр} (the same name that was substituted for, before \ru{это то же самое
имя которое подставлялось ранее}). It is bound to the result of interpreting the
argument \ru{Оно связано с результатом интерпретации аргумента} (because we’ve
decided on an eager application semantics \ru{потому что мы решили использовать
семантику жадного применения}). And finally, this extends the environment we
already have \ru{И наконец, оно расширяет среду которая у нас уже есть}.
Type-checking this helps to make sure \ru{Контроль типов помогает нам убедиться
что} we got all the little pieces right \ru{мы совместили все эти маленькие
кусочки правильно}.

Once we have a definition for lookup \ru{Как только мы получаем определение
для поиска}, we’d have a full interpreter \ru{мы получаем полный интерпретатор}.
So here’s one \ru{Вот он}:
\lsts{6/2/7.rkt}{rkt}

Observe that looking up a free identifier still produces an error \ru{Обратите
внимание что поиск свободного идентификатора все еще вызывает ошибку}, but it
has moved from the interpreter \ru{но она переместилась из интерпретатора}\ ---
which is by itself unable to determine whether or not an identifier is free
\ru{который сам по себе не способен определить свободен ли идентификатор}\ ---
to \ru{в} \verb|lookup|, which determines this based on the content of the
environment \ru{который определяет это на основе содержимого среды}.

Now we have a full interpreter \ru{Теперь у нас есть полный интерпретатор}. You
should of course test it make sure it works as you’d expect \ru{Естественно вы
должны протестировать его чтобы убедиться что он работает как вы ожидаете}.
For instance, these tests pass \ru{Например, с помощью этих тестов}:
\lsts{6/2/8_1.rkt}{rkt}
\lsts{6/2/8_2.rkt}{rkt}
\lsts{6/2/8_3.rkt}{rkt}
So we’re done, right \ru{Так что все сделано, правильно}\,?

\DoNow{
Spot the bug.\\\ru{Найдите ошибку.}
}

\input{4/desugaring}
 
\input{5/func}
\secrel{From Substitution to Environments\\
\ru{От подстановки к ср\'{е}дам}}\secdown

Though we have a working definition of functions \ru{Теперь, когда мы имеем
работающее определение функций}, you may feel a slight unease about it \ru{вы
можете почувствовать легкое беспокойство по этому поводу}. When the interpreter
sees an identifier \ru{Когда интерпретатор видит идентификатор}, you might have
had a sense \ru{вы возможно имеете ощущение} that it needs to \ru{что он должен}
``look it up'' \ru{``поискать`` его}. Not only did it not look up anything
\ru{Мало того, что он ничего не ищет}, we defined its behavior to be an error
\ru{мы определили его поведение с заложенной ошибкой}\,! While absolutely
correct \ru{Это совершенно точно}, this is also a little surprising \ru{и при
этом немного удивительно}. More importantly, \ru{Важнее то, что} we write
interpreters to \emph{understand} and \emph{explain} languages \ru{мы пишем
интерпретаторы для \emph{понимания} и \emph{объяснения} языков}, and this
implementation might strike you as not doing that \ru{и эта реализация может
показаться вам не делающей это}, because it doesn’t match our intuition
\ru{потому что она не соответствует нашим интуитивным предположениям}.

There’s another difficulty with using substitution \ru{Также есть другая
сложность с использованием подстановки}, which is the number of times we
traverse the source program \ru{это количество проходов по программе}.
It would be nice to have \ru{Было бы хорошо} to traverse only those parts of the
program \ru{проходить только по тем частям программы} that are actually
evaluated \ru{которые реально вычисляются}, and then, only when necessary \ru{и
притом только когда это необходимо}. But substitution traverses everything
\ru{Но подстановка обходит все}\ --- unvisited branches of conditionals, for
instance \ru{например неиспользуемые ветви условий}\ --- and forces the program
to be traversed \ru{и заставляет обходить программу дважды:} once for
substitution \ru{один раз для подстановки} and once again for interpretation
\ru{и еще раз для интерпретации}.

\Exercise{
Does substitution have implications for the time complexity of evaluation\,?\\
\ru{Какие последствия имеет подстановка с точки зрения временн\'{о}й сложности
вычисления\,?}
}

There’s yet another problem with substitution \ru{И есть еще одна проблема с
подстановкой}, which is that it is defined in terms of representations of the
program source \ru{она определена в терминах представления исходного кода
программы}. Obviously, our interpreter has and needs access to the source
\ru{Очевидно что наш интерпретатор имеет и принципиально должен иметь доступ к
исходному коду}, to interpret it \ru{для его интерпретации}. However, other
implementations \ru{В то же время, другие реализации языка}\ --- such as
compilers \ru{такие как компиляторы}\ --- have no need to store it for that
purpose \ru{не требуют хранить исходный код для этих целей}. It would be nice to
employ a mechanism \ru{Было бы хорошо реализовать механизм} that is more
portable across implementation strategies \ru{более переносимый между
стратегиями реализации}.
\note{Compilers might store versions of or information about the source for
other reasons, such as reporting runtime errors, and JITs may need it to
re-compile on demand.}
\note{\ru{Компиляторы могут хранить версии или информацию об исходном коде для
других целей, таких как вывод отчетов об ошибках времени выполнения, или эта
информация может использоваться JIT-компилятором для рекомпиляции по
требованию.}}

\secrel{Introducing the Environment\\\ru{Введение понятия ``Среда''}}

The intuition that addresses the first concern is to have \ru{Во-первых
интуитивно хочется иметь} the interpreter ``look up'' an identifier in some sort
of directory \ru{интерпретатор с процедурой ``поиска'' идентификатора в
некотором подобии каталога}. The intuition that addresses the second concern
\ru{Во-вторых интуиция подсказывает} is to \emph{defer} the substitution
\ru{\emph{избавиться} от подстановки}. Fortunately, these converge nicely in a
way that also addresses the third \ru{К счастью выполнение этих двух целей
сводится к третьей}. The directory records the \emph{intent to substitute}
\ru{Каталог хранит \emph{намерение замены}}, without actually rewriting the
program source \ru{без реального переписывания исходной программы}; by recording
the intent \ru{храня намерение замены}, rather than substituting immediately
\ru{без выполнения самой замены}, we can defer substitution \ru{мы можем
избавиться от замены}; and the resulting data structure \ru{и результирующая
структура}, which is called an \termdef{environment}{environment} \ru{которая
называется \termdef{среда}{среда}}, avoids the need for source-to-source
rewriting \ru{избавляет от необходимости source-to-source переписывания} and
maps nicely to low-level machine representations \ru{и хорошо отображается на
низкоуровневое машинное представление}. Each name association in the environment
\ru{Каждая ассоциация имени в среде} is called a \termdef{binding}{binding}
\ru{называется \termdef{связка}{связка}}.

Observe carefully \ru{Аккуратно рассмотрим} that what we are changing \ru{что мы
поменяли} is the \emph{implementation strategy} \ru{в \emph{стратегии
реализации}} for the programming language \ru{языка программирования}, \emph{not
the language itself} \ru{но \emph{не в самом языке}}. Therefore \ru{Таким
образом}, none of our datatypes for representing programs should change
\ru{никаких изменений не должно быть в наших типах данных представляющих
програму}, nor even should the answers \ru{также как и в результатах} that the
interpreter provides \ru{которые возвращает интерпретатор}. As a result \ru{В
результате}, we should think of the previous interpreter \ru{мы должны
рассматривать предыдущий интерпретатор} as a ''reference implementation''
\ru{как ``эталонную реализацию''} that the one we’re about to write should match
\ru{с поведением которой должно совпадать все что мы напишем}. Indeed \ru{В
самом деле}, we should create a generator \ru{нам следовало бы создать
генератор} that creates lots of tests \ru{который создает множество тестов},
runs them through both interpreters \ru{прогоняет их на обоих интерпретаторах},
and makes sure their answers are the same \ru{и подтверждает что их результаты
одинаковы}. Ideally \ru{В идеале}, we should \emph{prove} that the two
interpreters behave the same \ru{нам нужно \emph{доказать} что эти два
интерпретатора \term{эквивалентны}}, which is a good topic for advanced study
\ru{что является хорошей темой для отдельного исследования}.
\note{One subtlety is in defining precisely what “the same” means, especially
with regards to failure.}
\note{\ru{Одна из тонкостей\ --- что точно обозначает фраза ``то же
самое поведение'', особенно в случае ошибочных ситуаций.}}

Let’s first define our environment data structure \ru{Для начала давайте
определим структуру данных для нашей среды}.
An environment is a list of pairs of names associated with\ldots what\,?
\ru{Среда это список пар имен, ассоциированных с\ldots чем\,?}

\DoNow{
A natural question to ask here might be what the environment maps names to. But
a better, more fundamental, question is: How to determine the answer to the
“natural” question\,?\\
\ru{Здесь возникает естественный вопрос, на что именно среда отображает имена.
Но лучший, более фундаментальный вопрос: Как определить ответ на
``естественный'' вопрос\,?}
}

Remember that our environment was created to defer substitutions \ru{Вспомним,
что наша среда была создана чтобы избавиться от подстановки}. Therefore, the
answer lies in substitution \ru{Таким образом, ответ заключается в подстановке}.
We discussed earlier \ru{Ранее мы обсудили} \ref{waitmore} that we want
substitution to map names to answers \ru{что мы хотим чтобы подстановка
отображала имена на результаты вычислений}, corresponding to an eager function
application strategy \ru{в соответствие с жадной стратегией применения функций}.
Therefore, the environment should map names to answers \ru{Таким образом,
среда должна отображать имена на результаты вычислений}.
\lsts{6/1/1.rkt}{rkt}
\secrel{Interpreting with Environments\\
\ru{Интерпретация со средами}}

Now we can tackle the interpreter \ru{Теперь мы можем реализовать
интерпретатор}. One case is easy \ru{Одна ветка простая}, but we should revisit
all the others \ru{но другие нам нужно будет обсудить отдельно}:
\lsts{6/2/1.rkt}{rkt}
The arithmetic operations are easiest \ru{Арифметические опреации проще всего}.
Recall that before \ru{Как мы сказали ранее}, the interpreter recurred
\ru{интерпретатор вызывается рекурсивно} without performing any new
substitutions \ru{без выполнения новых подстановок}. As a result \ru{в
результате}, there are no new deferred substitutions to perform either \ru{нет
никаких отсроченных подстановок которые нужно выполнить}, which means the
environment does not change \ru{что значит что среда не меняется}:
\lsts{6/2/2.rkt}{rkt}

Now let’s handle identifiers \ru{Теперь обработаем идентификаторы}. Clearly
\ru{Ясно что}, encountering an identifier is no longer an error
\ru{неопределенный идентификатор больше не является ошибкой}:
this was the very motivation for this change \ru{это было одной из причин
изменения принципа интерпретации}. Instead, we must look up its value in the
directory \ru{Вместо этого мы должны выполнить поиск его значения в каталоге}:
\lsts{6/2/3.rkt}{rkt}

\DoNow{
Implement lookup.\\\ru{Реализуйте поиск.}
}

Finally, application \ru{И наконец, применение}. Observe that in the
substitution interpreter \ru{Обратите внимание что в интерпретатор с
подстановкой}, the only case that caused new substitutions to occur was
application \ru{единственным случаем вызывающим новые подстановки было
применение}. Therefore, this should be the case that constructs bindings
\ru{Таким образом, это должно быть случаем, когда создаются связки}. Let’s first
extract the function definition \ru{Давайте сначала выделим определение
функции}, just as before \ru{также как и раньше}:
\lsts{6/2/4.rkt}{rkt}
Previously, we substituted, then interpreted \ru{Раньше мы сначала подставляли,
а потом интерпретировали}. Because we have no substitution step \ru{Так как
теперь у нас нет шага подстановки}, we can proceed with interpretation \ru{мы
можем продолжать интерпретацию}, so long as we record the deferral of
substitution \ru{как только мы записали отсрочку подстановки}.
\lsts{6/2/5.rkt}{rkt}
That is \ru{То есть}, the set of function definitions remains unchanged
\ru{множество определений функций остается неизменным}; we’re interpreting the
body of the function, as before \ru{мы интерпретируем тело функции как и
раньше}; but we have to do it in an environment that binds the formal parameter
\ru{но мы должны делать это в среде, связывающей формальный параметр}. Let’s now
define that binding process \ru{Давайте определим процесс связывания}:
\lsts{6/2/6.rkt}{rkt}
the name being bound is the formal parameter \ru{имя было связано как формальный
параметр} (the same name that was substituted for, before \ru{это то же самое
имя которое подставлялось ранее}). It is bound to the result of interpreting the
argument \ru{Оно связано с результатом интерпретации аргумента} (because we’ve
decided on an eager application semantics \ru{потому что мы решили использовать
семантику жадного применения}). And finally, this extends the environment we
already have \ru{И наконец, оно расширяет среду которая у нас уже есть}.
Type-checking this helps to make sure \ru{Контроль типов помогает нам убедиться
что} we got all the little pieces right \ru{мы совместили все эти маленькие
кусочки правильно}.

Once we have a definition for lookup \ru{Как только мы получаем определение
для поиска}, we’d have a full interpreter \ru{мы получаем полный интерпретатор}.
So here’s one \ru{Вот он}:
\lsts{6/2/7.rkt}{rkt}

Observe that looking up a free identifier still produces an error \ru{Обратите
внимание что поиск свободного идентификатора все еще вызывает ошибку}, but it
has moved from the interpreter \ru{но она переместилась из интерпретатора}\ ---
which is by itself unable to determine whether or not an identifier is free
\ru{который сам по себе не способен определить свободен ли идентификатор}\ ---
to \ru{в} \verb|lookup|, which determines this based on the content of the
environment \ru{который определяет это на основе содержимого среды}.

Now we have a full interpreter \ru{Теперь у нас есть полный интерпретатор}. You
should of course test it make sure it works as you’d expect \ru{Естественно вы
должны протестировать его чтобы убедиться что он работает как вы ожидаете}.
For instance, these tests pass \ru{Например, с помощью этих тестов}:
\lsts{6/2/8_1.rkt}{rkt}
\lsts{6/2/8_2.rkt}{rkt}
\lsts{6/2/8_3.rkt}{rkt}
So we’re done, right \ru{Так что все сделано, правильно}\,?

\DoNow{
Spot the bug.\\\ru{Найдите ошибку.}
}

\secrel{Deferring Correctly\\
\ru{Правильная отсрочка}}

Here’s another test \ru{Вот другой тест}:
\lsts{6/3/1.rkt}{rkt}
In our interpreter \ru{В нашем интерпретаторе}, this evaluates to \ru{он
вычисляется в} 7. Should it \ru{Правильно ли это}\,?

Translated into \ru{При переводе на} \racket, this test corresponds to the
following two definitions and expression \ru{этот тест соответствует следующим
двум определениям и выражению}:
\lsts{6/3/2.rkt}{rkt}

What should this produce \ru{Что он должен вычислить}\,? \verb|(f1 3)|
substitutes \ru{подставляет} \verb|x| with \verb|3| in the body of
\ru{в теле} \verb|f1|, which then invokes \ru{что вызывает} \verb|(f2 4)|.
But notably, in \ru{Но заметим что} \verb|f2|, the identifier \ru{идентификатор}
\verb|x| is \emph{not bound} \ru{\emph{не связан}}\,! Sure enough, \racket\ will
produce an error \ru{Будьте уверены, \racket\ выдаст ошибку}.

In fact, so will our substitution-based interpreter \ru{Фактически то же
самое сделает наш интерпретатор с подстановкой}\,!

Why does the substitution process result in an error \ru{Почему процесс
подстановки приведет к ошибке}\,? It’s because \ru{Потому что}, when we replace
the representation of \ru{когда мы заменяем представление} \verb|x| with the
representation of \ru{на предславление} \verb|3| in the representation of \ru{в
представлении} \verb|f1|, we do so in \ru{мы это делаем только в} \verb|f1|
only.
\note{This “the representation of” is getting a little annoying, isn’t it\,?
Therefore, I’ll stop saying that, but do make sure you understand why I had to
say it. It’s an important bit of pedantry.}
\note{\ru{Это ``представление'' несколько надоедает, не так ли\,? Так что я
перестану говорить это, но убедитесь что понимаете почему я должен его говорить.
Это важный элемент педантизма.}}
(Obviously \ru{Очевидно}: \verb|x| is \ru{это параметр} \verb|f1|’s parameter;
even if another function had a parameter named \ru{даже если другая функция
имеет параметр} \verb|x|, that’s a \emph{different} \ru{это \emph{другой}}
\verb|x|.) Thus \ru{Так что}, when we get to evaluating the body of \ru{когда мы
вычисляем тело} \verb|f2|, its \ru{ее} \verb|x| hasn’t been substituted \ru{не
подставляется}, resulting in the error \ru{что приводит к ошибке}.

What went wrong when we switched to environments \ru{Что пошло не так, когда мы
переключились на среды}\,? Watch carefully \ru{Смотрите внимательнее}:
this is subtle \ru{это тонкая штука}. We can focus on applications \ru{Мы можем
сфокусироваться на применениях}, because only they affect the environment
\ru{потому что только они влияют на среду}. When we substituted the formal for
the value of the actual \ru{Когда мы заменили формальный параметр актуальным
значением}, we did so \ru{мы это сделали} by \emph{extending the current
environment} \ru{через \emph{расширение текущей среды}}. In terms of our example
\ru{В терминах нашего примера}, we asked the interpreter \ru{мы попросили
интерпретатор} to substitute not only \ru{заменить не только подстановку}
\verb|f2|’s substitution in \ru{в теле} \verb|f2|’s body, but also the current
ones \ru{но также и текущие вхождения} (those for the caller \ru{для
вызывающей}, \verb|f1|), and indeed all past ones as well \ru{а также вообще все
последующие}. That is, the environment only grows \ru{Так что среда только
растет}; it never shrinks \ru{она никогда не сокращается}.

Because we agreed that environments are only an alternate implementation
strategy for substitution \ru{Так как мы договорились что среды единственная
альтернативная стратегия реализации для подстановки}\ --- and in particular
\ru{в частности}, that the language’s meaning should not change \ru{семантика
языка не должна меняться}\ --- we have to alter the interpreter \ru{мы должны
поправить интерпретатор}. Concretely \ru{Конкретно}, we should not ask it to
carry around all past deferred substitution requests \ru{мы не должны просить
его хранить все отложенные запросы на подстановку}, but instead make it start
afresh for every new function \ru{вместо этого он должен начинать начисто для
каждой новой функции}, just as the substitution-based interpreter does \ru{также
как делает интерпретатор на подстановке}. This is an easy change \ru{Это простая
модификация}:
\lsts{6/3/3.rkt}{rkt}

Now we have truly reproduced the behavior of the substitution interpreter
\ru{Теперь мы на самом деле воспроизвели поведение подстановочного
интерпретатора}.
\note{In case you’re wondering how to write a test case that catches errors,
look up test/exn.}
\note{\ru{Если вам не понятно как написать тест который ловит ошибки,
посмотрите} test/exn}

\secrel{Scope\\\ru{Область видимости}}

The broken environment interpreter above \ru{Сломанный интерпретатор на средах
выше} implements what is known as \termdef{dynamic scope}{dynamic scope}
\ru{реализовывал то что называется \termdef{динамическая область
видимости}{динамическая область видимости}}. This means \ru{Это значит что} the
environment accumulates bindings as the program executes \ru{среда аккумулирует
связки в процессе выполнения программы}. As a result \ru{В результате}, whether
an identifier is even bound \ru{определение того является ли идентификатор
связанным} depends on the history of program execution \ru{зависит от истории
исполнения программы}. We should regard this unambiguously \ru{Мы должны
рассматривать это} as a flaw of programming language design \ru{как недостаток
дизайна языка программирования}. It adversely affects all tools \ru{Это
отрицательно влияет на все инструменты} that read and process programs
\ru{которые читают и обрабатывают программы}: compilers, IDEs, and humans
\ru{компиляторы, IDE и человеки}.

In contrast \ru{Наоборот}, substitution \ru{подстановка}\ --- and environments,
done correctly \ru{правильно реализованные среды}\ --- give us \termdef{lexical
scope}{lexical scope} or \termdef{static scope}{static scope} \ru{дают нам
\termdef{лексеческую}{лексическая область видимости} или
\termdef{статическую}{статическая область видимости} области видимости}.
``Lexical'' in this context means \ru{``Лексическая'' в этом контексте значит}
``as determined from the source program'' \ru{``определенная из исходного кода
программы''}, while ``static'' in computer science means \ru{в то время как
``статическая'' в информатике значит} ``without running the program'' \ru{``без
запуска программы''}, so these are appealing to the same intuition \ru{эти
названия следуют той же идее}. When we examine an identifier \ru{Когда мы
встречаем идентификатор}, we want to know two things \ru{мы хотим знать две
вещи}: (1) Is it bound \ru{Связан ли он}\,? (2) If so, where \ru{и если связан,
то где}\,? By “where” we mean \ru{``Где'' мы имеем в виду}: if there are
multiple bindings for the same name \ru{если существуют множественные связки для
одного имени}, which one governs this identifier \ru{какая из них контролирует
этот идентификатор}\,? Put differently \ru{Другими словами}, which one’s
substitution \ru{какая из подстановок} will give a value to this identifier
\ru{даст значение для этого идентификатора}\,? In general \ru{В общем}, these
questions cannot be answered statically \ru{на эти вопросы нельзя дать
статические ответы} in a dynamically-scoped language \ru{в языке с динамическими
областями видимости}: so your IDE \ru{так что ваша IDE}, for instance
\ru{например}, cannot overlay arrows to show you this information \ru{не может
расставить стрелки чтобы показать эту информацию} (as Dr\racket\ does \ru{как
это делает Dr\racket}).
\note{A different way to think about it is that in a dynamically-scoped
language, the answer to these questions is the same for all identifiers, and it
simply refers to the dynamic environment. In other words, it provides no useful
information.}
\note{\ru{Другой способ думать об этом для языка с динамической областью
видимости\ --- ответ на вопрос один для всех идентификаторов: так как указано
в динамической среде. То есть он не дает никакой полезной информации.}}
Thus \ru{Таким образом}, even though the rules of scope become more complex
\ru{по мере того как правила области видимости становятся сложнее} as the space
of names becomes richer \ru{и пространство типов имен становится богаче} (e.g.,
objects, threads, etc. \ru{например объекты, нити, и т.д.}), we should always
strive to preserve the spirit of static scoping \ru{мы всегда должны стремиться
сохранить дух статической области видимости}.

\secdown
\secrel{How Bad Is It\,?\\\ru{Насколько это плохо\,?}}

You might look at our running example \ru{Вы можете посмотреть на наш рабочий
пример} and wonder whether we’re creating a tempest in a teapot \ru{и удивиться
почему мы создаем бурю в стакане воды}. In return \ru{В свою очередь}, you
should consider two situations \ru{мы должны рассмотреть две ситуации}:
\begin{enumerate}

\item To understand the binding structure of your program \ru{Для понимания
структуры привязок в вашей программе}, you may need to look \emph{at the whole
program} \ru{вам потребуется рассматривать \emph{целиком всю программу}}. No
matter how much you’ve decomposed your program \ru{Не важно что вы разбили
программу} into small, understandable fragments \ru{на маленькие понятные
фрагменты}, it doesn’t matter if you have a free identifier anywhere \ru{не
важно если у вас есть где-то свободные идентификаторы}.

\item Understanding the binding structure \ru{Понимание структуры привязок} is
not only a function of the size of the program \ru{не только функция от размера
программы} but also of the complexity of its control flow \ru{но и от сложности
ее структур управления}. Imagine an interactive program with numerous callbacks
\ru{Представьте интерактивную программу со множеством обработчиков событий};
you’d have to track through every one of them \ru{вы должны отслеживать ее
поведение при срабатывании обработчиков в любом порядке}, too, to know which
binding governs an identifier \ru{чтобы понять как срабатывают привязки
идентификаторов}.

\end{enumerate}

Need a little more of a nudge \ru{Хотите еще пендаль}\,? Let’s replace the
expression of our example program with this one \ru{Давайте заменим выражение
в нашей программе-примере вот этим}:
\lsts{6/4/1.rkt}{rkt}
Suppose \ru{Предположим} \verb|moon-visible?| is a function \ru{это функция}
that presumably evaluates to \ru{которая предположительно вычисляется в}
\verb|false| on new-moon nights \ru{в ночи новолуния}, and \ru{и} \verb|true| at
other times \ru{в другое время}. Then \ru{Так что}, this program will evaluate
to an answer \ru{эта программа будет вычисляться до значения} except on new-moon
nights \ru{в зависимости от фазы луны}, when it will fail with an unbound
identifier error \ru{иногда она будет падать в ошибку несвязанного
идентификатора}.

\Exercise{
What happens on cloudy nights\,?\\\ru{Что случиться в облачную ночь}\,?
}

\secrel{6.4.2 The Top-Level Scope  . 31}

Matters become more complex when we contemplate top-level definitions in many
languages. For instance, some versions of Scheme (which is a paragon of lexical
scoping) allow you to write this:
\lsts{6/4/2.rkt}{rkt}
which seems to pretty clearly suggest where the y in the body of f will come
from, except:
\lsts{6/4/3.rkt}{rkt}
is legal and (f 10) produces 12. Wait, you might think, always take the last
one! But:
\lsts{6/4/4.rkt}{rkt}

Here, z is bound to the first value of y whereas the inner y is bound to the
second value. There is actually a valid explanation of this behavior in terms of
lexical scope, but it can become convoluted, and perhaps a more sensible option
is to prevent such redefinition. Racket does precisely this, thereby offering
the convenience of a top-level without its pain.
\note{Most “scripting” languages exhibit similar problems. As a result, on the
Web you will find enormous confusion about whether a certain language is
statically- or dynamically-scoped, when in fact readers are comparing behavior
inside functions (often static) against the top-level (usually dynamic).
Beware!}

\secup

\secrel{6.5 Exposing the Environment  . . 31}

If we were building the implementation for others to use, it would be wise and a
courtesy for the exported interpreter to take only an expression and list of
function definitions, and invoke our defined interp with the empty environment.
This both spares users an implementation detail, and avoids the use of an
interpreter with an incorrect environment. In some contexts, however, it can be
useful to expose the environment parameter. For instance, the environment can
represent a set of pre-defined bindings: e.g., if the language wishes to provide
pi automatically bound to 3.2 (in Indiana).

\secup


\secrel{Functions Anywhere\\
\ru{Функции везде}}\secdown

The introduction to the \ru{Введение в язык программирования} $Scheme$
programming language definition establishes this design principle \ru{предлагает
следующий принцип дизайна}:
\begin{framed}
Programming languages \ru{Языки программирования} should be designed \ru{должны
создаваться} not by piling feature on top of feature \ru{не нагромождением фичи
на фичу}, but by removing the weaknesses and restrictions \ru{но путем
устранения недостатков и ограничений} that make additional features appear
necessary \ru{которые делают необходимым появление дополнительных возможностей}.
\ref{}
\end{framed}
As design principles go, this one is hard to argue with \ru{с этим принципом
трудно спорить}. (Some restrictions \ru{Некоторые ограничения}, of course
\ru{конечно}, have good reason to exist \ru{имеют хорошо обоснованные причины
существования}, but this principle forces us to argue for them \ru{но этот
принцип заставляет нас аргументировать их}, not admit them by default \ru{а не
слепо следовать}.) Let’s now apply this to functions \ru{Давайте применим этот
принцип к функциям}.

One of the things \ru{Одна из вещей} we stayed coy about \ru{на которую мы не
обратили внимания} when introducing functions \ru{при введении функций}
\ref{sec5} is exactly where functions go \ru{это когда функции используются}.
We may have suggested we’re following the model \ru{Мы предполагали следовать
модели} of an idealized \ru{идеализированного} Dr\racket, with definitions and
their uses kept separate \ru{с раздельным определением функций, и их
использованием}. But, inspired by the \ru{Но вдохновляясь принципом} $Scheme$
design principle, let’s examine how necessary that is \ru{давайте ближе
рассмотрим насколько это необходимо}.

Why can’t functions definitions be expressions \ru{Почему бы не определять
функции через выражения}\,? In our current arithmetic-centric language \ru{В
нашем текущем арифметическом языке} we face the uncomfortable question \ru{мы
сталкиваемся с неудобным вопросом} ``What value does a function definition
represent \ru{Какое значение должно соответствовать определению функции}\,?'',
to which we don’t really have a good answer \ru{на который у нас на самом деле
нет подходящего ответа}. But a real programming language \ru{Но реальный язык
программирования} obviously computes more than numbers \ru{очевидно работает не
только с числами}, so we no longer need \ru{так что больше нет необходимости} to
confront the question in this form \ru{связываться с вопросом в такой
постановке}; indeed \ru{вместо этого}, the answer to the above can just as well
be \ru{ответ на него может быть}, ``A function value \ru{Объект-функция}''.
Let’s see how that might work out \ru{Давайте посмотрим как это может работать}.

What can we do with functions as values \ru{Что мы можем делать с функциями
как значениями}\,? Clearly \ru{Ясно что}, functions are a distinct kind of value
than a number \ru{функция это другой тип значения чем число}, so we cannot
\ru{так что мы не можем}, for instance \ru{например}, add them \ru{складывать
их}. But there is one evident thing we can do \ru{Есть особенная вещь которую мы
можем делать}: apply them to arguments \ru{применять их к аргументам}\,! Thus
\ru{Таким образом}, we can allow function values \ru{мы можем позволить появляться
функциональным значениям} to appear in the function position \ru{на месте имени
функции} of an application \ru{при применении}.
The behavior would, naturally, be to apply the function \ru{При этом поведение
будет, естественно, применение функции}.
Thus, we’re proposing a language \ru{итак, мы предлагаем язык} where the
following would be a valid program \ru{в котором следующий код будет валидной
программой} (where I’ve used brackets \ru{я использовал тут квадратные скобки}
so we can easily identify the function \ru{так что мы можем легко визуально
идентифицировать функцию})
\lsts{7/1.rkt}{rkt}
and would evaluate to \ru{что раскроется в} \verb|(+ 2 (* 4 3))|, or \ru{или}
14.\note{Did you see that I just used substitution\,?}\note{\ru{Вы заметили
что я только что использовал подстановку\,?}}

% \secrel{7.1 Functions as Expressions and Values  32}

Let’s first define the core language to include function definitions:
\lsts{7/1/1.rkt}{rkt}

For now, we’ll simply copy function definitions into the expression language.
We’re free to change this if necessary as we go along, but for now it at least
allows us to reuse our existing test cases.
\lsts{7/1/2.rkt}{rkt}

We also need to determine what an application looks like. What goes in the
function position of an application? We want to allow an entire function
definition, not just its name. Because we’ve lumped function definitions in with
all other expressions, let’s allow an arbitrary expression here, but with the
understanding that we want only function definition expressions:
\note{We might consider more refined datatypes that split function definitions
apart from other kinds of expressions. This amounts to trying to classify
different kinds of expressions, which we will return to when we study types.
\ref{}}
\lsts{7/1/3.rkt}{rkt}

With this definition of application, we no longer have to look up functions by
name, so the interpreter can get rid of the list of function definitions. If we
need it we can restore it later, but for now let’s just explore what happens
with function definitions are written at the point of application: so-called
immediate functions.

Now let’s tackle interp. We need to add a case to the interpreter for function
definitions, and this is a good candidate:
\lsts{7/1/4.rkt}{rkt}

\DoNow{
What happens when you add this?
}
Immediately, we see that we have a problem: the interpreter no longer always
returns numbers, so we have a type error.

We’ve alluded periodically to the answers computed by the interpreter, but never
bothered gracing these with their own type. It’s time to do so now.
\lsts{7/1/5.rkt}{rkt}

We’re using the suffix of V to stand for values, i.e., the result of evaluation.
The pieces of a funV will be precisely those of a fdC: the latter is input, the
former is output. By keeping them distinct we allow each one to evolve
independently as needed.

Now we must rewrite the interpreter. Let’s start with its type:
\lsts{7/1/6.rkt}{rkt}

This change naturally forces corresponding type changes to the Binding datatype
and to lookup.

\Exercise{
Modify Binding and lookup, appropriately.
}
\lsts{7/1/7.rkt}{rkt}

Clearly, numeric answers need to be wrapped in the appropriate numeric answer
constructor. Identifier lookup is unchanged. We have to slightly modify addition
and multiplication to deal with the fact that the interpreter returns Values,
not numbers:
\lsts{7/1/8.rkt}{rkt}

It’s worth examining the definition of one of these helper functions:
\lsts{7/1/9.rkt}{rkt}
Observe that it checks that both arguments are numbers before performing the
addition. This is an instance of a safe run-time system. We’ll discuss this
topic more when we get to types. \ref{}

There are two more cases to cover. One is function definitions. We’ve already
agreed these will be their own kind of value:
\lsts{7/1/10.rkt}{rkt}

That leaves one case, application. Though we no longer need to look up the
function definition, we’ll leave the code structured as similarly as possible:
\lsts{7/1/11.rkt}{rkt}

In place of the lookup, we reference f which is the function definition, sitting
right there. Note that, because any expression can be in the function definition
position, we really ought to harden the code to check that it is indeed a
function.

\DoNow{
What does is mean? That is, do we want to check that the function definition
position is syntactically a function definition (fdC), or only that it
evaluates to one (funV)? Is there a difference, i.e., can you write a program
that satisfies one condition but not the other?
}

We have two choices:
\begin{enumerate}[nosep]
  \item 
We can check that it syntactically is an fdC and, if it isn’t reject it as an
error.
  \item 
We can evaluate it, and check that the resulting value is a function (and signal
an error otherwise).
\end{enumerate}
We will take the latter approach, because this gives us a much more flexible
language. In particular, even if we can’t immediately imagine cases where we, as
humans, might need this, it might come in handy when a program needs to generate
code. And we’re writing precisely such a program, namely the desugarer! (See
section 7.5.) As a result, we’ll modify the application case to evaluate the
function position:
\lsts{7/1/12.rkt}{rkt}

\Exercise{
Modify the code to perform both versions of this check.
}

And with that, we’re done. We have a complete interpreter! Here, for instance,
are some of our old tests again:
\lsts{7/1/13.rkt}{rkt}

% \input{7/2/nested}
% \secrel{7.3 Implementing Closures   . 37}

We need to change our representation of values to record closures rather than
raw function text:
\lsts{7/3/1.rkt}{rkt}
While we’re at it, we might as well alter our syntax for defining functions to
drop the useless name. This construct is historically called a lambda:
\lsts{7/3/2.rkt}{rkt}

When encountering a function definition, the interpreter must now remember to
save the substitutions that have been applied so far:
\note{“Save the environment! Create a closure today!”\ --- Cormac Flanagan}
\lsts{7/3/3.rkt}{rkt}

This saved set, not the empty environment, must be used when applying a
function:
\lsts{7/3/4.rkt}{rkt}

There’s actually another possibility: we could use the environment present at
the point of application:
\lsts{7/3/5.rkt}{rkt}

\Exercise{
What happens if we extend the dynamic environment instead?
}

In retrospect, it becomes even more clear why we interpreted the body of a
function in the empty environment. When a function is defined at the top-level,
it is not “closed over” any identifiers. Therefore, our previous function
applications have been special cases of this form of application.

% \secrel{Substitution\\\ru{Подстановка}}

\termdef{Substitution}{substitution} is the act of replacing a name
\ru{\termdef{Подстановка}{подстановка} это процесс замены имени}
(in this case, that of the formal parameter
\ru{в этом конкретном случае, это имя формального параметра})
in an expression \ru{в выражении}
(in this case, the \termdef{body}{function body} of the function
\ru{в этом случае это \termdef{тело функции}{тело функции}})
with another expression
\ru{на другое выражение}
(in this case, the actual parameter
\ru{значение пареметра}).
Let’s define its type
\ru{Давайте определим его тип}:
\lst{5/3/1.rkt}
It helps to also give its parameters informative names
\ru{Это поможет нам также дать его параметрам информативные имена}:
\lsts{5/3/2.rkt}{rkt}
The first argument is \ru{Первый аргумент} \term{what} we want to replace the
name with \ru{на что мы хотим заменить имя};
the second \ru{второй} \term{is} for what name we want to perform substitution
\ru{для какого имени мы хотим выполнить подстановку};
and the third is \ru{и третий это } \term{in} which expression we want to
do it \ru{в каком выражения мы хотим сделать это}.
\DoNow{
Suppose we want to substitute \ru{Предположим что мы хотим подставить} $3$ for
the identifier \ru{вместо идентификатора} $x$ in the bodies of the three example
functions above \ru{в телах трех примеров функций выше}.
What should it produce \ru{Что должно получиться в результате}\,? } In \ru{В}
\verb|double|, this should produce \ru{должно получиться} \verb|(+ 3 3)|; in
\ru{в} \verb|quadruple|, it should produce \ru{результат}
\verb|(double (double 3))|; \clearpage and in \ru{и в} \verb|const5|, it should
produce \ru{должно получиться} \verb|5| (i.e., no substitution happens \ru{т.е.
никакой подстановки не происходит} because there are no instances of \ru{потому
что нет вхождений} \verb|x| in the body \ru{в теле функции}).
\note{A common mistake is to assume that the result of substituting, e.g., 3 for
x in double is (define (double x) (+ 3 3)). This is incorrect. We only
substitute \emph{at the point when we apply the function}, at which point the
\termdef{function’s invocation}{function invocation} is replaced by its body.
The \termdef{header}{function header} enables us to find the function and
ascertain the name of its parameter; but \emph{only its body remains for
evaluation}. Examine how substitution is used to notice the type error that
would result from returning a function definition.}
\note{\ru{Распространенной ошибкой является предположение того что результат
подстановки, например} 3 \ru{для} x \ru{в} double \ru{это} (define (double x) (+
3 3)). \ru{Это неправильно. Мы выполняем подстановку только \emph{в точке где мы
применяем функцию}, в том месте где \termdef{вызов функции}{вызов функции}
заменяется на ее тело. \termdef{Заголовок}{заголовок функции} позволяет нам
найти функцию и установить имя ее параметра; но \emph{только ее тело остается
для вычисления}. Рассмотрите внимательно как происходит подстановка, чтобы
запомнить типичную ошибку которая может быть результатом возврата определения
функции.}}

\clearpage
These examples already tell us what to do in almost all the cases. \ru{Эти
примеры показывают нам что делать в большинстве случаев.} Given a number,
there’s nothing to substitute. \ru{Для числа не нужна подстановка.} If it’s an
identifier \ru{Если это идентификатор}, we haven’t seen an example with a
\emph{different} identifier \ru{мы еще не видели пример с \emph{другим}
идентификатором}, but you’ve guessed what should happen \ru{но вы предполагаете
что должно произойти}: it stays unchanged \ru{он остается неизменным}. In the
other cases \ru{В других случаях}, descend into the sub-expressions \ru{обходим
подвыражения}, performing substitution \ru{выполняя подстановку}.

Before we turn this into code \ru{Перед тем как превратить эти правила в код},
there’s an important case to consider \ru{необходимо рассмотреть важный случай}.
Suppose the name we are substituting happens to be the name of a function
\ru{Предположим что подставляемое имя оказывается именем функции}. Then what
should happen \ru{Что должно при этом произойти}\,?
\DoNow{
What, indeed, should happen \ru{И шо-таки должно быть}\,?
}
There are many ways to approach this question \ru{Есть много подходов к этому
вопросу}. One is from a design perspective \ru{Один подход с точки зрения
дизайна}: function names live in their own ``world'' \ru{имена функций живут в
их собственном ``мире''}, distinct from ordinary program identifiers \ru{и
отличаются от обычных идентификаторов в прогремме}. Some languages \ru{Некоторые
языки} (such as C and Common Lisp \ru{такие как \ci\ и Common \lisp}, in
slightly different ways \ru{немного различными способами}) take this perspective
\ru{идут этим путем}, and partition identifiers into different
\termdef{namespaces}{namespace} \ru{разделяют идентификаторы в различные \termdef{пространства имен}{пространство имен}}
depending on how they are used \ru{в зависимости от того как они используются}.
In other languages \ru{В других языках}, there is no such distinction \ru{такого
разделения нет}; indeed, we will examine such languages soon \ru{на самом деле
мы рассмотрим такие языки в ближайшее время} \ref{}.

For now, we will take a pragmatic viewpoint \ru{Теперь примем практическую точку
зрения}. Because expressions evaluate to numbers \ru{Так как выражения
вычисляются в числа}, that means a function name could turn into a number
\ru{это значит что имя функции в итоге превращается в число}. However, numbers
cannot name functions \ru{Тем не менее, числа не могут именовать функции}, only
symbols can \ru{это могут делать только символы}. Therefore, it makes no sense
to substitute in that position \ru{Так что нет смысла делать подстановку в этой
позиции}, and we should leave the function name unmolested \ru{и мы должны
оставить имя функции нетронутым} irrespective of its relationship to the
variable being substituted \ru{независимо от ее отношения к подставляемой
переменной}. (Thus, a function could have a parameter named \ru{Таким образом
функция может иметь параметр названный} x as well as refer to another
\emph{function} called \ru{одновременно со ссылкой на другую \emph{функцию} с
именем} x, and these would be kept distinct \ru{и они будут различимы}.)

Now we’ve made all our decisions \ru{Теперь мы приняли все наши решения}, and we
can provide the body code \ru{и можем предоставить body-код}:
\lsts{5/3/3.rkt}{rkt}

\Exercise{
Observe that \ru{Заметим что}, whereas in the \ru{в случае} numC case the
interpreter returned \ru{интерпретатор возвращает} n, substitution returns
\ru{подстановка возвращает} in (i.e., the original expression \ru{т.е.
оригинальное выражение}, equivalent at that point to writing \ru{в этом месте
эквивалентное} (numC n). Why \ru{Зачем}\,?
}

% \secrel{7.5 Sugaring Over Anonymity  . . 39}

Now let’s get back to the idea of naming functions, which has evident value for
program understanding. Observe that we do have a way of naming things: by
passing them to functions, where they acquire a local name (that of the formal
parameter). Anywhere within that function’s body, we can refer to that entity
using the formal parameter name.

Therefore, we can take a collection of function definitions and name them using
other...functions. For instance, the Racket code
\lsts{7/5/1.rkt}{rkt}
could first be rewritten as the equivalent
\lsts{7/5/2.rkt}{rkt}
We can of course just inline the definition of double, but to preserve the name,
we could write this as:
\lsts{7/5/3.rkt}{rkt}
Indeed, this pattern—which we will pronounce as “left-left-lambda”—is a local
naming mechanism. It is so useful that in Racket, it has its own special syntax:
\lsts{7/5/4.rkt}{rkt}
where let can be defined by desugaring as shown above.

Here’s a more complex example:
\lsts{7/5/5.rkt}{rkt}
This could be rewritten as
\lsts{7/5/6.rkt}{rkt}
which works just as we’d expect; but if we change the order, it no longer works—
\lsts{7/5/7.rkt}{rkt}
—because quadruple can’t “see” double. so we see that top-level binding is
different from local binding: essentially, the top-level has an “infinite
scope”. This is the source of both its power and problems.

There is another, subtler, problem: it has to do with recursion. Consider the
simplest infinite loop:
\lsts{7/5/8.rkt}{rkt}
Let’s convert it to let:
\lsts{7/5/9.rkt}{rkt}
Seems fine, right? Rewrite in terms of lambda:
\lsts{7/5/10.rkt}{rkt}
Clearly, the loop-forever on the last line isn’t bound!

This is another feature we get “for free” from the top-level. To eliminate this
magical force, we need to understand recursion explicitly, which we will do soon
\ref{}.
\secup

\input{8/mutation}
% \secrel{9 Recursion and Cycles:\\Procedures and Data 62}\label{sec9}\secdown

Recursion is the act of self-reference. When we speak of recursion in
programming languages, we may have one of (at least) two meanings in mind:
recursion in data, and recursion in control (i.e., of program behavior—that is
to say, of functions).

\secrel{9.1 Recursive and Cyclic Data  . . 62}

Recursion in data can refer to one of two things. It can mean referring to
something of the same kind, or referring to the same thing itself.

Recursion of the same kind leads to what we traditionally call recursive data.
For instance, a tree is a recursive data structure: each vertex can have
multiple children, each of which is itself a tree. But if we write a procedure
to traverse the nodes of a tree, we expect it to terminate without having to
keep track of which nodes it has already visited. They are finite data
structures.

In contrast, a graph is often a cyclic datum: a node refers to another node,
which may refer back to the original one. (Or, for that matter, a node may refer
directly to itself.) When we traverse a graph, absent any explicit checks for
what we have already visited, we should expect a computation to diverge, i.e.,
not terminate. Instead, graph algorithms need a memory of what they have visited
to avoid repeating traversals.

Adding recursive data, such as lists and trees, to our language is quite
straightforward. We mainly require two things:
\begin{enumerate}[nosep]
  \item 
The ability to create compound structures (such as nodes that have references to
children).
  \item 
The ability to bottom-out the recursion (such as leaves).
\end{enumerate}

\Exercise{
Add lists and binary trees as built-in datatypes to the programming language.
}

Adding cyclic data is more subtle. Consider the simplest form of cyclic datum, a
cell referring back to itself:
\bigskip

Let’s try to define this in Racket. Here’s one attempt:
\lsts{9/1/1.rkt}{rkt}
But this doesn’t work: b on the right-hand side of the let isn’t bound. It’s
easy to see if we desugar it:
\lsts{9/1/2.rkt}{rkt}
and, for clarity, we can rename the b in the function:
\lsts{9/1/3.rkt}{rkt}
Now it’s patently clear that b is unbound.

Absent some magical Racket construct we haven’t yet seen, it becomes clear that
we can’t create a cyclic datum in one shot. Instead, we need to first create a
“place” for the datum, then refer to that place within itself. The use of
“then”—i.e., the introduction of time—should suggest a mutation operation.
Indeed, let’s try it with boxes.
\note{That construct would be shared, but virtually no other language has this
notational mechanism, so we won’t dwell on it here. In fact, what we are
studying is the main idea behind how shared actually works.}

Our plan is as follows. First, we want to create a box and bind it to some
identifier, say b. Now, we want to mutate the content of the box. What do we
want it to contain? A reference to itself. How does it obtain that reference? By
using the name, b, that is already bound to it. In this way, the mutation
creates the cycle:
\lsts{9/1/4.rkt}{rkt}
Note that this program will not run in Typed PLAI as written. We’ll return to
typing such programs later \ref{}. For now, run it in the untyped (\#lang plai)
language.

When the above program is Run, Racket prints this as: \verb|#0='#&#0#|. This
notation is in fact precisely what we want. Recall that \verb|#&| is how Racket
prints boxes. The \verb|#0=| (and similarly for other numbers) is how Racket
names pieces of cyclic data. Thus, Racket is saying, “\verb|#0| is bound to a
box whose content is \verb|#0#|, i.e., whatever is bound to \verb|#0|, i.e.,
itself”.

\Exercise{
Run the equivalent program through your interpreter for boxes and make sure it
produces a cyclic value. How do you check this?
}

The idea above generalizes to other datatypes. In this same way we can also
produce cyclic lists, graphs, and so on. The central idea is this two-step
process: first name an vacant placeholder; then mutate the placeholder so its
content is itself; to obtain “itself”, use the name previously bound. Of course,
we need not be limited to “self-cycles”: we can also have mutually-cyclic data
(where no one element is cyclic but their combination is).

\secrel{9.2 Recursive Functions   64}

In a shift in terminology, a recursive function is not a reference to a same
kind of function but rather to the same function itself. It’s useful to first
ensure we’ve first extended our language with conditionals (even of the kind
that only check for 0, as described earlier: section 5), so we can write
non-trivial programs that terminate.

Let’s now try to write a recursive factorial:
\lsts{9/2/1.rkt}{rkt}
But this doesn’t work at all! The inner fact gives an unbound identifier error,
just as in our cyclic datum example.

It is no surprise that we should encounter the same error, because it has the
same cause. Our traditional binding mechanism does not automatically make
function definitions cyclic (indeed, in some early programming languages, they
were not: misguidedly, recursion was considered a special feature). Instead, if
we want recursion—i.e., for a function definition to cyclically refer to
itself—we must implement it by hand.
\note{Because you typically write top-level definitions, you don’t encounter
this issue. At the top-level, every binding is implicitly a variable or a box.
As a result, the pattern below is more-or-less automatically put in place for
you. This is why, when you want a recursive local binding, you must use letrec
or local, not let.}

The means to do so is now clear: the problem is the same one we diagnosed
before, so we can reuse the same solution. We again have to follow a three-step
process: first create a placeholder, then refer to the placeholder where we want
the cyclic reference, and finally mutate the placeholder before use. Thus:
\lsts{9/2/2.rkt}{rkt}
In fact, we don’t even need fact-fun: I’ve used that binding just for clarity.
Observe that because it isn’t recursive, and we have identifiers rather than
variables, its use can simply be substituted with its value:
\lsts{9/2/3.rkt}{rkt}
There is the small nuisance of having to repeatedly unbox fact. In a language
with variables, this would be even more seamless:
\note{ Indeed, one use for variables is that they simplify the desguaring of the
above pattern, instead of requiring every use of a cyclically-bound identifier
to be unboxed. On the other hand, with a little extra effort the desugaring
process could take care of doing the unboxing, too.}
\lsts{9/2/4.rkt}{rkt}

\input{9/3/premat}
\input{9/4/wostate}
\secup

 
% \secrel{10 Objects 67}\secdown

When a language admits functions as values, it provides developers the most
natural way to represent a unit of computation. Suppose a developer wants to
parameterize some function f. Any language lets f be parameterized by passive
data, such as numbers and strings. But it is often attractive to parameterize it
over active data: a datum that can compute an answer, perhaps in response to
some information. Furthermore, the function passed to f can—assuming
lexically-scoped functions—refer to data from the caller without those data
having to be revealed to f, thus providing a foundation for security and
privacy. Thus, lexically-scoped functions are central to the design of many
secure programming techniques.

While a function is a splendid thing, it suffers from excessive terseness.
Sometimes we might want multiple functions to all close over to the same shared
data; the sharing especially matters if some of the functions mutate it and
expect the others to see the result of those mutations. In such cases, it
becomes unwieldly to send just a single function as a parameter; it is more
useful to send a group of functions. The recipient then needs a way to choose
between the different functions in the group. This grouping of functions, and
the means to select one from the group, is the essence of an object.
We are therefore perfectly placed to study objects having covered functions
(section \ref{sec7}) and mutation (section \ref{sec8})—and, it will emerge,
recursion (section \ref{sec9}).
\note{
I cannot hope to do justice to the enormous space of object systems.
Please read \href{http://users.dcc.uchile.cl/~etanter/ooplai/}{Object-Oriented
Programming Languages: Application and Interpretation} by Éric Tanter, which
goes into more detail and covers topics ignored here.
}

Let’s add this notion of objects to our language. Then we’ll flesh it out and
grow it, and explore the many dimensions in the design space of objects. We’ll
first show how to add objects to the core language, but because we’ll want to
prototype many different ideas quickly, we’ll soon shift to a desguaring-based
strategy. Which one you use depends on whether you think understanding them is
critical to understanding the essence of your language. One way to measure this
is how complex your desguaring strategy becomes, and whether by adding some key
core language enhancements, you can greatly reduce the complexity of desugaring.

\secrel{10.1 Objects Without Inheritance  . 67}

The simplest notion of an object—pretty much the only thing everyone who talks about
objects agrees about—is that an object is
\begin{itemize}[nosep]
  \item a value, that
  \item maps names to
  \item stuff: either other values or “methods”.
\end{itemize}
From a minimalist perspective, methods seem to be just functions, and since we
already have those in the language, we can put aside this distinction.
\note{We’re about to find out that “methods” are awfully close to functions but
differ in important ways in how they’re called and/or what’s bound in them.}

\secdown
% \secrel{14.2.3 Implementation in the Core  115}

Now that we’ve seen how CPS can be implemented through desguaring, we should ask
whether it can be put in the core instead.

Recall that we’ve said that CPS applies to all programs. We have one program we
are especially interested in: the interpreter. Sure enough, we can apply the CPS
transformation to it, making available what are effectively the same
continuations.

First, we’ll find it convenient to use a procedural representation of closures
\ref{}. We’ll have the interpreter take an extra argument, which consumes values
(those given to the continuation) and eventually returns them:
\lsts{14/2/3/1.rkt}{rkt}
In the easy cases, instead of returning a value we need to simply pass it to the
continuation argument:
\lsts{14/2/3/2.rkt}{rkt}
(Note that multC is handled entirely analogous to plusC.)

Let’s start with the easy case, plusC. First we interpret the left
sub-expression. The continuation for this evaluation interprets the right
sub-expression. The continuation for that adds the result. What should happen to
the result of addition? In interp, it was returned to whichever computation
caused the plusC to be interpreted. Now, remember, we no longer return values;
instead we pass them to the continuation:
\lsts{14/2/3/3.rkt}{rkt}
\Exercise{
Implement the code for multC.
}
This leaves the two difficult, and related, pieces.

In an application, we again have to interpret the two sub-expressions, and then
apply the resulting closure to the argument. But we’ve already agreed that every
application needs a continuation argument. Therefore, we have to update our
definition of a value:
\lsts{14/2/3/4.rkt}{rkt}

Now we have to decide what continuation to pass. In an application, it’s the
continuation given to the interpreter:
\lsts{14/2/3/5.rkt}{rkt}

Finally, the lamC case. We have to create a closV using a lambda, as before.
However, this procedure needs to take two arguments: the actual value of the
argument, and the continuation of the application. The critical question is,
what is this latter value?

We have essentially two choices. k represents the static continuation: the one
active at the point of closure construction. However, what we want is the
continuation at the point of closure invocation: the dynamic continuation.
\lsts{14/2/3/6.rkt}{rkt}

To test this revised interpreter, we need to invoke interp/k with some kind of
initial continuation value. This needs to be a procedure that represents nothing
remaining in the computation. A natural representation for this is the identity
function:
\lsts{14/2/3/7.rkt}{rkt}
To signify that this is strictly a top-level interface to interp/k, we’ve
dropped the environment parameter and pass the empty environment automatically.
If we want to be especially sure we haven’t accidentally used this procedure
recursively, we could insert a call to error at its end to prevent it from
returning and its return value being used.

% \input{10/1/2/desugar}
% \secrel{10.1.3 Objects as Named Collections  . . 69}

Let’s begin by reproducing the object language we had above. An object is just a
value that dispatches on a given name. For simplicity, we’ll use lambda to
represent the object and case to implement the dispatching.
\note{Observe that basic objects are a generalization of lambda to have multiple
“entry-points”. Conversely, a lambda is an object with just one entry-point, so
it doesn’t need a “method name” to disambiguate.}
\lsts{10/1/3/1.rkt}{rkt}

This is the same object we defined earlier, and we use its method in the same
way:
\lsts{10/1/3/2.rkt}{rkt}

Of course, writing method invocations with these nested function calls is
unwieldy (and is about to become even more so), so we’d be best off equipping
ourselves with a convenient syntax for invoking methods—the same one we saw
earlier (msgS), but here we can simply define it as a function:
\note{We’ve taken advantage of Racket’s variable-arity syntax: . a says “bind
all the remaining—zero or more—arguments to a list named a”. apply “splices” in
such lists of arguments to call functions.}
\lsts{10/1/3/3.rkt}{rkt}
This enables us to rewrite our test:
\lsts{10/1/3/4.rkt}{rkt}

\DoNow{
Something very important changed when we switched to the desguaring
strategy. Do you see what it is?
}

Recall the syntax definition we had earlier:
\lsts{10/1/3/5.rkt}{rkt}
The “name” position of a message was very explicitly a symbol. That is, the
developer had to write the literal name of the symbol there. In our desugared
version, the name position is just an expression that must evaluate to a symbol;
for instance, one could have written
\lsts{10/1/3/6.rkt}{rkt}
This is a general problem with desugaring: the target language may allow
expressions that have no counterpart in the source, and hence cannot be mapped
back to it. Fortunately we don’t often need to perform this inverse mapping,
though it does arise in some debugging and program comprehension tools. More
subtly, however, we must ensure that the target language does not produce values
that have no corresponding equivalent in the source.

Now that we have basic objects, let’s start adding the kinds of features we’ve
come to expect from most object systems.
% \secrel{10.1.4 Constructors   . . 70}

A constructor is simply a function that is invoked at object construction time.
We currently lack such a function. by turning an object from a literal into a
function that takes constructor parameters, we achieve this effect:
\lsts{10/1/4/1.rkt}{rkt}
In the first example, we pass 5 as the constructor’s argument, so adding 3
yields 8. The second is similar, and shows that the two invocations of the
constructors don’t interfere with one another.

% \secrel{10.1.5 State   71}

Many people believe that objects primarily exist to encapsulate state. We
certainly haven’t lost that ability. If we desugar to a language with variables
(we could equivalently use boxes, in return for a slight desugaring overhead),
we can easily have multiple methods mutate common state, such as a constructor
argument:
\note{Alan Kay, who won a Turing Award for inventing Smalltalk and modern object
technology, disagrees. In
\href{http://www.smalltalk.org/smalltalk/TheEarlyHistoryOfSmalltalk_Abstract.html}{The
Early History of Smalltalk}, he says, “[t]he small scale [motivation for OOP]
was to find a more flexible version of assignment, and then to try to eliminate
it altogether”. He adds, “It is unfortunate that much of what is called
‘object-oriented programming’ today is simply old style programming with fancier
constructs. Many programs are loaded with ‘assignment-style’ operations now done
by more expensive attached procedures.”}
\lsts{10/1/5/1.rkt}{rkt}
For instance, we can test a sequence of operations:
\lsts{10/1/5/2.rkt}{rkt}
and also notice that mutating one object doesn’t affect another:
\lsts{10/1/5/3.rkt}{rkt}

% \secrel{10.1.6 Private Members   71}

Another common object language feature is private members: ones that are visible
only inside the object, not outside it. These may seem like an additional
feature we need to implement, but we already have the necessary mechanism in the
form of locallyscoped, lexically-bound variables:
\note{Except that, in Java, instances of other classes of the same type are
privy to “private” members. Otherwise, you would simply never be able to
implement an Abstract Data Type.}
\lsts{10/1/6/1.rkt}{rkt}
The desugaring above provides no means for accessing count, and lexical scoping
ensures that it remains hidden to the world.

% \input{10/1/7/static}
% \input{10/1/8/selfref}
% \secrel{10.1.9 Dynamic Dispatch  . . 74}

Finally, we should make sure our objects can handle a characteristic attribute
of object systems, which is the ability to invoke a method without the caller
having to know or decide which object will handle the invocation. Suppose we
have a binary tree data structure, where a tree consists of either empty nodes
or leaves that hold a value. In traditional functions, we are forced to
implement the equivalent some form of conditional—either a cond or a type-case
or pattern-match or other moral equivalent—that exhaustively lists and selects
between the different kinds of trees. If the definition of a tree grows to
include new kinds of trees, each of these code fragments must be modified.
Dynamic dispatch solves this problem by making that conditional branch disappear
from the user’s program and instead be handled by the method selection code
built into the language. The key feature that this provides is an extensible
conditional. This is one dimension of the extensibility that objects provide.
\note{This property—which appears to make systems more black-box extensible
because one part of the system can grow without the other part needing to be
modified to accommodate those changes—is often hailed as a key benefit of
object-orientation.
While this is indeed an advantage objects have over functions, there is a dual
advantage that functions have over objects, and indeed many object programmers
end up contorting their code—using the Visitor pattern—to make it look more like
a function-based organization. Read
\href{http://cs.brown.edu/~sk/Publications/Papers/Published/kff-synth-fp-oo/}{Synthesizing
Object-Oriented and Functional Design to Promote Re-Use} for a running example
that will lay out the problem in its full glory. Try to solve it in your
favorite language, and see the
\href{http://www.cs.utah.edu/plt/publications/icfp98-ff/paper.shtml}{\racket\
solution}.}

Let’s now defined our two kinds of tree objects:
\lsts{10/1/9/1.rkt}{rkt}

With these, we can make a concrete tree:
\lsts{10/1/9/2.rkt}{rkt}

And finally, test it:
\lsts{10/1/9/3.rkt}{rkt}

Observe that both in the test case and in the add method of node, there is a
reference to 'add without checking whether the recipient is a mt or node.
Instead, the run-time system extracts the recipient’s add method and invokes it.
This missing conditional in the user’s program is the essence of dynamic
dispatch.

\secup

\input{10/2/member}
\secrel{10.3 What (Goes In) Else?   . . 75}

Until now, our case statements have not had an else clause. One reason to do so
would be if we had a variable set of members in an object, though that is
probably better handled through a different representation than a conditional: a
hash-table, for instance, as we’ve discussed above. In contrast, if an object’s
set of members is fixed, desugaring to a conditional works well for the purpose
of illustration (because it emphasizes the fixed nature of the set of member
names, which a hash table leaves open to interpretation—and also error). There
is, however, another reason for an else clause, which is to “chain” control to
another, parent, object. This is called inheritance.

Let’s return to our model of desugared objects above. To implement inheritance,
the object must be given “something” to which it can delegate method invocations
that it does not recognize. A great deal will depend on what that “something”
is.

One answer could be that it is simply another object.
\lsts{10/3/1.rkt}{rkt}

Due to our representation of objects, this application effectively searches for
the method in the parent object (and, presumably, recursively in its parents).
If a method matching the name is found, it returns through this chain to the
original call in msg that sought the method. If none is found, the final object
presumably signals a “message not found” error.

\Exercise{
Observe that the application (parent-object m) is like “half a msg”, just like
an l-value was “half a value lookup” \ref{}. Is there any connection?
}

Let’s try this by extending our trees to implement another method, size. We’ll
write an “extension” (you may be tempted to say “sub-class”, but hold off for
now!) for each node and mt to implement the size method. We intend these to
extend the existing definitions of node and mt, so we’ll use the extension
pattern described above.
\note{We’re not editing the existing definitions because that is supposed to be
the whole point of object inheritance: to reuse code in a black-box fashion.
This also means different parties, who do not know one another, can each extend
the same base code. If they had to edit the base, first they have to find out
about each other, and in addition, one might dislike the edits of the other.
Inheritance is meant to sidestep these issues entirely.}

\secdown
\input{10/3/1/classes}
\input{10/3/2/proto}
\input{10/3/3/multi}
\input{10/3/4/super}
\secrel{10.3.5 Mixins and Traits   79}

Let’s return to our “blobs”.

When we write a class in Java, what are we really defining between the opening
and closing braces? It is not the entire class: that depends on the parent that
it extends, and so on recursively. Rather, what we define inside the braces is a
class extension. It only becomes a full-blown class because we also identify the
parent class in the same place.

Naturally, we should ask: Why? Why not separate the act of defining an extension
from applying the extension to a base class? That is, suppose instead of
\lsts{10/3/5/1.rkt}{rkt}
we instead write:
\lsts{10/3/5/2.rkt}{rkt}
and separately
\lsts{10/3/5/3.rkt}{rkt}
where B is some already-defined class.

Thusfar, it looks like we’ve just gone to great lengths to obtain what we had
before. However, the function-application-like syntax is meant to be suggestive:
we can “apply” this extension to several different base classes. Thus:
\lsts{10/3/5/4.rkt}{rkt}
and so on. What we have done by separating the definition of E from that of the
class it extends is to liberate class extensions from the tyranny of the fixed
base class. We have a name for these extensions: they’re called mixins.
\note{The term “mixin” originated in Common Lisp, where it was a particular
pattern of using multiple inheritance.
Lipstick on a pig.}

Mixins make class definition more compositional. They provide many of the
benefits of multiple-inheritance (reusing multiple fragments of functionality)
but within the aegis of a single-inheritance language (i.e., no complicated
rules about lookup order). Observe that when desugaring, it’s actually quite
easy to add mixins to the language. A mixin is primarily a “function over
classes’;. Because we have already determined how to desugar classes, and our
target language for desugaring also has functions, and classes desugar to
expressions that can be nested inside functions, it becomes almost trivial to
implement a simple model of mixins.
\note{This is a case where the greater generality of the target language of
desugaring can lead us to a better construct, if we reflect it back into the
source language.}

In a typed language, a good design for mixins can actually improve
object-oriented programming practice. Suppose we’re defining a mixin-based
version of Java. If a mixin is effectively a class-to-class function, what is
the “type” of this “function”? Clearly, mixin ought to use interfaces to
describe what it expects and provides. Java already enables (but does not
require) the latter, but it does not enable the former: a class (extension)
extends another class—with all its members visible to the extension— not its
interface. That means it obtains all of the parent’s behavior, not a
specification thereof. In turn, if the parent changes, the class might break.

In a mixin language, we can instead write
\lsts{10/3/5/5.rkt}{rkt}
where I is an interface. Then M can only be applied to a class that satisfies
the interface I, and in turn the language can ensure that only members specified
in I are visible in M. This follows one of the important principles of good
software design.
\note{“Program to an interface, not an implementation.” —Design Patterns}

A good design for mixins can go even further. A class can only be used once in
an inheritance chain, by definition (if a class eventually referred back to
itself, there would be a cycle in the inheritance chain, causing potential
infinite loops). In contrast, when we compose functions, we have no qualms about
using the same function twice (e.g.: \verb|(map ... (filter ... (map ...))))|.
Is there value to using a mixin twice?
\note{There certainly is! See sections 3 and 4 of
\href{http://www.cs.brown.edu/~sk/Publications/Papers/Published/fkf-classes-mixins/}{Classes
and Mixins}.}

Mixins solve an important problem that arises in the design of libraries.
Suppose we have a dozen different features which can be combined in different
ways. How many classes should we provide? Furthermore, not all of these can be
combined with each other. It is obviously impractical to generate the entire
combinatorial explosion of classes. It would be better if the devleoper could
pick and choose the features they care about, with some mechanism to prevent
unreasonable combinations. This is precisely the problem that mixins solve: they
provide the class extensions, which the developers can combine, in an
interface-preserving way, to create just the classes they need.
\note{Mixins are used extensively in the Racket GUI library.
For instance, color:text-mixin consumes basic text editor interfaces and
implements the colored text editor interface. The latter is iself a basic text
editor interface, so additional basic text mixins can be applied to the result.}

\Exercise{
How does your favorite object-oriented library solve this problem?
}

Mixins do have one limitation: they enforce a linearity of composition. This
strictness is sometimes misplaced, because it puts a burden on programmers that
may not be necessary. A generalization of mixins called traits says that instead
of extending a single mixin, we can extend a set of them. Of course, the moment
we extend more than one, we must again contend with potential name-clashes. Thus
traits must be equipped with mechanisms for resolving name clashes, often in the
form of some name-combination algebra. Traits thus offer a nice complement to
mixins, enabling programmers to choose the mechanism that best fits their needs.
As a result, Racket provides both mixins and traits.

\secup

\secup

% \secrel{11 Memory Management 81}\secdown
\input{11/1/garbage}
\input{11/2/recovery}
\input{11/3/reclamation}
\input{11/4/automated}
\secrel{11.5 Convervative Garbage Collection  . . 86}

We’ve explained that the typical root set consists of the environment, global
variables, and some choice ephemerals. Where else might references reside?

In most languages, nowhere else. But some languages (I’m looking at you, C and
C++) allow references to be turned into arbitrary numbers, and arbitrary numbers
to be turned back into references. As a result, in principle, any numeric value
in the program (which, because of the nature of C and C++’s types, virtually any
value in the program) could potentially be treated as a reference.

This is problematic for two reasons. First, it means the GC can no longer limit
its attention to the small root set; instead, the entire store is now
potentially the root set. Second, if the GC tries to modify the object in any
way—e.g., to record a “visited” bit during traversal—then it is potentially
changing non-reference values: e.g., it might actually be changing an innocent
numeric constant in the program. As a result, the particular confluence of
features in languages like C and C++ conspire to make sound, efficient GC
extremely difficult.

But not impossible. A stimulating line of research called conservative GC has
managed to create reasonably effective GC systems for such languages. The
principle behind conservative GC notes that, while in principle every store
location might be a root, in practice many of them are not. It then proceeds
through a series of increasingly clever observations to deduce what must not be
a reference (the opposite of a traditional GC) and can hence be safely ignored:
for instance, on a word-aligned architecture, no odd number can never be a
reference. By skipping most of the store, by making some basic assumptions about
program behavior (e.g., that it will not manufacture certain kinds of
references), and by being careful to not modify the store—e.g., changing bits in
values, or moving data around—it can actually yield a reasonably effective GC
strategy.
\note{Nevertheless, it is a bit of a dog walking on its hind legs.}

Conservative GC is often popular with programming language implementations that
are written in, or rely on a base of code in, C or C++. For instance, early
versions of \racket\ relied exclusively on it. There are many good reasons for
this:
\begin{enumerate}
  \item 
It offers a quick bootstrapping technique, so the language implementor can focus
on other, more innovative, features.
  \item 
A language that controls all references (as Racket does) can easily create
memory representations that are especially conducive to increasing the
effectiveness of the GC (e.g., padding all true numbers with a one in the
least-significant-bit).
  \item 
It enables easy interoperation with useful libraries written in C and C++
(provided, of course, that they too meet the expectations of the technique).
\end{enumerate}

A word on vocabulary is in order. As we have argued [REF], all practical GC
techniques are “conservative” in that they approximate truth with reachability.
The word “conservative” has, however, become a term-of-art to refer to a GC
technique that operates in an uncooperative (and hopefully not hostile) run-time
system.

\input{11/6/precise}
\secup

% \secrel{12 Representation Decisions 87}\secdown

Go back and look again at our interpreter for function as values \ref{}. Do you
see something curiously non-uniform about it?

\DoNow{
No, really, do. Do you?
}

Consider how we chose to represent our two different kinds of values: numbers
and functions. Ignoring the superficial numV and closV wrappers, focus on the
underlying data representations. We represented the interpreted language’s
numbers as Racket numbers, but we did not represent the interpreted language’s
functions (closures) as Racket functions (closures).

That’s our non-uniformity. It would have been more uniform to use Racket’s
representations for both, or also to not use Racket’s representation for either.
So why did we make this particular choice?

We were trying to illustrate and point, and that point is what we will explore
right now.

\input{12/1/changing}
\input{12/2/errors}
\input{12/3/meanings}
\secrel{12.4 One More Example   90}

Let’s consider one more representation change. What is an environment?

An environment is a map from names to values (or locations, once we have
mutation). We’ve chosen to implement the mapping through a data structure,
but...do we have another way to represent maps? As functions, of course! An
environment, then, is a function that takes a name as an argument and return its
bound value (or an error):
\lsts{12/4/1.rkt}{rkt}
What is the empty environment? It’s the one that returns an error no matter what
name you try to look up:
\lsts{12/4/2.rkt}{rkt}
(In principle we should put a type annotation on the return, and it should be
Value, except of course thiis is vacuous.) Extending an environment with a
binding creates a function that takes a name and checks whether it is the name
just extended; if so it returns the corresponding value, otherwise it punts to
the environment being extended:
\lsts{12/4/3.rkt}{rkt}
Finally, how do we look up a name in an environment? We simply apply the
environment!
\lsts{12/4/4.rkt}{rkt}
And with that, we’re done!

\secup

% \input{13/desugaring}
% \secrel{14 Control Operations 102}\secdown

The term control refers to any programming language instruction that causes
evaluation to proceed, because it “controls” the program counter of the machine.
In that sense, even a simple arithmetic expression should qualify as “control”,
and operations such as sequential program execution, or function calls and
returns, most certainly do. However, in practice we use the term to refer
primarily to those operations that cause non-local transfer of control,
especially beyond that of mere functions and procedures, and the next step up,
namely exceptions. We will study such operations in this chapter.

As we study the following control operators, it’s worth remembering that even
without them, we still have languages that are Turing-complete, and therefore
have no more “power”. Therefore, what control operators do is change and
potentially improve the way we express our intent, and therefore enhance the
structure of programs. Thus, it pays to being our study by focusing on program
structure.

\input{14/1/onweb}
\input{14/2/contpass}
\secrel{14.3 Generators   . . 117}
\secdown
\secrel{14.3.1 Design Variations   117}
\secrel{14.3.2 Implementing Generators  . . 119}
\secup

\secrel{14.4 Continuations and Stacks   121}

Surprising as it may seem, CPS conversion actually provides tremendous insight
into the nature of the program execution stack. The first thing to understand is
that every continuation is actually the stack itself. This might seem odd, given
that stacks are low-level machine primitives while continuations are seemingly
complex procedures. But what is the stack, really?
\begin{itemize}
  \item 
It’s a record of what remains to be done in the computation. So is the
continuation.
  \item 
It’s traditionally thought of as a list of stack frames. That is, each frame has
a reference to the frames remaining after it finishes. Similarly, each
continuation is a small procedure that refers to—and hence closes over—its own
continuation. If we had chosen a different representation for program
instructions, combining this with the data structure representation of closures,
we would obtain a continuation representation that is essentially the same as
the machine stack.
  \item 
Each stack frame also stores procedure parameters. This is implicitly managed by
the procedural representation of continuations, whereas this was done explicitly
in the data stucture representation (using bind).
  \item 
Each frame also has space for “local variables”. In principle so does the
continuation, though by using the macro implementation of local binding, we’ve
effectively reduced everything to procedure parameters. Conceptually, however,
some of these are “true” procedure parameters while others are local bindings
turned into procedure parameters by a macro.
  \item 
The stack has references to, but does not close over, the heap. Thus changes to
the heap are visible across stack frames. In precisely the same way, closures
refer to, but do not close over, the store, so changes to the store are visible
across closures.
\end{itemize}
Therefore, traditionally the stack is responsible for maintaining lexical scope,
which we get automatically because we are using closures in a statically-scoped
language.

Now we can study the conversion of various terms to understand the mapping to
stacks. For instance, consider the conversion of a function application \ref{}:
\lsts{14/4/1.rkt}{rkt}
How do we ``read'' this? As follows:
\begin{itemize}
  \item 
Let’s use k to refer to the stack present before the function application begins
to evaluate.
  \item 
When we begin to evaluate the function position (f), create a new stack frame
((lambda (fv) ...)). This frame has one free identifier: k. Thus its closure
needs to record one element of the environment, namely the rest of the stack.
  \item 
The code portion of the stack frame represents what is left to be done once we
obtain a value for the function: evaluate the argument, and perform the
application, and return the result to the stack expecting the result of the
application: k.
  \item 
When evaluation of f completes, we begin to evaluate a, which also creates a
stack frame: (lambda (av) ...). This frame has two free identifiers: k and fv.
This tells us:
\begin{itemize}
  \item 
We no longer need the stack frame for evaluating the function position, but
  \item 
we now need a temporary that records the value—hopefully a function value—of
evaluating the function position.
\end{itemize}
  \item 
The code portion of this second frame also represents what is left to be done:
invoke the function value with the argument, in the stack expecting the value of
the application.
\end{itemize}
Let us apply similar reasoning to conditionals:
\lsts{14/4/2.rkt}{rkt}
It says that to evaluate the conditional expression we have to create a new
stack frame. This frame closes over the stack expecting the value of the entire
conditional. This frame makes a decision based on the value of the conditional
expression, and invokes one of the other expressions. Once we have examined this
value the frame created to evaluate the conditional expression is no longer
necessary, so evaluation can proceed in k.

Viewed through this lens, we can more easily provide an operational explanation
for generators. Each generator has its own private stack, and when execution
attempts to return past its end, our implementation raises an error. On
invocation, a generator stores a reference to the stack of the “rest of the
program” in where-to-go, and resumes its own stack, which is referred to by
resumer. On yielding, the system swaps references to stacks. Coroutines,
threads, and generators are all conceptually similar: they are all mechanisms to
create “many little stacks” instead of having a single, global stack.

\secrel{14.5 Tail Calls   . . 123}

Observe that the stack patterns above add a frame to the current stack, perform
some evaluation, and eventually always return to the current stack. In
particular, observe that in an application, we need stack space to evaluate the
function position and then the arguments, but once all these are evaluated, we
resume computation using the stack we started out with before the application.
In other words, function calls do not themselves need to consume stack space: we
only need space to compute the arguments.

However, not all languages observe or respect this property. In languages that
do, programmers can use recursion to obtain iterative behavior: i.e., a sequence
of function calls can consume no more stack space than no function calls at all.
This removes the need to create special looping constructs; indeed, loops can
simply be expressed as a syntactic sugar.

Of course, this property does not apply in general. If a call to f is performed
to compute an argument to a call to g, the call to f is still consuming space
relative to the context surrounding g. Thus, we should really speak of a
relationship between expressions: one expression is in tail position relative to
another if its evaluation requires no additional stack space beyond the other.
In our CPS macro, every expression that uses k as its continuation—such as a
function application after all the sub-expressions have been evaluated, or the
then- and else-branches of a conditional—are all in tail position relative to
the enclosing application (and perhaps recursively further up). In contrast,
every expression that has to create a new stack frame is not in tail position.

Some languages have special support for tail recursion: when a procedure calls
itself in tail position relative to its body. This is obviously useful, because
it enables recursion to efficiently implement loops. However, it hurts “loops”
that cannot be squeezed into a single recursive function. For instance, when
implementing a scanner or other state machine, it is most convenient to have a
set of functions each representing one state, and transitioning to other states
by making (tail) function calls. It is onerous (and misses the point) to turn
these into a single recursive function. If, however, a language recognizes tail
calls as such, it can optimize these cross-function calls just as much as it
does intra-function ones.

Racket, in particular, promises to implement tail calls without allocating
additional stack space. Though some people refer to this as “tail call
optimization”, this term is misleading: an optimization is optional, whereas
whether or not a language promises to properly implement tail calls is a
semantic feature. Developers need to know how the language will behave because
it affects how they program.

Because of this feature, observe something interesting about the program after
CPS transformation: all of its function applications are themselves tail calls!
You can see this starting with the read-number/suspend example that began this
chapter: any pending computation was put into the continuation argument.
Assuming the program might terminate at any call is tantamount to not using any
stack space at all (because the stack would get wiped out).

\Exercise{
How is the program able to function in the absence of a stack?
}

\secrel{14.6 Continuations as a Language Feature  124}\secdown

With this insight into the connection between continuations and stacks, we can
now return to the treatment of procedures: we ignored the continuation at the
point of closure creation and instead only used the one at the point of closure
invocation. This of course corresponds to normal procedure behavior. But now we
can ask, what if we use the creation-time continuation instead? This would
correspond to maintaining a reference to (a copy of) the stack at the point of
“procedure” creation, and when the procedure is applied, ignoring the dynamic
evaluation and going back to the point of procedure creation.

In principle, we are trying to leave lambda intact and instead give ourselves a
language construct that corresponds to this behavior:
\note{cc = “current continuation”}
\lsts{14/6/1.rkt}{rkt}

What this says is that either way, control will return to the expression that
immediately surrounds the let/cc: either by falling through (because the
continuation of the body, b, is k) or—more interestingly—by invoking the
continuation, which discards the dynamic continuation dyn/k by simply ignoring
it and returning to k instead.

Here’s the simplest test:
\lsts{14/6/2.rkt}{rkt}
This confirms that if we never use the continuation, evaluation of the body
proceeds as if the let/cc weren’t there at all (because of the ((cps b) k)). If
we use it, the value given to the continuation returns to the point of creation:
\lsts{14/6/3.rkt}{rkt}
This example, of course, isn’t revealing, but consider this one:
\lsts{14/6/4.rkt}{rkt}
This confirms that the addition actually happens. But what about the dynamic
continuation?
\lsts{14/6/5.rkt}{rkt}
This shows that the addition by 2 never happens, i.e., the dynamic continuation
is indeed ignored. And just to be sure that the continuation at the point of
creation is respected, observe:
\lsts{14/6/6.rkt}{rkt}

From these examples, you have probably noticed a familiar pattern: esc here is
behaving like an exception. That is, if you do not throw an exception (in this
case, invoke a continuation) it’s as if it’s not there, but if you do throw it,
all pending intermediate computation is ignored and computation returns to the
point of exception creation.

\Exercise{
Using let/cc and macros, create a throw/catch mechanism.
}

However, these examples only scratch the surface of available power, because the
continuation at the point of invocation is always an extension of one at the
point of creation: i.e., the latter is just earlier in the stack than the
former. However, nothing actually demands that k and dyn-k be at all related.
That means they are in fact free to be unrelated, which means each can be a
distinct stack, so we can in fact easily implement stack-switching procedures
with them.

\Exercise{
To be properly analogous to lambda, we should have introduced a construct
called, say, cont-lambda with the following expansion:
\lsts{14/6/7.rkt}{rkt}
Why didn’t we? Consider both the static typing implications, and also how we
might construct the above exception-like behaviors using this construct instead.
}

\secrel{14.6.1 Presentation in the Language  125}

\secrel{14.6.2 Defining Generators  . 126}

\secrel{14.6.3 Defining Threads   127}

Having done generators, let’s do another, similar primitive: threads. That is,
let’s assume we want to be able to write a program such as this:
\lsts{14/6/3/1.rkt}{rkt}
and expect the output
\lst{14/6/3/2.rkt}
We’ll build all the pieces necessary to achieve this.

Let’s start by defining the thread scheduler. It consumes a list of “threads”,
whose interface we assume will be a procedure that consumes a continuation to
which it eventually yields control. Each time the scheduler reactivates the
thread, it supplies it with a continuation. The scheduler might be choose
between threads in a simple round-robin manner, or it might use some more
complex algorithm; the details of how it chooses don’t concern us here.

As with generators, we’ll assume that yielding is done by invoking a procedure
named by the user: y, above. We could use name capture \ref{}\ to automatically
bind a name like yield.

More importantly, notice that the user of the thread system manually yields
control. This is called cooperative multitasking. Instead, we could have chosen
to have a timer or other intrinsic mechanism automtically yield without the
user’s permission; this is called preemptive multitasking (because the system
“pre-empts”—i.e., wrests control from—the thread). While the distinction is
important for buildling systems, it is not interesting from the perspective of
setting up the continuations.

\Exercise{
After we are done building cooperative multitasking, implement preemptive
multitasking. What changes?
}

With our stated constraints, we can write a first scheduler. It consumes a lists
of threads and continues executing so long as there are threads remaining. Each
time, it applies the thread procedure to a continuation that represents
returning to the scheduler and proceeding to the next thread:
\lsts{14/6/3/3.rkt}{rkt}
When the recipient thread invokes the continuation bound to after-thread,
control returns to the end of the first statement in the begin sequence. As a
result, the value supplied to the continuation is ignored (and can hence be any
dummy value; we’ll chose 'dummy, so that we can easily spot it if it shows up in
undesired places). Control then proceeds with the rest of the scheduler loop
after appending the most recently invoked thread to the end of the list of
threads (i.e., treating the list as a circular queue).

Now let’s define a thread. As we’ve said, it will be a procedure of one
argument, the scheduler’s continuation. Because the thread needs to resume,
i.e., continue where it left off, presumably it must store where it last was:
we’ll call this thread-resumer. Initially this is the entire thread body, but on
subsequent instances it will be a continuation: specifically, the continuation
of invoking yield. Thus, we obtain the following skeleton:
\lsts{14/6/3/4.rkt}{rkt}
That still leaves the yielder. It needs to be a procedure of no arguments that
stores the thread’s continuation in thread-resumer, and then invoke the
scheduler continuation with 'dummy. However, which scheduler continuation does
it need to invoke? Not the one provided on thread initiation, but rather the
most recent one. Thus, we must somehow “thread” the value in sched-k to the
yielder. There are many ways to accomplish it, but the simplest, perhaps most
brutal, way is to simply reconstruct the yielder on each thread resumption,
always closed over the most recent value of sched-k:
\lsts{14/6/3/5.rkt}{rkt}
When we run this ensemble, we get:
\lst{14/6/3/6.rkt}
Hey, that’s what we wanted! But it continues:
\lst{14/6/3/7.rkt}
Hmmm.

What’s happening? Well, we’ve been quiet all along about what needs to happen
when a thread reaches its end. In fact, control just returns to the thread
scheduler, which appends the thread to the end of the queue, and when the thread
comes to the head of the queue again, control resumes from the same previously
stored continuation: the one corresponding to printing the third value. This
prints, control returns, the thread is appended...ad infinitum.

Clearly, we need the thread scheduler to be notified when a thread has
terminated, so that the scheduler can remove it from the thread queue. We’ll
create a simple datatype to represent this signal:
\lsts{14/6/3/8.rkt}{rkt}
(In a real system, of course, these status messages might also carry informative
values from the computation.) We must now modify our scheduler to actually check
for and use these values:
\lsts{14/6/3/9.rkt}{rkt}
We have to now modify our thread representation in two ways: it must provide
Tsus- pended to the scheduler’s continuation on intermediate returns, and
provide Tdone when it terminates. Where does it terminate? After executing the
code in the body, b .... Observe, finally, that the termination process must be
sure to use the latest scheduler continuation, just as yielding does. Thus:
\lsts{14/6/3/10.rkt}{rkt}
If we now replace scheduler-loop-0 with scheduler-loop-1 and thread-0 with
thread-1 and re-run our example program above, we get just the output we desire.

\secrel{14.6.4 Better Primitives for Web Programming  131}

Finally, to tie the knot back to where we began, let’s return to read-number:
observe that, if the language running the server program has call/cc, instead of
having to CPS the entire program, we can simply capture the current continuation
and save it in the hash table, leaving the program structure again intact.

\secup

\secup


\secrel{15 Checking Program Invariants Statically: Types 131}\secdown
\secrel{15.1 Types as Static Disciplines  . . 133}
\secrel{15.2 A Classical View of Types  . . 134}
\secdown
\secrel{15.2.1 A Simple Type Checker  . . 134}
\secrel{15.2.2 Type-Checking Conditionals  139}
\secrel{15.2.3 Recursion in Code  . . 139}
\secrel{15.2.4 Recursion in Data   142}
\secrel{15.2.5 Types, Time, and Space  . . 144}
\secrel{15.2.6 Types and Mutation  . . 146}
\secrel{15.2.7 The Central Theorem: Type Soundness  147}
\secup
\secrel{15.3 Extensions to the Core   . 148}
\secdown
\secrel{15.3.1 Explicit Parametric Polymorphism  148}
\secrel{15.3.2 Type Inference   . 155}
\secrel{15.3.3 Union Types   . . 164}
\secrel{15.3.4 Nominal Versus Structural Systems  . . 170}
\secrel{15.3.5 Intersection Types  . . 171}
\secrel{15.3.6 Recursive Types   172}
\secrel{15.3.7 Subtyping   . 173}
\secrel{15.3.8 Object Types   . . 176}
\secup
\secup

\secrel{16 Checking Program Invariants Dynamically: Contracts 179}\secdown
\secrel{16.1 Contracts as Predicates   . 181}
\secrel{16.2 Tags, Types, and Observations on Values  . 182}
\secrel{16.3 Higher-Order Contracts   . 183}
\secrel{16.4 Syntactic Convenience   . 187}
\secrel{16.5 Extending to Compound Data Structures  . 188}
\secrel{16.6 More on Contracts and Observations  189}
\secrel{16.7 Contracts and Mutation   . 189}
\secrel{16.8 Combining Contracts   . . 190}
\secrel{16.9 Blame   . 191}
\secup

\secrel{17 Alternate Application Semantics 195}\secdown

Long ago \ref{}, we considered the question of what to substitute when
performing application. Now we are ready to consider some alternatives. At the
time, we suggested just one alternative; in fact there are many more. To
understand this, see whether you can answer this question:

Which of these is the same?
\begin{itemize}[nosep]
  \item 
(f x (current-seconds))
  \item 
(f x (current-seconds))
  \item 
(f x (current-seconds))
\item
(f x (current-seconds))
\end{itemize}

What we’re about to find is that this fragment of syntax can have wildly
different run-time behaviors. For instance, there is the distinction we have
already mentioned: variation in when (current-seconds) is evaluated. There is
variation in how many times it is evaluated (and hence f is run). There is even
variation even in whether values for x flow strictly from the caller to the
callee, or can even flow in the opposite direction!

\secrel{17.1 Lazy Application 196}
\secdown

Let’s start by considering when parameters are reduced to values. That is, do we
substitute formal parameters with the value of the actual parameter, or with the
actual parameter expression itself? If we define
\lsts{17/1/1.rkt}{rkt}
and invoke it as
\lsts{17/1/2.rkt}{rkt}
does that reduce to
\lsts{17/1/3.rkt}{rkt}
or to
\lsts{17/1/4.rkt}{rkt}
? The former is called eager application, while the latter is lazy. Of course we
don’t want to return to defining interpreters by substitution, but it is always
useful to think of substitution as a design principle.
\note{Some people also use the term strict for the former. A more arcane
terminology is applicative-order evaluation for the former and normal-order
evaluation for the latter. Or, call-by-value for the former and call-by-name or
call-by-need for the latter. The last two terms—by-name versus by-need—actually
represent a technical distinction we will see below.
This concludes our name-dump.}

\secrel{17.1.1 A Lazy Application Example . . . . . . . . . . . . . . . . . . 196}

\secrel{17.1.2 What Are Values? . . . . . . . . . . . . . . . . . . . . . . . 197}

\secrel{17.1.3 What Causes Evaluation? . 198}

Let us now return to discussing arithmetic expressions. On evaluating (+ 1 2), a
lazy application interpreter could return any number of things, including
(suspendV (+ 1 2) mt-env). In this way suspended computation could cascade on
suspended computation, and in the limiting case every program would return
immediately with an “answer”: the thunk representing the suspension of its
computation.
\note{It is legitimate to write mt-env here because even if the (+ 1 2)
expression was written in a non-empty environment, it has no free identifiers,
so it doesn’t need any of the environment’s bindings.}

Clearly, something must force a suspension to be lifted. (Lifting a suspension
means, of course, evaluating its body in the stored environment.) Those
expression positions that undo suspensions are called strictness points. The
most obvious strictness point is the interactive environment’s printer, because
a user clearly would not use such an environment if they did not wish to see
answers. We will embody the act of lifting suspension in the procedure strict:
\lsts{17/1/3/1.rkt}{rkt}
where the returned Value is guaranteed to not be a suspendV. We can imagine the
printer as wrapping strict around the result of evaluating the program, to
obtain a value to print.

\DoNow{
What impact would using closures to represent suspended computation have had?
}

The definition of strict above depends crucially on being able to distinguish
deferred computations—which are internally-constructed closures—from
user-defined closures. Had we conflated the two, then we would have to guess
what to do with zero-argument closures. If we fail to further process them, we
might incorrectly get an error (e.g., + might get a thunk rather than the
numeric value residing inside it). If we do process it further, we might
accidentally force a user-defined thunk prematurely. In short, we need a flag on
thunks telling us whether they are internal or user-defined. For clarity, our
interpreter uses a separate variant.

Let us now return to the interaction between strict and the interpreter.
Unfortunately, as we have defined things, this will cause an infinite loop. The
act of trying to interpret an addition creates a suspension, which strict tries
to undo by forcing the interpreter to interpret an addition, which.... Clearly,
therefore, we cannot have every expression simply suspend its computation;
instead, we will limit suspension to applications. This suffices to give us the
rich power of laziness, without making the language absurd.

\secrel{17.1.4 An Interpreter . . . . . . . . . . . . . . . . . . . . . . . . . . 199}

\secrel{17.1.5 Laziness and Mutation . . . . . . . . . . . . . . . . . . . . . 201}

\secrel{17.1.6 Caching Computation . . . . . . . . . . . . . . . . . . . . . 201}

\secup

\secrel{17.2 Reactive Application 201}\secdown

Now consider an expression like (current-seconds). When we evaluate it, it
returns a single number representing the current time. For instance,
\lst{17/2/1.rkt}
However, even as we stare at this value, it is already out-of-date! It
represents the time when the function application occurred, but does not stay
current.

\secrel{17.2.1 Motivating Example: A Timer . 202}

Suppose we were trying to implement a timer that measures elapsed time. Ideally,
we would like to write a program such as this:
\lsts{17/2/1/1.rkt}{rkt}
In JavaScript, we might write:
\lsts{17/2/1/2.java}{java}
On most machines this Racket expression, or the value of elapsed in JavaScript,
will evaluate to 0 or some other very small number. This is because these
programs represent one measure of the elapsed time: that at the second
invocation of the procedure that gets the current time. This gives us an
instanteous time split, but not an actual timer.

In most languages, to build an actual timer, we would have to create an instance
of some sort of timer object, and install a callback. Every time the clock
ticks, the timer object—representing the operating system—invokes the callback.
The callback is then responsible for updating values in the rest of the system,
and hopefully doing so globally and consistently. However, it cannot do so by
returning values, because it would return to the operating system, which is
agnostic to and does not care about our application; therefore, the callback is
forced to perform its action through mutation. In JavaScript, for instance:
\lsts{17/2/1/3.java}{java}
assuming we have an HTML page with an id named curTime, and that the onload or
other callback invokes startTimer.

One alternative to this spaghetti code is for the application program to
repeatedly poll the operating system for the current time. However:
\begin{itemize}
  \item 
Calling too frequently wastes resources, while calling too infrequently results
in incorrect answers. However, to call at just the right resolution, we would
need a timer signal in the first place!
  \item 
While it may be possible to create such a polling loop for regular events such
as timers, it is impossible to do so accurately for unpredictable behaviors such
as user input (whose frequency cannot, in general, be predicted).
  \item 
On top of all this, writing this loop pollutes the program’s structure and
forces the developer to sustain this extra burden.
\end{itemize}

The callback-based solution, however, demonstrates an inversion of control.
Instead of the application program calling the operating system, the operating
system has now been charged with calling (into) the application program. The
reactive behavior that should have been deeply nested, inside the display
expression, has instead been brought to the top-level, and its value drives the
other computations. The fundamental cause for this is that the world is in
control, not the program, so external stimuli determine when and how the program
should next run, not intrinsic program expressions.

\secrel{17.2.2 Callback Types are Four-Letter Words . . . . . . . . . . . . . 203}

\secrel{17.2.3 The Alternative: Reactive Languages . 204}

Consider the FrTime (pronounced “Father Time”) language in DrRacket. If we run
this expression at the interactions window, we still get 0 or some other very
small non-negative number:
\note{In DrRacket v5.3, you must select the language from the Language menu;
writing \#lang frtime will not provide the interesting interactions window
behavior.}
\lsts{17/2/3/1.rkt}{rkt}
In fact, we can try several other expressions and see that FrTime seems to have
exactly like traditional Racket.

However, it also binds a few additional identifiers. For instance, it provides a
value bound to seconds. If we type this into the interaction prompt, we get
something very interesting! First we see 1353030630, then a second later
1353030631, another second later 1353030632, and so on. This kind of value is
called a behavior: a value that changes over time. Except we haven’t written any
callbacks or other code to keep it current.

A behavior can be used in computations. For instance, we can write (- seconds
seconds), and this always evaluates to 0. Here are some more expressions to try
at the interaction prompt:
\lsts{17/2/3/2.rkt}{rkt}
As you can see, being a behavior is “sticky”: if any sub-expression is a
behavior, so is its enclosing expression.

Thanks to this evaluation model, every time seconds updates the entire
application happens afresh: as a result, even though we have written seemingly
simple expressions without any explicit loop-like control, the program still
“loops”. In other words, having explored an application semantics where
arguments are evaluated once, then another where they may be evaluated zero
times, now we have one where they are evaluated as many times as necessary, and
the entire corresponding function with them. As a consequence, reactive values
that are “inside” an expression no longer brought “outisde”; rather, they can
reside nested inside expressions, giving programmers a more natural means of
expression. This style of evaluation is called dataflow or functional reactive
programming.
\note{Historically, dataflow has tended to refer to languages with first-order
functions, whereas functional reactive languages support higher-order functions
too. See the \href{http://www.flapjax-lang.org/}{Flapjax Web site}.}

FrTime implements what we call transparent reactivity, whereby the programmer
can inject a reactive behavior anywhere in a program’s evaluation without
needing to make any syntactic changes to its context. This has the virtue of
making it easy to inject reactivity into existing programs, but it can make the
evaluation and cost model more complex for programmers. In other languages,
programmers can instead explicitly introduce behavior through appropriate
primitives, trading convenience for greater predictability. FrTime’s sister
language, Flapjax, an extension of JavaScript, provides both modes.

\secrel{17.2.4 Implementing Transparent Reactivity . 205}

To make an existing language implement transparent reactivity, we have to
(naturally) alter the semantics of function application. We will do this in two
steps. First we will rewrite reactive function applications into a more complex
form, then we will show how this more complex form enables reactive updates.

\secdown

\secrel{Dataflow Graph Construction}

The essence of making an application reactive is simple to explain through
desguaring. Assume we have defined a new constructor behavior. The constructor
takes a thunk that represents what computation to perform every time an argument
updates, and all the values that the expression depends on. The value it
produces stores the current value of the behavior. Then an expression like (f x
y) turns into
\lsts{17/2/4/1.rkt}{rkt}
where we assume, given a non-behavior constant, current-value behaves as the
identity function.

Let us look at two examples of using the above definition. Consider the trivial
case where neither parameter is a behavior, e.g., (+ 3 4). This desugars to
\lsts{17/2/4/2.rkt}{rkt}
Since both 3 and 4 are numbers, not behaviors, this reduces to (+ 3 4), which is
precisely what we would like. This reflects an important principle: when no
behaviors are present, programs behave exactly as they did in the non-reactive
version of the language.

If we compute (+ 1 seconds), this expands to
\lsts{17/2/4/3.rkt}{rkt}
Because seconds is a behavior, this reduces to
\lsts{17/2/4/4.rkt}{rkt}
Any expression that depends on this now sees its argument also become a
behavior, making the property “sticky” as we argued before.

\Exercise{
In what way, if any, did the above desugaring depend on eager evaluation?
}

\secrel{Dataflow Graph Update}

Of course, simply constructing behavior values is not enough. The key additional
information is in the extra arguments to behavior. The language filters out
those arguments that are themselves behaviors (e.g., seconds, above) and
registers this new behavior as one of that depends on those existing ones. This
registration process creates a graph of behavior expression dependencies, known
as a dataflow graph (since it reflects the paths along which data need to flow).

If the program did not evaluate to any behaviors, then evaluation simply
produces an answer, and there are no graphs created. If, however, there are
behavior dependencies, then evaluation produces not a traditional answer but a
behavior value, with dependencies already recorded. (In practice, it is useful
to also track which primitive behaviors are actually necessary, to avoid
unnecessarily evaluating primitives that no other behavior in the program refers
to.) In short, program execution generates a dataflow graph. Thus, we do not
need a special, new evaluator for the language; we instead embed the
graph-construction semantics in traditional evaluation.

Now a dataflow propagation algorithm begins to execute. Every time a primitive
behavior changes, the algorithm applies its stored thunk, obtains its new value,
stores it, and then signals each behavior dependent on it. For instance, if
seconds updates, it notifies the (+ 1 seconds) expression’s behavior. The latter
behavior now evaluates its thunk, ($\lambda$ () (+ (current-value 1)
(current-value seconds))). This adds 1 to the newest value of seconds, making
that the new value of this behavior—just as we would expect.

\secrel{Evaluation Order}

The discussion above presents too simple a view of graph update. Consider the
following program:
\lsts{17/2/4/5.rkt}{rkt}
This program has one primitive behavior, seconds, and constructs two more: one
for (add1 seconds) and one more for the entire expression.

We would expect this expression to always be true. However, when seconds
updates, depending on the order in which it handles updates, it might update the
whole expression before it does (add1 seconds). Suppose the old value of seconds
was 100, so the new one is 101. However, the node for (add1 seconds) is still
storing its old value (because it has not yet been updated), so it holds (add1
100) or 101. That means the > compares 101 with 1, which is false, making this
expression return a value that should simply never have ensued from its static
description. This situation is called a glitch.

There is an easy solution to avoiding glitches, which the above example
illustrates (and that a theorem can show is sufficient). This is to
topologically sort the nodes. Then, every node is only processed after it
depends on have been, so there is no danger of seeing outdated or inconsistent
values.

The problem becomes more difficult in the presence of cycles in the graph. In
those cases, we need special recursion operators that can take an initial value
for the cyclic behavior. This makes it possible to break the cyclic dependency,
reducing evaluation to the process that has already been defined.

There is much more to say about the evaluation of dataflow languages, such as
the treatment of conditionals and a dual notion to behaviors that is discrete
and streamlike. I hope you will read the literature on reactive languages to
learn more about these topics.

\Exercise{
Earlier we picked on a Haskell library. To be fair, however, the reactive
solution we have shown was enunciated in Haskell, whose lazy evaluation
makes this form of evaluation relatively easy to support.

Implement reactive evaluation using laziness.
}

\secup

\secup

\secup


\clearpage
\addcontentsline{toc}{chapter}{Index \ru{Предметный указатель}}\printindex

\end{document}
