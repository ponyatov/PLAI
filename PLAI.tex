\input{../texheader/ebook}

\usepackage{../texheader/lstrkt}\lstdefinestyle{rkt}{language=rkt}

\newcommand{\Exercise}[1]{
	\begin{description}
		\item{\textcolor{red}{Упражнение}}\\#1
	\end{description}
}

\newcommand{\DoNow}[1]{
	\begin{description}
		\item{\textcolor{red}{Сделайте\,!}}\\#1
	\end{description}
}

\title{{\Large{PLAI}}\\
{\Large{Programming Languages: Application and Interpretation}}\\
{\small{second edition}}\\
{\Large{\ru{языки программирования:\\применение и интерпретация}}}}

\author{\copyright\ Shriram Krishnamurthy\\
\ru{перевод Dmitry Ponyatov \email{dponyatov@gmail.com}}}

\begin{document}
\maketitle
\tableofcontents
\secdown

% \secrel{Introduction \ru{Введение}}\secdown
\secrel{Our Philosophy \ru{Наша философия}}

Please watch the video on \ru{Пожалуйста посмотрите это видео на}
\href{https://www.youtube.com/watch?v=3N__tvmZrzc}{YouTube}.

Someday there will be a textual description here instead.
\ru{Когда нибудь здесь будет полное текстовое описание того, о чем в нем
говориться.}

\secrel{The Structure of This Book \ru{Структура книги}}

В отличие от большинства учебников, эта книга не следует подходу "сверху вниз".
Скорее она имеет форму повествования с возвратами к предыдущим темам.
Мы часто будет строить программы инкрементно, так же как мы бы это делали в
парном программировании. Наш код будет включать ошибки, не потому что мы не
знаем правильный ответ, но потому что это лучший способ научить вас, 
углубляясь от поверхностного в детали. Включение намеренных ошибок делает
невозможным для вас читать материал пассивно: вы должны взаимодействовать с
ним, потому что вы никогда не можете быть уверены в правильности того что вы
читаете.

В конце концов вы всегда получите правильный ответ. Тем не меннее, это
нелинейное повествование немного раздражающе в краткосрочной перспективе (у вас
всегда будет соблазн сказать "Скажите же мне наконец ответ !"), и это также
делает эту книгу плохим справочником (вы не можете открыть произвольную
страницу и быть уверенным что на ней написана правда). Тем не менее, это чувство
разочарования\ --- ощущение обучения. Я не знаю другого способа.

\bigskip
В некоторых местах вы встретите следующие выделения:

\Exercise{Это упражнение. Попробуйте это сделать.}

Это традиционное для учебников упражнение.
Это то, что вам нужно сделать по своему усмотрению.
Если вы используете эту книгу как часть курса, это упражнение хорошо
задавать как домашнюю работу. В противоположность этому вы также можете найти
подобные вопросы, выделенные как 

\DoNow{Здесь предполагаются немедленные действия. Вы видите это ?}

Когда вы доберетесь до одного из этих блоков, остановитесь. Прочитайте,
подумайте, сформулируйте ответ перед тем как продолжить чтение. Вы должны
сделать это потому что это действительно упражнение, но ответ уже есть в книге,
чаще всего в тексте непосредственно после упражнения (т.е. в части которую вы
сейчас читаете) или это что-то, что вы можете получить самостоятельно,
запустив программу. Если вы просто продолжите читать, то вы увидете ответ без
его обдумывания (или не увидите его вообще, если это инструкции по запуску
программы), так что вы ни (а) проверите свои знания, ни (б) улучшите свое
понимание. Другими словами, это дополнительные, явные попытки стимулировать
ваше активное обучение. В конце концов, я могу только поощрять вас работать;
решение применять это или нет остается за вами.


\secrel{The Language of This Book \ru{Язык программирования используемый в
книге}}

The main programming language used in this book is
\ru{Язык программирования используемый в книге}\ --- 
\href{http://www.racket-lang.org/}{\racket}.
Like with all operating systems, however,
\ru{Аналогично операционным системам,}
\racket\ actually supports a host of programming languages,
\ru{\racket-система является исполняющей средой для целого ряда языков
программирования,}
so you must tell \racket\
\ru{так что вы должны указать \racket у}
\emph{which} language you’re programming in.
\ru{\emph{на каком} языке вы программируете.}
You inform the Unix shell by writing a line like
\ru{Например, в Unix вы пишете строку типа}

\begin{verbatim}
#!/bin/sh
\end{verbatim}
at the top of a script;
\ru{в первой строке shell-скрипта;}
you inform the browser by writing, say,
\ru{вы указываете веб-браузеру тип документа, добавляя заголовок}

\begin{verbatim}
<!DOCTYPE HTML PUBLIC "-//W3C//DTD HTML 4.01//EN" ...>
\end{verbatim}

Similarly, \racket\ asks that you declare which language you will be using.
\ru{Аналогично, \racket\ требует от вас указать какой язык вы будете
использовать.}
\racket\ languages can have the same parenthetical syntax as \racket\ but with a
different semantics;
\ru{Диалекты языков \racket\ имеют тот же скобочный синтаксис, что и сам
\racket, но другую семантику;}
the same semantics but a different syntax;
\ru{ту же семантику но другой синтаксис;}
or different syntax and semantics.
\ru{или различные синтаксис и семантику.}
Thus every \racket\ program
\ru{Так что каждая программа, которую может выполнять \racket-система,}
begins with \#lang followed by the name of some language:
\ru{начинается со строки \#lang за которой следует имя диалекта языка:}
by default, it’s \racket\ \ru{по умолчанию, это оригинальный \racket\ }
written as \ru{указыватся как} \verb|racket|).
In this book we’ll almost always use the language\note{In DrRacket v.5.3,
go to Language, then Choose Language, and select ``Use the language declared in
the source''.}
\ru{В этой книге мы почти всегда будем использовать диалект}\note{\ru{В DrRacket
v.6.6, выберите меню\\\menu{Язык > Выбрать язык\ldots > Start your program with
\#lang to specify the desired dialect}.}}
\begin{verbatim}
#lang plai-typed
\end{verbatim}
When we deviate we’ll say so explicitly,
\ru{Когда мы будем отклоняться от этого правила,}
so unless indicated otherwise, put
\ru{это будет указано особо, так что если не указано иное, добавляйте заголовок}
\verb|#lang plai-typed|
at the top of every file
\ru{в начало каждого файла программы}
(and assume I’ve done the same
\ru{предполагается что я тоже это
сделал})\note{В DrRacket v.6.6 требуется установить расширение
plai-typed:\\\menu{Файл>Install package\ldots>Package
Source:>\url{github://github.com/mflatt/plai-typed/master}>Install>\ldots>Закрыть}}.

The \termdef{Typed PLAI}{Typed PLAI}\ language differs from traditional \racket\
most importantly by being statically typed.
\ru{Язык \term{Typed PLAI}\ отличается от традиционного \racket\ в основном
\emph{статической типизацией}.}
It also gives you some useful new constructs:
\ru{Он также дает вам некоторые новые полезные конструкции:}
\verb|define-type| \ru{определение-типа}, \verb|type-case| \ru{выбор-по-типу},
and \verb|test|\note{There are additional commands for controlling the output
of testing, for instance. \ru{Также существуют дополнительные команды для
управления выводом тестов.} Be sure to read the documentation for the language.
\ru{Обязательно прочитайте документацию для языка.}
In DrRacket v.5.3, go to \menu{Help>Help Desk}, and in the Help Desk search bar,
type \menu{plai-typed}. \ru{В DrRacket v.6.6 идите в меню \menu{Help>Help
Desk}, и в поле поиска \menu{Help Desk} введите \menu{plai-typed}.}}
Here’s an example of each in use.
\ru{Вот примеры использования каждого из них.} 
We can introduce new datatypes
\ru{Мы можем создавать новые типы данных\note{запустить программу можно нажав
\keys{Ctrl+R}}}:
\lst{src/1/p8_1.rkt}
You can roughly think of this as analogous to the following in Java:
\ru{Вы можете примерно понять идею в терминах языка \java:}
an abstract class \term{абстрактный класс} \verb|MisspelledAnimal| and two
concrete sub-classes \ru{и два конкретизирующих подкласса}\ \verb|caml|
\ru{верблюд} and \verb|yacc| \ru{якк},
each of which has one numeric constructor argument named
\ru{каждый из которых имеет конструктор с числовым аргументом}
\verb|humps| \ru{горбы} and \verb|height| \ru{высота}, respectively
\ru{соответственно}.

In this language, we construct instances as follows:
\ru{На этом языке мы строим экземпляры классов следующим образом:}
\lst{src/1/p8_2.rkt}
As the name suggests \ru{Как следует из названия,}, \verb|define-type| creates a
type of the given name \ru{создает тип с заданным именем}.
We can use
this when, for instance, binding the above instances to names:
\ru{Мы можем это использовать например при связывании эксземпляров с именами:}
\lst{src/1/p8_3.rkt}
In fact you don’t need these particular type declarations, because \term{Typed
PLAI} will infer types for you here and in many other cases.
\ru{Фактически вам не нужны эти частные определения типов, так как \term{Typed
PLAI} в этом и других случаях будет сам делать для вас \term{вывод типов}.}
Thus you could just as well have written
\ru{Так что вы можете написать короче}

\lst{src/1/p8_4.rkt}

\noindent
but we prefer to write explicit type declarations as a matter of both discipline
and comprehensibility when we return to programs later.
\ru{но мы предпочтем писать полные объявления типов с точки зрения как
дисциплины, так и усвояемости, когда мы вернемся к программам позже.}

The type names can even be used recursively, as we will see repeatedly in this
book (for instance, section \ref{sec2_4}).
\ru{Имена типов даже могут быть использованы рекурсивно, как мы увидим
несколько позже в этой книге (например в разделе \ref{sec2_4}).}

The language provides a pattern-matcher for use when writing expressions, such
as a function’s body:
\ru{Язык предоставляет pattern-matcher для использования при написании
выражений, таких как тело функции:}

\lst{src/1/p9_1.rkt}

\noindent
In the expression \ru{Например в выражении} (>= humps 2), for instance,
\verb|humps| is the name given to whatever value was given as the argument to
the constructor \ru{имя humps соответствует любому значению, данному как
аргумент для конструктора} \verb|caml|.

Finally, you should write test cases, ideally before you’ve defined your
function, but also afterwards to protect against accidental changes:
\ru{И наконец, вы должны написать тесты, в идеале до того как вы ее реализовали,
или хотя бы после, чтобы защититься от внезапных несоответствий в ее поведении
при внесении изменений в код:}

\lst{src/1/p9_2.rkt}

\noindent
When you run the above program, the language will give you verbose output telling
you both tests passed.
\ru{При запуске тестов язык даст вам подробный отчет, что оба теста успешно
пройдены.}
Read the documentation to learn how to suppress most of these messages.
\ru{Прочитайте документацию, чтобы узнать как подавить вывод большей части
этих сообщений.}

Here’s something important that is obscured above.
\ru{Вот еще кое-что важное, что было неясно выше.}
We’ve used the same name,
\ru{Мы использовали одно и то же имя,}
humps (and height), in both the datatype definition and in the fields of the
patternmatch.
\ru{и в определении типа данных, и в полях объекта при проверке совпадения
шаблонов.}
This is absolutely unnecessary because the two are related by position, not
name.
\ru{Это совершенно необязательно, так как каждая пара связана по позиции, а не
по имени.}
Thus, we could have as well written the function as
\ru{Так что мы могли бы также написать эту функцию как}

\lst{src/1/p9_3.rkt}

\noindent
Because each h is only visible in the case branch in which it is introduced, the
two hs do not in fact clash.
\ru{Так как каждый h виден только в той case-секции, где он используется,
два h фактически не сталкиваются.}
You can therefore use convention and readability to
dictate your choices.
\ru{Таким образом вы можете использовать соглашения по оформлению кода для
улучшения читаемости, и диктовать свой выбор.}
In general, it makes sense to provide a long and
descriptive name when defining the datatype\note{because you probably won’t use
that name again}, but shorter names in the type-case because you’re
likely to use use those names one or more times.
\ru{В общем, имеет смысл использовать длинные описательные имена при определении
типа данных\note{\ru{потому что вы возможно больше никогда не будете
использовать это имя снова}}, и короткие имена в type-case, где они обычно
используются несколько раз.}

I did just say you’re unlikely to use the field descriptors introduced in the
datatype definition, but you can.
\ru{Также я хочу упомянуть декрипторы полей класса, которые вы возможно
захотите использовать.}
The language provides \term{selectors} to extract fields without the need for
pattern-matching: e.g., caml-humps.
\ru{Язык предоставляет \term{селекторы} для получения значений полей без
необходимости использовать pattern-matching, например caml-humps.}
Sometimes,
it’s much easier to use the selector directly rather than go through the
pattern-matcher.
\ru{Иногда намного проще использовать селектор, чем возиться с мэтчингом
шаблонов.}
It often isn’t, as when defining  above, but just to be
clear, let’s write it without pattern-matching:
\ru{Часто это не так, как в случае определения good?, но для ясности давайте
напишем без pattern-matching:}

\lst{src/1/p9_4.rkt}

\DoNow{
What happens if you mis-apply functions to the wrong kinds of values\,?\\
\ru{Что произойдет, если вы ошибочно примените функции к неправильным типам
значений\,?}

For instance, what if you give the caml constructor a string\,?\\
\ru{Например, что если вы дадите конструктору caml строковый аргумент\,?}

What if you send a number into each version of good? above\,?\\
\ru{Что произойдет если вы пошлете число к каждой версии good? описанных
выше\,?} }

\secup

% \secrel{Everything (We Will Say) About Parsing
\ru{Все (что мы будем говорить) о разборе}}\secdown
\clearpage

Parsing is the act of turning an input character stream into a more structured,
internal representation.
\ru{\termdef{Парсинг}{парсинг} или \termdef{разбор}{синтаксический разбор}\
--- процесс превращения входного потока одиночных символов в более
структурированное внутреннее представление\note{программы или данных, заданных в
текстовой синтаксической форме}.}
A common internal representation is as a tree, which programs can recursively
process.
\ru{Обычно используется внутреннее представление в виде дерева, которое может
быть обработано программой рекурсивно.}

For instance, given the stream
\ru{Например, для входного потока символов\note{включая пробелы, табуляции и
концы строк}}
\begin{verbatim}
23 + 5 - 6
\end{verbatim}
we might want a tree representing addition whose left (L) node represents the
number 23 and whose right (R) node represents subtraction of 6 from 5.
\ru{мы хотим получить деревянное представление сложения, в котором левая (L)
ветвь содержит число 23, а правая (R)\ --- вложенное представление вычитания 6
из 5.}
A parser is responsible for performing this transformation.
\ru{Парсер отвечает за выполнение такой транформации.}

\noindent
\begin{tabular}{c c}
\noindent\includegraphics[height=0.8\textheight]{tmp/2_p10_R.pdf}
&
\noindent\includegraphics[height=0.8\textheight]{tmp/2_p10_L.pdf}
\\
\emph{Право}ассоциативный разбор
&
(*) \emph{Лево}ассоциативный разбор
\\
\end{tabular}\bigskip

Parsing is a large, complex problem that is far from solved due to the
difficulties of ambiguity.
\ru{Парсинг\ --- большая проблема информатики, сложность которой
заключается в трудностях неоднозначности.}
For instance, an alternate parse tree (*) for the above input expression might
put subtraction at the top and addition below it.
\ru{Например, существует альтернативное (*) \termdef{дерево разбора}{дерево
разбора} для того же входного выражения, в котором мы можем поместить
вычитание на вершину дерева, а сложение будет вложенным поддеревом.}
We might also want to consider
\ru{Нас также может интересовать,}
whether this addition operation is commutative
\ru{является ли это сложение коммутативной операцией,}
and hence whether the order of arguments can be switched.
\ru{то есть можем ли мы изменить порядок аргументов\note{например для
оптимизации кода}.}
Everything only gets much, much worse when we get to full-fledged programming
languages (to say nothing of natural languages).
\ru{Все становится намного хуже для полноценных языков программирования
(не говоря о натуральных языках).}

\secrel{A Lightweight, Built-In First Half of a Parser
\ru{Легковесная встроенная часть парсера}}

These problems make parsing a worthy topic in its own right, and entire books,
tools, and courses are devoted to it.
\ru{Эти проблемы делают разбор достойной темой саму по себе, и ей посвящены
целые книги, утилиты и учебные курсы, например первая глава ``Книги Дракона''
\cite{dragon}.}
However, from our perspective parsing is mostly a distraction, because we want
to study the parts of programming languages that are not parsing.
\ru{Однако, с нашей точки зрения тема парсинга является сильным отвлечением,
так как есть другие более достойные темы, касающиеся реализации языков
программирования.}
We will therefore exploit a handy feature of Racket to manage the transformation
of input streams into trees: 
\ru{Поэтому мы сделаем финт ушами, и будем использовать встроенную фичу
\racket а для получения готовых деревьев разбора из входного потока: функцию}
\verb|read|.
\verb|read| is tied to the parenthetical form of the language, in that it parses
fully (and hence unambiguously) parenthesized terms into a built-in tree form.
\verb|read| \ru{привязана к скобочной форме языка, и полностью (и следовательно
однозначно) разбирает скобочные выражения во встроенное представление\ ---
дерево.}
For instance, running \ru{Например применение} \verb|(read)| on the
parenthesized form of the above input \ru{к следующему входному потоку символов
(включающему скобки)}\ ---
\begin{verbatim}
(+ 23 (- 5 6))
\end{verbatim}
--- will produce a list, whose first element is the symbol \ru{создаст список, в
котором первым элементом будет символ} \verb|'+|, second element is the number
\ru{вторым элементом число} 23, and third element is a list \ru{и третий
элемент список}: this list’s first element is the
symbol \ru{в котором первым элемент будет} \verb|'-|, second element is the
number \ru{второй элемент число} 5, and third element is the number \ru{и
третий элемент число} 6.

\fig{\ru{Дерево разбора для} (+ 23 (- 5 6))}{tmp/2_1.pdf}{height=0.7\textheight}

\lst{src/2/p10_1.rkt}

\secrel{A Convenient Shortcut \ru{Удобный трюк}}

As you know you need to test your programs extensively, which is hard to do when you
must manually type terms in over and over again.
\ru{Как вы знаете, нужно тщательно тестировать свои программы, что особенно
сложно, если вам нужно снова и снова вводить выражения вручную.}
Fortunately, as you might expect, the parenthetical syntax is integrated deeply
into \racket\ through the mechanism of quotation.
\ru{К счатью, как и следовало ожидать, скобочный синтаксис глубоко
интегрирован в \racket\ через механизм \termdef{квотирования}{квотирование}.}
That is, \ru{это то самое выражение} \verb|'<expr>|\ --—
which you saw a moment ago in the above example
\ru{которое вы видели только что при выполнении предыдущего примера}\ --- 
acts as if you had run \ru{действует так же, как если бы вы запустили}
\verb|(read)| and typed \ru{и ввели} <expr> at the prompt \ru{в текстовом поле
ввода} (and, of course, evaluates to the value the (read) would have
\ru{и конечно же вычисляется в то же значение, что дает} \verb|(read)|).

\secrel{Types for Parsing \ru{Типы для разбора}}

Actually, I’ve lied a little.
\ru{На самом деле, я немного соврал.}
I said that \ru{я сказал что} \verb|(read)|\ --- or equivalently, using
quotation \ru{или использование квотирования, что эквивалентно}\ --- will
produce a \emph{list}, etc. \ru{создаст \emph{список}, блаблабла.}
That’s true in regular \racket, but in $Typed PLAI$
\ru{Это так для оригинального \racket, но в $Typed PLAI$} the type it
returns a distinct type called an \ru{возвращается специальный тип, который
называется} \termdef{s-expression}{s-expression}
\ru{s-выражение\index{s-выражение}}, written in $Typed PLAI$ as
\ru{который в $Typed PLAI$ записывается как}
\verb|s-expression|:
\begin{verbatim}
> (read)
- s-expression
[type in (+ 23 (- 5 6))]
'(+ 23 (- 5 6))
\end{verbatim}
\racket\ has a very rich language of s-expressions
\ru{имеет очень богатый язык на s-выражениях}
(it even has notation to represent cyclic structures
\ru{он даже имеет нотацию для представления циклических структур}), 
but we will use only the simple fragment of it.
\ru{но мы будем использовать только простейшую часть этого синтаксиса.}

In the typed language, an s-expression is treated distinctly from the other
types, such as numbers and lists.
\ru{В типизированном языке s-выражения обрабатыватся обособленно от других
типов, таких как числа и списки.}
\begin{framed}
Underneath, an s-expression is a large
recursive datatype that consists of all the base printable values—numbers,
strings, symbols, and so on—and printable collections (lists, vectors, etc.) of
s-expressions.
\ru{Далее s-выражение рассматривается как большой рекурсивный тип данных,
который содержит все базовые отображаемые (представимые в тексте) значения\
--- числа, строки, символы и т.д.\ --- и коллекции (списки, вектора и т.д.)
других s-выражений}.
\end{framed}
As a result, base types like numbers, symbols, and strings are
\emph{both} their own type and an instance of s-expression.
\ru{В результате такие базовые типы как числа, символы и строки, могут
\emph{одновременно} являться как собственным типом (число,..), так и
экземпляром s-выражениея}.
Typing such data can be fairly problematic, as we will discuss later
\ru{Типизация таких данных может быть очень проблематична, детальнее мы обсудим
это позже} \ref{}.

$Typed PLAI$ takes a simple approach.
\ru{$Typed PLAI$ применяет более простой подход.}
When written on their own, values like numbers are of those respective types.
\ru{Когда значания простых типов, типа чисел, написаны сами по себе, они
являются собственными типами (число).}
But when written inside a complex
s-expression—in particular, as created by read or quotation—they have type
s-expression.
\ru{Но когда они включены в состав сложного s-выражения\ --- в частности,
созданы через (read) или квотацию\ --- они имеют тип s-выражения.}
You have to then cast them to their native types.
\ru{Вы должны привести их к нативному типу.}
For instance \ru{Например}:
\lstl{src/2/p11_1.rkt}
This is similar to the casting that a Java programmer would have to insert.
\ru{Это похоже на явное приведение типов, которое должен вставить программист
на \java.}
We will study casting itself later \ru{Мы обсудим само приведение типов позже}
\ref{}.

Observe that the first element of the list is still not treated by the
type checker as a symbol:
\ru{Отметим, что первый элемент списка все еще не распознается контролером
типов как символ:}
a list-shaped s-expression is a list of s-expressions.
\ru{списко-образное s-выражение является списком s-выражений.}
Thus \ru{Таким образом},
\lst{src/2/p11_2.rkt}
whereas again, casting does the trick:
\ru{и снова приведение типов решает проблему :}
\lst{src/2/p11_3.rkt}
The need to cast s-expressions is a bit of a nuisance,
\ru{Необходимость приведения s-выражений немного геморна,}
but some complexity is unavoidable because of what we’re trying to accomplish:
\ru{но некоторая сложность неизбежна из-за того что мы пытаемся достичь:}
to convert an \emph{untyped input} stream into a \emph{typed output} stream
\ru{преобразование \emph{нетипизированного} входного потока в
\emph{типизированный} выходной поток}
through robustly typed means.
\ru{через средства робастной типизации.}
Somehow we have to make explicit our assumptions about that input stream.
\ru{Каким-то образом мы должны делать явные предположения об этом входном
потоке.}

Fortunately we will use s-expressions only in our parser, and our goal is to
\emph{get away from parsing as quickly as possible\,!}
\ru{К счастью, мы будем использовать s-выражения только в нашем парсере, и наша
цель состоит в том, чтобы \emph{уйти от разбора как можно быстрее\,!}}
Indeed, if anything this should be inducement to get away even quicker.
\ru{В самом деле, все эти заморочки являются побуждением сделать этот уход еще
быстрее.}

\secrel{Completing the Parser \ru{Заканчиваем с парсером}}\label{sec2_4}

In principle, we can think of \ru{В принципе, мы можем думать о} \verb|read|\
as a complete parser \ru{как о законченном парсере}.
However, its output is generic \ru{Тем не менее, его вывод все еще сырой}:
it represents the token structure without offering any comment on its intent.
\ru{он содержит структуру токенов не предлагая каких-либо комментариев об их
назначении.}
We would instead prefer to have a representation
\ru{Вместо этого мы предпочли бы иметь представление,}
that tells us something about the \emph{intended meaning} of the terms in our
language,
\ru{которое говорит нам что-то о \emph{предполагаемом значении} термов нашего
языка,}
just as we wrote at the very beginning: “representing addition”, “represents a
number”, and so on.
\ru{так же как мы писали в самом начале: ``представление сложения'',
``представление числа'' и так далее.}

To do this, we must first introduce a datatype
\ru{Чтобы сделать это, мы сначала введем тип данных,}
that captures this representation.
\ru{который зафиксирует это представление.}
We will separately discuss \ru{Мы
отдельно рассмотрим} (section \ru{в разделе} \ref{sec31}) how and why we obtained this
datatype \ru{как и зачем мы применяем этот тип}, but for now let’s say it’s
given to us \ru{но сейчас пока будем считать, что он нам задан}:
\lstx{ArithC.rkt}{src/2/p12_1.rkt}{rkt}\label{arithc}
We now need a function that will convert s-expressions into instances of this
datatype.
\ru{Теперь нам нужна функция, которая преобразует s-выражение в структуру из
экземпляров этого типа.}
This is the other half of our parser \ru{Это вторая половина нашего парсера}:
\lstxl{ArithC.rkt}{src/2/p12_2.rkt}{rkt}

Thus\note{typing in \racket\ console \emph{after program run}} \ru{Таким
образом\note{\ru{введя выражение в \racket-консоли \emph{после выполнения
программы}}}}
\lst{src/2/v1.rkt}
\lstx{ArithC.rkt}{src/2/p12_3.rkt}{rkt}
\lst{src/2/v2.rkt}

Congratulations\,! \ru{Мои поздравления\,!}
You have just completed your first representation of a program.
\ru{Вы только что завершили ваше первое представление программы.}
From now on we can focus entirely on programs
\ru{С этого момента мы можем полностью сосредоточиться на программах,}
represented as recursive trees,
\ru{представленных в виде рекурсивных деревьев,}
ignoring the vagaries of surface syntax
\ru{не обращая внимания на капризы наносного синтаксиса}
and how to get them into the tree form.
\ru{и процесс получения из него дерева разбора.}
We’re finally ready to start studying programming languages\,!
\ru{Мы, наконец, готовы приступить к изучению языков программирования\,!}

\Exercise{
What happens if you forget to quote the argument to the 
\ru{Что случиться, если вы забудете заквотить аргумент вызова}
parser\,?
Why\,? \ru{Почему\,?}
}
\secrel{Coda \ru{Кода}}

\racket’s syntax, which it inherits from Scheme and \lisp, is controversial.
\ru{Синтаксис \racket а, который он наследует от Scheme и Lisp, спорен.}
Observe, however, something deeply valuable that we get from it.
\ru{Заметим, однако, что мы получаем от него нечто глубоко ценное.} 
While parsing traditional languages can be very complex,
\ru{В то время как парсинг традиционных языков может быть очень сложным,}
parsing this syntax is virtually trivial.
\ru{разбор этого синтаксиса практически тривиален.}
Given a sequence of tokens corresponding to the input,
\ru{Для заданной последовательности лексем, соответствующих входному потоку,}
it is absolutely straightforward to turn paren\-the\-sized sequences into
s-expressions;
\ru{абсолютно тривиально превратить скобочные последовательности в s-выражения;}
it is equally straightforward (as we see above) to turn sexpressions into proper
syntax trees.
\ru{столь же просто (как мы видим выше) преобразовать s-выражения в правильные
синтаксические деревья.}
I like to call such two-level languages \term{bicameral}, in loose analogy to
government legislative houses:
\ru{Мне нравится называть такие двухуровневые языки \term{двухпалатными}, в
свободной аналогии к государственным законодательным учреждениям:}
the lower-level does rudimentary well-formedness checking, while the upper-level
does deeper validity checking.
\ru{нижний уровень делает рудиментарную проверку правильности оформления, в то
время как верхний уровень выполняет глубокую проверку валидности.}
(We haven’t done any of the latter yet, but we will
\ru{Мы еще не делали последнего, но мы будем}
\ref{}.)

The virtues of this syntax are thus manifold.
\ru{Достоинства этого синтаксиса, таким образом, многообразны.}
The amount of code it requires is small, and can easily be embedded in many
contexts.
\ru{Объем кода, который он требует, очень мал, и может быть встроен во многих
контекстах.}
By integrating the syntax into the language, it becomes easy for programs to
manipulate representations of programs (as we will see more of in \ref{}).
\ru{Интеграция синтаксиса в язык делает простой программную манипуляцию
представлением программ (как мы увидим в \ref{}).}
It’s therefore no surprise that even though many Lisp-based syntaxes have had
wildly different semantics, they all share this syntactic legacy.
\ru{Поэтому неудивительно, что множество основанных на \lisp е синтаксисов,
имеющих дико разную семантику, все равно разделяют это общее синтаксическое
наследие.}

Of course, we could just use XML instead.
\ru{Конечно, мы могли бы использовать XML.}
That would be much better.
\ru{Это было бы намного лучше.}
Or JSON.
\ru{Или JSON.}
Because that wouldn’t be anything like an s-expression at all.
\ru{Потому что все равно это в итоге было бы тем же s-выражением.}

\secup

% \secrel{A First Look at Interpretation \ru{Первый взгляд на
интерпретацию}}\label{firstinterp}\secdown

Now that we have a representation of programs, there are many ways in which we
might want to manipulate them.
\ru{Теперь, когда мы имеем представление программ, существует множество
способов, которыми мы можем манипулировать ими.}
We might want to display a
program in an attractive way \ru{Мы можем захотеть выводить листинг программы в
красивом виде} (“pretty-print”), convert into code in some other format
\ru{преобразовать в код в какой-то другой формат} (“compilation”
\ru{``компиляция''/''трансляция''}), ask whether it obeys certain properties 
\ru{убедиться что она отвечает определенным требованиям}
(“verification” \ru{``верификация''}), and so on \ru{и так далее}.
For now, we’re going to focus on asking what value it corresponds to
(“\termdef{e\underline{val}uation}{evaluation}”\ --- the reduction of programs
to \emph{\underline{val}ues}).
\ru{Для начала, мы собираемся сфокусироваться на вопросе\ --- какому значению
соответствует программа (“\termdef{вычисление}{вычисление}"\ --- редукция
программы до \emph{значения})}

Let’s write an evaluator, in the form of an \termdef{interpreter}{interpreter},
for our arithmetic language.
\ru{Давайте напишем вычислитель, в форме
\termdef{интерпретатора}{интерпретатор}, для нашего арифметического языка.}
We choose arithmetic first for three reasons \ru{Мы выбрали арифметику прежде
всего по следующим трем причинам}:
\begin{itemize}[nosep]
  \item[(a)]
  you already know how it works, so we can focus on the mechanics of writing
evaluators;
\ru{вы уже знаете как работает арифметика, и мы можем сфокусироваться на
механике написания вычислителей;}
  \item[(b)]
  it’s contained in
every language we will encounter later, so we can build upwards and outwards from it;
\ru{она содержится в каждом языке, с которым мы столкнемся в дальнейшем, так
что мы будем расширять этот арифметический язык вверх и вширь;} and \ru{и}
  \item[(c)] 
it’s at once both small and big enough to illustrate many points
we’d like to get across.
\ru{этот язык минималистичен, но при этом достаточно большой, чтобы
проиллюстрировать многие моменты, которые мы хочем до вас донести.}
\end{itemize}

\secrel{Representing Arithmetic \ru{Представление арифметики}}\label{sec31}

Let’s first agree on how we will represent arithmetic expressions.
\ru{Давайте сначала договориться о том, как мы будем представлять арифметические
выражения.}
Let’s say we want to support only two operations\ --- addition and
multiplication\ --- in addition to primitive numbers.
\ru{Допустим, мы хотим поддерживать только две простые операции\ --- сложение и
умножение\ --- в дополнение к примитивным числам.}
We need to represent arithmetic \termdef{expressions}{expression}.
\ru{Нам необходимо представление для арифметических
\termdef{выражений}{выражение}}.
What are the rules that govern nesting of arithmetic expressions\,?
\ru{Какие правила регулируют вложенность арифметических выражений\,?} 
We’re actually free to nest any expression inside another.
\ru{На самом мы свободно можем владывать любое выражение внутрь любого другого.}

\DoNow{
Why did we not include division\,?
\ru{Почему мы не включили умножение\,?}
\\
What impact does it have on the remarks above\,?
\ru{Какое влияние это имеет на замечания выше\,?}
}

We’ve ignored division because it forces us into a discussion of what
expressions we might consider legal:
\ru{Мы игнорировали деление, потому что оно вовлекает нас в дикуссию о том,
какие выражения мы можем считать правильными:}
clearly the representation of $1/2$ ought to be legal;
\ru{ясно что представление $1/2$ должно быть легальным;} 
the representation of $1/0$ is much more debatable;
\ru{представление $1/0$ спорно;}
and that of \ru{и что-то типа} $1/(1-1)$ seems even more controversial.
\ru{кажется гораздо более спорным.}
 
We’d like to sidestep this controversy for now and return to it later
\ru{Мы хотели бы обойти сейчас это противоречие, в вернуться к нему позже}
\ref{}.

Thus, we want a representation for numbers and arbitrarily nestable
addition and multiplication.
\ru{Таким образом нам нужно представление для чисел и произвольно вложенных
сложений и умножений.} 
Here’s one we can use \ru{Вот то что мы можем использовать}
(used in \ru{использовано в} \ref{arithc} ):
\lstx{ArithC}{src/2/p12_1.rkt}{rkt}

\secrel{Writing an Interpreter \ru{Написание интерпретатора}}

Now let’s write an interpreter for this arithmetic language.
\ru{Теперь давайте напишем интерпретатор для этого арифметического языка.} 
First, we should think about what its type is.
\ru{Для начала нам надо подумать, какие типы он использует\,?} 
It clearly consumes a \verb|ArithC| value.
\ru{Совершенно ясно что на вход подается структура типа} \verb|ArithC|. 
What does it produce\,?
\ru{Что он возвращает\,?} 
Well, an interpreter evaluates
\ru{Ну, интерпретатор вычисляет} \ --- and what kind of value might arithmetic
expressions reduce to\,?
\ru{и к какому значению может редуцироваться арифметическое выражение\,?} 
Numbers, of course.
\ru{Конечно, числу.} 
So the interpreter is going to be a function from arithmetic expressions to
numbers.
\ru{Таким образом, интерпретатор должен быть функцией от арифметического
выражения, возвращающей число.}

\Exercise{
Write your test cases for the interpreter.
\ru{Напишите тесты для интерпретатора.}
}

Because we have a recursive datatype, it is natural to structure our interpreter
as a recursive function over it.
\ru{Так как мы имеем рекурсивный тип данных\note{допускаются произвольные
вложения того же типа}, нормально что структура нашего интерпретатора тоже
должна быть рекурсивной функций над выражением.}
Here’s a first template\note{Templates are
explained in great detail in \emph{How to Design Programs}.}
\ru{Вот первый шаблон\note{\ru{Шаблоны очень детально описаны в \emph{How to
Design Programs}}}} :
\lstx{ArithC.rkt}{src/3/p14_1.rkt}{rkt}
You’re probably tempted to jump straight to code, which you can:
\ru{Вероятно у вас есть соблазн сразу перейти к коду, который вы можете
написать:}
\lstx{ArithC.rkt}{src/3/p14_2.rkt}{rkt}

\DoNow{
Do you spot the errors\,?
\ru{Вы увидели ошибки\,?}
}

Instead, let’s expand the template out a step:
\ru{Вместо этого давайте расширим шаблон на один шаг:}
\lstx{ArithC.rkt}{src/3/p15_2.rkt}{rkt}
and now we can fill in the blanks:
\ru{и теперь мы можем заполнить пробелы:}
\lstx{ArithC.rkt}{src/3/p15_3.rkt}{rkt}

Later on \ru{Позже в} \ref{}, we’re going to wish we had returned a more complex
datatype than just numbers.
\ru{мы пожелаем возвращать более сложный тип данных, чем просто числа.}
But for now, this will do.
\ru{Но сейчас нам этого достаточно.}

Congratulations: you’ve written your first interpreter\,!
\ru{Поздравляем: вы только что написали свой первый интерпретатор\,!} 
I know, it’s very nearly an anticlimax\note{\ru{ситуация, когда проблема
казавшеяся очень сложной, решается с помощью чего-то тривиального //
Wikipedia}}\note{\ru{род морских улиток // там же}}.
\ru{Я знаю, это очень близко к разочарованию.}
But they’ll get harder\ --- much harder\ --- pretty soon, I promise.
\ru{Но все станет жестче\ --- намного жестче\ --- совсем скоро, я обещаю.}

\secrel{Did You Notice\,? \ru{Вы заметили\,?}}

I just slipped something by you:
\ru{Я только что утаил что-то от вас:}
\DoNow{
What is the ``meaning'' of addition and multiplication in this new language\,?
\ru{Каков ``смысл'' сложения и умножения в этом новом языке\,?}
}

That’s a pretty abstract question, isn’t it.
\ru{Это достаточно абстрактный вопрос, не так ли.} 
Let’s make it concrete.
\ru{Давайте его конкретизируем.} 
There are many kinds of addition in computer science:
\ru{В информатике существует множество видов сложения:}

\begin{itemize}
  \item 
First of all, there’s many different kinds of \termdef{numbers}{number}:
\ru{Прежде всего, существует множество видов \termdef{чисел}{число}:}
fixed-width \ru{фиксированной длины} (e.g., 32- bit \ru{например 32-битные})
integers \ru{целые}, signed fixed-width \ru{знаковые фиксированной длины} (e.g.,
31-bits plus a sign-bit \ru{например 31-битные плюс бит знака}) integers
\ru{целые}, arbitrary precision integers \ru{целые числа произвольной точности};
in some languages, rationals \ru{в некоторых языках\ --- натуральные дроби};
various formats of fixed- and floating-point numbers
\ru{различные форматы чисел с фиксированной и плавающей точкой}; in some
languages, complex numbers \ru{в некоторых языках комплектные числа}; and so on
\ru{и так далее}.
After the numbers have been chosen, addition may support only some combinations
of them.
\ru{После того как были выбраны определенные виды чисел, сложение может
поддерживать только некоторые их комбинации.}
  \item 
In addition, some languages permit the addition of datatypes such as matrices.
\ru{В дополнение, некоторые языки поддерживают сложение таких типов данных, как
матрицы.}
  \item 
Furthermore, many languages support ``addition'' of strings
\ru{Кроме того, многие языки поддерживают ``сложение'' строк} (
we use scare-quotes because we don’t really mean the mathematical concept of
addition, but rather the operation performed by an operator with the syntax +
\ru{Мы используем кавычки, так как предполагаем не математическую идею
сложения, а операцию, выполняемую оператором с синтаксисом +} ). 
In some languages this always means concatenation;
\ru{В некоторых языках это всегда значит конкатенацию строк;} 
in some others, it can result in numeric results (or numbers stored in strings).
\ru{в некоторых долбанутых языках иногда может получиться численный
результат (или числа хранимые в строках).}
\end{itemize}

These are all different meanings for addition. 
\ru{Все это является смыслом сложения.}
\termdef{Semantics}{semantics}
is the mapping of \emph{syntax} (e.g., +) to \emph{meaning} (e.g., some or all
of the above).
\ru{\termdef{Семантика}{семантика} это отображение \emph{синтаксиса} (например
+) на \emph{смысл} (например что-то или все из вышеперечисленного).}

This brings us to our first game of:
\ru{Это подводит нас к первоначальной игре:}
\begin{description}\item[\emph{Which of these is the same\,?} \ru{Что из
этого дает одинаковый результат\,?}]\
\\
\begin{itemize}[nosep]
  \item 1 + 2
  \item 1 + 2
  \item '1' + '2'
  \item '1' + '2'
\end{itemize}
\end{description}

Now return to the question above.
\ru{Теперь возвращаемся к предыдущему вопросу.} 
What semantics do we have\,?
\ru{Какую семантику мы имеем\,?}
We’ve adopted whatever semantics \racket\ provides, because we map + to
\racket’s +. 
\ru{Мы приняли ту же семантику, которую предоставляет \racket, потому что мы
отоборазили наш + на + в \racket е.}
In fact that’s not even quite true:
\ru{На самом деле это даже не совсем верно:} 
\racket\ may, for all we know, also enable +
to apply to strings, so we’ve chosen the restriction of \racket’s semantics to
numbers\note{though in fact \racket’s + doesn’t tolerate strings}.
\ru{\racket\ может, как все мы знаем, также использовать + к строкам, так что
мы выбрали ограничение семантики \racket а только для чисел\note{\ru{хотя на
самом деле \racket ский + нетолерантен к строкам}}.}

If we wanted a different semantics, we’d have to implement it explicitly.
\ru{Если мы хотим другую семантику, мы должны реализовать ее в явном виде.}

\Exercise{
What all would you have to change so that the number had signed- 32-bit
arithmetic\,?
\\
\ru{Что мы должны изменить, чтобы числа поддерживали знаковую 32-битную
арифметику\,?}
}

In general, we have to be careful about too readily borrowing from the host
language.
\ru{В общем, мы должны быть осторожными с заимствованиями из языка-носителя.}
We’ll return to this topic later \ru{Мы вернемся к этой теме позже} \ref{}.

\secrel{3.4 Growing the Language . . . . . . . . . . . . . . . . . . . . . . . . . 16}

\secup

% \secrel{4 A First Taste of Desugaring 16}\label{sec4}\secdown

We’ve begun with a very spartan arithmetic language.
\ru{Мы начали с очень спартанского арифметического языка.}
Let’s look at how we might extend it with more arithmetic operations that can
nevertheless be expressed in terms of existing ones.
\ru{Давайте посмотрим, как мы могли бы расширить его б\'ольшим количеством
арифметических операций, которые тем не менее могут быть выражены в терминах
существующих операторов.}
We’ll add just two, because these will suffice to illustrate the point.
\ru{Мы добавим только два, потому что этого будет достаточно для иллюстрации
этого метода.}

\secrel{4.1 Extension: Binary Subtraction . . . . . . . . . . . . . . . . . . . . . 17}
\secrel{4.2 Extension: Unary Negation . . . . . . . . . . . . . . . . . . . . . . . 18}
\secup


% \secrel{Adding Functions to the Language\\
\ru{Добавление функций в язык}}\secdown

Let’s start turning this into a real programming language.
\ru{Теперь давайте превратим нашу поделку в настоящий язык программирования.}
We could add intermediate features such as conditionals,
\ru{Мы можем добавить вспомогательные фичи типа условных конструкций,}
but to do almost anything interesting
\ru{но чтобы сделать что-то интересное}
we’re going to need functions or their moral equivalent, 
\ru{нам нужны \termdef{функции}{функция} или их эквивалент,}
so let’s get to it.
\ru{так что начнем.}

\Exercise{
Add conditionals to your language.
\ru{Добавьте условные конструкции в ваш язык.}
You can either add boolean datatypes
\ru{Вы можете добавить булевый тип данных}
or, if you want to do something quicker,
\ru{или, если вы хотите сделать это побыстрее,}
add a conditional that treats \emph{0} as \term{false}
\ru{добавьте условие что \emph{0} считается \term{ложью},}
and \emph{everything else} as \term{true}.
\ru{ и не-ноль\ --- \term{истиной}.}

What are the important test cases you should write\,?\\
\ru{Какие варианты тестов вы должны написать\,?}
}

Imagine, therefore, that we’re modeling a system like Dr\racket.
\ru{Представьте что мы моделируюем систему типа Dr\racket.}
The developer defines functions in the definitions window, 
\ru{Разработчик определяет функции в окне определений,}
and uses them in the interactions window.
\ru{и использует их к интерактивном окне.}
For now, let’s assume all definitions go in the definitions window only
\ru{Предполагается что все определения вводятя \emph{только} в окне определений}
(we’ll relax this soon \ru{позже мы ослабим это требование} \ref{}),
and all expressions in the interactions window only.
\ru{а все выражения только в интерактивном окне.}
Thus, running a program simply loads definitions.
\ru{Таким образом, запуск программы просто загружает определения.}
Because our interpreter corresponds to the interactions window prompt,
\ru{Так как наш интерпретатор соответствует строке ввода интерактивного окна,}
we’ll therefore assume it is supplied with a set of definitions.
\ru{мы также предполагаем что интерпретатор укомплектован набором этих
определений.}
\note{A \emph{set} of definitions suggests \emph{no ordering}, which means,
presumably, any definition can refer to any other. That’s what I intend here,
but when you are designing your own language, be sure to think about this.}
\note{\ru{\emph{Набор} определений предполагает \emph{отсутствие
упорядочивания}, то есть любое определение может ссылаться на любое другое. Вот
что я здесь предполагаю сделать, но когда вы разрабатываете ваш собственный
язык, подумайте об этом.}}
\clearpage

\secrel{Defining Data Representations\\\ru{Представление определения}}

To keep things simple, let’s just consider functions of one argument.
\ru{Для упрощения будем рассматривать только функции от одного
аргумента.\note{\ru{подход имеет право на жизнь\ --- в языке Ы несколько
параметров передаются в контейнере-списке в сочетании с оператором @, не говоря
о \termdef{карринге}{карринг}, когда применение функции к аргументу \#1
возвращает анонимную лямбда-\emph{функцию}, применямую к аргументу
\#2 и т.д.: $f(x,y,z) \sim
(f(x)_{\lambda_x}(y))_{\lambda_{x,y}}(z)$}}} Here are some
\racket\ examples:
\ru{Вот несколько примеров на \racket:}
\lsts{src/5/p19_1.rkt}{rkt}

\Exercise{
When a function has multiple arguments, what simple but important criterion
governs the names of those arguments\,?\\
\ru{Если функция имеет несколько аргументов, какой простой но важный
критерий определяет имена этих аргументов\,?} }

What are the parts of a \termdef{function definition}{function definition}\,?
\ru{Из каких частей состоит \termdef{определение функции}{определение
функции}\,?}
It has a name \ru{Это имя}
(above, \verb|double|, \verb|quadruple|, and \verb|const5|),
which we’ll represent as a symbol
\ru{которые мы будем представлять символом}
('double, etc.);
its \termdef{formal parameter}{formal parameter}
\ru{ее \termdef{формальный параметр}{формальный параметр}}
or \termdef{argument}{argument}
\ru{или \termdef{аргумент}{аргумент}}
has a name \ru{имеющий имя} (e.g., x),
which too we can model as a symbol
\ru{который мы тоже смоделируем как символ} ('x);
and it has a \termdef{body}{function body}. \ru{и ее \termdef{тело}{тело
функции}.}
We’ll determine the body’s representation in stages,
\ru{Мы определим представление тела пошагово,}
but let’s start to lay out a datatype for function definitions:
\ru{но для начала выделим тип данных для всех определений функций:}
\lstx{FunDefC(ore)}{src/5/p20_2.rkt}{rkt}

What is the body\,? \ru{Что такое \term{тело функции}\,?}
Clearly, it has the form of an arithmetic expression,
\ru{Очевидно, оно имеет форму арифметичского выражения,}
and sometimes it can even be represented using the existing
\ru{и иногда оно даже может быть представлено используя существующий} 
\verb|ArithC(ore)| language \ru{язык}:
for instance, the body of \ru{например, тело} \verb|const5|
can be represented as \ru{может быть представлено как } \verb|(numC 5)|.
But representing the body of \ru{Но представление тела}
\verb|double| requires something more: \ru{требует кое-чего еще:}
not just addition (which we have), \ru{не только сложение (которое у нас есть),} 
but also \ru{но также и} ''x''.
You are probably used to calling this a \termdef{variable}{variable},
\ru{Вы возможно уже назвали это \termdef{переменной}{переменная}} 
but we will \emph{not use} that term for now.
\ru{но мы пока \emph{не будем использовать} этот термин.} 
Instead, we will call it an \termdef{identifier}{identifier}.\note{I promise
we’ll return to this issue of nomenclature later \ref{}.}
\ru{Вместо этого мы назовем его
\termdef{идентификатор}{идентификатор}.\note{\ru{Я вам обещаю, что мы
вернемся к этому вопросу номенклатуры позже \ref{}.}}}

\DoNow{Anything else\,?\\\ru{Что-то еще\,?}}

Finally, let’s look at the body of
\ru{Давайте посмотрим на тело} \verb|quadruple|.
It has yet another new construct:
\ru{Оно имеет другой новый конструкт:}
a function \termdef{application}{function application}.
\ru{\termdef{применение}{применение функции} функции.}
Be very careful to distinguish between a function \term{definition},
\ru{Будьте очень осторожны, и отличайте \term{определение} функции,} 
which describes what the function is,
\ru{которое описывает что есть некоторая функция,} 
and an \term{application}, which uses it.
\ru{от \term{применения}, т.е. ее использования.}
These are uses. \ru{Это виды использования.}
The \term{argument} (or \term{actual parameter})
\ru{\term{Аргумент} (или \term{фактический параметр})}
in the inner application of \ru{во внутреннем применении}
\verb|double| is \verb|x|;
the argument in the outer application is \ru{аргумент во внешнем применении}
\verb|(double x)|.
Thus, the argument can be any complex expression.
\ru{Таким образом, аргумент может быть любым сложным выражением.}

Let’s commit all this to a crisp datatype.
\ru{Давайте применим все это к нашему сверкающему типу данных.}
Clearly we’re extending what we had before 
\ru{Расширяем то что у нас уже было раньше}
(because we still want all of arithmetic
\ru{потому что нам все еще нужна вся арифметика}).
We’ll give a new name to our datatype
\ru{Мы дадим новое имя нашему типу данных}
to signify that it’s growing up:
\ru{чтобы отметить что он расширяется:}
\lstx{ExprC(ore)}{src/5/p20_3.rkt}{rkt}

Identifiers are closely related to formal parameters.
\ru{Идентификаторы близко связаны с формальными параметрами.}
When we apply a function by giving it a value for its parameter,
\ru{когда при применяем функцию придавая значение ее параметру,} 
we are in effect asking it
\ru{фактически мы просим функцию} 
to replace all instances of that formal parameter in the body
\ru{заменить все вхождения этого формального параметра в ее теле}\ ---
i.e., the identifiers with the same name as the formal parameter
\ru{т.е., идентификаторы с тем же именем что и формальный параметр}
--- with that value. \ru{этим значением.}
To simplify this process of search-and-replace,
\ru{Для упрощения этого процесса поиска/замены,}
we might as well use the same datatype to represent both.
\ru{мы могли бы так же использовать один и тот же тип данных для представления
обоих.}
We’ve already chosen symbols to represent formal parameters, so:
\ru{Мы уже выбрали символы для представления формальных параметров, так что:}
\note{Observe that we are being coy about a few issues: what kind of ''value''
\ref{} and when to replace \ref{}.}
\note{\ru{Заметим что были скромны в нескольких вопросах: какого рода
``значения'' \ref{} и когда их заменять \ref{}}}
\lsts{src/5/p21_1.rkt}{rkt}

Finally, applications.
\ru{В заключение, применения.}
They have two parts: \ru{Они имеют две части:} 
the function’s name, \ru{имя функции,}
and its argument. \ru{и ее аргумент.}
We’ve already agreed that the argument can be any full-fledged expression
\ru{мы уже убедились, что аргумент может быть любым полноценным выражением}
(including identifiers and other applications
\ru{включая идентификаторы и другие применения}).
As for the function name,
\ru{что касается имени функции,}
it again makes sense to use the same datatype
\ru{снова имеет смысл использовать тот же тип данных,}
as we did when giving the function its name in a function definition.
\ru{как мы сделали когда давали функции имя в ее определении.}
Thus: \ru{Так что:}
\lsts{src/5/p21_2.rkt}{rkt}
identifying which function to apply, and providing its argument.
\ru{определяет какая функция применяется, и предоставляет ее аргумент.}

Using these definitions,
\ru{используя эти определения,}
it’s instructive to write out the representations of the examples
we defined above:
\ru{полезно выписать представления примеров которые мы рассматривалии выше:}
\lsts{src/5/p21_3.rkt}{rkt}
We also need to choose a representation for a set of function definitions. 
\ru{Нам также необходимо выбрать представление для набора определений функций.}
It’s convenient to represent these by a list.
\ru{Удобно определить из через списки.}

\secrel{5.2 Growing the Interpreter   . 21}
\secrel{5.3 Substitution   . 22}
\secrel{5.4 The Interpreter, Resumed   23}
\secrel{5.5 Oh Wait, There’s More!   . 25}
\secup

% \secrel{6 From Substitution to Environments 25}\secdown

Though we have a working definition of functions, you may feel a slight unease
about it. When the interpreter sees an identifier, you might have had a sense
that it needs to “look it up”. Not only did it not look up anything, we defined
its behavior to be an error! While absolutely correct, this is also a little
surprising. More importantly, we write interpreters to understand and explain
languages, and this implementation might strike you as not doing that, because
it doesn’t match our intuition.

There’s another difficulty with using substitution, which is the number of times
we traverse the source program. It would be nice to have to traverse only those
parts of the program that are actually evaluated, and then, only when necessary.
But substitution traverses everything—unvisited branches of conditionals, for
instance—and forces the program to be traversed once for substitution and once
again for interpretation.

\Exercise{
Does substitution have implications for the time complexity of evaluation?
}

There’s yet another problem with substitution, which is that it is defined in
terms of representations of the program source. Obviously, our interpreter has
and needs access to the source, to interpret it. However, other
implementations—such as compilers— have no need to store it for that purpose.
It would be nice to employ a mechanism that is more portable across
implementation strategies.
\note{Compilers might store versions of or information about the source for
other reasons, such as reporting runtime errors, and JITs may need it to
re-compile on demand.}

\secrel{6.1 Introducing the Environment  . 26}

The intuition that addresses the first concern is to have the interpreter “look
up” an identifier in some sort of directory. The intuition that addresses the
second concern is to defer the substitution. Fortunately, these converge nicely
in a way that also addresses the third. The directory records the intent to
substitute, without actually rewriting the program source; by recording the
intent, rather than substituting immediately, we can defer substitution; and the
resulting data structure, which is called an environment, avoids the need for
source-to-source rewriting and maps nicely to low-level machine representations.
Each name association in the environment is called a binding.

Observe carefully that what we are changing is the implementation strategy for
the programming language, not the language itself. Therefore, none of our
datatypes for representing programs should change, nor even should the answers
that the interpreter provides. As a result, we should think of the previous
interpreter as a “reference implementation” that the one we’re about to write
should match. Indeed, we should create a generator that creates lots of tests,
runs them through both interpreters, and makes sure their answers are the same.
Ideally, we should prove that the two interpreters behave the same, which is a
good topic for advanced study.

Let’s first define our environment data structure. An environment is a list of
pairs of names associated with...what?
\note{One subtlety is in defining precisely what “the same” means, especially
with regards to failure.}

\DoNow{
A natural question to ask here might be what the environment maps names to. But
a better, more fundamental, question is: How to determine the answer to the
“natural” question?
}

Remember that our environment was created to defer substitutions. Therefore, the
answer lies in substitution. We discussed earlier (Oh Wait, There’s More!) that
we want substitution to map names to answers, corresponding to an eager function
application strategy. Therefore, the environment should map names to answers.
\lsts{src/6/6_1_1.rkt}{rkt}
\secrel{6.2 Interpreting with Environments  27}

Now we can tackle the interpreter. One case is easy, but we should revisit all
the others:
\lsts{src/6/6_2_1.rkt}{rkt}
The arithmetic operations are easiest. Recall that before, the interpreter
recurred without performing any new substitutions. As a result, there are no new
deferred substitutions to perform either, which means the environment does not
change:
\lsts{src/6/6_2_2.rkt}{rkt}

Now let’s handle identifiers. Clearly, encountering an identifier is no longer
an error: this was the very motivation for this change. Instead, we must look up
its value in the directory:
\lsts{src/6/6_2_3.rkt}{rkt}

\DoNow{
Implement lookup.
}

Finally, application. Observe that in the substitution interpreter, the only
case that caused new substitutions to occur was application. Therefore, this
should be the case that constructs bindings. Let’s first extract the function
definition, just as before:
\lsts{src/6/6_2_4.rkt}{rkt}
Previously, we substituted, then interpreted. Because we have no substitution
step, we can proceed with interpretation, so long as we record the deferral of
substitution.
\lsts{src/6/6_2_5.rkt}{rkt}
That is, the set of function definitions remains unchanged; we’re interpreting
the body of the function, as before; but we have to do it in an environment that
binds the formal parameter. Let’s now define that binding process:
\lsts{src/6/6_2_6.rkt}{rkt}
the name being bound is the formal parameter (the same name that was substituted
for, before). It is bound to the result of interpreting the argument (because
we’ve decided on an eager application semantics). And finally, this extends the
environment we already have. Type-checking this helps to make sure we got all
the little pieces right.

Once we have a definition for lookup, we’d have a full interpreter. So here’s
one:
\lsts{src/6/6_2_7.rkt}{rkt}
Observe that looking up a free identifier still produces an error, but it has
moved from the interpreter—which is by itself unable to determine whether or not
an identifier is free—to lookup, which determines this based on the content of
the environment.

Now we have a full interpreter. You should of course test it make sure it works
as you’d expect. For instance, these tests pass:
\lsts{src/6/6_2_8.rkt}{rkt}
So we’re done, right?

\DoNow{
Spot the bug.
}

\secrel{6.3 Deferring Correctly   29}

\secrel{6.4 Scope   . 30}

The broken environment interpreter above implements what is known as dynamic
scope. This means the environment accumulates bindings as the program executes.
As a result, whether an identifier is even bound depends on the history of
program execution. We should regard this unambiguously as a flaw of programming
language design. It adversely affects all tools that read and process programs:
compilers, IDEs, and humans.

In contrast, substitution—and environments, done correctly—give us lexical scope
or static scope. “Lexical” in this context means “as determined from the source program”, while “static” in computer science means “without running the program”, so these are appealing to the same intuition. When we examine an identifier, we want to know two things: (1) Is it bound? (2) If so, where? By “where” we mean: if there are multiple bindings for the same name, which one governs this identifier? Put differently, which one’s substitution will give a value to this identifier? In general, these questions cannot be answered statically in a dynamically-scoped language: so your IDE, for instance, 
cannot overlay arrows to show you this information (as DrRacket does).
Thus, even though the rules of scope become more complex as the space of names
becomes richer (e.g., objects, threads, etc.), we should always strive to preserve the spirit of static 
scoping.
\note{A different way to think about it is that in a dynamically-scoped
language, the answer to these questions is the same for all identifiers, and it
simply refers to the dynamic environment. In other words, it provides no useful
information.}

\secdown
\secrel{6.4.1 How Bad Is It?   . 30}

You might look at our running example and wonder whether we’re creating a
tempest in a teapot. In return, you should consider two situations:
\begin{enumerate}

\item To understand the binding structure of your program, you may need to look
at the whole program. No matter how much you’ve decomposed your program into
small, understandable fragments, it doesn’t matter if you have a free identifier
anywhere.

\item Understanding the binding structure is not only a function of the size of
the program but also of the complexity of its control flow. Imagine an
interactive program with numerous callbacks; you’d have to track through every
one of them, too, to know which binding governs an identifier.

\end{enumerate}

Need a little more of a nudge? Let’s replace the expression of our example
program with this one:
\lsts{src/6/6_4_1.rkt}{rkt}

Suppose moon-visible? is a function that presumably evaluates to false on
new-moon nights, and true at other times. Then, this program will evaluate to an
answer except on new-moon nights, when it will fail with an unbound identifier
error.

\Exercise{
What happens on cloudy nights?
}

\secrel{6.4.2 The Top-Level Scope  . 31}

Matters become more complex when we contemplate top-level definitions in many
languages. For instance, some versions of Scheme (which is a paragon of lexical scoping) allow you to write this:
\lsts{src/6/6_4_2.rkt}{rkt}
which seems to pretty clearly suggest where the y in the body of f will come
from, except:
\lsts{src/6/6_4_3.rkt}{rkt}
is legal and (f 10) produces 12. Wait, you might think, always take the last
one! But:
\lsts{src/6/6_4_4.rkt}{rkt}

Here, z is bound to the first value of y whereas the inner y is bound to the
second value. There is actually a valid explanation of this behavior in terms of
lexical scope, but it can become convoluted, and perhaps a more sensible option is to
prevent such redefinition. Racket does precisely this, thereby offering the
convenience of a top-level without its pain.
\note{Most “scripting” languages exhibit similar problems. As a result, on the
Web you will find enormous confusion about whether a certain language is
statically- or dynamically-scoped, when in fact readers are comparing behavior
inside functions (often static) against the top-level (usually dynamic).
Beware!}

\secup

\secrel{6.5 Exposing the Environment  . . 31}

If we were building the implementation for others to use, it would be wise and a
courtesy for the exported interpreter to take only an expression and list of
function definitions, and invoke our defined interp with the empty environment.
This both spares users an implementation detail, and avoids the use of an
interpreter with an incorrect environment. In some contexts, however, it can be
useful to expose the environment parameter. For instance, the environment can
represent a set of pre-defined bindings: e.g., if the language wishes to provide
pi automatically bound to 3.2 (in Indiana).

\secup


% \secrel{7 Functions Anywhere 31}\secdown

The introduction to the Scheme programming language definition establishes this
design principle:
\begin{framed}
Programming languages should be designed not by piling feature on top of
feature, but by removing the weaknesses and restrictions that make additional
features appear necessary. \ref{}
\end{framed}
As design principles go, this one is hard to argue with. (Some restrictions, of
course, have good reason to exist, but this principle forces us to argue for
them, not admit them by default.) Let’s now apply this to functions.

One of the things we stayed coy about when introducing functions (section 5) is
exactly where functions go. We may have suggested we’re following the model of
an idealized DrRacket, with definitions and their uses kept separate. But,
inspired by the Scheme design principle, let’s examine how necessary that is.

Why can’t functions definitions be expressions? In our current
arithmetic-centric language we face the uncomfortable question “What value does
a function definition represent?”, to which we don’t really have a good answer.
But a real programming language obviously computes more than numbers, so we no
longer need to confront the question in this form; indeed, the answer to the
above can just as well be, “A function value”. Let’s see how that might work
out.

What can we do with functions as values? Clearly, functions are a distinct kind
of value than a number, so we cannot, for instance, add them. But there is one
evident thing we can do: apply them to arguments! Thus, we can allow function
values to appear in the function position of an application. The behavior would,
naturally, be to apply the function. Thus, we’re proposing a language where the
following would be a valid program (where I’ve used brackets so we can easily
identify the function)
\lsts{src/7/7.rkt}{rkt}
and would evaluate to \verb|(+ 2 (* 4 3))|, or 14.\note{Did you see that I just
used substitution\,?}

\secrel{7.1 Functions as Expressions and Values  32}

Let’s first define the core language to include function definitions:
\lsts{src/7/7_1_1.rkt}{rkt}

For now, we’ll simply copy function definitions into the expression language.
We’re free to change this if necessary as we go along, but for now it at least
allows us to reuse our existing test cases.
\lsts{src/7/7_1_2.rkt}{rkt}

We also need to determine what an application looks like. What goes in the
function position of an application? We want to allow an entire function
definition, not just its name. Because we’ve lumped function definitions in with
all other expressions, let’s allow an arbitrary expression here, but with the
understanding that we want only function definition expressions:
\note{We might consider more refined datatypes that split function definitions
apart from other kinds of expressions. This amounts to trying to classify
different kinds of expressions, which we will return to when we study types.
\ref{}}
\lsts{src/7/7_1_3.rkt}{rkt}

With this definition of application, we no longer have to look up functions by
name, so the interpreter can get rid of the list of function definitions. If we
need it we can restore it later, but for now let’s just explore what happens
with function definitions are written at the point of application: so-called
immediate functions.

Now let’s tackle interp. We need to add a case to the interpreter for function
definitions, and this is a good candidate:
\lsts{src/7/7_1_4.rkt}{rkt}

\DoNow{
What happens when you add this?
}
Immediately, we see that we have a problem: the interpreter no longer always
returns numbers, so we have a type error.

We’ve alluded periodically to the answers computed by the interpreter, but never
bothered gracing these with their own type. It’s time to do so now.
\lsts{src/7/7_1_5.rkt}{rkt}

We’re using the suffix of V to stand for values, i.e., the result of evaluation.
The pieces of a funV will be precisely those of a fdC: the latter is input, the
former is output. By keeping them distinct we allow each one to evolve
independently as needed.

Now we must rewrite the interpreter. Let’s start with its type:
\lsts{src/7/7_1_6.rkt}{rkt}

This change naturally forces corresponding type changes to the Binding datatype
and to lookup.

\Exercise{
Modify Binding and lookup, appropriately.
}
\lsts{src/7/7_1_7.rkt}{rkt}

Clearly, numeric answers need to be wrapped in the appropriate numeric answer
constructor. Identifier lookup is unchanged. We have to slightly modify addition
and multiplication to deal with the fact that the interpreter returns Values,
not numbers:
\lsts{src/7/7_1_8.rkt}{rkt}

It’s worth examining the definition of one of these helper functions:
\lsts{src/7/7_1_9.rkt}{rkt}
Observe that it checks that both arguments are numbers before performing the
addition. This is an instance of a safe run-time system. We’ll discuss this
topic more when we get to types. \ref{}

There are two more cases to cover. One is function definitions. We’ve already
agreed these will be their own kind of value:
\lsts{src/7/7_1_10.rkt}{rkt}

That leaves one case, application. Though we no longer need to look up the
function definition, we’ll leave the code structured as similarly as possible:
\lsts{src/7/7_1_11.rkt}{rkt}

In place of the lookup, we reference f which is the function definition, sitting
right there. Note that, because any expression can be in the function definition
position, we really ought to harden the code to check that it is indeed a
function.

\DoNow{
What does is mean? That is, do we want to check that the function definition
position is syntactically a function definition (fdC), or only that it
evaluates to one (funV)? Is there a difference, i.e., can you write a program
that satisfies one condition but not the other?
}

We have two choices:
\begin{enumerate}[nosep]
  \item 
We can check that it syntactically is an fdC and, if it isn’t reject it as an
error.
  \item 
We can evaluate it, and check that the resulting value is a function (and signal
an error otherwise).
\end{enumerate}
We will take the latter approach, because this gives us a much more flexible
language. In particular, even if we can’t immediately imagine cases where we, as
humans, might need this, it might come in handy when a program needs to generate
code. And we’re writing precisely such a program, namely the desugarer! (See
section 7.5.) As a result, we’ll modify the application case to evaluate the
function position:
\lsts{src/7/7_1_12.rkt}{rkt}

\Exercise{
Modify the code to perform both versions of this check.
}

And with that, we’re done. We have a complete interpreter! Here, for instance,
are some of our old tests again:
\lsts{src/7/7_1_13.rkt}{rkt}

\secrel{7.2 Nested What?   35}

\secrel{7.3 Implementing Closures   . 37}

We need to change our representation of values to record closures rather than raw
function text:
\lsts{src/7/7_3_1.rkt}{rkt}
While we’re at it, we might as well alter our syntax for defining functions to drop
the useless name. This construct is historically called a lambda:
\lsts{src/7/7_3_2.rkt}{rkt}

When encountering a function definition, the interpreter must now remember to
save the substitutions that have been applied so far:
\note{“Save the environment! Create a closure today!”\ --- Cormac Flanagan}
\lsts{src/7/7_3_3.rkt}{rkt}

This saved set, not the empty environment, must be used when applying a
function:
\lsts{src/7/7_3_4.rkt}{rkt}

There’s actually another possibility: we could use the environment present at the
point of application:
\lsts{src/7/7_3_5.rkt}{rkt}

\Exercise{
What happens if we extend the dynamic environment instead?
}

In retrospect, it becomes even more clear why we interpreted the body of a
function in the empty environment. When a function is defined at the top-level,
it is not “closed over” any identifiers. Therefore, our previous function
applications have been special cases of this form of application.

\secrel{7.4 Substitution, Again   38}

\secrel{7.5 Sugaring Over Anonymity  . . 39}

Now let’s get back to the idea of naming functions, which has evident value for
program understanding. Observe that we do have a way of naming things: by
passing them to functions, where they acquire a local name (that of the formal
parameter). Anywhere within that function’s body, we can refer to that entity
using the formal parameter name.

Therefore, we can take a collection of function definitions and name them using
other...functions. For instance, the Racket code
\lsts{src/7/5/1.rkt}{rkt}
could first be rewritten as the equivalent
\lsts{src/7/5/2.rkt}{rkt}
We can of course just inline the definition of double, but to preserve the name,
we could write this as:
\lsts{src/7/5/3.rkt}{rkt}
Indeed, this pattern—which we will pronounce as “left-left-lambda”—is a local
naming mechanism. It is so useful that in Racket, it has its own special syntax:
\lsts{src/7/5/4.rkt}{rkt}
where let can be defined by desugaring as shown above.

Here’s a more complex example:
\lsts{src/7/5/5.rkt}{rkt}
This could be rewritten as
\lsts{src/7/5/6.rkt}{rkt}
which works just as we’d expect; but if we change the order, it no longer works—
\lsts{src/7/5/7.rkt}{rkt}
—because quadruple can’t “see” double. so we see that top-level binding is
different from local binding: essentially, the top-level has an “infinite
scope”. This is the source of both its power and problems.

There is another, subtler, problem: it has to do with recursion. Consider the
simplest infinite loop:
\lsts{src/7/5/8.rkt}{rkt}
Let’s convert it to let:
\lsts{src/7/5/9.rkt}{rkt}
Seems fine, right? Rewrite in terms of lambda:
\lsts{src/7/5/10.rkt}{rkt}
Clearly, the loop-forever on the last line isn’t bound!

This is another feature we get “for free” from the top-level. To eliminate this
magical force, we need to understand recursion explicitly, which we will do soon
\ref{}.
\secup


\secrel{8 Mutation: Structures and Variables 41}\secdown

It’s time for another

\bigskip
\textbf{Which of these is the same?}
\begin{itemize}
  \item \verb|f = 3|
  \item \verb|o.f = 3|
  \item \verb|f = 3|
\end{itemize}

Assuming all three are in Java, the first and third could behave exactly like
each other or exactly like the second: it all depends on whether f is a local
identifier (such as a parameter) or a field of the object (i.e., the code is
really this.f = 3).

In either case, we are asking the evaluator to permanently change the value
bound to f. This has important implications for other observers. Until now, for
a given set of inputs, a computation always returned the same value. Now, the
answer depends on when it was invoked: above, it depends on whether it was
invoked before or after the value of f was changed. The introduction of time has
profound effects on reasoning about programs.

However, there are really two quite different notions of change buried in the
uniform syntax above. Changing the value of a field (o.f = 3 or this.f = 3) is
extremely different from changing that of an identifier (f = 3 where f is bound
inside the method, not by the object). We will explore these in turn. We’ll
tackle fields below, and return to identifiers in section \ref{8_2_vars}.

\secrel{8.1 Mutable Structures   41}
\secdown
\secrel{8.1.1 A Simple Model of Mutable Structures  41}

Objects are a generalization of structures, as we will soon see \ref{}.
Therefore, fields in objects are a generalization of fields in structures and to
understand mutation, it is mostly (but not entirely! \ref{}) sufficient to
understand mutable objects. To be even more reductionist, we don’t need a
structure to have many fields: a single one will suffice. We call this a box. In
Racket, boxes support just three operations:
\lsts{src/8/1/1.rkt}{rkt}
Thus, box takes a value and wraps it in a mutable container. unbox extracts the
current value inside the container. Finally, set-box! changes the value in the
container, and in a typed language, the new value is expected to be
type-consistent with what was there before. You can thus think of a box as
equivalent to a Java container class with parameterized type, which has a single
member field with a getter and setter: box is the constructor, unbox is the
getter, and set-box! is the setter. (Because there is only one field, its name
is irrelevant.)
\lsts{src/8/1/2.rkt}{rkt}

Because we must sometimes mutate in groups (e.g., removing money from one bank
account and depositing it in another), it is useful to be able to sequence a
group of mutable operations. In Racket, begin lets you write a sequence of
operations; it evaluates them in order and returns the value of the last one.

\Exercise{
Define begin by desugaring into let (and hence into lambda).
}

Even though it is possible to eliminate begin as syntactic sugar, it will prove
extremely useful for understanding how mutation works. Therefore, we will add a
simple, two-term version of sequencing to the core.

\secrel{8.1.2 Scaffolding   42}

First, let’s extend our core language datatype:
\note{This is an excellent illustration of the non-canonical nature of
desguaring. We’ve chosen to add to the core a construct that is certainly not
necessary. If our goal was to shrink the size of the interpreter— perhaps at
some cost to the size of the input program—we would not make this choice. But
our goal in this book is to study pedagogic interpreters, so we choose a larger
language because it is more instructive.}
\lsts{src/8/1/2_1.rkt}{rkt}
Observe that in a setboxC expression, both the box position and its new value
are expressions. The latter is unsurprising, but the former might be. It means
we can write programs such as this in corresponding Racket:
\lsts{src/8/1/2_2.rkt}{rkt}
This evaluates to a list of boxes, the first containing 1 and the second 2.
Observe that the first argument to the first set-box! instruction was (first l),
i.e., an expression that evaluated to a box, rather than just a literal box or
an identifier. This is precisely analogous to languages like Java, where one can
(taking some type liberties) write
\note{Your output may look like '(\#\&1 \#\&2). The \#\& notation is Racket’s
abbreviated syntactic prefix for “box”.}
\lsts{src/8/1/2_3.rkt}{rkt}
Observe that l.get(0) is a compound expression being used to find the
appropriate box, and evaluates to the box object on which set is invoked.

For convenience, we will assume that we have implemented desguaring to provide
us with (a) let and (b) if necessary, more than two terms in a sequence (which
can be desugared into nested sequences). We will also sometimes write
expressions in the original Racket syntax, both for brevity (because the core
language terms can grow quite large and unwieldy) and so that you can run these
same terms in Racket and observe what answers they produce. As this implies, we
are taking the behavior in Racket—which is similar to the behavior in just about
every mainstream language with mutable objects and structures—as the reference
behavior.

\secrel{8.1.3 Interaction with Closures  . . 43}

Consider a simple counter:
\lsts{src/8/1/3_1.rkt}{rkt}
Every time it is invoked, it produces the next integer:
\lst{src/8/1/3_1.log}
Why does this work? It’s because the box is created only once, and bound to n,
and then closed over. All subsequent mutations affect the same box. In contrast,
swapping two lines makes a big difference:
\lsts{src/8/1/3_2.rkt}{rkt}
Observe:
\lst{src/8/1/3_2.log}
In this case, a new box is allocated on every invocation of the function, so the
answer each time is the same (despite the mutation inside the procedure). Our
implementation of boxes should be certain to preserve this distinction.

The examples above hint at an implementation necessity. Clearly, whatever the
environment closes over in new-loc must refer to the same box each time. Yet
something also needs to make sure that the value in that box is different each
time! Look at it more carefully: it must be lexically the same, but dynamically
different. This distinction will be at the heart of our implementation.

\secrel{8.1.4 Understanding the Interpretation of Boxes . . 44}

Let’s begin by reproducing our current interpreter:
\lst{src/8/1/4_1.rkt}
Because we’ve introduced a new kind of value, the box, we have to update the set
of values:
\lst{src/8/1/4_2.rkt}
Two of these cases should be easy. When we’re given a box expression, we simply
evaluate it and return it wrapped in a boxV:
\lst{src/8/1/4_3.rkt}
Similarly, extracting a value from a box is easy:
\lst{src/8/1/4_4.rkt}

By now, you should be constructing a healthy set of test cases to make sure
these behave as you’d expect.

Of course, we haven’t done any hard work yet. All the interesting behavior is,
presumably, hidden in the treatment of setboxC. It may therefore surprise you
that we’re going to look at seqC first instead (and you’ll see why we included
it in the core).

Let’s take the most natural implementation of a sequence of two instructions:
\lst{src/8/1/4_5.rkt}

That is, we evaluate the first term, then the second, and return the result of
the second.

You should immediately spot something troubling. We bound the result of
evaluating the first term, but didn’t subsequently do anything with it. That’s
okay: presumably the first term contained a mutation expression of some sort,
and its value is uninteresting (indeed, note that set-box! returns a void
value). Thus, another implementation might be this:
\lst{src/8/1/4_6.rkt}

Not only is this slightly dissatisfying in that it just uses the analogous
Racket sequencing construct, it still can’t possibly be right! This can only
work only if the result of the mutation is being stored somewhere. But because
our interpreter only computes values, and does not perform any mutation itself,
any mutations in (interp b1 env) are completely lost. This is obviously not what
we want.

\secrel{8.1.5 Can the Environment Help?  46}

Here is another example that can help:
\lst{src/8/1/5_1.rkt}
In Racket, this evaluates to 2.

\Exercise{
Represent this expression in ExprC.
}

Let’s consider the evaluation of the inner sequence. In both cases, the
expression (the representation of (set-box! ...)) is exactly identical. Yet
something is changing underneath, because these cause the value of the box to go
from 0 to 2! We can “see” this even more clearly if instead we evaluate
\lst{src/8/1/5_2.rkt}
which evaluates to 3. Here, the two calls to interp in the rule for addition are
sending exactly the same textual expression in both cases. Yet somehow the
effects from the left branch of the addition are being felt in the right branch,
and we must rule out spukhafte Fernwirkung.

If the interpreter is being given precisely the same expression, how can it
possibly avoid producing precisely the same answer? The most obvious way is if
the interpreter’s other parameter, the environment were somehow different. As of
now the exact same environment is sent to both both branches of the sequence and
both arms of the addition, so our interpreter—which produces the same output
every time on a given input—cannot possibly produce the answers we want.

Here is what we know so far:
\begin{enumerate}[nosep]
  \item 
We must somehow make sure the interpreter is fed different arguments on calls
that are expected to potentially produce different results.
  \item 
We must return from the interpreter some record of the mutations made when
evaluating its argument expression.
\end{enumerate}

Because the expression is what it is, the first point suggests that we might try
to use the environment to reflect the differences between invocations. In turn,
the second point suggests that each invocation of the interpreter should also
return the environment, so it can be passed to the next invocation. Roughly,
then, the type of the interpreter might become:
\lst{src/8/1/5_3.rkt}
That is, the interpreter consumes an expression and environment; it evaluates in
that environment, updating it as it proceeds; when the expression is done
evaluating, the interpreter returns the answer (as it did before), along with an
updated environment, which in turn is sent to the next invocation of the
interpreter. And the treatment of setboxC would somehow impact the environment
to reflect the mutation.

Before we dive into the implementation, however, we should consider the
consequences of such a change. The environment already serves an important
purpose: it holds deferred substitutions. In that respect, it already has a
precise semantics—given by substitution—and we must be careful to not alter
that. One consequence of its tie to substitution is that it is also the
repository of lexical scope information. If we were to allow the extended
environment escape from one branch of addition and be used in the other, for
instance, consider the impact on the equivalent of the following program:
\lst{src/8/1/5_4.rkt}
It should be evident that this program has an error: b in the right branch of
the addition is unbound (the scope of the b in the left branch ends with the
closing of the let—if this is not evident, desugar the above expression to use
functions). But the extended environment at the end of interpreting the let
clearly has b bound in it.
\Exercise{
Work out the above problem in detail and make sure you understand it.
}

You could try various other related proposals, but they are likely to all have
similar failings. For instance, you may decide that, because the problem has to
do with additional bindings in the environment, you will instead remove all
added bindings in the returned environment. Sounds attractive? Did you remember
we have closures?
\Exercise{
Consider the representation of the following program:
\lst{src/8/1/5_5.rkt}
What problems does this example cause?
}

Rather, we should note that while the constraints described above are all valid,
the solution we proposed is not the only one. What we require are the two
conditions enumerated above; observe that neither one actually requires the
environment to be the responsible agent. Indeed, it is quite evident that the
environment cannot be the principal agent.

\secrel{8.1.6 Introducing the Store  . 48}

The preceding discussion tells us that we need two repositories to accompany the
expression, not one. One of them, the environment, continues to be responsible
for maintaining lexical scope. But the environment cannot directly map
identifiers to their value, because the value might change. Instead, something
else needs to be responsible for maintaining the dynamic state of mutated boxes.
This latter data structure is called the store.

Like the environment, the store is a partial map. Its domain could be any
abstract set of names, but it is natural to think of these as numbers, meant to
stand for memory locations. This is because the store in the semantics maps
directly onto (abstracted) physical memory in the machine, which is
traditionally addressed by numbers. Thus the environment maps names to
locations, and the store maps locations to values:
\lsts{src/8/1/6_1.rkt}{rkt}
We’ll also equip ourselves with a function to look up values in the store, just
as we already have one for the environment (which now returns locations
instead):
\lsts{src/8/1/6_2.rkt}{rkt}

With this, we can refine our notion of values to the correct one:
\lsts{src/8/1/6_3.rkt}{rkt}

\Exercise{
Fill in the bodies of lookup and fetch.
}

\secrel{8.1.7 Interpreting Boxes  . . 49}

Now we have something that the environment can return, updated, reflecting
mutations during the evaluation of the expression, without having to change the
environment in any way. Because a function can return only one value, let’s
define a data structure to hold the new result from the interpreter:
\lsts{src/8/1/7_1.rkt}{rkt}
Thus the interpreter’s type becomes:
\lsts{src/8/1/7_2.rkt}{rkt}

The easiest one to dispatch is numbers. Remember that we have to return the
store reflecting all mutations that happened while evaluating the given
expression. Because a number is a constant, no mutations could have happened, so
the returned store is the same as the one passed in:
\lsts{src/8/1/7_3.rkt}{rkt}

A similar argument applies to closure creation; observe that we are speaking of
the creation, not use, of closures:
\lsts{src/8/1/7_4.rkt}{rkt}

Identifiers are almost as straightforward, though if you are simplistic, you’ll
get a type error that will alert you that to obtain a value, you must now look
up both in the environment and in the store:
\lsts{src/8/1/7_5.rkt}{rkt}

Notice how lookup and fetch compose to produce the same result that lookup
alone produced before.

Now things get interesting.

Let’s take sequencing. Clearly, we need to interpret the two terms:
\lsts{src/8/1/7_6.rkt}{rkt}
Oh, but wait. The whole point was to evaluate the second term in the store
returned by the first one—otherwise there would have been no point to all these
changes. Therefore, instead we must evaluate the first term, capture the
resulting store, and use it to evaluate the second. (Evaluating the first term
also yields its value, but sequencing ignores this value and assumes the first
time was run purely for its potential mutations.) We will write this in a
stylized manner:
\lsts{src/8/1/7_7.rkt}{rkt}

This says to (interp b1 env sto); name the resulting value and store v-b1 and
s-b1, respectively; and evaluate the second term in the store from the first:
(interp b2 env s-b1). The result will be the value and store returned by the
second term, which is what we expect. The fact that the first term’s effect is
only on the store can be read from the code because, though we bind v-b1, we
never subsequently use it.
\DoNow{
Spend a moment contemplating the code above. You’ll soon need to adjust
your eyes to read this pattern fluently.
}

Now let’s move on to the binary arithmetic primitives. These are similar to
sequencing in that they have two sub-terms, but in this case we really do care
about the value from each branch. As usual, we’ll look at only plusC since multC
is virtually identical.
\lsts{src/8/1/7_8.rkt}{rkt}

Observe that we’ve unfolded the sequencing pattern out another level, so we can
hold on to both results and supply them to num+.

Here’s an important distinction. When we evaluate a term, we usually use the
same environment for all its sub-terms in accordance with the scoping rules of
the language. The environment thus flows in a recursive-descent pattern. In
contrast, the store is threaded: rather than using the same store in all
branches, we take the store from one branch and pass it on to the next, and take
the result and send it back out. This pattern is called store-passing style.

Now the penny drops. We see that store-passing style is our secret ingredient:
it enables the environment to preserve lexical scope while still giving a
binding structure that can reflect changes. Our intution told us that the
environment had to somehow participate in obtaining different results for the
same expression, and we can now see how it does: not directly, by itself
changing, but indirectly, by referring to the store, which updates. Now we only
need to see how the store itself “changes”.

Let’s begin with boxing. To store a value in a box, we have to first allocate a
new place in the store where its value will reside. The value corresponding to a
box will then remember this location, for use in box mutation.
\lsts{src/8/1/7_9.rkt}{rkt}

\DoNow{
Observe that we have relied above on new-loc, which is itself implemented in
terms of boxes! This is outright cheating. How would you modify the interpreter
so that we no longer need an mutating implementation of new-loc?
}

To eliminate this style of new-loc, the simplest option would be to add yet
another parameter to and return value from the interpreter, which represents the
largest address used so far. Every operation that allocates in the store would
return an incremented address, while all others would return it unchanged. In
other words, this is precisely another application of the store-passing pattern.
Writing the interpreter this way would make it extremely unwieldy and might
obscure the more important use of store-passing for the store itself, which is
why we have not done so. However, it is important to make sure that we can:
that’s what tells us that we are not reliant on boxes to add boxes to the
language.

Now that boxes are recording the location in memory, getting the value
corresponding to them is easy.
\lsts{src/8/1/7_10.rkt}{rkt}

It’s the same pattern we saw before, where we have to use fetch to obtain the
actual value residing at that location. Note that we are relying on Racket to
halt with an error if the underlying value isn’t actually a boxV; otherwise it
would be dangerous to not check, since this would be tantamount to dereferencing
arbitrary memory (as C programs can, sometimes with disastrous consequences).

Let’s now see how to update the value held in a box. First we have to evaluate
the box expression to obtain a box, and the value expression to obtain the new
value to store in it. The box’s value is going to be a boxV holding a location.

In principle, we want to “change”, or override, the value at that location in
the store. We can do this in two ways.
\begin{enumerate}[nosep]
  \item 
One is to traverse the store, find the old binding for that location, and
replace it with the new one, copying all the other store bindings unchanged.
  \item 
The other, lazier, option is to simply extend the store with a new binding for
that location, which works provided we always obtain the most recent binding for
a location (which is how lookup works in the environment, so fetch presumably
also does in the store).
\end{enumerate}

The code below is written to be independent of these options:
\lsts{src/8/1/7_11.rkt}{rkt}

However, because we’ve implemented override-store as cons above, we’ve actually
taken the lazier (and slightly riskier, because of its dependence on the
implementation of fetch) option.
\Exercise{
Implement the other version of store alteration, whereby we update an existing
binding and thereby avoid multiple bindings for a location in the store.
}
\Exercise{
When we look for a location to override the value stored at it, can the location
fail to be present? If so, write a program that demonstrates this. If not,
explain what invariant of the interpreter prevents this from happening.
}

Alright, we’re now done with everything other than application! Most of
application should already be familiar: evaluate the function position, evaluate
the argument position, interpret the closure body in an extension of the
closure’s environment...but how do stores interact with this?
\lsts{src/8/1/7_12.rkt}{rkt}

Let’s start by thinking about extending the closure environment. The name we’re
extending it with is obviously the name of the function’s formal parameter. But
what location do we bind it to? To avoid any confusion with already-used
locations (a confusion we will explicitly introduce later! [REF]), let’s just
allocate a new location. This location is used in the environment, and the value
of the argument resides at this location in the store:
\lsts{src/8/1/7_13.rkt}{rkt}

Because we have not said the function parameter is mutable, there is no real
need to have implemented procedure calls this way. We could instead have
followed the same strategy as before. Indeed, observe that the mutability of
this location will never be used: only setboxC changes what’s in an existing
store location (the override- store above is technically a store
initialization), and then only when they are referred to by boxVs, but no box is being allocated above. However, we have chosen to implement application this way
for uniformity, and to reduce the number of cases we’d have to handle.
\note{You could call this the useless app store.}

\Exercise{
It’s a useful exercise to try to limit the use of store locations only to boxes.
How many changes would you need to make?
}

\secrel{8.1.8 The Bigger Picture  . . 54}

\secup

\secrel{8.2 Variables   . . 57}\label{8_2_vars}

Now that we’ve got structure mutation worked out, let’s consider the other case:
variable mutation.

\secdown
\secrel{8.2.1 Terminology   . . 57}

First, our choice of terms. We’ve insisted on using the word “identifier” before because
we wanted to reserve “variable” for what we’re about to study. In Java, when we say
(assuming x is locally bound, e.g., as a method parameter)
\lsts{src/8/2/1.rkt}{java}
we’re asking to change the value of x. After the first assignment, the value of
x is 1; after the second one, it’s 3. Thus, the value of x varies over the
course of the execution of the method.

Now, we also use the term “variable” in mathematics to refer to function
parameters. For instance, in f(y) = y + 3 we say that y is a “variable”. That is
called a variable because it varies across invocations; however, within each
invocation, it has the same value in its scope. Our identifiers until now have
corresponded to this notion of a variable. In contrast, programming variables
can vary even within each invocation, like the Java x above.
\note{If the identifier was bound to a box, then it remained bound to the same
box value. It’s the content of the box that changed, not which box the
identifier was bound to.}

Henceforth, we will use variable when we mean an identifier whose value can
change within its scope, and identifier when this cannot happen. If in doubt, we
might play it safe and use “variable”; if the difference doesn’t really matter,
we might use either one. It is less important to get caught up in these specific
terms than to understand that they represent a distinction that matters \ref{}.

\secrel{8.2.2 Syntax   . . 57}

Whereas other languages overload the mutation syntax (= or :=), in Racket they
are kept distinct: set! is used to mutate variables. This forces Racket
programmers to confront the distinction we introduced at the beginning of
section 8. We will, of course, sidestep these syntactic issues in our core
language by using different constructs for boxes and for variables.

The first thing to note about variable mutation is that, although it too has two subterms
like box mutation (setboxC), its syntax is fundamentally different. To understand
why, let’s return to our Java fragment:
\lsts{src/8/2/2.rkt}{java}
In this setting, we cannot write an arbitrary expression in place of x: we must
literally write the name of the identifier itself. That is because, if it were
an expression position, then we could evaluate it, yielding a value: for
instance, if x were previously bound to 1, this would be tantamout to writing
the following statement:
\lsts{src/8/2/3.rkt}{java}
But this is, of course, nonsensical! We can’t assign a new value to 1, and
indeed 1 is pretty much the definition of immutable. Thus, what we instead want
is to find where x is in the store, and change the value held over there.

Here’s another way to see this. Suppose the local variable o were bound to some
String object; let’s call this object s. Say we write
\lsts{src/8/2/4.rkt}{java}
Are we trying to change s in any way? Certainly not: this statement intends to
leave s alone. It only wants to change the value that o is referring to, so that
subsequent references evaluate to this new string object instead.

\secrel{8.2.3 Interpreting Variables  . 58}

We’ll start by reflecting this in our syntax:
\lsts{src/8/2/5.rkt}{rkt}
Observe that we’ve jettisoned the box operations, but kept sequencing because
it’s handy around mutation. Importantly, we’ve now added the setC case, and its
first subterm is not an expression but the literal name of a variable. We’ve
also renamed idC to varC.

Because we’ve gotten rid of boxes, we can also get rid of the special box
values. When the only kind of mutation you have is variables, you don’t need new
kinds of values.
\lsts{src/8/2/6.rkt}{rkt}

As you might imagine, to support variables we need the same store-passing style
that we’ve seen before (section 8.1.7), and for the same reasons. What differs
is in precisely how we use it. Because sequencing is interpreted in just the
same way (observe that the code for it does not depend on boxes versus
variables), that leaves us just the variable mutation case to handle.

First, we might as well evaluate the value expression and obtain the updated
store:
\lsts{src/8/2/7.rkt}{rkt}

What now? Remember we just said that we don’t want to fully evaluate the
variable, because that would just give the value it is bound to. Instead, we
want to know which memory location it corresponds to, and update what is stored
at that memory location; this latter part is just the same thing we did when
mutating boxes:
\lsts{src/8/2/8.rkt}{rkt}

The very interesting new pattern we have here is this. When we added boxes, in
the idC case, we looked up an identifier in the environment, and immediately
fetched the value at that location from the store; the composition yielded a
value, just as it used to before we added stores. Now, however, we have a new
pattern: looking up an identifier in the environment without subsequently
fetching its value from the store. The result of invoking just lookup is
traditionally called an l-value, for “left-handside (of an assignment) value”.
This is a fancy way of saying “memory address”, and stands in contast to the
actual values that the store yields: observe that it does not directly
correspond to anything in the type Value.

And we’re done! We did all the hard work when we implemented store-passing style
(and also in that application allocated new locations for variables).


\secup

\secrel{8.3 The Design of Stateful Language Operations  . . 59}

Though most programming languages include one or both kinds of state we have
studied, their admission should not be regarded as a trivial or foregone matter.
On the one hand, state brings some vital benefits:

\begin{itemize}
  \item 
State provides a form of modularity. As our very interpreter demonstrates,
without explicit stateful operations, to achieve the same effect:
\begin{itemize}
  \item 
We would need to add explicit parameters and return values that pass the
equivalent of the store around.
  \item
These changes would have to be made to all procedures that may be involved in a
communication path between producers and consumers of state.
\end{itemize}

Thus, a different way to think of state in a programming language is that it is
an implicit parameter already passed to and returned from all procedures,
without imposing that burden on the programmer. This enables procedures to
communicate “at a distance” without all the intermediaries having to be aware of
the communication.

  \item
State makes it possible to construct dynamic, cyclic data structures, or at
least to do so in a relatively straightforward manner \ref{sec9}.

  \item
State gives procedures memory, such as new-loc above. If a procedure could not
remember things for itself, the callers would need to perform the remembering on
its behalf, employing the moral equivalent of store-passing. This is not only
unwieldy, it creates the potential for a caller to interfere with the memory for
its own nefarious purposes (e.g., a caller might purposely send back an old
store, thereby obtaining a reference already granted to some other party,
through which it might launch a correctness or security attack).

\end{itemize}

On the other hand, state imposes real costs on programmers as well as on
programs that process programs (such as compilers). One is “aliasing”, which we
discuss later \ref{}. Another is “referential transparency”, which too I
hope to return to \ref{}. Finally, we have described above how state provides a
form of modularity. However, this same description could be viewed as that of a
back-channel of communication that the intermediaries did not know and could not
monitor. In some (especially security and distributed system) settings, such
back-channels can lead to collusion, and can hence be extremely dangerous and
undesirable.

Because there is no optimal answer, it is probably wise to include mutation
operators but to carefully delinate them. In Standard ML, for instance, there is
no variable mutation, because it is considered unnecessary. Instead, the
language has the equivalent of boxes (called refs). One can easily simulate
variables using boxes (e.g., see new-loc and consider how it would be written
with variables instead), so no expressive power is lost, though it does create
more potential for aliasing than variables alone would have (aliasing
\ref{aliasing}) if the boxes are not used carefully.

In return, however, developers obtain expressive types: every data structure is
considered immutable unless it contains a ref, and the presence of a ref is a
warning to both developers and programs (such as compilers) that the underlying
value may keep changing. Thus, for instance, if b is a box, a developer should
be aware that replacing all instances of (unbox b) with v, where v is bound to
(unbox b), is unwise: the former always fetches the current value in the box,
while the latter may be referring to an older content. (Conversely, if the
developer wants the value at a certain point in time, oblivious to future
mutations to the box, they should be sure to retrieve and bind it rather than
always use unbox.)

\secrel{8.4 Parameter Passing   . 60}

In our current implementation, on every function call, we allocate a fresh
location in the store for the parameter. This means the following program
\lsts{src/8/4/1.rkt}{rkt}
evaluates to 5, not 3. That is because the value of the formal parameter x is
held at a different location than that of the actual parameter y, so the
mutation affects the location of x, leaving y unscathed.

Now suppose, instead, that application behaved as follows. When the actual
parameter is a variable, and hence has a location in memory, instead of
allocating a new location for the value, it simply passes along the existing one
for the variable. Now the formal parameter is referring to the same store
location as the actual: i.e., they are variable aliases. Thus any mutation on
the formal will leak back out into the calling context; the above program would
evaluate to 3 rather than 5. These is called a call-by-reference
parameter-passing strategy.
\note{Instead, our interpreter implements call-by-value, and this is the same
strategy followed by languages like Java. This causes confusion because when the
value is itself mutable, changes made to the value in the callee are observed by
the caller. However, that is simply an artifact of mutable values, not of the
calling strategy. Please avoid this confusion!}

For some years, this power was considered a good idea. It was useful because
programmers could write abstractions such as swap, which swaps the value of two
variables in the caller. However, the disadvantages greatly outweigh the
advantages:
\begin{itemize}
  \item 
A careless programmer can alias a variable in the caller and modify it without
realizing they have done so, and the caller may not even realize this has
happened until some obscure condition triggers it.
  \item 
Some people thought this was necessary for efficiency: they assumed the
alternative was to copy large data structures. However, call-by-value is
compatible with passing just the address of the data structure. You only need
make a copy if (a) the data structure is mutable, (b) you do not want the caller
to be able to mutate it, and (c) the language does not itself provide
immutability annotations or other mechanisms.
  \item 
It can force non-uniform and hence non-modular reasoning. For instance, suppose
we have the procedure:
\lsts{src/8/4/2.rkt}{rkt}
If the language were to permit by-reference parameter passing, then the
programmer cannot locally—i.e., just from the above code—determine what the
value of x will be in the ellipses.
\end{itemize}

At the very least, then, if the language is going to permit by-reference
parameters, it should let the caller determine whether to pass the
reference—i.e., let the callee share the memory address of the caller’s
variable—or not. However, even this option is not quite as attractive as it may
sound, because now the callee faces a symmetric problem, not knowing whether its
parameters are aliased or not. In traditional, sequential programs this is less
of a concern, but if the procedure is reentrant, the callee faces precisely the
same predicaments.

At some point, therefore, we should consider whether any of this fuss is
worthwhile. Instead, callers who want the callee to perform a mutation could
simply send a boxed value to the callee. The box signals that the caller
accepts—indeed, invites—the callee to perform a mutation, and the caller can
extract the value when it’s done. This does obviate the ability to write a
simple swapper, but that’s a small price to pay for genuine software engineering
concerns.

\secup


\secrel{9 Recursion and Cycles: Procedures and Data 62}\secdown
\secrel{9.1 Recursive and Cyclic Data  . . 62}
\secrel{9.2 Recursive Functions   64}
\secrel{9.3 Premature Observation   . 65}
\secrel{9.4 Without Explicit State   . . 66}
\secup

\secrel{10 Objects 67}\secdown
\secrel{10.1 Objects Without Inheritance  . 67}
\secdown
\secrel{10.1.1 Objects in the Core  . . 68}
\secrel{10.1.2 Objects by Desugaring  69}
\secrel{10.1.3 Objects as Named Collections  . . 69}
\secrel{10.1.4 Constructors   . . 70}
\secrel{10.1.5 State   71}
\secrel{10.1.6 Private Members   71}
\secrel{10.1.7 Static Members   . 72}
\secrel{10.1.8 Objects with Self-Reference  72}
\secrel{10.1.9 Dynamic Dispatch  . . 74}
\secup
\secrel{10.2 Member Access Design Space  75}
\secrel{10.3 What (Goes In) Else?   . . 75}
\secdown
\secrel{10.3.1 Classes   . . 76}
\secrel{10.3.2 Prototypes   78}
\secrel{10.3.3 Multiple Inheritance  . 78}
\secrel{10.3.4 Super-Duper!   . . 79}
\secrel{10.3.5 Mixins and Traits   79}
\secup
\secup

\secrel{11 Memory Management 81}\secdown
\secrel{11.1 Garbage   81}
\secrel{11.2 What is “Correct” Garbage Recovery?  . . 81}
\secrel{11.3 Manual Reclamation   . . 82}
\secdown
\secrel{11.3.1 The Cost of Fully-Manual Reclamation  82}
\secrel{11.3.2 Reference Counting  . 83}
\secup
\secrel{11.4 Automated Reclamation, or Garbage Collection  84}
\secdown
\secrel{11.4.1 Overview   . 84}
\secrel{11.4.2 Truth and Provability  . 85}
\secrel{11.4.3 Central Assumptions  . 85}
\secup
\secrel{11.5 Convervative Garbage Collection  . . 86}
\secrel{11.6 Precise Garbage Collection  . . 87}
\secup

\secrel{12 Representation Decisions 87}\secdown
\secrel{12.1 Changing Representations  . . 87}
\secrel{12.2 Errors   . 89}
\secrel{12.3 Changing Meaning   89}
\secrel{12.4 One More Example   90}
\secup

\secrel{13 Desugaring as a Language Feature 91}\secdown
\secrel{13.1 A First Example   . . 91}
\secrel{13.2 Syntax Transformers as Functions  . 93}
\secrel{13.3 Guards   . 95}
\secrel{13.4 Or: A Simple Macro with Many Features  95}
\secdown
\secrel{13.4.1 A First Attempt   . 95}
\secrel{13.4.2 Guarding Evaluation  . 97}
\secrel{13.4.3 Hygiene   . . 98}
\secup
\secrel{13.5 Identifier Capture   . 99}
\secrel{13.6 Influence on Compiler Design  101}
\secrel{13.7 Desugaring in Other Languages  . . 101}
\secup

\secrel{14 Control Operations 102}\secdown
\secrel{14.1 Control on the Web   102}
\secdown
\secrel{14.1.1 Program Decomposition into Now and Later 104}
\secrel{14.1.2 A Partial Solution   104}
\secrel{14.1.3 Achieving Statelessness  . . 106}
\secrel{14.1.4 Interaction with State  . 107}
\secup
\secrel{14.2 Continuation-Passing Style  . . 109}
\secdown
\secrel{14.2.1 Implementation by Desugaring  . . 110}
\secrel{14.2.2 Converting the Example  . . 114}
\secrel{14.2.3 Implementation in the Core  115}
\secup
\secrel{14.3 Generators   . . 117}
\secdown
\secrel{14.3.1 Design Variations   117}
\secrel{14.3.2 Implementing Generators  . . 119}
\secup
\secrel{14.4 Continuations and Stacks   121}
\secrel{14.5 Tail Calls   . . 123}
\secrel{14.6 Continuations as a Language Feature  124}
\secdown
\secrel{14.6.1 Presentation in the Language  125}
\secrel{14.6.2 Defining Generators  . 126}
\secrel{14.6.3 Defining Threads   127}
\secrel{14.6.4 Better Primitives for Web Programming  131}
\secup
\secup

\secrel{15 Checking Program Invariants Statically: Types 131}\secdown
\secrel{15.1 Types as Static Disciplines  . . 133}
\secrel{15.2 A Classical View of Types  . . 134}
\secdown
\secrel{15.2.1 A Simple Type Checker  . . 134}
\secrel{15.2.2 Type-Checking Conditionals  139}
\secrel{15.2.3 Recursion in Code  . . 139}
\secrel{15.2.4 Recursion in Data   142}
\secrel{15.2.5 Types, Time, and Space  . . 144}
\secrel{15.2.6 Types and Mutation  . . 146}
\secrel{15.2.7 The Central Theorem: Type Soundness  147}
\secup
\secrel{15.3 Extensions to the Core   . 148}
\secdown
\secrel{15.3.1 Explicit Parametric Polymorphism  148}
\secrel{15.3.2 Type Inference   . 155}
\secrel{15.3.3 Union Types   . . 164}
\secrel{15.3.4 Nominal Versus Structural Systems  . . 170}
\secrel{15.3.5 Intersection Types  . . 171}
\secrel{15.3.6 Recursive Types   172}
\secrel{15.3.7 Subtyping   . 173}
\secrel{15.3.8 Object Types   . . 176}
\secup
\secup

\secrel{16 Checking Program Invariants Dynamically: Contracts 179}\secdown
\secrel{16.1 Contracts as Predicates   . 181}
\secrel{16.2 Tags, Types, and Observations on Values  . 182}
\secrel{16.3 Higher-Order Contracts   . 183}
\secrel{16.4 Syntactic Convenience   . 187}
\secrel{16.5 Extending to Compound Data Structures  . 188}
\secrel{16.6 More on Contracts and Observations  189}
\secrel{16.7 Contracts and Mutation   . 189}
\secrel{16.8 Combining Contracts   . . 190}
\secrel{16.9 Blame   . 191}
\secup

\secrel{17 Alternate Application Semantics 195}\secdown
\secrel{17.1 Lazy Application . . . . . . . . . . . . . . . . . . . . . . . . . . . . 196}
\secdown
\secrel{17.1.1 A Lazy Application Example . . . . . . . . . . . . . . . . . . 196}
\secrel{17.1.2 What Are Values? . . . . . . . . . . . . . . . . . . . . . . . 197}
\secrel{17.1.3 What Causes Evaluation? . . . . . . . . . . . . . . . . . . . 198}
\secrel{17.1.4 An Interpreter . . . . . . . . . . . . . . . . . . . . . . . . . . 199}
\secrel{17.1.5 Laziness and Mutation . . . . . . . . . . . . . . . . . . . . . 201}
\secrel{17.1.6 Caching Computation . . . . . . . . . . . . . . . . . . . . . 201}
\secup
\secrel{17.2 Reactive Application . . . . . . . . . . . . . . . . . . . . . . . . . . 201}
\secdown
\secrel{17.2.1 Motivating Example: A Timer . . . . . . . . . . . . . . . . . 202}
\secrel{17.2.2 Callback Types are Four-Letter Words . . . . . . . . . . . . . 203}
\secrel{17.2.3 The Alternative: Reactive Languages . . . . . . . . . . . . . 204}
\secrel{17.2.4 Implementing Transparent Reactivity . . . . . . . . . . . . . 205}
\secup
\secup



\clearpage
\addcontentsline{toc}{chapter}{Index \ru{Предметный указатель}}\printindex

\end{document}
