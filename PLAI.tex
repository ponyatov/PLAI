\input{../texheader/ebook}

\usepackage{../texheader/lstrkt}\lstdefinestyle{rkt}{language=rkt}

\newcommand{\Exercise}[1]{
	\begin{description}
		\item{\textcolor{red}{Упражнение}}\\#1
	\end{description}
}

\newcommand{\DoNow}[1]{
	\begin{description}
		\item{\textcolor{red}{Сделайте\,!}}\\#1
	\end{description}
}

\title{{\Huge{PLAI}}\\
Programming Languages: Application and Interpretation\\
{\small{second edition}}\\
{\Huge{языки программирования}}\\{\Huge{применение и реализация}}}

\author{\copyright\ Шрирам Кришнамурти\\
перевод Dmitry Ponyatov \email{dponyatov@gmail.com}}

\begin{document}
\maketitle
\tableofcontents
\secdown

\secrel{Введение}\secdown
\secrel{Наша философия}

Пожалуйста посмотрите это
\href{https://www.youtube.com/watch?v=3N__tvmZrzc}{видео на YouTube}.

Когда нибудь здесь будет полное текстовое описание того, о чем в нем говориться.


\secrel{Структура книги}

В отличие от большинства учебников, эта книга не следует подходу "сверху вниз".
Скорее она имеет форму повествования с возвратами к предыдущим темам.
Мы часто будет строить программы инкрементно, так же как мы бы это делали в
парном программировании. Наш код будет включать ошибки, не потому что мы не
знаем правильный ответ, но потому что это лучший способ научить вас, 
углубляясь от поверхностного в детали. Включение намеренных ошибок делает
невозможным для вас читать материал пассивно: вы должны взаимодействовать с
ним, потому что вы никогда не можете быть уверены в правильности того что вы
читаете.

В конце концов вы всегда получите правильный ответ. Тем не меннее, это
нелинейное повествование немного раздражающе в краткосрочной перспективе (у вас
всегда будет соблазн сказать "Скажите же мне наконец ответ !"), и это также
делает эту книгу плохим справочником (вы не можете открыть произвольную
страницу и быть уверенным что на ней написана правда). Тем не менее, это чувство
разочарования\ --- ощущение обучения. Я не знаю другого способа.

\bigskip
В некоторых местах вы встретите следующие выделения:

\Exercise{Это упражнение. Попробуйте это сделать.}

Это традиционное для учебников упражнение.
Это то, что вам нужно сделать по своему усмотрению.
Если вы используете эту книгу как часть курса, это упражнение хорошо
задавать как домашнюю работу. В противоположность этому вы также можете найти
подобные вопросы, выделенные как 

\DoNow{Здесь предполагаются немедленные действия. Вы видите это ?}

Когда вы доберетесь до одного из этих блоков, остановитесь. Прочитайте,
подумайте, сформулируйте ответ перед тем как продолжить чтение. Вы должны
сделать это потому что это действительно упражнение, но ответ уже есть в книге,
чаще всего в тексте непосредственно после упражнения (т.е. в части которую вы
сейчас читаете) или это что-то, что вы можете получить самостоятельно,
запустив программу. Если вы просто продолжите читать, то вы увидете ответ без
его обдумывания (или не увидите его вообще, если это инструкции по запуску
программы), так что вы ни (а) проверите свои знания, ни (б) улучшите свое
понимание. Другими словами, это дополнительные, явные попытки стимулировать
ваше активное обучение. В конце концов, я могу только поощрять вас работать;
решение применять это или нет остается за вами.


\secrel{Язык программирования используемый в книге}

Язык программирования используемый в книге\ --- 
\href{http://www.racket-lang.org/}{Racket}. Аналогично операционным системам,
Racket-система является исполняющей средой для целого ряда языков
программирования, так что \emph{вы должны указать Racketу на каком языке 
вы программируете}. Например, в Unix вы указываете в строку типа

\begin{verbatim}
#!/bin/sh
\end{verbatim}

в первой строке shell-скрипта; вы указываете веб-браузеру
тип документа, добавляя заголовок

\begin{verbatim}
<!DOCTYPE HTML PUBLIC "-//W3C//DTD HTML 4.01//EN" ...>
\end{verbatim}

Аналогично, Racket требует от вас указать какой язык вы будете использовать.
Диалекты языков Racket имеют тот же скобочный синтаксис, что и сам Racket,
но другую семантику; ту же семантику но другой синтаксис; или и то и то.
Так что каждая программа, которую может выполнять Racket-система, начинается со
строки \#lang за которой следует имя диалекта языка: по умолчанию, 
это оригинальный Racket (указыватся как \verb|racket|). В этой книге мы 
почти всегда будем использовать диалект\note{В DrRacket v.6.6,
выберите меню \menu{Язык > Выбрать язык\ldots > Start your program with \#lang
to specify the desired dialect}.}

\begin{verbatim}
#lang plai-typed
\end{verbatim}

Когда мы будем отклоняться от этого правила, это будет указано особо, так что
если не указано иное, добавляйте заголовок \verb|#lang plai-typed| в начало
каждого файла программы (предполагается что я тоже это сделал).



\secup

%\refstepcounter{chapter}
\secrel{Everything (We Will Say) About Parsing
\ru{Все (что мы будем говорить) о разборе}}\secdown
\clearpage

Parsing is the act of turning an input character stream into a more structured,
internal representation.
\ru{\termdef{Парсинг}{парсинг} или \termdef{разбор}{синтаксический разбор}\
--- процесс превращения входного потока одиночных символов в более
структурированное внутреннее представление\note{программы или данных, заданных в
текстовой синтаксической форме}.}
A common internal representation is as a tree, which programs can recursively
process.
\ru{Обычно используется внутреннее представление в виде дерева, которое может
быть обработано программой рекурсивно.}

For instance, given the stream
\ru{Например, для входного потока символов\note{включая пробелы, табуляции и
концы строк}}
\begin{verbatim}
23 + 5 - 6
\end{verbatim}
we might want a tree representing addition whose left (L) node represents the
number 23 and whose right (R) node represents subtraction of 6 from 5.
\ru{мы хотим получить деревянное представление сложения, в котором левая (L)
ветвь содержит число 23, а правая (R)\ --- вложенное представление вычитания 6
из 5.}
A parser is responsible for performing this transformation.
\ru{Парсер отвечает за выполнение такой транформации.}

\noindent
\begin{tabular}{c c}
\noindent\includegraphics[height=0.8\textheight]{tmp/2_p10_R.pdf}
&
\noindent\includegraphics[height=0.8\textheight]{tmp/2_p10_L.pdf}
\\
\emph{Право}ассоциативный разбор
&
(*) \emph{Лево}ассоциативный разбор
\\
\end{tabular}\bigskip

Parsing is a large, complex problem that is far from solved due to the
difficulties of ambiguity.
\ru{Парсинг\ --- большая проблема информатики, сложность которой
заключается в трудностях неоднозначности.}
For instance, an alternate parse tree (*) for the above input expression might
put subtraction at the top and addition below it.
\ru{Например, существует альтернативное (*) \termdef{дерево разбора}{дерево
разбора} для того же входного выражения, в котором мы можем поместить
вычитание на вершину дерева, а сложение будет вложенным поддеревом.}
We might also want to consider whether this addition operation is commutative
and hence whether the order of arguments can be switched.
\ru{Нас также может интересовать, является ли это сложение коммутативной
операцией, то есть можем ли мы изменить порядок аргументов\note{например для
оптимизации кода}.}
Everything only gets much, much worse when we get to full-fledged programming
languages (to say nothing of natural languages).
\ru{Все становится намного хуже по мере того, как мы пробираемся в сторону
полноценных языков программирования (даже не говоря о натуральных языках).}

\secrel{2.1 A Lightweight, Built-In First Half of a Parser . . . . . . . . . . . . . 10}

\input{2_2_shortcut}
\input{2_3_types}
\secrel{Completing the Parser \ru{Заканчиваем с парсером}}\label{sec2_4}

In principle, we can think of \ru{В принципе, мы можем думать о} \verb|read|\
as a complete parser \ru{как о законченном парсере}.
However, its output is generic \ru{Тем не менее, его вывод все еще сырой}:
it represents the token structure without offering any comment on its intent.
\ru{он содержит структуру токенов не предлагая каких-либо комментариев об их
назначении.}
We would instead prefer to have a representation that tells us something about
the \emph{intended meaning} of the terms in our language, just as we wrote at
the very beginning: “representing addition”, “represents a number”, and so on.
\ru{Вместо этого мы предпочли бы иметь представление, которое говорит нам
что-то о \emph{предполагаемом значении} термов нашего языка, так же как мы
писали в самом начале: ``представление сложения'', ``представление числа'' и
так далее.}

To do this, we must first introduce a datatype that captures this
representation. \ru{Чтобы сделать это, мы сначала введем тип данных, который
зафиксирует это представление.} We will separately discuss \ru{Мы отдельно
рассмотрим} (section \ru{в разделе} \ref{sec31}) how and why we obtained this
datatype \ru{как и зачем мы применяем этот тип}, but for now let’s say it’s
given to us \ru{но сейчас пока будем считать, что он нам задан}:
\lstx{ArithC.rkt}{src/2/p12_1.rkt}{rkt}
We now need a function that will convert s-expressions into instances of this
datatype.
\ru{Теперь нам нужна функция, которая преобразует s-выражение в структуру из
экземпляров этого типа.}
This is the other half of our parser \ru{Это вторая половина нашего парсера}:
\lstx{ArithC.rkt}{src/2/p12_2.rkt}{rkt}

Thus\note{typing in \racket\ console \emph{after program run}} \ru{Таким
образом\note{\ru{введя выражение в \racket-консоли \emph{после выполнения
программы}}}}
\begin{verbatim}
> (parse '(+ (* 1 2) (+ 2 3)))
- ArithC
(plusC
    (multC (numC 1) (numC 2))
    (plusC (numC 2) (numC 3)))
\end{verbatim}
\lstx{ArithC.rkt}{src/2/p12_3.rkt}{rkt}
\begin{verbatim}
(good
  (parse '(+ (* 1 2) (+ 2 3)))
  (plusC (multC (numC 1) (numC 2)) (plusC (numC 2) (numC 3)))
  (plusC (multC (numC 1) (numC 2)) (plusC (numC 2) (numC 3)))
  "at line 26")
\end{verbatim}

Congratulations\,! \ru{Мои поздравления\,!}
You have just completed your first representation of a program.
\ru{Вы только что завершили ваше первое представление программы.}
From now on we can focus entirely on programs represented as recursive trees,
ignoring the vagaries of surface syntax and how to get them into the tree form.
\ru{С этого момента мы можем полностью сосредоточиться на программах,
представленных в виде рекурсивных деревьев, не обращая внимания на капризы
наносного синтаксиса и процесс получения из него дерева разбора.}
We’re finally ready to start studying programming languages\,!
\ru{Мы, наконец, готовы приступить к изучению языков программирования\,!}

\Exercise{
What happens if you forget to quote the argument to the 
\ru{Что случиться, если вы забудете заквотить аргумент вызова}
parser\,?
Why\,? \ru{Почему\,?}
}
\input{2_5_coda}
\secup

%\refstepcounter{chapter}
\secrel{A First Look at Interpretation \ru{Первый взгляд на
интерпретацию}}\secdown

Now that we have a representation of programs, there are many ways in which we
might want to manipulate them.
\ru{Теперь, когда мы имеем представление программ, существует множество
способов, которыми мы можем манипулировать ими.}
We might want to display a
program in an attractive way \ru{Мы можем захотеть выводить листинг программы в
красивом виде} (“pretty-print”), convert into code in some other format
\ru{преобразовать в код в какой-то другой формат} (“compilation”
\ru{``компиляция''/''трансляция''}), ask whether it obeys certain properties 
\ru{убедиться что она отвечает определенным требованиям}
(“verification” \ru{``верификация''}), and so on \ru{и так далее}.
For now, we’re going to focus on asking what value it corresponds to
(“\termdef{e\underline{val}uation}{evaluation}”\ --- the reduction of programs
to \emph{\underline{val}ues}).
\ru{Для начала, мы собираемся сфокусироваться на вопросе\ --- какому значению
соответствует программа (“\termdef{вычисление}{вычисление}"\ --- редукция
программы до \emph{значения})}

Let’s write an evaluator, in the form of an \termdef{interpreter}{interpreter},
for our arithmetic language.
\ru{Давайте напишем вычислитель, в форме
\termdef{интерпретатора}{интерпретатор}, для нашего арифметического языка.}
We choose arithmetic first for three reasons \ru{Мы выбрали арифметику прежде
всего по следующим трем причинам}:
\begin{itemize}[nosep]
  \item[(a)]
  you already know how it works, so we can focus on the mechanics of writing
evaluators;
\ru{вы уже знаете как работает арифметика, и мы можем сфокусироваться на
механике написания вычислителей;}
  \item[(b)]
  it’s contained in
every language we will encounter later, so we can build upwards and outwards from it;
\ru{она содержится в каждом языке, с которым мы столкнемся в дальнейшем, так
что мы будем расширять этот арифметический язык вверх и вширь;} and \ru{и}
  \item[(c)] 
it’s at once both small and big enough to illustrate many points
we’d like to get across.
\ru{этот язык минималистичен, но при этом достаточно большой, чтобы
проиллюстрировать многие моменты, которые мы хочем до вас донести.}
\end{itemize}

\secrel{Representing Arithmetic \ru{Представление арифметики}}\label{sec31}

\input{3_2_interp}
\input{3_3_notice}
\secrel{3.4 Growing the Language . . . . . . . . . . . . . . . . . . . . . . . . . 16}

\secup

% \refstepcounter{chapter}

\secrel{4 A First Taste of Desugaring 16}\secdown
\secrel{4.1 Extension: Binary Subtraction . . . . . . . . . . . . . . . . . . . . . 17}
\secrel{4.2 Extension: Unary Negation . . . . . . . . . . . . . . . . . . . . . . . 18}
\secup

\secrel{5 Adding Functions to the Language 19}\secdown
\secrel{5.1 Defining Data Representations . . . . . . . . . . . . . . . . . . . . . 19}
\secrel{5.2 Growing the Interpreter . . . . . . . . . . . . . . . . . . . . . . . . . 21}
\secrel{5.3 Substitution . . . . . . . . . . . . . . . . . . . . . . . . . . . . . . . 22}
\secrel{5.4 The Interpreter, Resumed . . . . . . . . . . . . . . . . . . . . . . . . 23}
\secrel{5.5 Oh Wait, There’s More! . . . . . . . . . . . . . . . . . . . . . . . . . 25}
\secup

\secrel{6 From Substitution to Environments 25}\secdown
\secrel{6.1 Introducing the Environment . . . . . . . . . . . . . . . . . . . . . . 26}
\secrel{6.2 Interpreting with Environments . . . . . . . . . . . . . . . . . . . . . 27}
\secrel{6.3 Deferring Correctly . . . . . . . . . . . . . . . . . . . . . . . . . . . 29}
\secrel{6.4 Scope . . . . . . . . . . . . . . . . . . . . . . . . . . . . . . . . . . 30}
\secdown
\secrel{6.4.1 How Bad Is It? . . . . . . . . . . . . . . . . . . . . . . . . . 30}
\secrel{6.4.2 The Top-Level Scope . . . . . . . . . . . . . . . . . . . . . . 31}
\secup
\secrel{6.5 Exposing the Environment . . . . . . . . . . . . . . . . . . . . . . . 31}
\secup

\secrel{7 Functions Anywhere 31}\secdown
\secrel{7.1 Functions as Expressions and Values . . . . . . . . . . . . . . . . . . 32}
\secrel{7.2 Nested What? . . . . . . . . . . . . . . . . . . . . . . . . . . . . . . 35}
\secrel{7.3 Implementing Closures . . . . . . . . . . . . . . . . . . . . . . . . . 37}
\secrel{7.4 Substitution, Again . . . . . . . . . . . . . . . . . . . . . . . . . . . 38}
\secrel{7.5 Sugaring Over Anonymity . . . . . . . . . . . . . . . . . . . . . . . 39}
\secup

\secrel{8 Mutation: Structures and Variables 41}\secdown
\secrel{8.1 Mutable Structures . . . . . . . . . . . . . . . . . . . . . . . . . . . 41}
\secdown
\secrel{8.1.1 A Simple Model of Mutable Structures . . . . . . . . . . . . 41}
\secrel{8.1.2 Scaffolding . . . . . . . . . . . . . . . . . . . . . . . . . . . 42}
\secrel{8.1.3 Interaction with Closures . . . . . . . . . . . . . . . . . . . . 43}
\secrel{8.1.4 Understanding the Interpretation of Boxes . . . . . . . . . . . 44}
\secrel{8.1.5 Can the Environment Help? . . . . . . . . . . . . . . . . . . 46}
\secrel{8.1.6 Introducing the Store . . . . . . . . . . . . . . . . . . . . . . 48}
\secrel{8.1.7 Interpreting Boxes . . . . . . . . . . . . . . . . . . . . . . . 49}
\secrel{8.1.8 The Bigger Picture . . . . . . . . . . . . . . . . . . . . . . . 54}
\secup
\secrel{8.2 Variables . . . . . . . . . . . . . . . . . . . . . . . . . . . . . . . . 57}
\secdown
\secrel{8.2.1 Terminology . . . . . . . . . . . . . . . . . . . . . . . . . . 57}
\secrel{8.2.2 Syntax . . . . . . . . . . . . . . . . . . . . . . . . . . . . . 57}
\secrel{8.2.3 Interpreting Variables . . . . . . . . . . . . . . . . . . . . . . 58}
\secup
\secrel{8.3 The Design of Stateful Language Operations . . . . . . . . . . . . . . 59}
\secrel{8.4 Parameter Passing . . . . . . . . . . . . . . . . . . . . . . . . . . . . 60}
\secup

\secrel{9 Recursion and Cycles: Procedures and Data 62}\secdown
\secrel{9.1 Recursive and Cyclic Data . . . . . . . . . . . . . . . . . . . . . . . 62}
\secrel{9.2 Recursive Functions . . . . . . . . . . . . . . . . . . . . . . . . . . . 64}
\secrel{9.3 Premature Observation . . . . . . . . . . . . . . . . . . . . . . . . . 65}
\secrel{9.4 Without Explicit State . . . . . . . . . . . . . . . . . . . . . . . . . . 66}
\secup

\secrel{10 Objects 67}\secdown
\secrel{10.1 Objects Without Inheritance . . . . . . . . . . . . . . . . . . . . . . 67}
\secdown
\secrel{10.1.1 Objects in the Core . . . . . . . . . . . . . . . . . . . . . . . 68}
\secrel{10.1.2 Objects by Desugaring . . . . . . . . . . . . . . . . . . . . . 69}
\secrel{10.1.3 Objects as Named Collections . . . . . . . . . . . . . . . . . 69}
\secrel{10.1.4 Constructors . . . . . . . . . . . . . . . . . . . . . . . . . . 70}
\secrel{10.1.5 State . . . . . . . . . . . . . . . . . . . . . . . . . . . . . . 71}
\secrel{10.1.6 Private Members . . . . . . . . . . . . . . . . . . . . . . . . 71}
\secrel{10.1.7 Static Members . . . . . . . . . . . . . . . . . . . . . . . . . 72}
\secrel{10.1.8 Objects with Self-Reference . . . . . . . . . . . . . . . . . . 72}
\secrel{10.1.9 Dynamic Dispatch . . . . . . . . . . . . . . . . . . . . . . . 74}
\secup
\secrel{10.2 Member Access Design Space . . . . . . . . . . . . . . . . . . . . . 75}
\secrel{10.3 What (Goes In) Else? . . . . . . . . . . . . . . . . . . . . . . . . . . 75}
\secdown
\secrel{10.3.1 Classes . . . . . . . . . . . . . . . . . . . . . . . . . . . . . 76}
\secrel{10.3.2 Prototypes . . . . . . . . . . . . . . . . . . . . . . . . . . . 78}
\secrel{10.3.3 Multiple Inheritance . . . . . . . . . . . . . . . . . . . . . . 78}
\secrel{10.3.4 Super-Duper! . . . . . . . . . . . . . . . . . . . . . . . . . . 79}
\secrel{10.3.5 Mixins and Traits . . . . . . . . . . . . . . . . . . . . . . . . 79}
\secup
\secup

\secrel{11 Memory Management 81}\secdown
\secrel{11.1 Garbage . . . . . . . . . . . . . . . . . . . . . . . . . . . . . . . . . 81}
\secrel{11.2 What is “Correct” Garbage Recovery? . . . . . . . . . . . . . . . . . 81}
\secrel{11.3 Manual Reclamation . . . . . . . . . . . . . . . . . . . . . . . . . . 82}
\secdown
\secrel{11.3.1 The Cost of Fully-Manual Reclamation . . . . . . . . . . . . 82}
\secrel{11.3.2 Reference Counting . . . . . . . . . . . . . . . . . . . . . . 83}
\secup
\secrel{11.4 Automated Reclamation, or Garbage Collection . . . . . . . . . . . . 84}
\secdown
\secrel{11.4.1 Overview . . . . . . . . . . . . . . . . . . . . . . . . . . . . 84}
\secrel{11.4.2 Truth and Provability . . . . . . . . . . . . . . . . . . . . . . 85}
\secrel{11.4.3 Central Assumptions . . . . . . . . . . . . . . . . . . . . . . 85}
\secup
\secrel{11.5 Convervative Garbage Collection . . . . . . . . . . . . . . . . . . . . 86}
\secrel{11.6 Precise Garbage Collection . . . . . . . . . . . . . . . . . . . . . . . 87}
\secup

\secrel{12 Representation Decisions 87}\secdown
\secrel{12.1 Changing Representations . . . . . . . . . . . . . . . . . . . . . . . 87}
\secrel{12.2 Errors . . . . . . . . . . . . . . . . . . . . . . . . . . . . . . . . . . 89}
\secrel{12.3 Changing Meaning . . . . . . . . . . . . . . . . . . . . . . . . . . . 89}
\secrel{12.4 One More Example . . . . . . . . . . . . . . . . . . . . . . . . . . . 90}
\secup

\secrel{13 Desugaring as a Language Feature 91}\secdown
\secrel{13.1 A First Example . . . . . . . . . . . . . . . . . . . . . . . . . . . . . 91}
\secrel{13.2 Syntax Transformers as Functions . . . . . . . . . . . . . . . . . . . 93}
\secrel{13.3 Guards . . . . . . . . . . . . . . . . . . . . . . . . . . . . . . . . . . 95}
\secrel{13.4 Or: A Simple Macro with Many Features . . . . . . . . . . . . . . . 95}
\secdown
\secrel{13.4.1 A First Attempt . . . . . . . . . . . . . . . . . . . . . . . . . 95}
\secrel{13.4.2 Guarding Evaluation . . . . . . . . . . . . . . . . . . . . . . 97}
\secrel{13.4.3 Hygiene . . . . . . . . . . . . . . . . . . . . . . . . . . . . . 98}
\secup
\secrel{13.5 Identifier Capture . . . . . . . . . . . . . . . . . . . . . . . . . . . . 99}
\secrel{13.6 Influence on Compiler Design . . . . . . . . . . . . . . . . . . . . . 101}
\secrel{13.7 Desugaring in Other Languages . . . . . . . . . . . . . . . . . . . . 101}
\secup

\secrel{14 Control Operations 102}\secdown
\secrel{14.1 Control on the Web . . . . . . . . . . . . . . . . . . . . . . . . . . . 102}
\secdown
\secrel{14.1.1 Program Decomposition into Now and Later . . . . . . . . . 104}
\secrel{14.1.2 A Partial Solution . . . . . . . . . . . . . . . . . . . . . . . . 104}
\secrel{14.1.3 Achieving Statelessness . . . . . . . . . . . . . . . . . . . . 106}
\secrel{14.1.4 Interaction with State . . . . . . . . . . . . . . . . . . . . . . 107}
\secup
\secrel{14.2 Continuation-Passing Style . . . . . . . . . . . . . . . . . . . . . . . 109}
\secdown
\secrel{14.2.1 Implementation by Desugaring . . . . . . . . . . . . . . . . . 110}
\secrel{14.2.2 Converting the Example . . . . . . . . . . . . . . . . . . . . 114}
\secrel{14.2.3 Implementation in the Core . . . . . . . . . . . . . . . . . . 115}
\secup
\secrel{14.3 Generators . . . . . . . . . . . . . . . . . . . . . . . . . . . . . . . . 117}
\secdown
\secrel{14.3.1 Design Variations . . . . . . . . . . . . . . . . . . . . . . . . 117}
\secrel{14.3.2 Implementing Generators . . . . . . . . . . . . . . . . . . . . 119}
\secup
\secrel{14.4 Continuations and Stacks . . . . . . . . . . . . . . . . . . . . . . . . 121}
\secrel{14.5 Tail Calls . . . . . . . . . . . . . . . . . . . . . . . . . . . . . . . . 123}
\secrel{14.6 Continuations as a Language Feature . . . . . . . . . . . . . . . . . . 124}
\secdown
\secrel{14.6.1 Presentation in the Language . . . . . . . . . . . . . . . . . . 125}
\secrel{14.6.2 Defining Generators . . . . . . . . . . . . . . . . . . . . . . 126}
\secrel{14.6.3 Defining Threads . . . . . . . . . . . . . . . . . . . . . . . . 127}
\secrel{14.6.4 Better Primitives for Web Programming . . . . . . . . . . . . 131}
\secup
\secup

\secrel{15 Checking Program Invariants Statically: Types 131}\secdown
\secrel{15.1 Types as Static Disciplines . . . . . . . . . . . . . . . . . . . . . . . 133}
\secrel{15.2 A Classical View of Types . . . . . . . . . . . . . . . . . . . . . . . 134}
\secdown
\secrel{15.2.1 A Simple Type Checker . . . . . . . . . . . . . . . . . . . . 134}
\secrel{15.2.2 Type-Checking Conditionals . . . . . . . . . . . . . . . . . . 139}
\secrel{15.2.3 Recursion in Code . . . . . . . . . . . . . . . . . . . . . . . 139}
\secrel{15.2.4 Recursion in Data . . . . . . . . . . . . . . . . . . . . . . . . 142}
\secrel{15.2.5 Types, Time, and Space . . . . . . . . . . . . . . . . . . . . 144}
\secrel{15.2.6 Types and Mutation . . . . . . . . . . . . . . . . . . . . . . . 146}
\secrel{15.2.7 The Central Theorem: Type Soundness . . . . . . . . . . . . 147}
\secup
\secrel{15.3 Extensions to the Core . . . . . . . . . . . . . . . . . . . . . . . . . 148}
\secdown
\secrel{15.3.1 Explicit Parametric Polymorphism . . . . . . . . . . . . . . . 148}
\secrel{15.3.2 Type Inference . . . . . . . . . . . . . . . . . . . . . . . . . 155}
\secrel{15.3.3 Union Types . . . . . . . . . . . . . . . . . . . . . . . . . . 164}
\secrel{15.3.4 Nominal Versus Structural Systems . . . . . . . . . . . . . . 170}
\secrel{15.3.5 Intersection Types . . . . . . . . . . . . . . . . . . . . . . . 171}
\secrel{15.3.6 Recursive Types . . . . . . . . . . . . . . . . . . . . . . . . 172}
\secrel{15.3.7 Subtyping . . . . . . . . . . . . . . . . . . . . . . . . . . . . 173}
\secrel{15.3.8 Object Types . . . . . . . . . . . . . . . . . . . . . . . . . . 176}
\secup
\secup

\secrel{16 Checking Program Invariants Dynamically: Contracts 179}\secdown
\secrel{16.1 Contracts as Predicates . . . . . . . . . . . . . . . . . . . . . . . . . 181}
\secrel{16.2 Tags, Types, and Observations on Values . . . . . . . . . . . . . . . . 182}
\secrel{16.3 Higher-Order Contracts . . . . . . . . . . . . . . . . . . . . . . . . . 183}
\secrel{16.4 Syntactic Convenience . . . . . . . . . . . . . . . . . . . . . . . . . 187}
\secrel{16.5 Extending to Compound Data Structures . . . . . . . . . . . . . . . . 188}
\secrel{16.6 More on Contracts and Observations . . . . . . . . . . . . . . . . . . 189}
\secrel{16.7 Contracts and Mutation . . . . . . . . . . . . . . . . . . . . . . . . . 189}
\secrel{16.8 Combining Contracts . . . . . . . . . . . . . . . . . . . . . . . . . . 190}
\secrel{16.9 Blame . . . . . . . . . . . . . . . . . . . . . . . . . . . . . . . . . . 191}
\secup

\secrel{17 Alternate Application Semantics 195}\secdown
\secrel{17.1 Lazy Application . . . . . . . . . . . . . . . . . . . . . . . . . . . . 196}
\secdown
\secrel{17.1.1 A Lazy Application Example . . . . . . . . . . . . . . . . . . 196}
\secrel{17.1.2 What Are Values? . . . . . . . . . . . . . . . . . . . . . . . 197}
\secrel{17.1.3 What Causes Evaluation? . . . . . . . . . . . . . . . . . . . 198}
\secrel{17.1.4 An Interpreter . . . . . . . . . . . . . . . . . . . . . . . . . . 199}
\secrel{17.1.5 Laziness and Mutation . . . . . . . . . . . . . . . . . . . . . 201}
\secrel{17.1.6 Caching Computation . . . . . . . . . . . . . . . . . . . . . 201}
\secup
\secrel{17.2 Reactive Application . . . . . . . . . . . . . . . . . . . . . . . . . . 201}
\secdown
\secrel{17.2.1 Motivating Example: A Timer . . . . . . . . . . . . . . . . . 202}
\secrel{17.2.2 Callback Types are Four-Letter Words . . . . . . . . . . . . . 203}
\secrel{17.2.3 The Alternative: Reactive Languages . . . . . . . . . . . . . 204}
\secrel{17.2.4 Implementing Transparent Reactivity . . . . . . . . . . . . . 205}
\secup
\secup



\clearpage
\addcontentsline{toc}{chapter}{Index \ru{Предметный указатель}}\printindex

\end{document}
