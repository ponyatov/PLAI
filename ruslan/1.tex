\secrel{Part 1}

“If you don’t know how compilers work, then you don’t know how computers work.
If you’re not 100\% sure whether you know how compilers work, then you don’t
know how they work.”\ --- Steve Yegge

There you have it. Think about it. It doesn’t really matter whether you’re a
newbie or a seasoned software developer: if you don’t know how compilers and
interpreters work, then you don’t know how computers work. It’s that simple.

So, do you know how compilers and interpreters work? And I mean, are you 100\%
sure that you know how they work? If you don’t. 
\begin{framed}\end{framed}
Or if you don’t and you’re really agitated about it. 
\begin{framed}\end{framed}

Do not worry. If you stick around and work through the series and build an
interpreter and a compiler with me you will know how they work in the end. And
you will become a confident happy camper too. At least I hope so.
\begin{framed}\end{framed}

Why would you study interpreters and compilers? I will give you three reasons.
\begin{enumerate}
  \item 
To write an interpreter or a compiler you have to have a lot of technical skills
that you need to use together. Writing an interpreter or a compiler will help
you improve those skills and become a better software developer. As well, the
skills you will learn are useful in writing any software, not just interpreters
or compilers.
  \item 
You really want to know how computers work. Often interpreters and compilers
look like magic. And you shouldn’t be comfortable with that magic. You want to
demystify the process of building an interpreter and a compiler, understand how
they work, and get in control of things.
  \item 
You want to create your own programming language or domain specific language. If
you create one, you will also need to create either an interpreter or a compiler
for it. Recently, there has been a resurgence of interest in new programming
languages. And you can see a new programming language pop up almost every day:
Elixir, Go, Rust just to name a few.
\end{enumerate}

Okay, but what are interpreters and compilers?

The goal of an interpreter or a compiler is to translate a source program in
some high-level language into some other form. Pretty vague, isn’t it? Just bear
with me, later in the series you will learn exactly what the source program is
translated into.

At this point you may also wonder what the difference is between an interpreter
and a compiler. For the purpose of this series, let’s agree that if a translator
translates a source program into machine language, it is a compiler. If a
translator processes and executes the source program without translating it into
machine language first, it is an interpreter. Visually it looks something like
this:
\begin{framed}\end{framed}

I hope that by now you’re convinced that you really want to study and build an
interpreter and a compiler. What can you expect from this series on
interpreters?

Here is the deal. You and I are going to create a simple interpreter for a large
subset of Pascal language. At the end of this series you will have a working
Pascal interpreter and a source-level debugger like Python’s pdb.

You might ask, why Pascal? For one thing, it’s not a made-up language that I
came up with just for this series: it’s a real programming language that has
many important language constructs. And some old, but useful, CS books use
Pascal programming language in their examples (I understand that that’s not a
particularly compelling reason to choose a language to build an interpreter for,
but I thought it would be nice for a change to learn a non-mainstream language
:)

Here is an example of a factorial function in Pascal that you will be able to
interpret with your own interpreter and debug with the interactive source-level
debugger that you will create along the way:
\lstx{program factorial}{ruslan/pas.pas}{Pascal}

The implementation language of the Pascal interpreter will be Python, but you
can use any language you want because the ideas presented don’t depend on any
particular implementation language. Okay, let’s get down to business. Ready,
set, go!

You will start your first foray into interpreters and compilers by writing a
simple interpreter of arithmetic expressions, also known as a calculator. Today
the goal is pretty minimalistic: to make your calculator handle the addition of
two single digit integers like $3+5$. Here is the source code for your
calculator, sorry, interpreter:
\lstx{}{ruslan/py.py}{Python}

Save the above code into calc1.py file or download it directly from GitHub.
Before you start digging deeper into the code, run the calculator on the command
line and see it in action. Play with it! Here is a sample session on my laptop
(if you want to run the calculator under Python3 you will need to replace
raw\_input with input):
\lstx{}{ruslan/py.py}{Python}

For your simple calculator to work properly without throwing an exception, your
input needs to follow certain rules:
\begin{itemize}[nosep]
  \item 
Only single digit integers are allowed in the input
  \item 
The only arithmetic operation supported at the moment is addition
  \item 
No whitespace characters are allowed anywhere in the input
\end{itemize}

Those restrictions are necessary to make the calculator simple. Don’t worry,
you’ll make it pretty complex pretty soon.

Okay, now let’s dive in and see how your interpreter works and how it evaluates
arithmetic expressions.

When you enter an expression $3+5$ on the command line your interpreter gets a
string “3+5”. In order for the interpreter to actually understand what to do
with that string it first needs to break the input “3+5” into components called
tokens. A token is an object that has a type and a value. For example, for the
string “3” the type of the token will be INTEGER and the corresponding value
will be integer 3.

The process of breaking the input string into tokens is called lexical analysis.
So, the first step your interpreter needs to do is read the input of characters
and convert it into a stream of tokens. The part of the interpreter that does it
is called a lexical analyzer, or lexer for short. You might also encounter other
names for the same component, like scanner or tokenizer. They all mean the same:
the part of your interpreter or compiler that turns the input of characters into
a stream of tokens.

The method \verb|get_next_token| of the Interpreter class is your lexical
analyzer. Every time you call it, you get the next token created from the input
of characters passed to the interpreter. Let’s take a closer look at the method
itself and see how it actually does its job of converting characters into
tokens. The input is stored in the variable text that holds the input string and
pos is an index into that string (think of the string as an array of
characters). pos is initially set to 0 and points to the character ‘3’. The
method first checks whether the character is a digit and if so, it increments
pos and returns a token instance with the type INTEGER and the value set to the
integer value of the string ‘3’, which is an integer 3:

\img{ruslan/png.png}{height=0.2\textheight}