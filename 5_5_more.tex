\secrel{Oh Wait, There’s More\,!\\\ru{Подождите, есть кое-что еще\,!}}

Earlier, we gave the following type to subst:
\lsts{src/5/5_5_1.rkt}{rkt}
Sticking to surface syntax for brevity, suppose we apply double to (+ 1 2). This
would substitute (+ 1 2) for each x, resulting in the following expression—(+ (+
1 2) (+ 1 2))—for interpretation. Is this necessarily what we want?

When you learned algebra in school, you may have been taught to do this
differently: first reduce the argument to an answer (in this case, 3), then
substitute the answer for the parameter. This notion of substitution might have
the following type instead:
\lsts{src/5/5_5_2.rkt}{rkt}
Careful now: we can’t put raw numbers inside expressions, so we’d have to
constantly wrap the number in an invocation of numC. Thus, it would make sense
for subst to have a helper that it invokes after wrapping the first parameter.
(In fact, our existing subst would be a perfectly good candidate: because it
accepts any ExprC in the first parameter, it will certainly work just fine with
a numC.)
\note{In fact, we don’t even have substitution quite right! The version of
substitution we have doesn’t scale past this language due to a subtle problem
known as “name capture”. Fixing substitution is complex, subtle, and an exciting
intellectual endeavor, but it’s not the direction I want to go in here. We’ll
instead sidestep this problem in this book. If you’re interested, however, read
about the lambda calculus, which provides the tools for defining substitution
correctly.}

\Exercise{
Modify your interpreter to substitute names with answers, not expressions.
}

We’ve actually stumbled on a profound distinction in programming languages. The
act of evaluating arguments before substituting them in functions is called
eager application, while that of deferring evaluation is called lazy—and has
some variations. For now, we will actually prefer the eager semantics, because
this is what most mainstream languages adopt. Later \ref{}, we will return to
talking about the lazy application semantics and its implications.
