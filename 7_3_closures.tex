\secrel{7.3 Implementing Closures   . 37}

We need to change our representation of values to record closures rather than raw
function text:
\lsts{src/7/7_3_1.rkt}{rkt}
While we’re at it, we might as well alter our syntax for defining functions to drop
the useless name. This construct is historically called a lambda:
\lsts{src/7/7_3_2.rkt}{rkt}

When encountering a function definition, the interpreter must now remember to
save the substitutions that have been applied so far:
\note{“Save the environment! Create a closure today!”\ --- Cormac Flanagan}
\lsts{src/7/7_3_3.rkt}{rkt}

This saved set, not the empty environment, must be used when applying a
function:
\lsts{src/7/7_3_4.rkt}{rkt}

There’s actually another possibility: we could use the environment present at the
point of application:
\lsts{src/7/7_3_5.rkt}{rkt}

\Exercise{
What happens if we extend the dynamic environment instead?
}

In retrospect, it becomes even more clear why we interpreted the body of a
function in the empty environment. When a function is defined at the top-level,
it is not “closed over” any identifiers. Therefore, our previous function
applications have been special cases of this form of application.
